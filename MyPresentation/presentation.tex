\ifdefined\ishandout
  %% \documentclass[handout,xcolor=dvipsnames]{beamer}
  \documentclass[xcolor=dvipsnames]{beamer}
\else
  \documentclass[xcolor=dvipsnames]{beamer}
\fi

\mode<presentation>
{
%% \usetheme{Frankfurt}
%% \useoutertheme{split}
	%% \usetheme{CambridgeUS}
	%% \usetheme{Hannover}
  \usetheme{Singapore}
	\usecolortheme{rose}
}
\usepackage{amsmath,stmaryrd}
\usepackage{amsfonts}
\usepackage{amssymb}
\usepackage{pifont}
\usepackage{cmap}
\usepackage{listings}
\usepackage{lmodern}
\usepackage{xparse}
\usepackage{color}
\usepackage{minted}
\usepackage{graphicx}
\usepackage{tikz}
\usetikzlibrary{positioning}
\usetikzlibrary{fadings,decorations.pathmorphing,decorations.pathreplacing}
\usetikzlibrary{shapes}
\usetikzlibrary{shapes.multipart,matrix,calc}
\usetikzlibrary{arrows}
\usetikzlibrary{patterns}
\usetikzlibrary{shapes.callouts} 
\usetikzlibrary{shadows,calc}
\usepackage{wrapfig}
\usepackage{stmaryrd}
\usepackage{hyperref}
\usepackage{mathtools}
\usepackage[absolute,overlay]{textpos}
\usepackage{xspace}
\usepackage[nomessages]{fp}% http://ctan.org/pkg/fp
\usepackage{wasysym}

\usepackage[labelformat=empty]{caption}
\usepackage{fontspec}
%% \usepackage{polyglossia}
%% \setdefaultlanguage{russian}

\setmainfont[Ligatures=TeX]{CMU Serif}
\setsansfont[Ligatures=TeX]{CMU Sans Serif}
%% \setmonofont{Nimbus Mono PS}

\tikzset{
    invisible/.style={opacity=0,text opacity=0},
    visible on/.style={alt=#1{}{invisible}},
    alt/.code args={<#1>#2#3}{%
      \alt<#1>{\pgfkeysalso{#2}}{\pgfkeysalso{#3}} % \pgfkeysalso doesn't change the path
    },
}

\NewDocumentCommand{\mycallout}{r<> O{center} O{opacity=1.0,text opacity=1} m m m +m}{%
\tikz[remember picture, overlay]\node[align=#2, fill=#4!20, %text width=3cm,
#3,visible on=<#1>, rounded corners,
draw,rectangle callout,anchor=pointer,callout relative pointer={#6}
]
at #5 {#7};
}

\newcommand{\tikzmark}[1]{\tikz[overlay,remember picture,baseline=-0.5ex] \node (#1) {};}
\newcommand{\prevFr}[2]{\FPeval{#1}{clip(#2-1)}}
\newcommand{\nextFr}[2]{\FPeval{#1}{clip(#2+1)}}
\newcommand{\toGray}[3]{\only<-#1>{#3} \only<#2->{\textcolor{gray!30}{#3}}}

\newcommand{\toShadow}[3]{
  \only<#1>{#3}\only<#2>{\textcolor{colorSHADE}{#3}}
}

\definecolor{myGray}{RGB}{50,50,50}

\newcommand{\itab}[1]{\hspace{0em}\rlap{#1}}
\newcommand{\tab}[1]{\hspace{.2\textwidth}\rlap{#1}}

\newcommand{\inarr}[1]{\begin{array}{@{}l@{}}#1\end{array}}
\newcommand{\inarrc}[1]{\begin{array}{@{}c@{}}#1\end{array}}
\newcommand{\inarrIIId}[3]{\begin{array}{@{}l@{~}||@{~}l@{~}||@{~}l@{}}\inarr{#1}&\inarr{#2}&\inarr{#3}\end{array}}

\newcommand{\myblock}[1]{\begin{block}{} #1 \end{block}}


\definecolor{CommentColor}{rgb}{0.0,0.4,0.1}

\newcommand{\commenta}[1]{\color{CommentColor}{\texttt{/*}\textit{ #1 }\texttt{*/}}}
\newcommand{\comment}[1]{{\color{CommentColor}~\texttt{/\!\!/}\,{#1}}}
\newcommand{\nocomment}[1]{{\color{red}~~\texttt{/\!\!/}\,{#1}}}
\newcommand{\progval}[1]{{\color{CommentColor} #1}}

\newcommand\semState[2]{\llbracket #1 \rrbracket _{#2}}

\newcommand{\storageThreeThreads}[3]{
    \draw (0,0.5) rectangle (3,1) node [pos=.5] {Thread 1};
    \draw (3.5,0.5) rectangle (6.5 ,1) node [pos=.5] {Thread 2};
    \draw (7.0,0.5) rectangle (10.0,1) node [pos=.5] {Thread 3};
    \draw (1,0.5) -- (1, 0.5 -#1) -- (5.0, 0.5 -#1) -- (5.0,0.5);
    \draw (3.0, 0.5 -#1) -- (3.0, 0.5 -#1 - #2) -- (8.5, 0.5 -#1 - #2) -- (8.5, 0.5);
    \draw (5.75, 0.5 -#1 - #2) -- (5.75, 0.5 -#1 - #2 - #3);

    \node at (5.75, 0.3 -#1 - #2 - #3) {The main memory};
    \draw (4.75, 0.1 -#1 - #2 - #3) -- (6.75, 0.1 -#1 - #2 - #3);
}

\newcommand{\storageTwoThreads}[2]{
    \draw (0,#1) rectangle (2,#1 + 0.5) node [pos=.5] {Thread 1};
    \draw (2.5,#1) rectangle (4.5,#1 + 0.5) node [pos=.5] {Thread 2};
    \draw (1,#1) -- (1,-0.5) -- (3.5,-0.5) -- (3.5,#1);
    \draw (2.25, -0.5) -- (2.25, -#2);

    \node at (2.25, -#2 - 0.2) {The main memory};
    \draw (1.25, -#2 - 0.4) -- (3.25, -#2 - 0.4);
}

\newcommand{\executed}[1]{\colorbox{gray}{\ensuremath{#1}}}
\newcommand{\markExecuted}[3]{
  \only<-#1>{#3}
  \only<#2->{\executed{#3}}
}
\newcommand{\light}[1]{\textcolor{gray!30}{#1}}

\makeatletter
\def\ft@overlay{}

\addtobeamertemplate{footline}{}%
{%
  \lineskiplimit0pt
  \begin{tikzpicture}[remember picture,overlay]%
  \ft@overlay
  \end{tikzpicture}%
  \gdef\ft@overlay{}%
}

\newcommand<>{\addtooverlay}[1]{%
  \only#2{%
  \expandafter\gdef\expandafter\ft@overlay\expandafter{\ft@overlay #1}%
  }%
}

\def\shadowshift{3pt,-3pt}
\def\shadowradius{6pt}

\colorlet{innercolor}{black!60}
\colorlet{outercolor}{blue!05}

% this draws a shadow under a rectangle node
\newcommand\drawshadow[1]{
    \begin{pgfonlayer}{shadow}
        \shade[outercolor,inner color=innercolor,outer color=outercolor] ($(#1.south west)+(\shadowshift)+(\shadowradius/2,\shadowradius/2)$) circle (\shadowradius);
        \shade[outercolor,inner color=innercolor,outer color=outercolor] ($(#1.north west)+(\shadowshift)+(\shadowradius/2,-\shadowradius/2)$) circle (\shadowradius);
        \shade[outercolor,inner color=innercolor,outer color=outercolor] ($(#1.south east)+(\shadowshift)+(-\shadowradius/2,\shadowradius/2)$) circle (\shadowradius);
        \shade[outercolor,inner color=innercolor,outer color=outercolor] ($(#1.north east)+(\shadowshift)+(-\shadowradius/2,-\shadowradius/2)$) circle (\shadowradius);
        \shade[top color=innercolor,bottom color=outercolor] ($(#1.south west)+(\shadowshift)+(\shadowradius/2,-\shadowradius/2)$) rectangle ($(#1.south east)+(\shadowshift)+(-\shadowradius/2,\shadowradius/2)$);
        \shade[left color=innercolor,right color=outercolor] ($(#1.south east)+(\shadowshift)+(-\shadowradius/2,\shadowradius/2)$) rectangle ($(#1.north east)+(\shadowshift)+(\shadowradius/2,-\shadowradius/2)$);
        \shade[bottom color=innercolor,top color=outercolor] ($(#1.north west)+(\shadowshift)+(\shadowradius/2,-\shadowradius/2)$) rectangle ($(#1.north east)+(\shadowshift)+(-\shadowradius/2,\shadowradius/2)$);
        \shade[outercolor,right color=innercolor,left color=outercolor] ($(#1.south west)+(\shadowshift)+(-\shadowradius/2,\shadowradius/2)$) rectangle ($(#1.north west)+(\shadowshift)+(\shadowradius/2,-\shadowradius/2)$);
        %% \filldraw ($(#1.south west)+(\shadowshift)+(\shadowradius/2,\shadowradius/2)$) rectangle ($(#1.north east)+(\shadowshift)-(\shadowradius/2,\shadowradius/2)$);
    \end{pgfonlayer}
}

% create a shadow layer, so that we don't need to worry about overdrawing other things
\pgfdeclarelayer{shadow} 
\pgfsetlayers{shadow,main}

\newsavebox\mybox
\newlength\mylen

%% \newcommand\shadowimage[2][]{%
%% \begin{tikzpicture}
%% \node[anchor=south west,inner sep=0] (image) at (0,0) {\colorbox{white}{\includegraphics[#1]{#2}}};
%% \drawshadow{image}
%% \end{tikzpicture}}

\newcommand\shadowimage[2][]{%
\setbox0=\hbox{\includegraphics[#1]{#2}}
\setlength\mylen{\wd0}
\ifnum\mylen<\ht0
\setlength\mylen{\ht0}
\fi
\divide \mylen by 120
\def\shadowshift{\mylen,-\mylen}
\def\shadowradius{\the\dimexpr\mylen+\mylen+\mylen\relax}
\begin{tikzpicture}
  \node[anchor=south west,inner sep=0] (image) at (0,0) {\colorbox{white}{\includegraphics[#1]{#2}}};
  \begin{pgfonlayer}{shadow}
     \drawshadow{image}
  \end{pgfonlayer}
\end{tikzpicture}}

\makeatother

% SET OVERLAY NUMBER ORDER
\usepackage{etoolbox}
\usepackage{xstring}
\newcounter{mycounter}

% #1 is of the form <name>[=-|n], e.g. foo=- or foo=10 or foo
% Stores in #2 the overlay specification for <name> s.t. it can be appended to the content of <name>
\newcommand*{\getNewOverlayContent}[2]{%
    \getItemSpec{#1}{itemSpec}%
    \IfStrEq{\itemSpec}{-}{%
        \csedef{#2}{\arabic{beamerpauses}-}%
    }{%
        \IfStrEq{\itemSpec}{}{%
            \csedef{#2}{\arabic{beamerpauses}}%
        }{%
            \IfInteger{\itemSpec}{%
%               \mycounter=\
                \setcounter{mycounter}{\arabic{beamerpauses}}%
                \addtocounter{mycounter}{\itemSpec}%
                \addtocounter{mycounter}{-1}%
                \csedef{#2}{\arabic{beamerpauses}-\arabic{mycounter}}%
            }{%
                \PackageError{setorder}{Argument has illegal format}{Argument was #1}%
            }%
        }%
    }%
%   input: #1, itemspec:\itemSpec, beamervalue: \arabic{beamerpauses}, content: \csuse{#2} \\
}
% #1 is of the form 'foo=1' or 'foo=-' or 'foo'. 
% #2 Is the name of the macro which should hold the result
% This macro stores the part infront '=' (the name) in #2.
\newcommand*{\getItemName}[2]{% 
    \IfSubStr{#1}{=}{%
        \StrBefore{#1}{=}[\tmp]%
        \csdef{#2}{\tmp}%
    }{%
        \csdef{#2}{#1}% 
    }%
}
% #1 is of the form 'foo=1' or 'foo=-' or 'foo'. 
% #2 Is the name of the macro which should hold the result
% This macro stores the part behind '=' (the overlay spec) in #2. The stored part is empty iff there is no '=' in #1
\newcommand*{\getItemSpec}[2]{%
    \StrBehind{#1}{=}[\tmp]%
    \csdef{#2}{\tmp}%
}
% #2 is the name where content should be appended. 
% It has been ensured previously that #2 is a defined macro
% #1 is the content to append
% Depending on whether #2 is empty or not a (,) is added 
% before appending #1
\newcommand*{\appendToOverlaySpecification}[2]{%
    \IfStrEq{\csexpandonce{#2}}{}{%
        % #1 i.e. <name> is empty
        \cseappto{#2}{\csname#1\endcsname}%
    }{%
        \cseappto{#2}{,\csname#1\endcsname}%
    }%
}

\newcommand*{\setorderItem}[1]{%
    \getNewOverlayContent{#1}{overlaycontent}%
    \getItemName{#1}{cmdname}%
    \appendToOverlaySpecification{overlaycontent}{\cmdname}%
}

\newcommand*{\setorderList}[1]{%
    \forcsvlist{\setorderItem}{#1}%
    \stepcounter{beamerpauses}%
}
\newcommand*{\setorder}[1]{%    
    \clearNamesListofLists{#1}%
    \forcsvlist{\setorderList}{#1}%
    %% \createBef{#1}%
}

%% \newcommand*{\createBefItem}[1]{%
%%     \getItemName{#1}{cmdname}%
%%     \getItemName{bef#1}{cmdnameBef}%
%%     \csdef{\cmdnameBef}{}% 
%%     \prevFr{\cmdnameBef}{\cmdname}%
%% }

%% \newcommand*{\createBefList}[1]{%
%%     \forcsvlist{\createBefItem}{#1}%
%% }

%% \newcommand*{\createBef}[1]{%
%%     \forcsvlist{\createBefList}{#1}%
%% }

% takes a list of lists of the form: {foo=1, bla},{gar=-} and then defines empty macros for each name
\newcommand*{\clearNamesListofLists}[1]{%
    \forcsvlist{    \clearNamesList}{#1}%
}
\newcommand*{\clearNamesList}[1]{%
    \forcsvlist{    \clearName}{#1}%
}
\newcommand*{\clearName}[1]{%
    \getItemName{#1}{cmdname}%
    \csdef{\cmdname}{}% 
}

\lstdefinelanguage
   [sparc]{Assembler}     % add a "x64" dialect of Assembler
   [x86masm]{Assembler} % based on the "x86masm" dialect
   % with these extra keywords
   {morekeywords={ldw, stw, LD, ADD, SUB, ST, OP, BR, BLTZ, BGTZ, HALT, CMOZ, BL}} % etc.
\lstset{language={[sparc]Assembler}}

\newcommand{\complScheme}{\mathsf{compl}}
\newcommand{\lwsync}{\mathsf{lwsync}}
\newcommand{\hwsync}{\mathsf{hwsync}}
\newcommand{\opt}{\mathsf{opt}}
\newcommand{\correctOpt}{\mathsf{CorrectOpt}}
\newcommand{\MM}{\mathcal{M}}
\newcommand{\LL}{\mathsf{L}}
\newcommand{\LowLL}{\mathsf{L'}}
\newcommand{\LowMM}{\mathcal{M'}}
\newcommand{\PP}{\mathsf{P}}

\newcommand{\valuecom}[1]{\textcolor{green!60!black}{// #1}}
\newcommand{\valuefail}[1]{\textcolor{red!60!black}{// #1}}
\newcolumntype{L}{>{$}l<{$}} % math-mode version of "l" column type

\declaretheorem[name=Теорема,style=default,numberwithin=section]{thm}
\declaretheorem[name=Лемма,style=default,numberwithin=section]{lem}
%% \declaretheorem[name=Определение,style=default,sibling=thm]{mydefinition}
\declaretheorem[name=Определение,style=definition,numbered=no]{mydefinition}

% enviorments
\theoremstyle{plain}
%% \newtheorem{notation}[theorem]{Notation}
\newtheorem{notation}{Notation}
\newtheorem*{notation*}{Notation}

\newtheorem{definition}{Определение}
\newtheorem*{definition*}{Определение}
\newtheorem{proposition}{Утверждение}
\newtheorem*{proposition*}{Утверждение}
\newtheorem{theorem}{Теорема}
\newtheorem*{theorem*}{Теорема}
\newtheorem{lemma}{Лемма}
\newtheorem*{lemma*}{Лемма}

\crefformat{section}{#2\S{}#1#3}
\Crefname{section}{Section}{Section}
\Crefformat{section}{Section #2#1#3}
\Crefname{figure}{\text{Figure}}{\text{Figures}}
\crefname{corollary}{\text{Corollary}}{\text{corollaries}}
\Crefname{corollary}{\text{Corollary}}{\text{Corollaries}}
\crefname{lemma}{\text{Lemma}}{\text{Lemmas}}
\Crefname{lemma}{\text{Lemma}}{\text{Lemmas}}
\crefname{proposition}{\text{Prop.}}{\text{Propositions}}
\Crefname{proposition}{\text{Proposition}}{\text{Propositions}}
\crefname{definition}{\text{Def.}}{\text{Definitions}}
\Crefname{definition}{\text{Definition}}{\text{Definitions}}
\crefname{notation}{\text{Notation}}{\text{Notations}}
\Crefname{notation}{\text{Notation}}{\text{Notations}}
\crefname{theorem}{\text{Theorem}}{\text{Theorems}}
\Crefname{theorem}{\text{Theorem}}{\text{Theorems}}


\newenvironment{claim}[1]{\par\noindent\underline{Claim:}\space#1}{}
\newenvironment{claimproof}[1]{\par\noindent\underline{Proof:}\space#1}
{\leavevmode\unskip\penalty9999 \hbox{}\nobreak\hfill\quad\hbox{$\square$}}

\newcommand{\textdom}[1]{\mathsf{#1}}
\newcommand{\textcode}[1]{\texorpdfstring{\texttt{#1}}{#1}}
\newcommand{\kw}[1]{\textbf{\textcode{#1}}}
\newcommand{\skipc}{\kw{skip}}
\newcommand{\ite}[3]{\kw{if}\;#1\:\kw{then}\;#2\;\\ \kw{else}\;#3\;\kw{fi}}
\newcommand{\iteml}[3]{
  \kw{if} \; #1\\
  \begin{array}[t]{@{}l@{}l}
    \kw{then}& \begin{array}[t]{l} #2 \end{array} \\
    \kw{else}& \begin{array}[t]{l} #3 \end{array} \\
  \end{array}\\
  \kw{fi}
}
\newcommand{\itne}[2]{\kw{if}\;#1\:\kw{then}\\ \quad\;{#2}}
\newcommand{\while}[2]{\kw{while}\;#1\;\kw{do}\;#2}
\newcommand{\ALT}{\;\;|\;\;}

\newcommand{\ie}{\emph{i.e.,} }
\newcommand{\eg}{\emph{e.g.,} }
\newcommand{\etal}{\emph{et~al.}}
\newcommand{\wrt}{w.r.t.~}

\newcommand{\inarrC}[1]{\begin{array}{@{}c@{}}#1\end{array}}
\newcommand{\inpar}[1]{\left(\begin{array}{@{}l@{}}#1\end{array}\right)}
\newcommand{\inset}[1]{\left\{\begin{array}{@{}l@{}}#1\end{array}\right\}}
\newcommand{\inarr}[1]{\begin{array}{@{}l@{}}#1\end{array}}
\newcommand{\inarrII}[2]{\begin{array}{@{}l@{~~}||@{~~}l@{}}\inarr{#1}&\inarr{#2}\end{array}}
\newcommand{\inarrIII}[3]{\begin{array}{@{}l@{~~}||@{~~}l@{~~}||@{~~}l@{}}\inarr{#1}&\inarr{#2}&\inarr{#3}\end{array}}
\newcommand{\inarrIV}[4]{\begin{array}{@{}l@{~~}||@{~~}l@{~~}||@{~~}l@{~~}||@{~~}l@{}}\inarr{#1}&\inarr{#2}&\inarr{#3}&\inarr{#4}\end{array}}
\newcommand{\inarrV}[5]{\begin{array}{@{}l@{~~}||@{~~}l@{~~}||@{~~}l@{~~}||@{~~}l@{~~}||@{~~}l@{}}\inarr{#1}&\inarr{#2}&\inarr{#3}&\inarr{#4}&\inarr{#5}\end{array}}

%% \renewcommand{\comment}[1]{\color{teal}{~~\texttt{/\!\!/}\textit{#1}}}


\newcommand{\set}[1]{\{{#1}\}}
\newcommand{\sem}[1]{\llbracket #1 \rrbracket}
\newcommand{\pfn}{\rightharpoonup}
\newcommand{\suchthat}{\; | \;}
\newcommand{\N}{{\mathbb{N}}}
\newcommand{\dom}[1]{\textit{dom}{({#1})}}
\newcommand{\codom}[1]{\textit{codom}{({#1})}}
\newcommand{\before}[2]{{#1}_{#2}^\uparrow}
\newcommand{\after}[2]{{#1}_{#2}^\downarrow}
%\newcommand{\fv}[1]{fv{[{#1}]}}
\newcommand{\tup}[1]{{\langle{#1}\rangle}}
\newcommand{\nin}{\not\in}
\newcommand{\suq}{\subseteq}
\newcommand{\sqsuq}{\sqsubseteq}
\newcommand{\sqsu}{\sqsubset}
\newcommand{\sqslq}{\sqsupseteq}
\newcommand{\size}[1]{|{#1}|}
\newcommand{\block}[1]{\langle {#1}\rangle}
\newcommand{\true}{\top}
\newcommand{\maketil}[1]{{#1}\ldots{#1}}
\newcommand{\til}{\maketil{,}}
\newcommand{\cuptil}{\maketil{\cup}}
\newcommand{\uplustil}{\maketil{\uplus}}
\renewcommand*{\mathellipsis}{\mathinner{{\ldotp}{\ldotp}{\ldotp}}}
\newcommand{\rst}[1]{|_{#1}}
\newcommand{\imm}[1]{{#1}{\rst{\text{imm}}}}
\renewcommand{\succ}[2]{\text{succ}_{#1}(#2)}
\newcommand{\aite}[3]{(#1?#2,#3)}
%\newcommand{\defeq}{\mathrel{\stackrel{\mathsf{def}}{=}}}
\newcommand{\defeq}{\triangleq}
\newcommand{\powerset}[1]{\mathcal{P}({#1})}
\newcommand{\finpowerset}[1]{\mathcal{P}_{<\omega}({#1})}
\renewcommand{\implies}{\Rightarrow}


\colorlet{colorPO}{gray!60!black}
\colorlet{colorRF}{green!60!black}
\colorlet{colorMO}{orange}
\colorlet{colorFR}{purple}
\colorlet{colorECO}{red!80!black}
\colorlet{colorSYN}{green!40!black}
\colorlet{colorHB}{blue}
\colorlet{colorPPO}{magenta}
\colorlet{colorPB}{olive}
\colorlet{colorSBRF}{olive}
\colorlet{colorRMW}{olive!70!black}
\colorlet{colorRSEQ}{blue}
\colorlet{colorSC}{violet}
\colorlet{colorPSC}{violet}
\colorlet{colorREL}{olive}
\colorlet{colorCONFLICT}{olive}
\colorlet{colorRACE}{olive}
\colorlet{colorWB}{orange!70!black}
\colorlet{colorPSC}{violet}
\colorlet{colorSCB}{violet}
\colorlet{colorDEPS}{violet}

\tikzset{
   every path/.style={>=stealth},
   sb/.style={->,color=colorSB,thin,shorten >=-0.5mm,shorten <=-0.5mm},
   sw/.style={->,color=colorSYN,shorten >=-0.5mm,shorten <=-0.5mm},
   rf/.style={->,color=colorRF,dashed,,shorten >=-0.5mm,shorten <=-0.5mm},
   hb/.style={->,color=colorHB,thick,shorten >=-0.5mm,shorten <=-0.5mm},
   mo/.style={->,color=colorMO,dotted,very thick,shorten >=-0.5mm,shorten <=-0.5mm},
   no/.style={->,dotted,thick,shorten >=-0.5mm,shorten <=-0.5mm},
   fr/.style={->,color=colorFR,dotted,thick,shorten >=-0.5mm,shorten <=-0.5mm},
   deps/.style={->,color=colorDEPS,dotted,thick,shorten >=-0.5mm,shorten <=-0.5mm},
   rmw/.style={->,color=colorRMW,thick,shorten >=-0.5mm,shorten <=-0.5mm},
}

%% Orders
\newcommand{\na}{\mathtt{na}}
\newcommand{\pln}{\mathtt{pln}}
\newcommand{\rlx}{\mathtt{rlx}}
\newcommand{\rel}{{\mathtt{rel}}}
\newcommand{\acq}{{\mathtt{acq}}}
\newcommand{\con}{{\mathtt{con}}}
\newcommand{\acqrel}{{\mathtt{acqrel}}}
\newcommand{\sco}{{\mathtt{sc}}}
\newcommand{\sto}{{\mathtt{st}}}
\newcommand{\full}{{\mathtt{sy}}}
\newcommand{\ld}{{\mathtt{ld}}}
\newcommand{\isb}{{\mathtt{isb}}}

%% Event labels

\newcommand{\evlab}[4]{{#1}({#3},{#4})}
\newcommand{\evflab}[1]{{\flab}^{#1}}
\newcommand{\evulab}[4]{{\ulab}^{#1}({#2},{#3},{#4})}
\newcommand{\rlab}[3]{{\lR}^{#1}({#2},{#3})}
\newcommand{\erlab}[4]{{\lR}^{#1}({#2},{#3},{#4})}
\newcommand{\wlab}[3]{{\lW}^{#1}({#2},{#3})}
\newcommand{\flab}[1]{{\lF}(#1)}
%\newcommand{\ulab}[4]{{\lU}^{#1}({#2},{#3},{#4})}

\newcommand{\lE}{{\mathtt{E}}}
\newcommand{\lC}{{\mathtt{C}}}
%\newcommand{\lP}{{\mathit{P}}}
\newcommand{\lM}{{\mathtt{M}}}
\newcommand{\lS}{{\mathtt{S}}}
\newcommand{\lR}{{\mathtt{R}}}
\newcommand{\lW}{{\mathtt{W}}}
\newcommand{\lA}{{\mathtt{A}}}
\newcommand{\lQ}{{\mathtt{Q}}}
\newcommand{\lL}{{\mathtt{L}}}
\newcommand{\lU}{{\mathtt{RMW}}}
\newcommand{\lF}{{\mathtt{F}}}
\newcommand{\lRES}{{\mathtt{Res}}}
\newcommand{\lAT}{{\mathtt{At}}}
\newcommand{\lATR}{{\mathtt{AtR}}}

\newcommand{\lLAB}{{\mathtt{lab}}}
\newcommand{\lTID}{{\mathtt{tid}}}
\newcommand{\lTYP}{{\mathtt{typ}}}
\newcommand{\lLOC}{{\mathtt{loc}}}
\newcommand{\lMOD}{{\mathtt{mod}}}
\newcommand{\lVALR}{{\mathtt{val_r}}}
\newcommand{\lVALW}{{\mathtt{val_w}}}
\newcommand{\lELAB}{{\mathtt{elab}}}
\newcommand{\lAVALS}{{\mathtt{atvals}}}

%% Relations

\newcommand{\lX}{\mathtt{X}}
\newcommand{\lPO}{{\color{colorPO}\mathtt{po}}}
\newcommand{\lRF}{{\color{colorRF} \mathtt{rf}}}
\newcommand{\lRMW}{{\color{colorRMW} \mathtt{rmw}}}
\newcommand{\lMO}{{\color{colorMO} \mathtt{mo}}}
\newcommand{\lMOx}{{\color{colorMO} \mathtt{mo}}_x}
\newcommand{\lMOy}{{\color{colorMO} \mathtt{mo}}_y}
\newcommand{\lCO}{{\color{colorMO} \mathtt{co}}}
\newcommand{\lCOx}{{\color{colorMO} \mathtt{co}}_x}
\newcommand{\lCOy}{{\color{colorMO} \mathtt{co}}_y}
\newcommand{\lFR}{{\color{colorFR} \mathtt{rb}}}
\newcommand{\lFRx}{{\color{colorFR} \mathtt{rb}}_x}
\newcommand{\lFRy}{{\color{colorFR} \mathtt{rb}}_y}
\newcommand{\lECO}{{\color{colorECO} \mathtt{eco}}}
\newcommand{\lSBRF}{{\color{colorSBRF} \mathtt{sbrf}}}
\newcommand{\lRSEQ}{{\color{colorRSEQ}\mathtt{rseq}}}
\newcommand{\lSW}{{\color{colorSYN}\mathtt{sw}}}
\newcommand{\lASW}{{\color{colorSYN}\mathtt{asw}}}
\newcommand{\lSO}{{\color{colorSYN}\mathtt{so}}}
\newcommand{\lHB}{{\color{colorHB}\mathtt{hb}}}
%\newcommand{\lWB}{{\color{colorWB} \mathtt{wb}}}
\newcommand{\lDOB}{{\mathtt{dob}}}
\newcommand{\lBOB}{{\mathtt{bob}}}
\newcommand{\lAOB}{{\mathtt{aob}}}
\newcommand{\lOBS}{{\mathtt{obs}}}
\newcommand{\lEORD}{{\mathtt{eord}}}
\newcommand{\lTORD}{{\mathtt{tord}}}
\newcommand{\lSC}{{\color{colorSC}{\mathtt{sc}}}}

\newcommand{\lSCB}{{\color{colorSCB} \mathtt{scb}}}
\newcommand{\lPSC}{{\color{colorPSC} \mathtt{psc}}}
\newcommand{\lPSCB}{\lPSC_{\rm base}}
\newcommand{\lPSCF}{\lPSC_\lF}
\newcommand{\lCONFLICT}{{\color{colorCONFLICT} \mathtt{conflict}}}
\newcommand{\lRACE}{{\color{colorRACE} \mathtt{race}}}
\newcommand{\lNARACE}{{\color{colorRACE} \mathtt{na-race}}}

\newcommand{\lmakeW}[1]{\mathtt{w}#1}
\newcommand{\lWFR}{\lmakeW{\lFR}}
\newcommand{\lWECO}{\lmakeW{\lECO}}
\newcommand{\lWSCB}{\lmakeW{\lSCB}}
\newcommand{\lWPSCB}{\lmakeW{\lPSCB}}
\newcommand{\lWPSCF}{\lmakeW{\lPSCF}}
\newcommand{\lWPSC}{\lmakeW{\lPSC}}
\newcommand{\lWB}{\lmakeW{\lMO}}

\newcommand{\lDEPS}{{{\color{colorDEPS}\mathtt{deps}}}}
\newcommand{\lCTRL}{{{\color{colorDEPS}\mathtt{ctrl}}}}
\newcommand{\lCTRLISYNC}{{{\color{colorDEPS}\mathtt{ctrl_{isync}}}}}
\newcommand{\lDATA}{{{\color{colorDEPS}\mathtt{data}}}}
\newcommand{\lADDR}{{{\color{colorDEPS}\mathtt{addr}}}}

\newcommand{\lmakeE}[1]{#1\mathtt{e}}
\newcommand{\lRFE}{\lmakeE{\lRF}}
\newcommand{\lCOE}{\lmakeE{\lCO}}
\newcommand{\lFRE}{\lmakeE{\lFR}}
\newcommand{\lMOE}{\lmakeE{\lMO}}
\newcommand{\lmakeI}[1]{#1\mathtt{i}}
\newcommand{\lRFI}{\lmakeI{\lRF}}
\newcommand{\lCOI}{\lmakeI{\lCO}}
\newcommand{\lFRI}{\lmakeI{\lFR}}

\newcommand{\Tid}{\mathsf{Tid}}
\newcommand{\Loc}{\mathsf{Loc}}
\newcommand{\Val}{\mathsf{Val}}
\newcommand{\Lab}{\mathsf{Lab}}
\newcommand{\Mod}{\mathsf{Mod}}
\newcommand{\Modr}{\mathsf{Mod}_{\lR}}
\newcommand{\Modw}{\mathsf{Mod}_{\lW}}
\newcommand{\Modf}{\mathsf{Mod}_{\lF}}
\newcommand{\Modrmw}{\mathsf{Mod}_{\lU}}

\newcommand{\reorder}[1]{\mathsf{Reorder}({#1})}
%\newcommand{\deorder}[1]{\mathsf{Deorder}({#1})}
\newcommand{\remove}[1]{\mathsf{RemoveWR}({#1})}
\newcommand{\reorderWR}[1]{\mathsf{ReorderWR}({#1})}

\newcommand{\travConfigStep}{\rightarrow_{\rm TC}}
\newcommand{\ARM}{\ensuremath{\mathsf{ARM}}\xspace}
\newcommand{\correctTmap}{{\rm \bf correct\text{-}tmap}}
\newcommand{\rmwCovering}{{\rm \bf rmw\text{-}covering}}
\newcommand{\invWsc}{{\rm \bf write\text{-}sc\text{-}covering}}
\newcommand{\invViewRel}{\mathcal{I}_{\rm \bf view\text{-}rel}}
\newcommand{\setView}{{\rm set\text{-}view}}
\newcommand{\domView}{{\rm dom\text{-}view}}
\newcommand{\scSet}{{\rm sc\text{-}set}}
\newcommand{\scSetF}{{\rm sc\text{-}set\text{-}fun}}
\newcommand{\scSynchSet}{{\rm sc\text{-}synch}}
\newcommand{\acqSynchSet}{{\rm acq\text{-}synch}}
\newcommand{\relSynchFSet}{{\rm rel\text{-}synchf}}
\newcommand{\acqSet}{{\rm acq\text{-}set}}
\newcommand{\acqSetF}{{\rm acq\text{-}set\text{-}fun}}
\newcommand{\curSet}{{\rm cur\text{-}set}}
\newcommand{\curSetF}{{\rm cur\text{-}set\text{-}fun}}
\newcommand{\relSetFF}{{\rm rel\text{-}setf\text{-}fun}}
\newcommand{\msgSetF}{{\rm msg\text{-}set\text{-}fun}}
\newcommand{\msgSet}{{\rm msg\text{-}set}}
\newcommand{\relFunSet}{{\rm rel\text{-}setf}}
\newcommand{\lessView}{{\sf \bf less\text{-}view}}
\newcommand{\TSf}{\ensuremath{\mathcal{TS}}\xspace}
\newcommand{\TS}{\ensuremath{\mathit{TS}}\xspace}
\newcommand{\View}{\ensuremath{\mathcal{V}}\xspace}
\newcommand{\SCview}{\ensuremath{\mathcal{S}}\xspace}
\newcommand{\letdef}[2]{\kw{let} \; #1 \defeq #2 \; \kw{in}}
\newcommand{\nextset}{{\sf Next}}
\newcommand{\coverable}{{\sf Coverable}}
\newcommand{\promisable}{{\sf Promisable}}
%% \newcommand{\addToMemory}[2]{#1 \buildrel{\text{A}}\over\hookleftarrow #2}
\newcommand{\addToMemory}[2]{#1 \cup \{#2\}}

\newcommand{\stepa}{\leadsto_{\ARM}}
\newcommand{\stepp}{\promStepgen{}}
%% \leadsto_{\Promise}}
\newcommand{\stepptid}{\leadsto_{\Promise \; tid}}
\newcommand{\lISB}{\mathtt{F^{\isb}}}
\newcommand{\lDMB}{\mathtt{DMB}}
\newcommand{\lDMBSY}{\flab{\SY}}
\newcommand{\lDMBLD}{\flab{\LD}}
\newcommand{\lDMBST}{\flab{\ST}}
\newcommand{\SY}{\mathtt{sy}}
\newcommand{\LD}{\mathtt{ld}}
\newcommand{\ST}{\mathtt{st}}
\newcommand{\removeCR}[1]{\mathsf{RemoveCR}({#1})}
%% \spnewtheorem{defn}{Definition}{\bfseries}{}
%% \crefname{defn}{Definition}{Definitions}
\newcommand{\Timestamp}{\mathtt{Time}}
\newcommand{\TimestampMap}{\mathit{TimeMap}}

\newcommand{\Prog}{\ensuremath{Prog}\xspace}
\newcommand{\ProgARM}{\ensuremath{Prog_\ARM}\xspace}

% Promise definitions
\newcommand{\msg}[4]{\tup{#1:#2@#3,#4}}
\newcommand{\Promise}{\ensuremath{\mathsf{Promise}}\xspace}
%% \newcommand\PromSet{\mathit{promises}}
\newcommand\PromSet{\mathtt{P}}
\newcommand\PromState{\mathit{st}}
\newcommand{\V}{V}
\newcommand{\view}{view}
%% \newcommand{\viewCur}{view_{\sf cur}}
%% \newcommand{\viewAcq}{view_{\sf acq}}
%% \newcommand{\viewRel}{view_{\sf rel}}
\newcommand{\viewCur}{{\sf cur}}
\newcommand{\viewAcq}{{\sf acq}}
\newcommand{\viewRel}{{\sf rel}}
\newcommand{\viewfRel}{{\sf rel}}
\newcommand{\loc}{\ell}
\newcommand{\val}{v}
%% \newcommand{\tstamp}{\tau}
\newcommand{\tstampRange}[2]{(#1,#2]}
\newcommand{\simRel}{\mathcal{I}}
\newcommand{\invMemOne}{\mathcal{I}_{\rm \bf mem1}}
\newcommand{\invMemTwo}{\mathcal{I}_{\rm \bf mem2}}
\newcommand{\invMemThree}{\mathcal{I}_{\rm \bf mem3}}
\newcommand{\invView}{\mathcal{I}_{\rm \bf view}}
\newcommand{\invState}{\mathcal{I}_{\rm \bf state}}
\newcommand{\pstate}{\sigma}
\newcommand{\labelF}{{\rm label}}
\newcommand{\ltsReq}{{\rm lts\text{-}req}}
\newcommand{\maxF}{{\rm max}}
\newcommand{\compileF}{{\rm compile}}
\newcommand{\compileMod}{{\rm mod\text{-}compile}}
\newcommand{\compileFenceMod}{{\rm fence\text{-}mod\text{-}compile}}
%% \newcommand{\writeLbl}[3]{W \; (#1, #2, #3)}
\newcommand{\writeLbl}[2]{W \; (#1, #2)}
\newcommand{\writeBlankLbl}{W \; \_}
\newcommand{\fenceLbl}[1]{F \; #1}
%% \newcommand{\readLbl}[3]{R \; (#1, #2, #3)}
\newcommand{\readLbl}[2]{R \; (#1, #2)}
\newcommand{\readBlankLbl}{R \; \_}
\newcommand{\casLbl}[5]{U \; (#1, #2, #3, #4, #5)}
\newcommand{\svaleq}{\asymp}
\newcommand{\strans}[1]{\xrightarrow{#1}}

%% \newcommand\readWriteTau{\textsf{access-tmap}}
%% \newcommand\readsView{\textsf{reads-view}}

%\newcommand{\RC}{\ensuremath{\mathsf{RC11}}\xspace}
%\newcommand{\WRC}{\ensuremath{\mathsf{WRC11}}\xspace}

\newcounter{mylabelcounter}

\makeatletter
\newcommand{\labelAxiom}[2]{%
\hfill{\normalfont\textsc{(#1)}}\refstepcounter{mylabelcounter}
\immediate\write\@auxout{%
  %% \string\newlabel{#2}{{1}{\thepage}{{#1}}{mylabelcounter.\number\value{mylabelcounter}}{}}
  \string\newlabel{#2}{{\unexpanded{\normalfont\textsc{#1}}}{\thepage}{{\unexpanded{\normalfont\textsc{#1}}}}{mylabelcounter.\number\value{mylabelcounter}}{}}
}%
}
\makeatother

\newcommand{\squishlist}[1][$\bullet$]{%[$\tinybullet$]{
 \begin{list}{#1}
  { \setlength{\itemsep}{0pt}
     \setlength{\parsep}{0pt}
     \setlength{\topsep}{1pt}
     \setlength{\partopsep}{0pt}
     \setlength{\leftmargin}{1.2em}
     \setlength{\labelwidth}{0.5em}
     \setlength{\labelsep}{0.4em} } }
\newcommand{\squishend}{
  \end{list}  }

\newcommand{\app}[1]{{\color{blue}\textbf{ANTON: #1}}}

\newcommand{\optarrow}{\rightsquigarrow}

%% Old defs.tex content

%% \newcommand{\excl}[1]{{\color{purple} #1}}
\newcommand{\ifext}[2]{\ifdefined\extflag{#1}\else{#2}\fi}

\newcommand{\Reads}{{\sf Reads}}
%% \newcommand{\Reads}{\{ \angled{\tId, \cpath, \loc} \mid \tId, \cpath, \loc \}}

\newcommand{\excl}[1]{}
\newcommand\tT{\mathbf{t}}

\newcommand{\ListOf}[1]{\mathit{List}\;{#1}}
\newcommand{\Label}{\mathit{Label}}
\newcommand{\lab}{\mathit{lab}}
\newcommand{\Pset}{\mathbb{P}}
\newcommand{\dep}{\mathit{dep}}
\newcommand{\regs}{\mathrm{regs}}

\newcommand\viewf{\textbf{\sf viewf}}
\newcommand{\nthf}{\mathrm{nth}}

\newcommand{\deltaToView}{\delta\textup{\sf -to-view}}
\newcommand{\deltaHmap}{\textup{\sf comb-time}}
\newcommand{\invDeltaDefOne}{\inv_{\delta\textup{\sf -con-1}}}
\newcommand{\invDeltaDefTwo}{\inv_{\delta\textup{\sf -con-2}}}
\newcommand{\invDeltaDefThree}{\inv_{\delta\textup{\sf -con-3}}}
\newcommand{\invDeltaDefFour}{\inv_{\delta\textup{\sf -con-4}}}

\newcommand{\tIdState}{\textup{\sf thread-state}}
\newcommand{\lastCommittedWrite}{\textup{\sf last-write-com}}

\newcommand\certifiableTid{\mathsf{certifiable}_{\tId}}
\newcommand\certifiable{\mathsf{certifiable}}

\newcommand\lastInstr[1]{#1.\textup{\sf last}}

\newcommand\length{\textup{\sf length}}
\newcommand\prefix{\textup{\sf prefix}}

\newcommand{\ExclReadType}{\mathit{er}}
\newcommand{\ExclWriteType}{\mathit{ew}}
\newcommand{\ComWriteState}{\mathit{im}}

\newcommand\InMemory{im}
\newcommand{\NotInMemory}{\textup{\sf no-mem}}
\newcommand{\IssuedToMemory}{\textup{\sf mem}}
\newcommand{\ExclIssuedToMemory}{\textup{\sf excl-mem}}

\newcommand{\readInst }[3][\ExclReadType]{#2 \;:=_{\excl{#1}}\;[#3]}
\newcommand{\fenceInst}[1]{\fence{#1}}
\newcommand{\dmbSY}{\fenceInst{\SY}}
\newcommand{\dmbLD}{\fenceInst{\LD}}

\newcommand{\readInstParam}[3]{#2 \;:=_{#1}\;[#3]}
\newcommand{\writeInstParam}[3]{[#2] \;:=_{#1}\;#3}
\newcommand{\repeatReadInst}[2]{\kw{repeat} \;[#2]_{#1}\; \kw{end}}
\newcommand{\casInstParam}[6]{#3\;:=\;\kw{cas}_{#1,#2}(#4, #5, #6)}

\newcommand\ifGoto{\textsf{if}-\textsf{goto}}
\newcommand{\ifGotoInst}[2]{\textsf{if} \; #1 \; \textsf{goto} \; #2}
\newcommand{\writeInst}[2]{[#1]\;:=\;#2}
\newcommand{\writeExclInst}[3]{\textup{\sf atomic-write}(#1, [#2], #3)}
\newcommand{\assignInst}[2]{#1\;:=\;#2}
\newcommand{\repeatInst}[1]{\kw{repeat} \;#1\; \kw{end}}

\newcommand{\restrict}[2]{#1{\restriction_{#2}}}

%% \newcommand{\writeRI}[3][, \ExclType]{#2:#3\excl{, #1}}
\newcommand{\writeRI}[3][]{#2:#3\excl{#1}}

%% \newcommand\ARM{\mathrm{ARM}}
\newcommand\ARMt{\mathrm{ARM}{+}\tau}
%% \newcommand\Promise{\mathrm{Promise}}

\newcommand{\tstamp}[1]{\mbox{\small\color{brown!60!black}\bf{#1}}}

\newcommand{\OrdPrevRequest}{\textup{\sf prev-Ord-req}}

\newcommand{\uniqueTimeLoc}{\textup{\sf uniq-time-loc}}
\newcommand{\ordPrevRequest}{\textup{\sf ord-prev-req}}
\newcommand{\coherentThread}{\textup{\sf coherent-thread}}
\newcommand{\timeRangeCondition}{\textup{\sf time-range}}
\newcommand{\sameMemory}{\textup{\sf same-memory}}
\newcommand{\instToLbl}{\textup{\sf inst-to-lbl}}
\newcommand{\cmdsToLbls}{\textup{\sf cmds-to-lbls}}
\newcommand{\cmdsToLblsAux}{\textup{\sf cmds-to-lbls-aux}}

\newcommand{\instToVrtx}{\textup{\sf inst-to-vertex}}
\newcommand{\cmdsToVrtxs}{\textup{\sf cmds-to-vertices}}
\newcommand{\cmdsToVrtxsAux}{\textup{\sf cmds-to-vertices-aux}}

\newcommand{\prevInstrCommitted}{\textup{\sf prev-instr-committed}}
\newcommand{\prevReadsCommitted}{\textup{\sf prev-reads-committed}}
\newcommand{\prevFencesCommitted}{\textup{\sf prev-fences-committed}}
\newcommand{\prevBrCommitted}{\textup{\sf prev-branches-committed}}
\newcommand{\prevBrFencesCommitted}{\textup{\sf prev-branches-and-fences-committed}}
\newcommand{\prevCmdDetermined}{\textup{\sf prev-fully-determined}}
\newcommand{\prevNoRestart}{\textup{\sf no-prev-restartable-reads-from-loc}}
\newcommand{\prevExclCommitted}{\textup{\sf prev-excl-to-loc-committed}}
\newcommand{\noFollowingWcom}{\textup{\sf no-following-com-writes-to-loc}}

%% \newcommand\reorderableRel{\leftrightarrow} 
%% \newcommand\notReorderableRel{\not \leftrightarrow} 
\newcommand\reorderableRel[2]{#1 \hookrightarrow #2} 
\newcommand\notReorderableRel[2]{#1 \not \hookrightarrow #2} 
%% \newcommand\ReorderingFunction{is\_reorderable}
%% \newcommand{\checkReorderings}[1]{\textup{\sf check-reorderings}(#1)}
\newcommand{\checkReorderings}[1]{#1 \setminus {\reorderableRel{}{}}}

\newcommand{\deleteUpdReads}{\textup{\sf delete-upd-reads}}
\newcommand{\acceptRequest}{\textup{\sf accept-request}}
\newcommand{\acceptExclWrite}{\textup{\sf accept-excl-write}}

\newcommand{\readsBetweenCommitted}{\textup{\sf reads-in-between-committed}}
\newcommand{\noWritesBetween}{\textup{\sf no-writes-to-loc-in-between}}
\newcommand{\noDiffReadsBetween}{\textup{\sf no-different-write-reads-in-between}}
\newcommand{\samePropagated}{\textup{\sf propagated-to-same-threads}}
\newcommand{\fullyPropagated}{\textup{\sf fully-propagated}}
\newcommand{\getNewTapeCell}{\textup{\sf get-new-tapecell}}
\newcommand{\prevReadFromOther}{\textup{\sf prev-read-from-other-write}}
%\newcommand{\nextPath}{\textup{\sf next-path}}
\newcommand{\tapeUpdRestart}{\textup{\sf tape-upd-restart}}
\newcommand{\doesntPreventExcl}{\textup{\sf doesnt-prevent-excl}}
\newcommand{\tapeUpdWcom}{\textup{\sf tape-upd-Wcom}}
\newcommand{\tapeUpdRsat}{\textup{\sf tape-upd-Rsat}}
\newcommand{\tapeUpdIf}{\textup{\sf tape-upd-IfGoto}}
\newcommand{\noExclInBetween}{\textup{\sf no-excl-in-between}}
\newcommand{\getLoc}{\textup{\sf get-loc}}

%% PROBLEMS WITH: \C, \next, \a
  \newcommand\armStepWriteCommit{\armStepgen{\transenv{Write commit} \; \tId \; \cpath \; x \; \stval \; \tau}}
  \newcommand\armStepWriteCommitLoc{\armStepgen{\transenv{Write commit} \; \tId \; \cpath \; \loc \; \stval \; \tau}}
  \newcommand\armStepWriteCommitPrime{\armStepgen{\transenv{Write commit} \; \tId' \; \cpath' \; y \; \stval' \; \tau}}
  \newcommand\armStepWriteCommitP{\armStepPgen{\transenv{Write commit} \; \tId \; \cpath \; x \; \stval}}
  \newcommand\armStepWriteCommitPLoc{\armStepPgen{\transenv{Write commit} \; \tId \; \cpath \; \loc \; \stval}}
  \newcommand\armStepWriteCommitPrimeP{\armStepPgen{\transenv{Write commit} \; \tId' \; \cpath' \; y \; \stval'}}


  \newcommand\promStepBranch{\promStepgen{\transenv{Branch commit} \; \tId}}
  \newcommand\promTStepBranch{\promTStepgen{\transenv{Branch commit}}}

  \newcommand\promStepAcquire{\promStepgen{\transenv{Acquire fence commit} \; \tId}}
  \newcommand\promTStepAcquire{\promTStepgen{\transenv{Acquire fence commit}}}
  \newcommand\promStepRelease{\promStepgen{\transenv{Release fence commit} \; \tId}}
  \newcommand\promTStepRelease{\promTStepgen{\transenv{Release fence commit}}}

  \newcommand\promStepRead{\promStepgen{\transenv{Read from memory} \; \tId \; \writeEvt{x}{\stval}{\tau}{\R}}}
  \newcommand\promTStepRead{\promTStepgen{\transenv{Read from memory} \; \writeEvt{x}{\stval}{\tau}{\R}}}
  \newcommand\promTStepReadLoc{\promTStepgen{\transenv{Read from memory} \; \writeEvt{\loc}{\stval}{\tau}{\R}}}

  \newcommand\promStepPromise{\promStepgen{\transenv{Promise write} \; \tId \; \writeEvt{x}{\stval}{\tau}{\R}}}
  \newcommand\promStepPromiseRPrime{\promStepgen{\transenv{Promise write} \; \tId \; \writeEvt{x}{\stval}{\tau}{\R'}}}
  \newcommand\promTStepPromise{\promTStepgen{\transenv{Promise write} \; \writeEvt{x}{\stval}{\tau}{\R}}}
  \newcommand\promTStepPromiseLoc{\promTStepgen{\transenv{Promise write} \; \writeEvt{\loc}{\stval}{\tau}{\R}}}

  \newcommand\promStepFulfill{\promStepgen{\transenv{Fulfill promise} \; \tId \; \writeEvt{x}{\stval}{\tau}{\R}}}
  \newcommand\promTStepFulfill{\promTStepgen{\transenv{Fulfill promise} \; \writeEvt{x}{\stval}{\tau}{\R}}}
  \newcommand\promTStepFulfillLoc{\promTStepgen{\transenv{Fulfill promise} \; \writeEvt{\loc}{\stval}{\tau}{\R}}}

  \newcommand\promStepAssign{\promStepgen{\transenv{Local variable assignment} \; \tId}}
  \newcommand\promTStepAssign{\promTStepgen{\transenv{Local variable assignment}}}

  \newcommand\promStepNop{\promStepgen{\transenv{Execution of $\nop$} \; \tId}}
  \newcommand\promTStepNop{\promTStepgen{\transenv{Execution of $\nop$}}}

  \newcommand{\tapeToCertificate}{\mathsf{tape}\textsf{-}\mathsf{to}\textsf{-}\mathsf{certificate}}
  %% \newcommand{\Timestamp}{\mathit{Time}}

  \newcommand{\graybox}[1]{$\colorbox{gray!20}{$#1\!$}$}

  %\newcommand\Rarm{\R_{\ARM}}
  \newcommand\Rarm{\mathsf{view}_{\ARM}}
  \newcommand\transenv[1]{\textcolor{darkgray}{\textup{\textsf{\textbf{\mathversion{bold}#1}}}}}

  \newcommand\promMessage{\mathit{msg}}
  \newcommand\promMessageSet{\mathit{Msg}}

  \newcommand\FtypeARM{\mathit{fmod}_{\ARM}}
  %% \newcommand\FtypeProm{\mathit{ftype}_{\Promise}}
  \newcommand\FtypeProm{\mathit{fmod}}

  \newcommand\RtypeProm{\mathit{rtype}_{\Promise}}
  \newcommand\WtypeProm{\mathit{wtype}_{\Promise}}

  %% \newcommand\V{\mathit{V}}
  \newcommand\R{\mathit{view}}
  %% \newcommand\View{\mathit{View}}
  \newcommand\Rsc{\R_{\mathrm{sc}}}
  \newcommand\Rcur{\R_{\mathrm{cur}}}
  \newcommand\Racq{\R_{\mathrm{acq}}}
  \newcommand\Rrel{\R_{\mathrm{rel}}}
  \newcommand\Rna{\R_{\mathrm{na}}}
  \newcommand\Rwrite{\R_{\mathrm{write}}}
  
  \newcommand\lessUpToDelta[3]{#2 <_{#1} #3}

  \newcommand\StmtARM{S}
  \newcommand\StmtProm{S_{\Promise}}


  %% \newcommand\C{C}
  \newcommand\Cf{\mathit{Prog}}%\textsf{Prog}
  \newcommand\Carm{{cmds}}
  \newcommand\CARM{{Cmds}}
  \newcommand\Cfarm{\Cf}
  \newcommand\Cprom{{cmds}}
  \newcommand\Cfprom{\Cf}
  %% \newcommand\Carm{\textsf{C}_{\ARM}}
  %% \newcommand\Cfarm{\Cf_{\ARM}}
  %% \newcommand\Cprom{\textsf{C}_{\Promise}}
  %% \newcommand\Cfprom{\Cf_{\Promise}}
  
  %% \newcommand\PromSet{\mathit{promises}}
  %% \newcommand\PromState{\mathit{st}}

  \newcommand\scAcqRead[2]{#1 := [#2]_{\sf LDAR}}
  \newcommand\scRelWrite[2]{[#1]_{\sf STLR} := #2}

  \newcommand\acqFence{\fence{\sf acquire}}
  \newcommand\relFence{\fence{\sf release}}
  \newcommand\scFence{\fence{\sf sc}}
  \newcommand\fence[1]{\mathsf{fence}({#1})}

  \newcommand\syFence{\fence{\sf sy}}
  \newcommand\ldFence{\fence{\sf ld}}
  
  \newcommand\reqInfoRead[1]{{\sf rd} \; #1}
  \newcommand\reqInfoWrite[2]{{\sf wr} \; #1:#2}
  \newcommand\reqInfoFence{{\sf dmb}}

  \newcommand\stRequest[3]{\angled{#1, #2, #3}}
  \newcommand\stRequestWrite[4]{\angled{#1, #2, \reqInfoWrite{#3}{#4}}}
  \newcommand\stRequestRead[3]{\angled{#1, #2, \reqInfoRead{#3}}}
  \newcommand\stRequestFence[2]{\angled{#1, #2, \reqInfoFence}}

  \newcommand\moTau{\textsf{tedges}}
  \newcommand\invTT{\textsf{inv}}

  % \newcommand\opstau{op\_\loc\_\tau}
  \newcommand\opstau{\textsf{com-writes-time}}
  \newcommand\readsSatisfiedR{\textsf{sat-reads-view}}
  \newcommand\readsCommittedR{\textsf{com-reads-view}}

  \newcommand\hmap{\mathit{H}}
  \newcommand\tmap{\mathit{H}_{\tau}}
  \newcommand\rmap{\mathit{H}_{\mathsf{view}}}
  \newcommand\omap{\mathit{H}_{\le}}

  \newcommand\StateARM{\mathsf{State}_{\ARM}}
  \newcommand\StateARMtau{\mathsf{State}_{\ARMt}}
  \newcommand\StateProm{\mathsf{State}_{\Promise}}
  \newcommand\TStateProm{\mathsf{TState}_{\Promise}}

  \newcommand\armStep{\xrightarrow[\ARMt]{}}
  \newcommand\armStepl{\xrightarrow[\ARMt]{}}
  \newcommand\armStepgen[1]{\xrightarrow[\ARMt]{#1}}
  \newcommand\armStepP{\xrightarrow[\ARM]{}}
  \newcommand\armStepPgen[1]{\xrightarrow[\ARM]{#1}}

  \newcommand\armStepPrSat{\armStepPgen{\transenv{Read satisfy} \; \tId \; \cpath \; \tId' \; \cpath' \; x \; \stval}}
  \newcommand\armStepPrSatLoc{\armStepPgen{\transenv{Read satisfy} \; \tId \; \cpath \; \tId' \; \cpath' \; \loc \; \stval}}
  \newcommand\armStepRSat{\armStepgen{\transenv{Read satisfy} \; \tId \; \cpath \; \tId' \; \cpath' \; x \; \stval}}
  \newcommand\armStepPrSatFail{\armStepPgen{\transenv{Read satisfy (fail)} \; \tId \; \cpath \; \tId' \; \cpath' \; x \; \stval}}
  \newcommand\armStepPrSatFailLoc{\armStepPgen{\transenv{Read satisfy (fail)} \; \tId \; \cpath \; \tId' \; \cpath' \; \loc \; \stval}}
  \newcommand\armStepRSatFail{\armStepgen{\transenv{Read satisfy (fail)} \; \tId \; \cpath \; \tId' \; \cpath' \; x \; \stval}}
  \newcommand\armStepPrInFlightSat{\armStepPgen{\transenv{Read satisfy from in-flight write} \; \tId \; \cpath \; \cpath' \; x \; \stval}}
  \newcommand\armStepPrInFlightSatLoc{\armStepPgen{\transenv{Read satisfy from in-flight write} \; \tId \; \cpath \; \cpath' \; \loc \; \stval}}
  \newcommand\armStepRInFlightSat{\armStepgen{\transenv{Read satisfy from in-flight write} \; \tId \; \cpath \; \cpath' \; x \; \stval}}

  \newcommand\promStep{\xrightarrow[\Promise]{}}
  \newcommand\promStepl{\xrightarrow[\Promise]{}}
  \newcommand\promStepgen[1]{\xrightarrow[\Promise]{#1}}
  \newcommand\promTStepgen[1]{\xrightarrow[\Promise \; \tId]{#1}}
  \newcommand\promTStep{\xrightarrow[\Promise \; \tId]{}}

  \newcommand\semState[2]{|[#1|] ^{#2}}
  \newcommand\semf[2]{|[#1|] ^{#2}}
  \newcommand\semfcom[2]{|[#1|] ^{#2}_\mathsf{com}}
  \newcommand\sembr[1]{|[#1|]}
  \newcommand\PreExecutions{{\rm PreExecs}}

  \newcommand\textIf{\text{\underline{if}} \;}
  \newcommand\textElif{\text{\underline{elif}} \;}
  \newcommand\textElse{\text{\underline{else}} \;}
  \newcommand\textThen{\text{\underline{then}} \;}
  \newcommand\textLet{\text{\underline{let}} \;}
  \newcommand\textIn{\text{\underline{in}}}

  %% \newcommand\nextPathCom[3]{nextPath_{com}(#1, #2, #3)}
  \newcommand\nextPathCom[3]{\mathsf{next}\textsf{-}\mathsf{path}(#1, #2, #3)}
  \newcommand\nextPathProm{\mathsf{next}\textsf{-}\mathsf{path}_{\Promise}}

  \newcommand\comShift{\phantom{{}_\mathsf{com}}}

  \newcommand\extendExpr{\expr\_extend}

  \newcommand\regst{\mathsf{regf}}
  \newcommand\regstcom{\mathsf{regf}_\mathsf{com}}
  \newcommand\regf{\mathit{regf}}
  
  \newcommand\Mpop{\mathit{M}_{\mathrm{POP}}}
  %% \newcommand\Mprom{\mathit{M}_{\mathrm{Promise}}}
  \newcommand\Mprom{\mathit{M}}
  \newcommand{\Mcomp}[3]{\angled{#1, #2, #3}}


  \newcommand\Request{\textit{req}}
  \newcommand\RequestSet{\mathit{ReqSet}}
  \newcommand\RequestInfo{\mathit{reqinfo}}
  \newcommand\RequestInfoSet{\mathit{ReqInfoSet}}
  \newcommand\Evt{\mathit{Evt}}
  \newcommand\Ord{\mathit{Ord}}
  \newcommand\Prop{\mathit{Prop}}
  \newcommand\ExclMap{\mathit{Excl}}

  \newcommand\IssuingOrder{\mathit{iord}}
  \newcommand\IssuingOrderf{\mathit{iordf}}

  \newcommand\Issued[1]{\mathsf{requested} \; #1}
  \newcommand\Satisfied{\mathsf{sat}}
  \newcommand\SatisfiedInFlight{\mathsf{inflight}}
  \newcommand\Committed{\mathsf{com}}
  \newcommand\Plain{\mathsf{pln}}
  \newcommand\Exclusive{\mathsf{excl}}
  \newcommand\None{\mathsf{none}}
  \newcommand\Any{\mathsf{any}}

  %% \newcommand\LD{\mathsf{LD}}
  %% \newcommand\SY{\mathsf{SY}}
  %% \newcommand\ST{\mathsf{ST}}

  \newcommand\tId{\mathit{tid}}
  %% \newcommand\Tid{\mathit{Tid}}

  \newcommand\Tape{\mathit{Tape}}
  \newcommand\tape{\mathit{tape}}
  \newcommand\tapef{\mathit{tapef}}
  \newcommand\TapeCell{\mathit{TapeCell}}
  \newcommand\tapeCell{\mathit{tapecell}}
  \newcommand\Taken{\mathsf{taken}}
  \newcommand\Ignored{\mathsf{ignored}}

  \newcommand\cpath{\mathit{path}}
  \newcommand\cpathSY{\cpath^{\SY}}
  \newcommand\cpathLD{\cpath^{\LD}}
  \newcommand\cpathLDSY{\cpath^{\LD\SY}}
  \newcommand\Path{Path}
  
  \newcommand\tapeRead[1]{\textsf{R} \; #1}
  \newcommand\tapeFence[2]{\textsf{F} \; #1 \; #2}
  \newcommand\tapeWrite[1]{\textsf{W} \; #1}
  \newcommand\tapeIfGoto[2]{\textsf{If} \; #1 \; #2}
  \newcommand\tapeNop{\textsf{Nop}}
  \newcommand\tapeAssign{\textsf{Assign}}

  \newcommand\tapeSatisfied[2]{\Satisfied \; #1 \; #2}
  \newcommand\satisfiedState{\mathit{sat\text{-}state}}

  \newcommand\tapePending[2]{\textsf{pending} \; #1 \; #2}
  \newcommand\tapeWriteCommitted[3]{\Committed \; #1 \; #2 \; #3}

  \newcommand\Fstate{\mathit{st}_{\textup{\rm fence}}}
  \newcommand\Rstate{\mathit{st}_{\textup{\rm read}}}
  \newcommand\Wstate{\mathit{st}_{\textup{\rm write}}}
  \newcommand\IfState{\mathit{st}_{\textup{\rm ifgoto}}}

  \newcommand\locvar{\iota}
  %% \newcommand\loc{\ell}
  %% \newcommand\Loc{Loc}
  \newcommand\Reg{\mathit{Reg}}
  \newcommand\reg{\mathit{reg}}
  %% \newcommand\val{\mathit{val}}
  \newcommand\stval{\mathit{val}}
  \newcommand\Stval{\mathit{Val}}
  \newcommand\expr{\mathit{expr}}
  \newcommand\Expr{\mathit{Expr}}
  \newcommand\z{\mathit{k}}

  \newcommand\dmb{\textsf{dmb}}
  \newcommand\nop{\textsf{nop}}

  \newcommand\ImmediateEdge{ImmediateEdge}
   
  \newcommand\armState[1]{\ARM_{state}(#1)}
  \newcommand\angled[1]{\langle #1 \rangle}
  %% \newcommand\TSfprom{TSf}
  %% \newcommand\TSfprom{TSf_{\Promise}}
  %% \newcommand\TS{TS_{\Promise}}
  %% \newcommand\TSfprom{\textsf{tsf}}
  %% \newcommand\TS{\textsf{ts}}
  %% \newcommand\TSfprom{\mathit{tsf}}
  %% \newcommand\TS{\mathit{ts}}

  %% \newcommand\M{M}
  \newcommand\e{e}
  \newcommand\w{w}
  %% \newcommand\dom[1]{\mathsf{dom}(#1)}
  \newcommand\taumapping{\mathsf{map}_{\tau}}

  \newcommand\Nop{Nop}
  \newcommand\Write[1]{Write \; #1}
  \newcommand\WritePending[1]{\Write(\Pending \; #1)}
  \newcommand\Read[1]{Read \; #1}
  \newcommand\ReadIssued[1]{Read \; (\Issued \; #1)}
  \newcommand\ReadSatisfied[4]{Read \; (\Satisfied \; #1 \; #2 \; #3 \; #4)}
  %% \newcommand\next[2]{next(#1, #2)}
  \newcommand\nextPath[2]{\mathsf{next}\textsf{-}\mathsf{path}(#1, #2)}
  \newcommand\lastIndex{\mathsf{last}\textsf{-}\mathsf{index}}
  \newcommand\last[1]{last(#1)}
  \newcommand\lastSY{\mathsf{last}\SY}
  \newcommand\lastCF{\mathsf{lastCF}}
  \newcommand\lastLD{\mathsf{last}\LD}
  \newcommand\lastLDSY{\mathsf{last}\LD\SY}
  \newcommand\Fence[2]{Fence \; #1 \; #2}
  \newcommand\FenceSY[1]{\Fence{#1}{SY}}
  \newcommand\FenceLD[1]{\Fence{#1}{LD}}
  \newcommand\IfGoto[2]{IfGoto \; #1 \; #2}
  \newcommand\IfGotoK[1]{\IfGoto{#1}{k}}
  \newcommand\tick{\ding{51}} % ✓
  \newcommand{\tickP}{\tick}
  \newcommand{\tickPP}{\tick}

  \newcommand\fail{\ding{55}} % ✗
  \newcommand\Certificate{certificate}
  \newcommand\CommandState{instrPlan}
  %% \newcommand\State{State}
  \newcommand\writeEvt[4]{\angled{#1:#2@#3,#4}}
  %% \newcommand\writeEvt[4]{#1:#2@#3,#4}
  \newcommand\simrel{\mathcal{I}}
  \newcommand\simrelPre{\mathcal{I}_{\textup{\rm pre}}}
  \newcommand\simrelBase{\mathcal{I}_{\textup{\rm base}}}
  \newcommand\inv{\mathcal{I}}
  \newcommand\s{\mathbf{s}}
  \newcommand\sfst{\mathbf{s}_0}
  \newcommand\ssnd{\mathbf{s}_1}
  \newcommand\aT{\mathbf{a}}
  \newcommand\afst{\mathbf{a}_0}
  \newcommand\asnd{\mathbf{a}_1}
  \newcommand\p{\mathbf{p}}
  \newcommand\ptid{\mathbf{p}\textsf{-}\tId}
%  \newcommand\ordPlusMO{S_{\Ord \cup mo}}
  \newcommand\ordPlusMO{\textsf{S-Ord-mo}}

  \newcommand\finalStateP{\mathsf{Final}^{\ARM}}
  \newcommand\finalStateA{\mathsf{Final}^{\ARMt}}
  \newcommand\finalStateProm{\mathsf{Final}^{\Promise}}

  \newcommand\sinit{\mathbf{s}^{\rm init}}
  \newcommand\ainit{\mathbf{a}^{\rm init}}
  \newcommand\pinit{\mathbf{p}^{\rm init}}

  \newcommand\invI[1]{\inv_{\textup{\rm #1}}}
  \newcommand\invARM[1]{\inv^\ARM_{\textup{\rm #1}}}

  \newcommand\invTidWriteComCERT{\inv^{\tId}_{\mathsf{w-cert}}}
  \newcommand\invStateCERT{\invI{state-cert}}
  \newcommand\invViewDeltaCERT{\invI{view-cert}}
  \newcommand\invViewWriteCERT{\invI{view-write-cert}}
  \newcommand\invViewRelCERT{\invI{write-rel-cert}}
  \newcommand\invViewReadCERT{\invI{view-read-cert}}
  \newcommand\invWriteTimestampCERT{\invI{write-time-cert}}
  %% \newcommand\invViewCERT{\invI{view-cert}}
  \newcommand\invMemZeroCERT{\invI{mem-1-tid-cert}}
  \newcommand\invMemOneCERT{\invI{mem-1-com-cert}}
  \newcommand\invMemTwoCERT{\invI{mem-2-cert}}
  \newcommand\simrelBaseCERT{\mathcal{I}_{\textup{\rm base-cert}}}

  \newcommand\invCf{\invI{prg}}
  \newcommand\invTId{\invI{tid}}
  \newcommand\invPrefix{\invI{prefix}}
  \newcommand\invSPrefix{\invI{strong prefix}}
  %% \newcommand\invMemOne{\invI{mem1}}
  %% \newcommand\invMemTwo{\invI{mem2}}
  %% \newcommand\invMemThree{\invI{mem3}}
  %% \newcommand\invView{\invI{view}}
  %% \newcommand\invState{\invI{state}}
  \newcommand\invReach{\invI{reach}}
  \newcommand\invComWrite{\invI{com-SY}}
  \newcommand\invCert{\invI{cert}} %\Certificate}}
  \newcommand\invCertTid{\inv_{cert \; \tId}} %\Certificate}}
  \newcommand\simrelTid{\inv_{exec \; \tId}} %\Certificate}}
  \newcommand\invApartialOrderOrd{\inv^{ARM}_{\Ord \; \text{is a partial order}}}
  \newcommand\invAuniqWrite{\invARM{unique write}}
  \newcommand\invAtransClosedOrd{\invARM{\Ord = transitive\_closure(\Ord)}}
  \newcommand\invAimmediateEdge{\invARM{Immediate edge}}
  \newcommand\invAimmediatePath{\invARM{Immediate path}}
  \newcommand\invAmaxPath{\invARM{max \; \cpath}}
  \newcommand\invPromUptoARM{\invI{Promise is up to ARM}}
  \newcommand\invPromUptoARMtId{\inv^{\tId}_{\textup{Promise is up to ARM}}}
  \newcommand\invPromUptoARMnot{\invI{Promise isn't up to ARM}}


  \newcommand\correctStateA{\inv^{\ARMt}_{\textup{\rm correct}}}
  \newcommand\correctStateP{\invARM{correct}}
  \newcommand\invATapeCf{\invARM{tape-Prg}}
  \newcommand\invATapeCfState{\invARM{{tape-Prg-State}}}
  \newcommand\invAReadWrite{\invARM{{Read-Write}}}
  \newcommand\invAReadRead{\invARM{{Read-Read}}}
  \newcommand\invAWriteWriteRead{\invARM{{Write-Write-Read}}}
  \newcommand\invAReadCommittedWrite{\invARM{{Read-Write-Committed}}}
  \newcommand\invAview{\invARM{{View}}}
  \newcommand\invAviewWrite{\invARM{{View-Write}}}
  \newcommand\invAviewRead{\invARM{{View-Read}}}
  \newcommand\invAWriteView{\invARM{{Write-View}}}
  \newcommand\invAnextCommitted{\invARM{{Next-Committed}}}
  \newcommand\invACf{\invARM{Prg}}
  \newcommand\invAtId{\invARM{tid}}
  \newcommand\invAldRead{\invARM{LD-Read}}
  \newcommand\invAtypePreservation{\invARM{{tape-Type}}}
  \newcommand\invAcommittedPreservation{\invARM{{Committed-Preserve}}}
  \newcommand\invAstatePreservation{\invARM{{State-Preserve}}}
  \newcommand\invAstateCom{\invARM{S-Scom}}
  \newcommand\invAcomFences{\invARM{{Committed-Fences}}}
  \newcommand\invAordMOacyclic{\invARM{{Ord-mo-acyclic}}}
  \newcommand\invAordProp{\invARM{{Ord-Prop}}}
  \newcommand\invApropOrd{\invARM{{Prop-Ord}}}
  \newcommand\invAord{\invARM{{Ord-acyclic}}}
  \newcommand\invAevtTape{\invARM{{Evt-tape}}}
  \newcommand\invAtapeEvt{\invARM{{tape-Evt}}}
  \newcommand\invAtapeOrd{\invARM{{tape-Ord}}}
  \newcommand\invAReadWriteOne{\invARM{{Read-Write-1}}}
  \newcommand\invAReadWriteTwo{\invARM{{Read-Write-2}}}

  \newcommand\invPMP{\inv^{\Promise}_{\textsf{M-P}}}
  \newcommand\invPmessageView{\inv^{\Promise}_{\textup{\rm Message-View}}}

  \newcommand\armStepFetch{\overset{fetch \; \tId \; \cpath}{\armStepl}}
  \newcommand\armStepFetchPrime{\overset{fetch \; \tId' \; \cpath'}{\armStepl}}
  \newcommand\armStepProp{\overset{e \rightsquigarrow \tId}{\armStepl}}
  \newcommand\armStepPropPrime{\overset{e' \rightsquigarrow \tId'}{\armStepl}}
  \newcommand\armStepWritePending{\overset{\dashrightarrow \stRequestWrite{\tId}{ \cpath}{ x}{\stval}}{\armStepl}}
  \newcommand\armStepWritePendingPrime{\overset{\dashrightarrow \stRequestWrite{\tId'}{ \cpath'}{ y}{\stval'}}{\armStepl}}

  \newcommand\armStepReadRequest{\overset{\dashrightarrow \stRequestRead{\tId}{ \cpath}{ x}}{\armStepl}}
  \newcommand\armStepReadRequestPrime{\overset{\dashrightarrow \stRequestRead{\tId'}{ \cpath'}{ y}}{\armStepl}}
  \newcommand\armStepCondBranch{\overset{\dashrightarrow \tId, \cpath, \text{choose branch}}{\armStepl}}
  \newcommand\armStepCondBranchPrime{\overset{\dashrightarrow \tId', \cpath', \text{choose branch}}{\armStepl}}
  \newcommand\armStepFenceCommit{\overset{\dashrightarrow \stRequestFence{\tId}{ \cpath}}{\armStepl}}
  \newcommand\armStepFenceCommitPrime{\overset{\dashrightarrow \stRequestFence{\tId'}{ \cpath'}}{\armStepl}}
  \newcommand\armStepReadSatisfy{\overset{\dashrightarrow \stRequestRead{\tId}{ \cpath}{ x}, \stRequestWrite{\tId'}{ \cpath'}{ x}{\stval}}{\armStepl}}
  \newcommand\armStepReadSatisfyPrime{\overset{\dashrightarrow \stRequestRead{\tId'}{ \cpath'}{ y}, \stRequestWrite{\tId''}{ \cpath''}{ y}{\stval'}}{\armStepl}}
  \newcommand\armStepReadSatisfyFail{\overset{\not \dashrightarrow \stRequestRead{\tId}{ \cpath}{ x}, \stRequestWrite{\tId'}{ \cpath'}{ x}{\stval}}{\armStepl}}
  \newcommand\armStepReadSatisfyFailPrime{\overset{\not \dashrightarrow \stRequestRead{\tId'}{ \cpath'}{ y}, \stRequestWrite{\tId''}{ \cpath''}{ y}{\stval'}}{\armStepl}}
  \newcommand\armStepReadSatisfyInF{\overset{\dashrightarrow \stRequestRead{\tId}{ \cpath}{ x}, \stRequestWrite{\tId}{ \cpath'}{ x}{\stval}}{\armStepl}}
  \newcommand\armStepReadSatisfyInFPrime{\overset{\dashrightarrow \stRequestRead{\tId'}{ \cpath'}{ y}, \stRequestWrite{\tId'}{ \cpath''}{ y}{\stval'}}{\armStepl}}
  \newcommand\armStepReadCommit{\overset{\dashrightarrow \tId, \cpath, \text{read commit}}{\armStepl}}
  \newcommand\armStepReadCommitPrime{\overset{\dashrightarrow \tId', \cpath', \text{read commit}}{\armStepl}}

  \newcommand\armStepFetchP{\overset{\text{fetch} \; \tId \; \cpath}{\armStepP}}
  \newcommand\armStepFetchPrimeP{\overset{\text{fetch} \; \tId' \; \cpath'}{\armStepP}}
  \newcommand\armStepPropP{\overset{e \rightsquigarrow \tId}{\armStepP}}
  \newcommand\armStepPropPrimeP{\overset{e' \rightsquigarrow \tId'}{\armStepP}}
  \newcommand\armStepWritePendingP{\overset{\dashrightarrow \stRequestWrite{\tId}{ \cpath}{ x}{\stval}}{\armStepP}}
  \newcommand\armStepWritePendingPrimeP{\overset{\dashrightarrow \stRequestWrite{\tId'}{ \cpath'}{ y}{\stval'}}{\armStepP}}
  \newcommand\armStepReadRequestP{\overset{\dashrightarrow \stRequestRead{\tId}{ \cpath}{ x}}{\armStepP}}
  \newcommand\armStepReadRequestPrimeP{\overset{\dashrightarrow \stRequestRead{\tId'}{ \cpath'}{ y}}{\armStepP}}
  \newcommand\armStepCondBranchP{\overset{\dashrightarrow \tId, \cpath, \text{choose branch}}{\armStepP}}
  \newcommand\armStepCondBranchPrimeP{\overset{\dashrightarrow \tId', \cpath', \text{choose branch}}{\armStepP}}
  \newcommand\armStepFenceCommitP{\overset{\dashrightarrow \stRequestFence{\tId}{ \cpath}}{\armStepP}}
  \newcommand\armStepFenceCommitPrimeP{\overset{\dashrightarrow \stRequestFence{\tId'}{ \cpath'}}{\armStepP}}
  \newcommand\armStepReadSatisfyP{\overset{\dashrightarrow \stRequestRead{\tId}{ \cpath}{ x}, \stRequestWrite{\tId'}{ \cpath'}{ x}{\stval}}{\armStepP}}
  \newcommand\armStepReadSatisfyPrimeP{\overset{\dashrightarrow \stRequestRead{\tId'}{ \cpath'}{ y}, \stRequestWrite{\tId''}{ \cpath''}{ y}{\stval'}}{\armStepP}}
  \newcommand\armStepReadSatisfyFailP{\overset{\not \dashrightarrow \stRequestRead{\tId}{ \cpath}{ x}, \stRequestWrite{\tId'}{ \cpath'}{ x}{\stval}}{\armStepP}}
  \newcommand\armStepReadSatisfyFailPrimeP{\overset{\not \dashrightarrow \stRequestRead{\tId'}{ \cpath'}{ y}, \stRequestWrite{\tId''}{ \cpath''}{ y}{\stval'}}{\armStepP}}
  \newcommand\armStepReadSatisfyInFP{\overset{\dashrightarrow \stRequestRead{\tId}{ \cpath}{ x}, \stRequestWrite{\tId}{ \cpath'}{ x}{\stval}}{\armStepP}}
  \newcommand\armStepReadSatisfyInFPrimeP{\overset{\dashrightarrow \stRequestRead{\tId'}{ \cpath'}{ y}, \stRequestWrite{\tId'}{ \cpath''}{ y}{\stval'}}{\armStepP}}
  \newcommand\armStepReadCommitP{\overset{\dashrightarrow \tId, \cpath, \text{read commit}}{\armStepP}}
  \newcommand\armStepReadCommitPrimeP{\overset{\dashrightarrow \tId', \cpath', \text{read commit}}{\armStepP}}


\newcommand{\event}[3]{#1#2#3}
\tikzset{
   every path/.style={>=stealth},
   po/.style={->,color=brown,,shorten >=-0.5mm,shorten <=-0.5mm},
   rf/.style={->,color=green!60!black,dashed,,shorten >=-0.5mm,shorten <=-0.5mm},
   fr/.style={->,color=red,thick,shorten >=-0.5mm,shorten <=-0.5mm},
   mo/.style={->,color=orange!60!red,dotted,thick,shorten >=-0.5mm,shorten <=-0.5mm},
   no/.style={->,dotted,thick,shorten >=-0.5mm,shorten <=-0.5mm},
   deps/.style={->,color=violet,dotted,thick,shorten >=-0.5mm,shorten <=-0.5mm},
}

\newcommand{\IssuedSet}{I}
\newcommand{\issuable}{{\sf Issuable}}

%% \newcommand{\comment}[1]{\color{teal}{~~\texttt{/\!\!/}\textit{#1}}}

\newcommand{\lEID}{\lE_{tid}}
\newcommand{\TransSet}{{\rm Transitions}}
\newcommand{\LabelSet}{{\rm Labels}}

\newcommand{\pseudoCompileF}{{\rm pseudo\text{-}compile}}
\newcommand{\pseudoCompileReq}{{\rm pseudo\text{-}compile\text{-}req}}
\newcommand{\compileReq}{{\rm compile\text{-}req}}

\newcommand{\scRel}{{\rm sc\text{-}rel}}
\newcommand{\scRelF}{{\rm sc\text{-}rel\text{-}fun}}
\newcommand{\acqRel}{{\rm acq\text{-}rel}}
\newcommand{\acqRelF}{{\rm acq\text{-}rel\text{-}fun}}
\newcommand{\curRel}{{\rm cur\text{-}rel}}
\newcommand{\curRelF}{{\rm cur\text{-}rel\text{-}fun}}
\newcommand{\relRelFF}{{\rm rel\text{-}rel\text{-}fun}}
\newcommand{\relRel}{{\rm rel\text{-}rel}}
\newcommand{\msgRelF}{{\rm msg\text{-}rel\text{-}fun}}
\newcommand{\msgRel}{{\rm msg\text{-}rel}}

\newcommand{\step}{\longrightarrow}
\newcommand{\astep}[1]{\xrightarrow{#1}}
\newcommand{\bstep}[1]{\xRightarrow{#1}}
\newcommand{\cstep}[2]{\xrightarrow[{#1}]{{#2}}}
\newcommand{\dstep}[2]{\dhxrightarrow[{#1}]{{#2}}}
\newcommand{\Pfin}{\txtsub{P}{final}}
\newcommand{\fin}{\text{final}}
\newcommand{\rulename}[1]{{\textsc{({#1})}}}
\newcommand{\onestep}[2]{{\textsc{({#1}}{({#2})}\textsc{)}}}

\newcommand{\mrel}{R}
\newcommand{\lmrel}{\texttt{view}}
\newcommand{\msc}{S}

\newcommand{\I}{{\cal I}}

\newcommand{\tmin}{\in}
\newcommand{\tmle}{\leqslant}
\newcommand{\tcom}{{\cal V}}
%\newcommand{\tcom}{V}
\newcommand{\gsco}{{\cal S}}
%\newcommand{\gsco}{F}
%\newcommand{\mem}{{\cal M}}
\newcommand{\mem}{M}
\newcommand{\scmap}{S}
\newcommand{\msgid}{i}
\newcommand{\gts}{\mathcal{T\!S}}
\newcommand{\lts}{\mathit{TS}}
\newcommand{\tconf}{{\mathbf{TC}}}
\newcommand{\mconf}{{\mathbf{MS}}}
\newcommand{\lstate}{\texttt{st}}
\newcommand{\lview}{\texttt{view}}
\newcommand{\lprmem}{\texttt{prm}}
\newcommand{\lmem}{\txtsub{\mem}{p}}
%\newcommand{\lprom}{{\cal P}}
\newcommand{\lprom}{P}
\newcommand{\limitedmem}{{\cal L}}
\newcommand{\ogts}{\txtsub{\gts}{fut}}
\newcommand{\ogsco}{\txtsub{\gsco}{future}}
\newcommand{\olts}{\txtsub{\lts}{fut}}
\newcommand{\omem}{\txtsub{\mem}{future}}
\newcommand{\omsc}{\txtsub{\msc}{future}}
\newcommand{\lr}{{\text{r}}}
\newcommand{\lw}{{\text{w}}}
\newcommand{\cur}{cur}
%% \newcommand{\acq}{acq}
%% \newcommand{\rel}{rel}
%% \newcommand{\lsc}{sc}
%% \newcommand{\lscm}{s\!f}

\newcommand{\suc}{{succ}}
\newcommand{\lsuc}{\texttt{succ}}

\newcommand{\locprm}{P}
\newcommand{\llocprm}{\texttt{promise}}
\newcommand{\SUBS}{\mathit{sub}}
\newcommand{\lSUBS}{\texttt{sub}}

\newcommand{\relo}{{\texttt{rel}}}
\newcommand{\acqo}{{\texttt{acq}}}
%% \newcommand{\sco}{{\texttt{sc}}}
%% \newcommand{\na}{\texttt{na}}
%% \newcommand{\pln}{\texttt{pln}}
%% \newcommand{\atm}{\texttt{atm}}
\newcommand{\ra}{\texttt{ra}}
%% \newcommand{\nf}{\texttt{nf}}
%% \newcommand{\rlx}{\texttt{rlx}}
%% %\newcommand{\unord}{\texttt{uno}}
%% \newcommand{\relacqo}{{\texttt{relacq}}}

%% \newcommand{\Typ}{{\textdom{Typ}}}
%% \newcommand{\Lab}{{\textdom{Lab}}}
%% \newcommand{\Loc}{{\textdom{Loc}}}
%% \newcommand{\Val}{{\textdom{Val}}}
\newcommand{\AVal}{{\textdom{G}}}
%% \newcommand{\Tid}{{\textdom{Tid}}}
%% \newcommand{\Ord}{{\textdom{Ord}}}
%% \newcommand{\Reg}{{\textdom{Reg}}}
%% \newcommand{\Time}{{\textdom{Time}}}
%% \newcommand{\Timemap}{{\textdom{Timemap}}}
%% \newcommand{\Id}{{\textdom{Id}}}

\newcommand{\writeInstVolatile}[2]{[#1]\;:=_{\rm volatile}\;#2}
\newcommand{\readInstVolatile}[2]{#1 \;:=_{\rm volatile}\;[#2]}

\newcommand{\nullPtr}{\kw{null}}
\newcommand{\funcSt}[1]{{\rm #1}}

\newcommand{\stPhi}{\phi} 
\newcommand{\stAlpha}{\alpha}
\newcommand{\stGamma}{\gamma} 

\newcommand{\stPostOp}{\beta}
\newcommand{\stObservedWrites}{\omega}
% Shortcuts
\newcommand{\arrayBlock}[1]{\begin{array}{c}#1\end{array}}

\newcommand{\vName}{\mathsf{x}}


%% \selectlanguage{russian}
%% \newtranslation[to=russian]{Section}{Раздел}
%% \AtBeginSection{\frame{\sectionpage}}
\AtBeginSection[]
{
  \begin{frame}<beamer>
    \large
    \LARGE
    %% \frametitle{Outline for section \thesection}
    \tableofcontents[currentsection]
  \end{frame}
}

\addtobeamertemplate{navigation symbols}{}{%
    \usebeamerfont{footline}%
    \usebeamercolor[fg]{footline}%
    \hspace{1em}%
    {\bf \huge \insertframenumber}
}

\definecolor{darkGreen}{RGB}{0,127,0}
\definecolor{darkYellow}{RGB}{127,127,0}
\definecolor{darkRed}{RGB}{127,0,0}
\definecolor{darkPurple}{RGB}{127,0,127}
\colorlet{colorFETCH}{gray!40!white}
\colorlet{colorSAT}{yellow!40!white}
\colorlet{colorCOM}{green!40!white}
\colorlet{colorBUG}{red!20!white}
\colorlet{colorPROP}{green!80!black}
\colorlet{colorNPROP}{red!80!black}
\colorlet{colorVAL}{green!80!black}
\colorlet{colorPROM}{darkGreen}
\colorlet{colorPOSTPONE}{magenta!80!black}
\colorlet{colorSHADE}{gray!70}
\colorlet{colorCOV}{magenta!20}
\colorlet{colorISS}{green!30!white}

\colorlet{colorPO}{gray!60!black}
\colorlet{colorRF}{green!60!black}
\colorlet{colorMO}{orange}
\colorlet{colorFR}{purple}
\colorlet{colorECO}{red!80!black}
\colorlet{colorSYN}{green!40!black}
\colorlet{colorHB}{blue}
\colorlet{colorPPO}{magenta}
\colorlet{colorPB}{olive}
\colorlet{colorSBRF}{olive}
\colorlet{colorRMW}{olive!70!black}
\colorlet{colorRSEQ}{blue}
\colorlet{colorSC}{violet}
\colorlet{colorPSC}{violet}
\colorlet{colorREL}{olive}
\colorlet{colorCONFLICT}{olive}
\colorlet{colorRACE}{olive}
\colorlet{colorWB}{orange!70!black}
\colorlet{colorPSC}{violet}
\colorlet{colorSCB}{violet}
\colorlet{colorDEPS}{violet}

\tikzset{
   every path/.style={>=stealth},
   po/.style={->,color=brown,,shorten >=-0.5mm,shorten <=-0.5mm},
   rf/.style={->,color=green!60!black,dashed,,shorten >=-0.5mm,shorten <=-0.5mm},
   fr/.style={->,color=red,thick,shorten >=-0.5mm,shorten <=-0.5mm},
   mo/.style={->,color=orange!60!red,dotted,thick,shorten >=-0.5mm,shorten <=-0.5mm},
   no/.style={->,dotted,thick,shorten >=-0.5mm,shorten <=-0.5mm},
   deps/.style={->,color=violet,dotted,thick,shorten >=-0.5mm,shorten <=-0.5mm},
}

\newcommand{\lX}{\mathtt{X}}
\newcommand{\lPO}{{\color{colorPO}\mathtt{po}}}
\newcommand{\lRF}{{\color{colorRF} \mathtt{rf}}}
\newcommand{\lRMW}{{\color{colorRMW} \mathtt{rmw}}}
\newcommand{\lMO}{{\color{colorMO} \mathtt{mo}}}
\newcommand{\lMOx}{{\color{colorMO} \mathtt{mo}}_x}
\newcommand{\lMOy}{{\color{colorMO} \mathtt{mo}}_y}
\newcommand{\lCO}{{\color{colorMO} \mathtt{co}}}
\newcommand{\lCOx}{{\color{colorMO} \mathtt{co}}_x}
\newcommand{\lCOy}{{\color{colorMO} \mathtt{co}}_y}
\newcommand{\lFR}{{\color{colorFR} \mathtt{fr}}}
\newcommand{\lFRx}{{\color{colorFR} \mathtt{fr}}_x}
\newcommand{\lFRy}{{\color{colorFR} \mathtt{fr}}_y}
\newcommand{\lECO}{{\color{colorECO} \mathtt{eco}}}
\newcommand{\lSBRF}{{\color{colorSBRF} \mathtt{sbrf}}}
\newcommand{\lRSEQ}{{\color{colorRSEQ}\mathtt{rseq}}}
\newcommand{\lSW}{{\color{colorSYN}\mathtt{sw}}}
\newcommand{\lHB}{{\color{colorHB}\mathtt{hb}}}
%\newcommand{\lWB}{{\color{colorWB} \mathtt{wb}}}
\newcommand{\lDOB}{{\mathtt{dob}}}
\newcommand{\lBOB}{{\mathtt{bob}}}
\newcommand{\lAOB}{{\mathtt{aob}}}
\newcommand{\lOBS}{{\mathtt{obs}}}
\newcommand{\lEORD}{{\mathtt{eord}}}
\newcommand{\lTORD}{{\mathtt{tord}}}
\newcommand{\lSC}{{\mathtt{sc}}}

\tikzset{
    ncbar angle/.initial=90,
    ncbar/.style={
        to path=(\tikztostart)
        -- ($(\tikztostart)!#1!\pgfkeysvalueof{/tikz/ncbar angle}:(\tikztotarget)$)
        -- ($(\tikztotarget)!($(\tikztostart)!#1!\pgfkeysvalueof{/tikz/ncbar angle}:(\tikztotarget)$)!\pgfkeysvalueof{/tikz/ncbar angle}:(\tikztostart)$)
        -- (\tikztotarget)
    },
    ncbar/.default=0.5cm,
}

\tikzset{square left brace/.style={ncbar=0.3cm}}
\tikzset{square right brace/.style={ncbar=-0.3cm}}

\tikzset{round left paren/.style={ncbar=0.5cm,out=120,in=-120}}
\tikzset{round right paren/.style={ncbar=0.5cm,out=60,in=-60}}

%% Custom TikZ patterns

% defining the new dimensions and parameters
\newlength{\hatchspread}
\newlength{\hatchthickness}
\newlength{\hatchshift}
\newcommand{\hatchcolor}{}
% declaring the keys in tikz
\tikzset{hatchspread/.code={\setlength{\hatchspread}{#1}},
         hatchthickness/.code={\setlength{\hatchthickness}{#1}},
         hatchshift/.code={\setlength{\hatchshift}{#1}},% must be >= 0
         hatchcolor/.code={\renewcommand{\hatchcolor}{#1}}}
% setting the default values
\tikzset{hatchspread=10pt,
         hatchthickness=4pt,
         hatchshift=0pt,% must be >= 0
         hatchcolor=black}
% declaring the pattern
\pgfdeclarepatternformonly[\hatchspread,\hatchthickness,\hatchshift,\hatchcolor]% variables
   {custom north west lines}% name
   {\pgfqpoint{\dimexpr-2\hatchthickness}{\dimexpr-2\hatchthickness}}% lower left corner
   {\pgfqpoint{\dimexpr\hatchspread+2\hatchthickness}{\dimexpr\hatchspread+2\hatchthickness}}% upper right corner
   {\pgfqpoint{\dimexpr\hatchspread}{\dimexpr\hatchspread}}% tile size
   {% shape description
    \pgfsetlinewidth{\hatchthickness}
    \pgfpathmoveto{\pgfqpoint{0pt}{\dimexpr\hatchspread+\hatchshift}}
    \pgfpathlineto{\pgfqpoint{\dimexpr\hatchspread+0.15pt+\hatchshift}{-0.15pt}}
    \ifdim \hatchshift > 0pt
      \pgfpathmoveto{\pgfqpoint{0pt}{\hatchshift}}
      \pgfpathlineto{\pgfqpoint{\dimexpr0.15pt+\hatchshift}{-0.15pt}}
    \fi
    \pgfsetstrokecolor{\hatchcolor}
%    \pgfsetdash{{1pt}{1pt}}{0pt}% dashing cannot work correctly in all situation this way
    \pgfusepath{stroke}
   }

\pgfdeclarepatternformonly[\hatchspread,\hatchthickness,\hatchshift,\hatchcolor]% variables
   {custom north east lines}% name
   {\pgfqpoint{\dimexpr-2\hatchthickness}{\dimexpr-2\hatchthickness}}% lower left corner
   {\pgfqpoint{\dimexpr\hatchspread+2\hatchthickness}{\dimexpr\hatchspread+2\hatchthickness}}% upper right corner
   {\pgfqpoint{\dimexpr\hatchspread}{\dimexpr\hatchspread}}% tile size
   {% shape description
    \pgfsetlinewidth{\hatchthickness}
    \pgfpathmoveto{\pgfqpoint{\dimexpr\hatchshift-0.15pt}{-0.15pt}}
    \pgfpathlineto{\pgfqpoint{\dimexpr\hatchspread+0.15pt}{\dimexpr\hatchspread-\hatchshift+0.15pt}}
    \ifdim \hatchshift > 0pt
      \pgfpathmoveto{\pgfqpoint{-0.15pt}{\dimexpr\hatchspread-\hatchshift-0.15pt}}
      \pgfpathlineto{\pgfqpoint{\dimexpr\hatchshift+0.15pt}{\dimexpr\hatchspread+0.15pt}}
    \fi
    \pgfsetstrokecolor{\hatchcolor}
%    \pgfsetdash{{1pt}{1pt}}{0pt}% dashing cannot work correctly in all situation this way
    \pgfusepath{stroke}
   }

\begin{document}

\title
{Операционные методы в приложении к слабым моделям памяти}
\author
[А.В. Подкопаев]
{\textbf{Подкопаев Антон Викторович}}
\institute{
\begin{tabular}{r l}
Научный руководитель: & д.т.н., доцент Д.В. Кознов (СПбГУ)
\end{tabular}

\vspace{2cm}
Кафедра Системного Программирования, МатМех, СПбГУ
%% Семинар ИСП РАН
}
\date[16-11-17]{16.11.2017}

\newcommand{\fr} [2]{\begin{frame}{#1} #2 \end{frame}}
\newcommand{\frN}[2]{\begin{frame}[noframenumbering]{#1} #2 \end{frame}}
\newcommand{\ite}[1]{\begin{itemize} #1 \end{itemize}}
\newcommand{\iteN}[1]{\begin{enumerate} #1 \end{enumerate}}
\newcommand{\colu}[2]{\begin{column}{#1} #2 \end{column}}
\newcommand{\ex}{\textbf{Ex: }}
\newcommand{\exN}[1]{\textbf{Ex #1: }}
\newcommand{\df}{\textbf{Def: }}
\newcommand{\pitem}{\pause \item}
\newcommand{\lstasm}[1]{\lstinputlisting[language={[sparc]Assembler}]{codes/#1}}
\newcommand{\lstinlineasm}[1]{\lstinline[language={[sparc]Assembler}]{#1}}
\newcommand{\fri}[2]{\fr{#1}{\ite{#2}}}
\newcommand{\fre}[2]{\fr{#1}{\iteN{#2}}}
\newcommand{\frc}[2]{\fr{#1}{\begin{columns} #2 \end{columns}}}
\newcommand{\colui}[2]{\colu{#1}{\ite{#2}}}
\newcommand{\phrase}[1]{
  \fr{}{
  \begin{center}
    \Huge #1
  \end{center}
  }
}
\newcommand{\phraseL}[1]{
  \fr{}{
  \begin{center}
    \Large #1
  \end{center}
  }
}
\newcommand{\alertcolor}{red!60!black}
\setbeamercolor{alerted text}{fg=\alertcolor}
\newcommand\tick{{\color{green!50!black} \ding{51}}}
\newcommand\fail{{\color{red!50!black} \ding{55}}}

\newcommand{\relo}{{\texttt{rel}}}
\newcommand{\acqo}{{\texttt{acq}}}
\newcommand{\sco}{{\texttt{sc}}}
\newcommand{\na}{\texttt{na}}
\newcommand{\pln}{\texttt{pln}}
\newcommand{\atm}{\texttt{atm}}
\newcommand{\ra}{\texttt{ra}}
\newcommand{\rlx}{\texttt{rlx}}
%\newcommand{\unord}{\texttt{uno}}
\newcommand{\relacqo}{{\texttt{relacq}}}
\newcommand{\rlxmsg}[3]{\tup{#1\mathbin{:}#2\text{\smaller@}#3}}
\newcommand{\ts}[1]{\color{blue!60!green}{\mbox{\smaller#1}}}

\newcommand{\rlab}{\texttt{R}}
\newcommand{\wlab}{\texttt{W}}
\newcommand{\ulab}{\texttt{U}}
\newcommand{\slab}{\texttt{S}}
\newcommand{\tlab}{\texttt{T}}
\newcommand{\valw}{\mathit{val}_{\sf w}}
\newcommand{\flab}{\texttt{F}}

\newcommand{\dimslide}[1]{
   \addtooverlay<#1>{%
     \draw[fill=black,opacity=0.70] 
     (current page.north east) rectangle (current page.south west);
   }
}
\NewDocumentCommand{\dimalert}{r<> m}{%
\only<#1>{
\dimslide{#1}
\begin{textblock*}{\textwidth}(0.1\textwidth,0.4\textheight)
\begin{alertblock}{}
  \LARGE #2
\end{alertblock}
\end{textblock*}
}
}
\newcommand{\cntrd}[1]{ \begin{center} #1 \end{center} }

\newcommand{\instBackground}[2]{
    \fill[#1, rounded corners=3pt] ($(#2)  + (-1.1,0.35)$) rectangle ++(2.2,-0.7);
}
\newcommand{\parLines}[1]{
      \draw[-,ultra thick] ($(#1) + (-0.1,0.5)$) -- ($(#1) + (-0.1,-0.5-0.8)$);
      \draw[-,ultra thick] ($(#1) + ( 0.1,0.5)$) -- ($(#1) + ( 0.1,-0.5-0.8)$);
}

\newcommand{\indepCommentLeft}[2]{
      \draw[-,red,ultra thick] ($(#1) + (0.1, 0)$)
        to[out=0,in=0] ($(#2) + (0.1, 0)$);
      \node[anchor=west] at ($(#1) + (0.3, -0.4)$)  {\Large \textcolor{red}{Independent}};
}

\newcommand{\indepCommentRight}[2]{
      \draw[-,red,ultra thick] ($(#1) + (0.1, 0)$)
        to[out=180,in=180] ($(#2) + (0.1, 0)$);
      \node[anchor=east] at ($(#1) + (-0.1, -0.4)$)  {\Large \textcolor{red}{Independent}};
}

\newcommand{\promMsg}[3]{\angled{#1:#2@\tstampWOsize{#3}}}

\newcommand{\promisedTXT}{Обещана} % Promised
\newcommand{\postponedTXT}{Отложена} % Postponed
\newcommand{\lviewTXT}{Л.фронт} % LView
\newcommand{\rviewTXT}{П.фронт} % RView

\newcommand{\ptrPromise}[1]{
    \draw[->,ultra thick] ($(#1)  + (-3.0,0.39)$) -- +(1.9, 0.0);
    \node at ($(#1) + (-2.1,0.62)$) {\large \Promise};
}
\newcommand{\ptrPromiseLeft}[1]{
    \draw[->,ultra thick] ($(#1)  + (-1.2,0.39)$) -- +(1.6, 0.0);
}
\newcommand{\ptrPromiseRight}[1]{
  \ptrPromiseLeft{#1}
    %% \draw[->,ultra thick] ($(#1)  + (3.0,0.39)$) -- +(-1.9, 0.0);
}
\newcommand{\postponedBorder}[1]{
    \draw[colorPOSTPONE, ultra thick] ($(#1)  + (-1.1,0.3)$) rectangle ++(2.2,-0.6);
}
\newcommand{\postponedCommentLeft}[1]{
    \node[anchor=east] at ($(#1) - (1.4, 0.0)$) {\Large \textcolor{colorPOSTPONE}{\postponedTXT}};
}
\newcommand{\promisedBorder}[1]{
    \draw[colorPROM, ultra thick, rounded corners=3pt] ($(#1)  + (-1.1,0.35)$) rectangle ++(2.2,-0.7);
}
\newcommand{\promisedCommentDown}[1]{
    \node at ($(#1) + (0, -0.6)$) {\Large \textcolor{colorPROM}{\promisedTXT}};
}
\newcommand{\promisedCommentLeft}[1]{
    \node[anchor=east] at ($(#1) - (1.4, 0.0)$) {\Large \textcolor{colorPROM}{\promisedTXT}};
}
\newcommand{\promisedCommentRight}[1]{
    \node[anchor=west] at ($(#1) + (1.4, 0.0)$) {\Large \textcolor{colorPROM}{\promisedTXT}};
}

\newcommand{\traverseSingleBorder}[2]{
    \draw[pattern=custom north east lines, hatchcolor=#1, thick, rounded corners] ($(#2)  + (-0.35,-0.3)$) rectangle ++(0.7,0.6);
}

\newcommand{\traverseTwoVertBorder}[3]{
    \draw[pattern=custom north east lines, hatchcolor=#1, thick, rounded corners] ($(#2)  + (-0.45,0.4)$) rectangle ($(#3)  + (0.45,-0.4)$);
}

{\setbeamertemplate{footline}{}\setbeamertemplate{headline}{}\frame[plain,noframenumbering]{\titlepage}}
%% \begin{frame}[plain,noframenumbering]
%%   \titlepage
%% \end{frame}

%% \fri{\huge Что нужно для быстрого ПО?}{
%% \LARGE
%% \pause
%% \item \only<-4>{Хороший Алгоритм} \only<5->{\textcolor{gray!30}{Хороший Алгоритм}}
%% \vfill
%% \pause
%% \item \only<-5>{Эффективный}\only<6>{{\bf ?Эффективный?}}\only<7->{{\bf Оптимизирующий}} Компилятор
%% \vfill
%% \pause
%% \item \only<-7>{Производительный}\only<8>{{\bf ?Производительный?}}\only<9->{{\bf Оптимизирующий}} Процессор
%% }

\section{Контекст, мотивация, задачи}
\phrase{
  {\bf Модель памяти} \\ (memory model, MM) --- семантика многопоточной системы\\
}

\phrase{
  Последовательная консистентность (SC) \cite{Lamport:TC79}
  \vfill
  \pause
  \emph{семантика системы --- попеременное исполнение потоков}
}

\newcommand{\loadBufferValues}[2]{
      \node (memoryValues) at ($(middleBuffer) + (2.0, -3.2)$) {};

      \onslide<\storageAppearFr->{
        %% \storageTwoThreadsRelative{(leftBuffer)}{(rightBuffer)}{1.7}{0.5}{-3.0}
        \draw[draw,fill=yellow!20] ($(memoryValues) + (-1.9,0.8)$) rectangle ($(memoryValues) + (1.9,-0.3)$) ;
        \node[anchor=north] at ($(memoryValues) + (0,0.85)$) {\Large Память};
      }

      \node at ($(memoryValues) + (-0.05, -0.1)$) {\Large $;$}; 
      \node<-\befPropWriteRightFr>[anchor=east] at (memoryValues) {\Large $\writeReq{x}{0}$}; 
      \node<\propWriteRightFr->[anchor=east] at (memoryValues) {\Large $\writeReq{x}{\textcolor{colorVAL}{1}}$}; 
      \node<-\befPropWriteLeftFr>[anchor=west] at (memoryValues) {\Large $\writeReq{y}{0}$}; 
      \node<\propWriteLeftFr->[anchor=west] at (memoryValues) {\Large $\writeReq{y}{\textcolor{colorVAL}{1}}$}; 

      \node (values) at ($(memoryValues) + (-4.0, 0.0)$) {};
      
      \onslide<\firstFr->
        { \draw[draw,fill=yellow!20] ($(values) + (-1.9,  0.8)$) rectangle
                                     ($(values) + ( 1.9, -0.3)$) ;
          \node[anchor=north] at ($(values) + (0,  0.85)$) {\Large Регистры}; }
      \node at ($(values) + (-0.05, -0.1)$) {\Large $;$}; 
      \node<-\befPropReadLeftFr>[anchor=east] at (values) {\Large $a = \bot$}; 
      \node<\propReadLeftFr->[anchor=east] at (values) {\Large $a = \textcolor{colorVAL}{#1}$}; 
      \node<-\befPropReadRightFr>[anchor=west] at (values) {\Large $b = \bot$}; 
      \node<\propReadRightFr->[anchor=west] at (values) {\Large $b = \textcolor{colorVAL}{#2}$}; 
}

\fr{\huge Исполнение в SC}{
  \setorder{{firstFr, storageAppearFr,pointerAppearFr},
            {propReadLeftFr},
            {propReadRightFr},
            {comRightFr, propWriteRightFr},
            {comLeftFr, propWriteLeftFr},
            {noWeakFr}}
  \prevFr{\befPropWriteLeftFr}{\propWriteLeftFr}
  \prevFr{\befPropReadLeftFr}{\propReadLeftFr}
  \prevFr{\befPropWriteRightFr}{\propWriteRightFr}
  \prevFr{\befPropReadRightFr}{\propReadRightFr}

  \cntrd{
    \begin{tikzpicture}
      \node (leftFirstInst)   {};
      \node (leftSecondInst)  [below of= leftFirstInst, node distance = 0.8cm] {};
      \node (leftThirdInst)   [below of= leftSecondInst, node distance = 0.8cm] {};
      \node (rightFirstInst)  [right of=  leftFirstInst, node distance = 3.0cm] {};
      \node (rightSecondInst) [below of= rightFirstInst, node distance = 0.8cm] {};
      \node (rightThirdInst)  [below of= rightSecondInst, node distance = 0.8cm] {};

      \node (instMiddle)   at ($.5*(leftFirstInst) + .5*(rightFirstInst)$) {};
      \node (leftBuffer)   at (leftSecondInst)  {};
      \node (rightBuffer)  at (rightSecondInst) {};
      \node (middleBuffer) at ($.5*(leftBuffer) + .5*(rightBuffer)$) {};
      \node (finalValues)  at ($(middleBuffer) + (0, -4.5)$) {};

      \onslide<\pointerAppearFr-\befPropReadLeftFr>
        { \ptrPromiseLeft{leftFirstInst} } 
      \onslide<\propReadLeftFr-\befPropWriteLeftFr>
        { \ptrPromiseLeft{leftSecondInst} } 
      \onslide<\propWriteLeftFr->
        { \ptrPromiseLeft{leftThirdInst} } 

      \onslide<\pointerAppearFr-\befPropReadRightFr>
        { \ptrPromiseRight{rightFirstInst} }
      \onslide<\propReadRightFr-\befPropWriteRightFr>
        { \ptrPromiseRight{rightSecondInst} } 
      \onslide<\propWriteRightFr->
        { \ptrPromiseRight{rightThirdInst} } 

      \loadBufferValues{0}{0}

      \node at (leftFirstInst)   {\Large $\readInst{a}{x};$};
      \node at (leftSecondInst)  {\Large $\writeInst{y}{1};$};
      \node at (rightFirstInst)  {\Large $\readInst{b}{y};$};
      \node at (rightSecondInst) {\Large $\writeInst{x}{1};$};

      \draw[-,ultra thick] ($(instMiddle) + (-0.1,0.5)$) -- ($(instMiddle) + (-0.1,-0.5-0.8)$);
      \draw[-,ultra thick] ($(instMiddle) + ( 0.1,0.5)$) -- ($(instMiddle) + ( 0.1,-0.5-0.8)$);
      
      \onslide<\noWeakFr>
        { \draw[fill=red!30,rounded corners=3pt]
            ($(middleBuffer) + (-5.4, -1.0)$) rectangle
            ($(middleBuffer) + ( 5.4, -2.0)$);
          \node at ($(middleBuffer) + (0, -1.5)$)
          { \huge Невозможно получить $a = b = 1$ }; }
    \end{tikzpicture}
  }
}

\phrase{
  \LARGE
  Не-SC исполнения --- {\bf слабые}
  \vfill
  \pause
  {\bf Слабые} MM разрешают \\ слабые исполнения 
  \vfill
  \pause
  Реальные системы имеют слабые MM \\
  \pause
  {\Large (x86, Power, ARM, C++, Java)}
}

\phrase{
  Почему системы имеют слабые MM?
}

\fr{\huge Процессоры и компиляторы \newline {\bf оптимизируют} программы}{
  \setorder{{firstFr}, {listFr}, {correctFr}, {oneThreadFr}, {weakFr}}
  \LARGE
  
  \uncover<\listFr->{
    \ite{
    \large
      \item переупорядочивание инструкций \tikzmark{reorder}
      \item кэш
      \item буферы
      \item удаление чтения после записи 
      \item спекулятивное исполнение \tikzmark{fakeelim}
      %% \item удаление антизависимости 
      \item {\Huge \dots} \tikzmark{dots}
    }
  }

  \vfill
  
  \onslide<\weakFr->{
    Приводит к слабым исполнениям
  }
  
  \onslide<\correctFr>{
    \tikz[remember picture, overlay]{
      \draw [decorate,decoration={brace,amplitude=10pt},xshift=-4pt,yshift=0pt,thick]
            let \p1 = (reorder) in
            let \p2 = (dots) in
            let \p3 = (reorder) in
            (\x3, \y1) -- (\x3, \y2)
            node [anchor=west,black,midway,xshift=0.4cm] {\Large Корректны};
    }
  }

  \onslide<\oneThreadFr->{
    \tikz[remember picture, overlay]{
      \draw [decorate,decoration={brace,amplitude=10pt},xshift=-4pt,yshift=0pt,thick]
            let \p1 = (reorder) in
            let \p2 = (dots) in
            let \p3 = (reorder) in
            (\x3, \y1) -- (\x3, \y2)
            node [anchor=west,black,midway,xshift=0.4cm] (test) {\Large Корректны};
      \node [anchor=west,below of=test, node distance=0.45cm,xshift=0.1cm] {\Large для {\bf одного}};
      \node [anchor=west,below of=test, node distance=0.9cm,xshift=-0.5cm] {\Large потока};
    }
  }
}

%% \fr{}{
%% \LARGE
%% $$\begin{array}{c}
%% \uncover<7->{\hspace{12pt} [x] := 0; [y] := 0} \\
%% \begin{array}{l||l}
%% \only<-10,13->{\inarr{
%%   \tikzmark{reorder1} {} [x] := 1;  \\
%%   \tikzmark{reorder2} a := [y]
%% }}
%% \only<11-12>{\inarr{
%%   a := [y]; \\
%%   {} [x] := 1
%% }}
%% \uncover<7->{& \inarr{
%%   {} [y] := 1; \\
%%   b := [x]
%% }}
%% \end{array}\end{array}$$

%% \mycallout<2>{green}{(reorder1)}{(-0.5cm, -0.5cm)}{{\bf Компилятор:} \\ Независимые обращения. \\ Можно переупорядочить.}
%% \mycallout<3>{blue}{(reorder2)}{(-0.5cm, 1.0cm)}{{\bf Процессор:} \\ Независимые обращения. \\ Можно выполнить не по порядку.}

%% \dimalert<5>{Всегда ли корректны такие преобразования?}

%% \only<8-11>{\cntrd{\textcolor{red}{a = b = 0}}}

%% \dimalert<9>{А если переупорядочить?}

%% \only<12->{\cntrd{\textcolor{darkGreen}{a = b = 0}}}
%% \only<14->{\cntrd{\bf \alert{Такое поведение наблюдается для GCC + x86!}}}

%%     \tikz[remember picture, overlay]{ \draw<10,13->[<->,ultra thick] (reorder2) to[out=178,in=182] (reorder1); }
%% }

%% \fr{}{
%%   \huge
%%   Подобные поведения называются {\bf слабыми}\pause,
%%   а семантики --- {\bf слабыми моделями памяти}
%% }

%% \fr{}{

%%   Модели памяти процессоров
%%   \uncover<2->{
%%     \ite{
%%     \item \toGray{7}{8}{x86, \cite{Owens-al:TPHOL09}} \tikzmark{x86}
%%     \item \toGray{7}{8}{Power, \cite{Alglave-al:TOPLAS14}} \tikzmark{Power}
%%     \item ARM,\tikzmark{ARM} \cite{Flur-al:POPL16}
%%     \item ...
%%     }
%%   }
%%   \vfill

%%   Модели памяти ЯП
%%   \uncover<3->{
%%     \ite{
%%     %% \item \toGray{4}{5}{Последовательная консистентность, \cite{Lamport:TC79}}
%%     \item \toGray{4}{5}{C/C++11, \cite{Batty-al:POPL11}}
%%     \item \toGray{4}{5}{Java\tikzmark{jmm}, \cite{Manson-al:POPL05}}
%%     \uncover<5->{ \item ``Обещающая''\tikzmark{Promise} семантика, \cite{Kang-al:POPL17} }
%%     }

%%     \mycallout<4>{red}{(jmm)}{(-2.0cm, 0.5cm)}{Имеют ряд существенных недостатков}
%%   }

%%   %% \uncover<6>{%
%%     \tikz[remember picture, overlay]{
%%       \draw<6-8>[-,ultra thick]
%%         let \p1 = (x86) in
%%         let \p2 = (Power) in
%%         (\x2, \y1) -- (\x2 ++ 0.5cm, \y1) --
%%         (\x2 ++ 0.5cm, \y2) -- (\x2, \y2);

%%         \node<7-8>[above right = -0.8cm and 1.0cm of x86, align=left, fill=green!20, rounded corners, draw,rectangle] (e) 
%%           {Корректность компиляции \\ показана в \cite{Kang-al:POPL17}};
%%         \node<8>[below right = 0.8cm and 1.0cm of x86, align=left, fill=red!20, rounded corners, draw,rectangle] (f) 
%%           {Та же схема доказательства \\ не подходит для ARM!};
%%         \draw<9>[->,red,ultra thick] (Promise) to[out=30,in=-30] (ARM);
%%     }
%%   %% }
%% }
\phrase{
  Модель памяти для ЯП? \pause \\
  Должна соответствовать \\ \textbf{3} критериям
}

\fr{\huge 1. Эффективная компиляция}{
  \setorder{{firstFr},{programFr}, {scmmFr}, {powermmFr}, {compiledFr}, {noteffFr}}
  %% \LARGE
  %% {\bf Ex.} SC MM requires fences

  \cntrd{
    \begin{tikzpicture}
      \node (leftFirstInst)   {};
      \node (leftSecondInst)  [below of= leftFirstInst, node distance = 0.8cm] {};
      \node (leftThirdInst)   [below of= leftSecondInst, node distance = 0.8cm] {};
      \node (rightFirstInst)  [right of=  leftFirstInst, node distance = 3.0cm] {};
      \node (rightSecondInst) [below of= rightFirstInst, node distance = 0.8cm] {};
      \node (rightThirdInst)  [below of= rightSecondInst, node distance = 0.8cm] {};

      \node (instMiddle)   at ($.5*(leftFirstInst) + .5*(rightFirstInst)$) {};
      
      \uncover<\programFr->{
        \node at (leftFirstInst)   {\Large $\readInst{a}{x};$};
        \node at (leftSecondInst)  {\Large $\writeInst{y}{1};$};
        \node at (rightFirstInst)  {\Large $\readInst{b}{y};$};
        \node at (rightSecondInst) {\Large $\writeInst{x}{1};$};

        \draw[-,ultra thick] ($(instMiddle) + (-0.1,0.5)$) -- ($(instMiddle) + (-0.1,-0.5-0.8)$);
        \draw[-,ultra thick] ($(instMiddle) + ( 0.1,0.5)$) -- ($(instMiddle) + ( 0.1,-0.5-0.8)$);
      }
      
      \uncover<\scmmFr->{
        \node[anchor=west] at
          ($.5*(leftFirstInst) + .5*(leftSecondInst) + (-7.0, 0.0)$) {\large Исходная модель (SC MM)};
      }

      \node (leftFirstInstC)   [below of= leftFirstInst, node distance = 4cm] {};
      \node (leftSecondInstC)  [below of= leftFirstInstC, node distance = 0.8cm] {};
      \node (leftThirdInstC)   [below of= leftSecondInstC, node distance = 0.8cm] {};
      \node (rightFirstInstC)  [right of=  leftFirstInstC, node distance = 3.0cm] {};
      \node (rightSecondInstC) [below of= rightFirstInstC, node distance = 0.8cm] {};
      \node (rightThirdInstC)  [below of= rightSecondInstC, node distance = 0.8cm] {};

      \node (instMiddleC)   at ($.5*(leftFirstInstC) + .5*(rightFirstInstC)$) {};
      
      \uncover<\compiledFr->{
        \node at (leftFirstInstC)   {\Large $\readInst{a}{x};$};
        \node at (leftSecondInstC)  {\Large ${\rm lwsync};$};
        \node at (leftThirdInstC)  {\Large $\writeInst{y}{1};$};
        \node at (rightFirstInstC)  {\Large $\readInst{b}{y};$};
        \node at (rightSecondInstC) {\Large ${\rm lwsync};$};
        \node at (rightThirdInstC) {\Large $\writeInst{x}{1};$};

        \draw[-,ultra thick] ($(instMiddleC) + (-0.1,0.5)$) -- ($(instMiddleC) + (-0.1,-0.5-1.6)$);
        \draw[-,ultra thick] ($(instMiddleC) + ( 0.1,0.5)$) -- ($(instMiddleC) + ( 0.1,-0.5-1.6)$);
      }

      \node<\powermmFr->[anchor=west] at ($(leftSecondInstC) + (-7.0, 0.0)$) {\large Целевая модель (Power MM)};

      \draw<\compiledFr->[->,thick] ($(instMiddle) + (0.0,-1.5)$) -- ($(instMiddleC) + (0.0,1.0)$);
      \node<\noteffFr->[anchor=east] at ($.5*(instMiddle) + .5*(instMiddleC) + (0.0,-0.2)$)
        {\Large \bf \textcolor{darkRed}{Не эффективно!}};
    \end{tikzpicture}
  }
}

\fr{\huge 2. Компиляторные оптимизации}{
  \setorder{{firstFr},{programFr}, {scmmFr}, {powermmFr, compiledFr}, {behFr}}

  \cntrd{
    \begin{tikzpicture}
      \node (leftFirstInst)   {};
      \node (leftSecondInst)  [below of= leftFirstInst, node distance = 0.8cm] {};
      \node (leftThirdInst)   [below of= leftSecondInst, node distance = 0.8cm] {};
      \node (rightFirstInst)  [right of=  leftFirstInst, node distance = 3.0cm] {};
      \node (rightSecondInst) [below of= rightFirstInst, node distance = 0.8cm] {};
      \node (rightThirdInst)  [below of= rightSecondInst, node distance = 0.8cm] {};

      \node (instMiddle)   at ($.5*(leftFirstInst) + .5*(rightFirstInst)$) {};
      
      \uncover<\programFr->{
        \node at (leftFirstInst)   {\Large $\readInst{a}{x};$};
        \node at (leftSecondInst)  {\Large $\writeInst{y}{1};$};
        \node at (rightFirstInst)  {\Large $\readInst{b}{y};$};
        \node at (rightSecondInst) {\Large $\writeInst{x}{1};$};

        \draw[-,ultra thick] ($(instMiddle) + (-0.1,0.5)$) -- ($(instMiddle) + (-0.1,-0.5-0.8)$);
        \draw[-,ultra thick] ($(instMiddle) + ( 0.1,0.5)$) -- ($(instMiddle) + ( 0.1,-0.5-0.8)$);
      }
      
      \uncover<\scmmFr->{
        \node[anchor=west] at
          ($.5*(leftFirstInst) + .5*(leftSecondInst) + (-7.0, 0.0)$) {\large Исходная};
      }

      \node (leftFirstInstC)   [below of= leftFirstInst, node distance = 4cm] {};
      \node (leftSecondInstC)  [below of= leftFirstInstC, node distance = 0.8cm] {};
      \node (leftThirdInstC)   [below of= leftSecondInstC, node distance = 0.8cm] {};
      \node (rightFirstInstC)  [right of=  leftFirstInstC, node distance = 3.0cm] {};
      \node (rightSecondInstC) [below of= rightFirstInstC, node distance = 0.8cm] {};
      \node (rightThirdInstC)  [below of= rightSecondInstC, node distance = 0.8cm] {};

      \node (instMiddleC)   at ($.5*(leftFirstInstC) + .5*(rightFirstInstC)$) {};
      
      \uncover<\compiledFr->{
        \node at (leftFirstInstC)   {\Large $\writeInst{y}{1};$};
        \node at (leftSecondInstC)  {\Large $\readInst{a}{x};$};
        \node at (rightFirstInstC)  {\Large $\readInst{b}{y};$};
        \node at (rightSecondInstC) {\Large $\writeInst{x}{1};$};

        \draw[-,ultra thick] ($(instMiddleC) + (-0.1,0.5)$) -- ($(instMiddleC) + (-0.1,-0.5-0.8)$);
        \draw[-,ultra thick] ($(instMiddleC) + ( 0.1,0.5)$) -- ($(instMiddleC) + ( 0.1,-0.5-0.8)$);
      }

      \uncover<\powermmFr->{
        \node[anchor=west] at
          ($.5*(leftFirstInstC) + .5*(leftSecondInstC) + (-7.0, 0.0)$) {\large Оптимизированная};
      }
      
      \uncover<\behFr->{
        \draw [thick] ($(leftFirstInst) + (-0.9, 0.5)$)
          to [square right brace] ($(leftSecondInst) + (-0.9, -0.5)$);
        \draw [thick] ($(leftFirstInst) + (-0.9-0.15, 0.5)$)
          to [square right brace] ($(leftSecondInst) + (-0.9-0.15, -0.5)$);
        \draw [thick] ($(rightFirstInst) + (0.9, 0.5)$)
          to [square left brace] ($(rightSecondInst) + (0.9, -0.5)$);
        \draw [thick] ($(rightFirstInst) + (0.9+0.15, 0.5)$)
          to [square left brace] ($(rightSecondInst) + (0.9+0.15, -0.5)$);

        \draw [thick] ($(leftFirstInstC) + (-0.9, 0.5)$)
          to [square right brace] ($(leftSecondInstC) + (-0.9, -0.5)$);
        \draw [thick] ($(leftFirstInstC) + (-0.9-0.15, 0.5)$)
          to [square right brace] ($(leftSecondInstC) + (-0.9-0.15, -0.5)$);
        \draw [thick] ($(rightFirstInstC) + (0.9, 0.5)$)
          to [square left brace] ($(rightSecondInstC) + (0.9, -0.5)$);
        \draw [thick] ($(rightFirstInstC) + (0.9+0.15, 0.5)$)
          to [square left brace] ($(rightSecondInstC) + (0.9+0.15, -0.5)$);
      }

      \node<\behFr->[rotate=90] at ($.5*(instMiddle) + .5*(instMiddleC) + (0.0,-0.5)$) {\Huge $\subseteq$};
      %% \node<\noteffFr->[anchor=east] at ($.5*(instMiddle) + .5*(instMiddleC) + (0.0,-0.2)$)
      %%   {\Large \bf \textcolor{darkRed}{Not efficient}};
    \end{tikzpicture}
  }
}

\fr{\huge 3. Отсутствие ``значений из воздуха''}{
  \setorder{{firstFr}, {programFr}, {ootaFr}}

  \cntrd{
    \begin{tikzpicture}
      \node (leftFirstInst)   {};
      \node (leftSecondInst)  [below of= leftFirstInst, node distance = 0.8cm] {};
      \node (leftThirdInst)   [below of= leftSecondInst, node distance = 0.8cm] {};
      \node (rightFirstInst)  [right of=  leftFirstInst, node distance = 3.0cm] {};
      \node (rightSecondInst) [below of= rightFirstInst, node distance = 0.8cm] {};
      \node (rightThirdInst)  [below of= rightSecondInst, node distance = 0.8cm] {};

      \node (instMiddle)   at ($.5*(leftFirstInst) + .5*(rightFirstInst)$) {};
      
      \uncover<\programFr->{
        \node at (leftFirstInst)   {\Large $\readInst{a}{x};$};
        \node at (leftSecondInst)  {\Large $\writeInst{y}{a};$};
        \node at (rightFirstInst)  {\Large $\readInst{b}{y};$};
        \node at (rightSecondInst) {\Large $\writeInst{x}{b};$};

        \draw[-,ultra thick] ($(instMiddle) + (-0.1,0.5)$) -- ($(instMiddle) + (-0.1,-0.5-0.8)$);
        \draw[-,ultra thick] ($(instMiddle) + ( 0.1,0.5)$) -- ($(instMiddle) + ( 0.1,-0.5-0.8)$);
      }

      \onslide<\ootaFr>
        { \draw[fill=red!30,rounded corners=3pt]
            ($(instMiddle) + (-5.4, -3.0)$) rectangle
            ($(instMiddle) + ( 5.4, -4.0)$);
          \node at ($(instMiddle) + (0, -3.5)$)
          { \huge C/C++11 разрешает $a = b = 8$ }; }
    \end{tikzpicture}
  }
}

\begin{frame}<-7>{\huge MM для языков программирования}
  \setorder{{firstFr},{reqFr},{appearFr},
            {notionEffImplFr, effImplFr},
            {notionComplOptFr, complOptFr},
            {notionHighLevelFr, ootaFr, highLevelFr},
            {ootaCppFr},
            {notionUndefBehFr}, {undefBehFr},
            {promiseFr},{complCorFr}}

  %% \large

  \uncover<\appearFr->{
  \begin{tabular}{@{}l l | l | l | l}
    \uncover<\effImplFr->{& {\bf ЭК}}   \uncover<\complOptFr->{& {\bf КО}}
    \uncover<\highLevelFr->{& Нет {\bf ЗВ}} \uncover<\undefBehFr->{& No {\bf UB}} \\
    %% \tabitem
    SC MM, \cite{Lamport:TC79}
      \uncover<\effImplFr->{& \fail}   \uncover<\complOptFr->{& \fail}
      \uncover<\highLevelFr->{& \tick} \uncover<\undefBehFr->{& \tick} \\
    %% \tabitem
    Java MM\tikzmark{jmm}, \cite{Manson-al:POPL05} \tikzmark{java}
      \uncover<\effImplFr->{& \tick}   \uncover<\complOptFr->{& \fail}
      \uncover<\highLevelFr->{& \tick} \uncover<\undefBehFr->{& \tick} \\
    %% \tabitem
    C/C++11 MM, \cite{Batty-al:POPL11} \tikzmark{cpp}
      \uncover<\effImplFr->{& \tick}   \uncover<\complOptFr->{& \tick$^{*}$}
      \uncover<\highLevelFr->{& \fail\tikzmark{ootaCppMark}} \uncover<\undefBehFr->{& \fail} \\
    \\
    \uncover<\promiseFr->{%
      %% \tabitem
      Proposed solution \cite{Kang-al:POPL17}\tikzmark{Promise}
      \uncover<\effImplFr->{& \tick\tikzmark{effImplMark}}   \uncover<\complOptFr->{& \tick}
      \uncover<\highLevelFr->{& \tick} \uncover<\undefBehFr->{& \tick} \\
        $\quad ~~~\; \Promise$ MM, for C/C++ and Java
    } \\
  \end{tabular}
  }
  %% \vfill
  
  \uncover<\reqFr->{
  Требования:
  \ite{
    \item возможность {\bf Э}ффективной {\bf К}омпиляции \\ (x86, Power, ARM)
    \item разрешение {\bf К}омпиляторных {\bf О}птимизаций
    \item отсутствие {\bf З}начений из {\bf В}оздуха
    %% \uncover<\notionUndefBehFr-> {\item avoid {\bf U}ndefined {\bf B}ehavior}
  }
  }


  \addtooverlay<\ootaCppFr>{%
    \draw[ultra thick, rounded corners=3pt] ($(ootaCppMark) + (-0.35,-0.25)$) rectangle ++(0.5,0.55);
  }
  %% \addtooverlay<\complCorFr>{%
  %%   \draw[colorPROM, ultra thick, rounded corners=3pt] ($(effImplMark) + (-0.45,-0.3)$) rectangle ++(0.6,0.6);
  %% }

 %%  \mycallout<\ootaFr>{yellow}{(oota)}{(-1.0cm, -1.0cm)}{
 %%    $\comment{Out\text{-}Of\text{-}Thin\text{-}Air}$ \\
 %%    $\begin{array}{l || l}
 %%    a := [x]; \comment{8} & b := [y]; \comment{8} \\
 %% {} [y] := a; & [x] := b; \\
 %%    \end{array}$
 %%  }
\end{frame}

\phrase{C/C++11 MM --- {\bf аксиоматическая} MM}

\fr{\huge Исполнения в C/C++11 MM}{
  \setorder{{firstFr}, {lbHighlightFr}, {bigLBFr}, {axiomsFr}, {removeValueFr}, {randomValueFr}}
  \prevFr{\befBigLBFr}{\bigLBFr}
  \prevFr{\befRemoveValueFr}{\removeValueFr}
  \prevFr{\befRandomValueFr}{\randomValueFr}

  \cntrd{
    \vspace{-1.5cm}
    \begin{tikzpicture}[every node/.style={transform shape}]
      \node (leftFirstInst)   {};
      \node (leftSecondInst)  [below of= leftFirstInst, node distance = 0.8cm] {};
      \node (rightFirstInst)  [right of= leftFirstInst, node distance = 3.0cm] {};
      \node (rightSecondInst) [below of=rightFirstInst, node distance = 0.8cm] {};

      \node (instMiddle)   at ($.5*(leftFirstInst) + .5*(rightFirstInst)$) {};
      \node (leftBuffer)   at (leftSecondInst)  {};
      \node (rightBuffer)  at (rightSecondInst) {};
      \node (middleBuffer) at ($.5*(leftBuffer) + .5*(rightBuffer)$) {};
      \node (finalValues)  at ($(middleBuffer) + (0, -4.5)$) {};
      
      %% \node (ptrLeft)  at ($.5*(rightFirstInst) + .5*(rightSecondInst) + (1.5, 0)$) {};
      %% \node (ptrRight) at ($(ptrLeft) + (1.0, 0)$) {};
      \node (ptrLeft)  at ($(instMiddle) + (0.0, -1.7)$) {};
      \node (ptrRight) at ($(ptrLeft) + (0.0, -1.5)$) {};

      %% \node (leftCurlyCenter)  at ($(ptrRight) + (1.0, 0)$) {};
      \node (leftCurlyCenter)  at ($(leftSecondInst) +  (-3.25, -4.0)$) {};
      \node (rightCurlyCenter) at ($(rightSecondInst) + ( 3.25, -4.0)$) {};
      
      \node (curlyTopShift) at (0, -1.5) {};
      \node (curlyBotShift) at (0,  1.5) {};
      \node (leftCurlyTop)  at ($(leftCurlyCenter)  + (curlyTopShift)$) {};
      \node (rightCurlyTop) at ($(rightCurlyCenter) + (curlyTopShift)$) {};

      \node (leftCurlyBot)  at ($(leftCurlyCenter)  + (curlyBotShift)$) {};
      \node (rightCurlyBot) at ($(rightCurlyCenter) + (curlyBotShift)$) {};
      
      \node at (leftFirstInst)   {\Large $\readInst{a}{x};$};
      \node<-\befRemoveValueFr> at (leftSecondInst)  {\Large $\writeInst{y}{1};$};
      \node<\removeValueFr->    at (leftSecondInst)  {\Large $\writeInst{y}{a};$};
      \node at (rightFirstInst)  {\Large $\readInst{b}{y};$};
      \node<-\befRemoveValueFr> at (rightSecondInst) {\Large $\writeInst{x}{1};$};
      \node<\removeValueFr->    at (rightSecondInst) {\Large $\writeInst{x}{b};$};

      \draw[-,thick] ($(instMiddle) + (-0.1,0.5)$) -- ($(instMiddle) + (-0.1,-0.5-0.8)$);
      \draw[-,thick] ($(instMiddle) + ( 0.1,0.5)$) -- ($(instMiddle) + ( 0.1,-0.5-0.8)$);
      
  \uncover<-\befBigLBFr>{
      \draw[->, ultra thick] (ptrLeft) -- (ptrRight);
      \draw [decorate,decoration={brace,amplitude=10pt,raise=4pt},yshift=0pt,thick]
        (leftCurlyTop) -- (leftCurlyBot) node [black,midway,xshift=-0.8cm] {};
      \draw [decorate,decoration={brace,amplitude=10pt,mirror,raise=4pt},yshift=0pt,thick]
        (rightCurlyTop) -- (rightCurlyBot) node [black,midway,xshift=0.8cm] {};

  \node (aGraphCenter) at ($(leftCurlyCenter) + (1.0, 0.0)$) {};
  \node (a1)  at ($(aGraphCenter) + (-0.5, 0.5)$) {$\rlab{}{x}{0}$ };
  \node (a2)  at ($(a1)  + (0, -1)$)  {$\wlab{}{y}{1}$ };
  \node (a11) at ($(a1)  + (1,  0)$)  {$\rlab{}{y}{0}$ };
  \node (a12) at ($(a11) + (0, -1)$)  {$\wlab{}{x}{1}$ };
  \draw[po] (a1)  edge  (a2);
  \draw[po] (a11) edge (a12);
  \draw[fr] (a11)  edge node[below] {\small $\lFR$} (a2);
  \draw[fr] (a1)  edge node[below] {} (a12);
  %% \draw[fr,bend right=30] (12) edge node[below] {\small $\lFR$} (1);

  \node at ($(aGraphCenter) + (1.1, -0.3)$) {\Huge ,};

  \node (bGraphCenter) at ($(aGraphCenter) + (2.5, 0.0)$) {};
  \node (b1)  at ($(bGraphCenter) + (-0.5, 0.5)$) {$\rlab{}{x}{0}$ };
  \node (b2)  at ($(b1)  + (0, -1)$)  {$\wlab{}{y}{1}$ };
  \node (b11) at ($(b1)  + (1,  0)$)  {$\rlab{}{y}{1}$ };
  \node (b12) at ($(b11) + (0, -1)$)  {$\wlab{}{x}{1}$ };
  \draw[po] (b1)  edge  (b2);
  \draw[po] (b11) edge (b12);
  \draw[fr] (b1)  edge node[below] {\small $\lFR$} (b12);
  \draw[rf] (b2)  edge node[above] {\small $\lRF$} (b11);

  \node at ($(bGraphCenter) + (1.1, -0.3)$) {\Huge ,};

  \node (cGraphCenter) at ($(bGraphCenter) + (2.5, 0.0)$) {};
  \node (c1)  at ($(cGraphCenter) + (-0.5, 0.5)$) {$\rlab{}{x}{1}$ };
  \node (c2)  at ($(c1)  + (0, -1)$)  {$\wlab{}{y}{1}$ };
  \node (c11) at ($(c1)  + (1,  0)$)  {$\rlab{}{y}{0}$ };
  \node (c12) at ($(c11) + (0, -1)$)  {$\wlab{}{x}{1}$ };
  \draw[po] (c1)  edge  (c2);
  \draw[po] (c11) edge (c12);
  \draw[fr] (c11) edge node[below] {\small $\lFR$} (c2);
  \draw[rf] (c12) edge node[above] {\small $\lRF$} (c1);

  \node at ($(cGraphCenter) + (1.1, -0.3)$) {\Huge ,};
  

  \node (dGraphCenter) at ($(cGraphCenter) + (2.5, 0.0)$) {};
  \draw<\lbHighlightFr>[colorPROM, ultra thick, rounded corners=3pt] ($(dGraphCenter) + (-1.0,-1.0)$) rectangle ++(2.0,2.0);

  \node (d1)  at ($(dGraphCenter) + (-0.5, 0.5)$) {$\rlab{}{x}{1}$ };
  \node (d2)  at ($(d1)  + (0, -1)$)  {$\wlab{}{y}{1}$ };
  \node (d11) at ($(d1)  + (1,  0)$)  {$\rlab{}{y}{1}$ };
  \node (d12) at ($(d11) + (0, -1)$)  {$\wlab{}{x}{1}$ };
  \draw[po] (d1)  edge (d2);
  \draw[po] (d11) edge (d12);
  \draw[rf] (d2)  edge node[below] {\small $\lRF$} (d11);
  \draw[rf] (d12) edge node[below] {} (d1);
  }
  

  \uncover<\bigLBFr->{
  \node at ($(ptrRight) + (1.9, -1.0)$) {
    \begin{tikzpicture}[scale=2.0,every node/.style={transform shape}]
  \node (dGraphCenter) at ($(0.0, 0.0)$) {};
  
  \uncover<-\befRandomValueFr>{
    \node (d1)  at ($(dGraphCenter) + (-0.5, 0.5)$) {$\rlab{}{x}{1}$ };
    \node (d2)  at ($(d1)  + (0, -1)$)  {$\wlab{}{y}{1}$ };
    \node (d11) at ($(d1)  + (1,  0)$)  {$\rlab{}{y}{1}$ };
    \node (d12) at ($(d11) + (0, -1)$)  {$\wlab{}{x}{1}$ };
  }

  \uncover<\randomValueFr->{
    \node (d1)  at ($(dGraphCenter) + (-0.5, 0.5)$) {$\rlab{}{x}{8}$ };
    \node (d2)  at ($(d1)  + (0, -1)$)  {$\wlab{}{y}{8}$ };
    \node (d11) at ($(d1)  + (1,  0)$)  {$\rlab{}{y}{8}$ };
    \node (d12) at ($(d11) + (0, -1)$)  {$\wlab{}{x}{8}$ };
  }
  \draw[po, ultra thick] (d1)  edge node[left]  {\small $\lPO$} (d2);
  \draw[po, ultra thick] (d11) edge node[right] {\small $\lPO$} (d12);
  \draw[rf, ultra thick] (d2)  edge node[below] {\small $\lRF$} (d11);
  \draw[rf, ultra thick] (d12) edge node[below] {} (d1);
    \end{tikzpicture}
    };
  }
    
  \uncover<\axiomsFr->{
  \node[anchor=west] at ($(ptrRight) + (-5.5,  0.0)$)
    {\LARGE Аксиомы:};
  \node[anchor=west] at ($(ptrRight) + (-5.5, -1.0)$)
    {\LARGE 1. $\lHB$ ациклично};
  \node[anchor=west] at ($(ptrRight) + (-5.5, -2.0)$)
    {\LARGE \dots};
  }

    \end{tikzpicture}
  }
}

\fr{\huge Постановка задачи}{
\Large
\begin{block}{Проблема}
  Не существует MM промышленного ЯП, удовлетворяющей основным критериям
\end{block}
\pause
\begin{block}{Цель}
  Разработать операционный аналог C/C++11 MM без значений из воздуха, \\ удовлетворяющий основным критериям
\end{block}
}

\section{Описание результатов}

\begin{frame}<-8>[label=resultsFrame]{\huge Результаты}
  \setorder{{firstFr},{opCppFr},
    {promiseDiscussionFr},{promisePropertiesFr},{promisePerspectiveFr},{promiseARMHoleFr},
    {compilationFr},{compilationAxFr},
    {opCppSectionFr},{armPOPSectionFr},{armAxSectionFr},{fullResultFr},{interpreterLinkFr},{thanksFr}}
  \prevFr{\befCompilationFr}{\compilationFr}
  \nextFr{\nextPromisePropertiesFr}{\promisePropertiesFr}

  \prevFr{\befOpCppSectionFr}{\opCppSectionFr}
  \nextFr{\nextArmAxSectionFr}{\armAxSectionFr}

  \large
  \begin{itemize}
    \uncover<\opCppFr->{\item
      \toShadow
        {-\befOpCppSectionFr,\opCppSectionFr,\nextArmAxSectionFr-}
        {\armPOPSectionFr,\armAxSectionFr}
      {Предложен операционный аналог модели памяти C/С++11 и реализован интерпретатор для него}
    } \\
    \only<\interpreterLinkFr->{ {\scriptsize [Интерпретатор: \url{github.com/anlun/OperationalSemanticsC11}]} }
    \vfill
    
    \only<\promiseDiscussionFr-\promiseARMHoleFr>{
      \begin{block}{Конкурирующая работа}
        \cntrd{ ``Обещающая'' семантика \cite{Kang-al:POPL17} }
        \uncover<\promisePropertiesFr->{
          \begin{itemize}
            \item Операционная
            \item Доказана корректность компиляции в MM x86 и Power
            \item Разрешает оптимизации
            \item Не имеет значений из воздуха
            \uncover<\promisePerspectiveFr->{
              \item Может стать частью стандартов C/C++ и Java
            }
            \uncover<\promiseARMHoleFr->{
              \vfill 
              \item \alert{Нет д-ва корректности компиляции в MM ARM}!
            }
          \end{itemize}
        }
      \end{block}
    }
    
    \uncover<\compilationFr->{
      \item 
        \toShadow
          {-\befOpCppSectionFr,\armPOPSectionFr,\nextArmAxSectionFr-}
          {\opCppSectionFr,\armAxSectionFr}
        {Доказана корректность компиляции из существенного подмножества\tikzmark{subset} ``обещающей'' модели (ККОМ) в операционную \ARMpop}
    }
    \vfill
    \uncover<\compilationAxFr->{
      %% \item
      %%   \toShadow
      %%     {-\befOpCppSectionFr,\armAxSectionFr,\nextArmAxSectionFr-}
      %%     {\opCppSectionFr,\armPOPSectionFr}
      %%   {Разработан метод доказательства ККОМ в аксиоматические модели}
      %% \vfill
      \item
        \toShadow
          {-\befOpCppSectionFr,\armAxSectionFr,\nextArmAxSectionFr-}
          {\opCppSectionFr,\armPOPSectionFr}
        {Доказана ККОМ в аксиоматическую $\ARMax$ с помощью метода, основанного на построении операционной семантики по аксиоматической}
    }
  \end{itemize}

  %% \mycallout<2>[left]{green}{(subset)}{(-0.5cm, 1.0cm)}{Расслабленные (relaxed) \\ чтения и записи,\\
  %%                                                       высвобождающие (release) и \\
  %%                                                       приобретающие (acquire) \\ барьеры памяти}

  \uncover<\thanksFr->{\cntrd{\huge \textcolor{red}{Спасибо!}}}
\end{frame}

\againframe<9>{resultsFrame}

\fr{\Large Исполнение в oперационной модели (\OpCpp)}{
  \setorder{{firstFr, memoryAppearFr},{viewAppearFr},
            {pointerAppearFr},
            {postponeReadFr,propReadLeftFr},
            {promLeftFr,fulfillLeftFr},
            {readRightFr, propReadRightFr},{fulfillRightFr},
            {readLeftFr}}
  \prevFr{\befReadLeftFr}{\readLeftFr}
  \prevFr{\befPromLeftFr}{\promLeftFr}
  \prevFr{\befPropReadLeftFr}{\propReadLeftFr}
  \prevFr{\befFulfillLeftFr}{\fulfillLeftFr}

  \prevFr{\befReadRightFr}{\readRightFr}
  \prevFr{\befPropReadRightFr}{\propReadRightFr}
  \prevFr{\befFulfillRightFr}{\fulfillRightFr}

  \prevFr{\befPostponeReadFr}{\postponeReadFr}

  \cntrd{
    \begin{tikzpicture}
      \node (leftFirstInst)   {};
      \node (leftSecondInst)  [below of= leftFirstInst , node distance = 0.8cm] {};
      \node (leftThirdInst)   [below of= leftSecondInst, node distance = 0.8cm] {};
      \node (rightFirstInst)  [right of= leftFirstInst , node distance = 3.0cm] {};
      \node (rightSecondInst) [below of=rightFirstInst , node distance = 0.8cm] {};
      \node (rightThirdInst)  [below of=rightSecondInst, node distance = 0.8cm] {};

      \node (instMiddle)   at ($.5*(leftFirstInst) + .5*(rightFirstInst)$) {};
      \node (leftBuffer)   at (leftSecondInst)  {};
      \node (rightBuffer)  at (rightSecondInst) {};
      \node (middleBuffer) at ($.5*(leftBuffer) + .5*(rightBuffer)$) {};

      \node (memory)           at ($(middleBuffer) + (0.0,-2.4)$) {};
      \node (memorySecondLine) at ($(memory)       + (-1.35,-0.6)$) {};
      
      \onslide<\postponeReadFr-\befReadLeftFr>
        { \postponedBorder{leftFirstInst}
          \postponedCommentLeft{leftFirstInst} }
      %% \onslide<\promRightFr-\readRightFr>
      %%   { \promisedBorder{rightSecondInst}
      %%     \promisedCommentRight{rightSecondInst} }
      
      \onslide<\pointerAppearFr-\befPostponeReadFr>
        { \ptrPromiseLeft{leftFirstInst} } 
      \onslide<\postponeReadFr-\befFulfillLeftFr>
        { \ptrPromiseLeft{leftSecondInst} } 
      \onslide<\fulfillLeftFr->
        { \ptrPromiseLeft{leftThirdInst} } 

      \onslide<\pointerAppearFr-\befReadRightFr>
        { \ptrPromiseRight{rightFirstInst} }
      \onslide<\readRightFr-\befFulfillRightFr>
        { \ptrPromiseRight{rightSecondInst} } 
      \onslide<\fulfillRightFr->
        { \ptrPromiseRight{rightThirdInst} } 
      
      \node (memoryBackLeft)     at ($(memory) + (-5.5,1.5)$) {};
      \node (memoryBackRightOne) at ($(memory) + (5.5,-0.5)$) {};
      \node (memoryBackRightTwo) at ($(memory) + (5.5,-1.0)$) {};

      \onslide<\memoryAppearFr-\befPromLeftFr>{
        \draw[draw,fill=yellow!20] (memoryBackLeft) rectangle (memoryBackRightOne);
        \node at (memory) {\Large Память: $[\promMsg{x}{0}{0},\promMsg{y}{0}{0}]$};
      }
      \onslide<\promLeftFr->{
        \draw[draw,fill=yellow!20] (memoryBackLeft) rectangle (memoryBackRightTwo);
        \node at (memory) {\Large Память: $[\promMsg{x}{0}{0},\promMsg{y}{0}{0},$};
      }
      \onslide<\promLeftFr-\befFulfillRightFr>
        { \node [anchor=west] at (memorySecondLine) {\Large $\promMsg{y}{1}{1}]$}; }
      \onslide<\fulfillRightFr->
        { \node [anchor=west] at (memorySecondLine) {\Large $\promMsg{y}{1}{1}, \promMsg{x}{1}{1}]$}; }

      \node<\viewAppearFr-\befFulfillLeftFr> at ($(memory) + (0,1.0)$) {\Large
        \lviewTXT: $[x@\tstampWOsize{0},y@\tstampWOsize{0}] \quad$
        \rviewTXT: $[x@\tstampWOsize{0},y@\tstampWOsize{0}]$
      };
      \node<\fulfillLeftFr-\befReadRightFr> at ($(memory) + (0,1.0)$) {\Large
        \lviewTXT: $[x@\tstampWOsize{0},y@\tstampWOsize{1}] \quad$
        \rviewTXT: $[x@\tstampWOsize{0},y@\tstampWOsize{0}]$
      };
      \node<\readRightFr-\befFulfillRightFr> at ($(memory) + (0,1.0)$) {\Large
        \lviewTXT: $[x@\tstampWOsize{0},y@\tstampWOsize{1}] \quad$
        \rviewTXT: $[x@\tstampWOsize{0},y@\tstampWOsize{1}]$
      };
      \node<\fulfillRightFr-\befReadLeftFr> at ($(memory) + (0,1.0)$) {\Large
        \lviewTXT: $[x@\tstampWOsize{0},y@\tstampWOsize{1}] \quad$
        \rviewTXT: $[x@\tstampWOsize{1},y@\tstampWOsize{1}]$
      };
      \node<\readLeftFr-> at ($(memory) + (0,1.0)$) {\Large
        \lviewTXT: $[x@\tstampWOsize{1},y@\tstampWOsize{1}] \quad$
        \rviewTXT: $[x@\tstampWOsize{1},y@\tstampWOsize{1}]$
      };

      \node at (leftFirstInst)   {\Large $\readInst{a}{x};$};
      \node at (leftSecondInst)  {\Large $\writeInst{y}{1};$};
      \node at (rightFirstInst)  {\Large $\readInst{b}{y};$};
      \node at (rightSecondInst) {\Large $\writeInst{x}{1};$};

      \draw[-,ultra thick] ($(instMiddle) + (-0.1,0.5)$) -- ($(instMiddle) + (-0.1,-0.5-0.8)$);
      \draw[-,ultra thick] ($(instMiddle) + ( 0.1,0.5)$) -- ($(instMiddle) + ( 0.1,-0.5-0.8)$);
      
      
      \node (values)  at ($(middleBuffer) + (0, -4.5)$) {};
      \onslide<\firstFr->
        { \draw[draw,fill=yellow!20] ($(values) + (-1.9,  0.8)$) rectangle
                                     ($(values) + ( 1.9, -0.3)$) ;
          \node[anchor=north] at ($(values) + (0,  0.85)$) {\Large Регистры}; }
      \node at ($(values) + (-0.05, -0.1)$) {\Large $;$}; 
      \node<-\befReadLeftFr>[anchor=east] at (values) {\Large $a = \bot$}; 
      \node<\readLeftFr->[anchor=east] at (values) {\Large $a = \textcolor{colorVAL}{1}$}; 
      \node<-\befReadRightFr>[anchor=west] at (values) {\Large $b = \bot$}; 
      \node<\readRightFr->[anchor=west] at (values) {\Large $b = \textcolor{colorVAL}{1}$}; 
    \end{tikzpicture}
  }
}

\fri{\huge Операционная модель \OpCpp}{
  \Large
  \item Поддержаны все основные конструкции \\
    \textcolor{gray!50!white}{(relaxed, release, acquire, SC, fences, CAS)}
  \vfill
  \item Интерпретатор
  \begin{itemize}
    \large
    \item Реализован на Racket с помощью PLT/Redex
    \item \url{github.com/anlun/OperationalSemanticsC11}
  \end{itemize}
  \vfill
  \item Апробирована на 40 тестах
  \begin{itemize}
    \large
    \item Совпадает с C/C++11 MM на 36 тестах
    \item \tick: 2 теста --- ``значения из воздуха''
    \item \fail: 2 теста --- синтаксическое ограничение
  \end{itemize}
}

\againframe<10>{resultsFrame}

\fr{\Huge Корректность компиляции}{
  \LARGE

  $\onslide<3->{compile :} \tikzmark{proglang}S \onslide<3->{\rightarrow} \tikzmark{hardlang}\onslide<2->{T}$\\
  \onslide<4->{
  $\onslide<6->{\forall Prog \in S.} \\$
  $\onslide<5->{\quad \semState{\onslide<6->{compile(\tikzmark{hardsem}Prog)}}{T}}$
  $\onslide<6->{\subseteq}$
  $\semState{\onslide<6->{Prog}\tikzmark{progsem}}{S}$.
  }

  \mycallout<1>{green}{(proglang)}{(0.0cm, 1.0cm)}{Исходный язык}
  \mycallout<2>{green}{(hardlang)}{(0.0cm, 1.0cm)}{Целевой язык}

  \mycallout<4>{green}{(progsem)}{(0.0cm, 1.0cm)}{Исходная MM}
  \mycallout<5>{green}{(hardsem)}{(0.0cm, 1.0cm)}{Целевая MM}
}

\fr{\Large Исполнение в обещающей модели (\Promise)}{
  \setorder{{firstFr, memoryAppearFr,viewAppearFr,pointerAppearFr},
            {promLeftFr},
            {readRightFr, propReadRightFr},{fulfillRightFr},
            {readLeftFr, propReadLeftFr},{fulfillLeftFr}}
  \prevFr{\befReadLeftFr}{\readLeftFr}
  \prevFr{\befPropReadLeftFr}{\propReadLeftFr}
  \prevFr{\befFulfillLeftFr}{\fulfillLeftFr}

  \prevFr{\befReadRightFr}{\readRightFr}
  \prevFr{\befPropReadRightFr}{\propReadRightFr}
  \prevFr{\befFulfillRightFr}{\fulfillRightFr}

  \cntrd{
    \begin{tikzpicture}
      \node (leftFirstInst)   {};
      \node (leftSecondInst)  [below of= leftFirstInst , node distance = 0.8cm] {};
      \node (leftThirdInst)   [below of= leftSecondInst, node distance = 0.8cm] {};
      \node (rightFirstInst)  [right of= leftFirstInst , node distance = 3.0cm] {};
      \node (rightSecondInst) [below of=rightFirstInst , node distance = 0.8cm] {};
      \node (rightThirdInst)  [below of=rightSecondInst, node distance = 0.8cm] {};

      \node (instMiddle)   at ($.5*(leftFirstInst) + .5*(rightFirstInst)$) {};
      \node (leftBuffer)   at (leftSecondInst)  {};
      \node (rightBuffer)  at (rightSecondInst) {};
      \node (middleBuffer) at ($.5*(leftBuffer) + .5*(rightBuffer)$) {};

      \node (memory)           at ($(middleBuffer) + (0.0,-2.4)$) {};
      \node (memorySecondLine) at ($(memory)       + (-1.35,-0.6)$) {};
      
      \onslide<\promLeftFr-\readLeftFr>
        { \promisedBorder{leftSecondInst}
          \promisedCommentLeft{leftSecondInst} }
      %% \onslide<\promRightFr-\readRightFr>
      %%   { \promisedBorder{rightSecondInst}
      %%     \promisedCommentRight{rightSecondInst} }
      
      \onslide<\pointerAppearFr-\befReadLeftFr>
        { \ptrPromiseLeft{leftFirstInst} } 
      \onslide<\readLeftFr-\befFulfillLeftFr>
        { \ptrPromiseLeft{leftSecondInst} } 
      \onslide<\fulfillLeftFr->
        { \ptrPromiseLeft{leftThirdInst} } 

      \onslide<\pointerAppearFr-\befReadRightFr>
        { \ptrPromiseRight{rightFirstInst} }
      \onslide<\readRightFr-\befFulfillRightFr>
        { \ptrPromiseRight{rightSecondInst} } 
      \onslide<\fulfillRightFr->
        { \ptrPromiseRight{rightThirdInst} } 
      
      \node (memoryBackLeft)     at ($(memory) + (-5.5,1.5)$) {};
      \node (memoryBackRightOne) at ($(memory) + (5.5,-0.5)$) {};
      \node (memoryBackRightTwo) at ($(memory) + (5.5,-1.0)$) {};

      \onslide<\memoryAppearFr-\pointerAppearFr>{
        \draw[draw,fill=yellow!20] (memoryBackLeft) rectangle (memoryBackRightOne);
        \node at (memory) {\Large Память: $[\promMsg{x}{0}{0},\promMsg{y}{0}{0}]$};
      }
      \onslide<\promLeftFr->{
        \draw[draw,fill=yellow!20] (memoryBackLeft) rectangle (memoryBackRightTwo);
        \node at (memory) {\Large Память: $[\promMsg{x}{0}{0},\promMsg{y}{0}{0},$};
      }
      \onslide<\promLeftFr-\befFulfillRightFr>
        { \node [anchor=west] at (memorySecondLine) {\Large $\promMsg{y}{1}{1}]$}; }
      \onslide<\fulfillRightFr->
        { \node [anchor=west] at (memorySecondLine) {\Large $\promMsg{y}{1}{1}, \promMsg{x}{1}{1}]$}; }

      \node<\viewAppearFr-\befReadRightFr> at ($(memory) + (0,1.0)$) {\Large
        \lviewTXT: $[x@\tstampWOsize{0},y@\tstampWOsize{0}] \quad$
        \rviewTXT: $[x@\tstampWOsize{0},y@\tstampWOsize{0}]$
      };
      \node<\readRightFr-\befFulfillRightFr> at ($(memory) + (0,1.0)$) {\Large
        \lviewTXT: $[x@\tstampWOsize{0},y@\tstampWOsize{0}] \quad$
        \rviewTXT: $[x@\tstampWOsize{0},y@\tstampWOsize{1}]$
      };
      \node<\fulfillRightFr-\befReadLeftFr> at ($(memory) + (0,1.0)$) {\Large
        \lviewTXT: $[x@\tstampWOsize{0},y@\tstampWOsize{0}] \quad$
        \rviewTXT: $[x@\tstampWOsize{1},y@\tstampWOsize{1}]$
      };
      \node<\readLeftFr-\befFulfillLeftFr> at ($(memory) + (0,1.0)$) {\Large
        \lviewTXT: $[x@\tstampWOsize{1},y@\tstampWOsize{0}] \quad$
        \rviewTXT: $[x@\tstampWOsize{1},y@\tstampWOsize{1}]$
      };
      \node<\fulfillLeftFr-> at ($(memory) + (0,1.0)$) {\Large
        \lviewTXT: $[x@\tstampWOsize{1},y@\tstampWOsize{1}] \quad$
        \rviewTXT: $[x@\tstampWOsize{1},y@\tstampWOsize{1}]$
      };

      \node at (leftFirstInst)   {\Large $\readInst{a}{x};$};
      \node at (leftSecondInst)  {\Large $\writeInst{y}{1};$};
      \node at (rightFirstInst)  {\Large $\readInst{b}{y};$};
      \node at (rightSecondInst) {\Large $\writeInst{x}{1};$};

      \draw[-,ultra thick] ($(instMiddle) + (-0.1,0.5)$) -- ($(instMiddle) + (-0.1,-0.5-0.8)$);
      \draw[-,ultra thick] ($(instMiddle) + ( 0.1,0.5)$) -- ($(instMiddle) + ( 0.1,-0.5-0.8)$);
      
      
      \node (values)  at ($(middleBuffer) + (0, -4.5)$) {};
      \onslide<\firstFr->
        { \draw[draw,fill=yellow!20] ($(values) + (-1.9,  0.8)$) rectangle
                                     ($(values) + ( 1.9, -0.3)$) ;
          \node[anchor=north] at ($(values) + (0,  0.85)$) {\Large Регистры}; }
      \node at ($(values) + (-0.05, -0.1)$) {\Large $;$}; 
      \node<-\befPropReadLeftFr>[anchor=east] at (values) {\Large $a = \bot$}; 
      \node<\propReadLeftFr->[anchor=east] at (values) {\Large $a = \textcolor{colorVAL}{1}$}; 
      \node<-\befPropReadRightFr>[anchor=west] at (values) {\Large $b = \bot$}; 
      \node<\propReadRightFr->[anchor=west] at (values) {\Large $b = \textcolor{colorVAL}{1}$}; 
    \end{tikzpicture}
  }
}

\newcommand{\storageTwoThreadsRelative}[5]{
    \draw
      ($#1 + (0,-0.5)$) -- ($#1 + (0,-0.5-#3)$) -- ($#2 + (0,-0.5-#3)$) -- ($#2 + (0,-0.5)$);
      
    \draw
        let \p1 = #1 in
        let \p2 = #2 in
        (.5*\x1 + .5*\x2, #5) -- +(0,-#4);
}

\fr{\huge Исполнение в \ARMpop}{
  \setorder{{firstFr},{storageAppearFr},{grayBoxFr},
            {issueReadRightFr}, {comRightFr},
            {reorderArrowRightFr}, {reorderRightFr},
            {propWriteRightFr},
            {comLeftFr}, {propWriteLeftFr},
            {propReadRightFr},
            {issueReadLeftFr},{propReadLeftFr}}
  \prevFr{\befPropWriteLeftFr}{\propWriteLeftFr}
  \prevFr{\befPropReadLeftFr}{\propReadLeftFr}
  \prevFr{\befPropWriteRightFr}{\propWriteRightFr}
  \prevFr{\befPropReadRightFr}{\propReadRightFr}

  \prevFr{\befReorderRightFr}{\reorderRightFr}

  \cntrd{
    \begin{tikzpicture}
      \node (leftFirstInst)   {};
      \node (leftSecondInst)  [below of= leftFirstInst, node distance = 0.8cm] {};
      \node (rightFirstInst)  [right of= leftFirstInst, node distance = 3.0cm] {};
      \node (rightSecondInst) [below of=rightFirstInst, node distance = 0.8cm] {};

      \node (instMiddle)   at ($.5*(leftFirstInst) + .5*(rightFirstInst)$) {};
      \node (leftBuffer)   at (leftSecondInst)  {};
      \node (rightBuffer)  at (rightSecondInst) {};
      \node (middleBuffer) at ($.5*(leftBuffer) + .5*(rightBuffer)$) {};
      \node (finalValues)  at ($(middleBuffer) + (0, -4.5)$) {};

      \node (memoryValues) at ($(middleBuffer) + (0, -3.2)$) {};

      \node (firstLeftReq)   at ($(leftBuffer)  + (0.0, -1.0)$) {};
      \node (secondLeftReq)  at ($(leftBuffer)  + (0.0, -1.7)$) {};
      \node (firstRightReq)  at ($(rightBuffer) + (0.0, -1.0)$) {};
      \node (secondRightReq) at ($(rightBuffer) + (0.0, -1.7)$) {};
      
      \onslide<\storageAppearFr->{
        \storageTwoThreadsRelative{(leftBuffer)}{(rightBuffer)}{1.7}{0.5}{-3.0}
        \draw[draw,fill=yellow!20] ($(middleBuffer) + (-1.9,-2.4)$) rectangle ($(middleBuffer) + (1.9,-3.5)$) ;
        \node at ($(middleBuffer) + (0,-2.7)$) {\Large Память};
      }

      \onslide<\storageAppearFr-\befPropWriteRightFr>
        { \node at (memoryValues) 
          {\Large $\writeReq{x}{0}; \writeReq{y}{0}$}; }
      \onslide<\propWriteRightFr-\befPropWriteLeftFr>
        { \node at (memoryValues)
          {\Large $\writeReq{x}{\textcolor{colorVAL}{1}}; \writeReq{y}{0}$}; }
      \onslide<\propWriteLeftFr->
        { \node at (memoryValues)
          {\Large $\writeReq{x}{\textcolor{colorVAL}{1}}; \writeReq{y}{\textcolor{colorVAL}{1}}$}; }

      \onslide<\grayBoxFr->{
        \instBackground{colorFETCH}{leftFirstInst}
        \instBackground{colorFETCH}{leftSecondInst}
        \instBackground{colorFETCH}{rightFirstInst}
        \instBackground{colorFETCH}{rightSecondInst}
      }

      \onslide<\comLeftFr->
        {\instBackground{colorCOM  }{leftSecondInst}}
      \onslide<\comRightFr->
        {\instBackground{colorCOM  }{rightSecondInst}}

      \onslide<\issueReadLeftFr->
        {\instBackground{colorSAT  }{leftFirstInst}}
      \onslide<\issueReadRightFr->
        {\instBackground{colorSAT  }{rightFirstInst}}

      \onslide<\propReadLeftFr->
        {\instBackground{colorCOM  }{leftFirstInst}}
      \onslide<\propReadRightFr->
        {\instBackground{colorCOM  }{rightFirstInst}}
      
      \onslide<\comLeftFr-\befPropWriteLeftFr>
        { \node [anchor=east] at (firstLeftReq) {\Large $\writeReq{y}{1}$}; }
      \onslide<\issueReadLeftFr-\befPropReadLeftFr>
        { \node [anchor=east] at (firstLeftReq)  {\Large $\readReq{a}{x}$}; }

      \onslide<\comRightFr-\befReorderRightFr>
        { \node [anchor=west] at (firstRightReq) {\Large $\writeReq{x}{1}$}; }
      \onslide<\issueReadRightFr-\befReorderRightFr>
        { \node [anchor=west] at (secondRightReq)  {\Large $\readReq{b}{y}$}; }
      \onslide<\reorderRightFr-\befPropWriteRightFr>
        { \node [anchor=west] at (secondRightReq) {\Large $\writeReq{x}{1}$}; }
      \onslide<\reorderRightFr-\befPropReadRightFr>
        { \node [anchor=west] at (firstRightReq)  {\Large $\readReq{b}{y}$}; }
      
      \onslide<\reorderArrowRightFr-\befReorderRightFr>
        { \draw[-,red,ultra thick] ($(firstRightReq) + (0.1, 0)$)
            to[out=180,in=180] ($(secondRightReq) + (0.1, 0)$);
          \node[anchor=east] at ($(firstRightReq) + (-0.1, -0.4)$)  {\Large \textcolor{red}{Независимы}}; }

      \node at (leftFirstInst)   {\Large $\readInst{a}{x};$};
      \node at (leftSecondInst)  {\Large $\writeInst{y}{1};$};
      \node at (rightFirstInst)  {\Large $\readInst{b}{y};$};
      \node at (rightSecondInst) {\Large $\writeInst{x}{1};$};

      \draw[-,ultra thick] ($(instMiddle) + (-0.1,0.5)$) -- ($(instMiddle) + (-0.1,-0.5-0.8)$);
      \draw[-,ultra thick] ($(instMiddle) + ( 0.1,0.5)$) -- ($(instMiddle) + ( 0.1,-0.5-0.8)$);
      
      \onslide<-\befPropReadRightFr>
        { \node at (finalValues)
          { \Large $a = \_$, $b = \_$}; }
      \onslide<\propReadRightFr-\befPropReadLeftFr>
        { \node at (finalValues)
          { \Large $a = \_$, $b = \textcolor{colorVAL}{1}$}; }
      \onslide<\propReadLeftFr->
        { \node at (finalValues) 
          { \Large $a = \textcolor{colorVAL}{1}$, $b = \textcolor{colorVAL}{1}$}; }
    \end{tikzpicture}
  }
}

\fr{\huge Основные отличия \newline $\Promise$ от \ARMpop}{

  {\LARGE
  \begin{enumerate}
  \item В $\Promise$ только запись исполняется не по порядку  \\
  \vfill
  \item В $\ARMpop$ нет порядка над записями в одну локацию 
  \end{enumerate}
  }
}

\begin{frame}<1,2>[label=mainingred]{\Huge \toShadow{1}{2,3-}{Основа доказательства}}
  \begin{center}
  { \LARGE
    \toShadow{1,2}{3-}{1. ``Запаздывающая'' симуляция} \\
    \vfill
    \toShadow{1,3-}{2}{2. \ARMpop + метки времени}
  }
  \end{center}
\end{frame}

\fr{\huge ``Запаздывающая'' симуляция} {
  \setorder{{firstFr},{justInstFr},{grayBoxFr,grayBoxFEFr,grayBoxPEFr,grayBoxNEFr},{promPtrFr},
            {simRelOneFr}, {simRelTwoFr}, {simRelThreeFr},
            {satFirstFr}, {comFirstFr}, {promFirstFr}, %{satSecondFr}, 
            {comFifthFr}, {promWriteFifthFr}, %{satFourthOneFr},
            {comSecondFr}, {promSecondFr}, %{comFourthFr},
            %% {promFourthFr},
            {promFifthFr}}

  \prevFr{\befComFifthFr}{\comFifthFr}

  \cntrd{
    \begin{tikzpicture}
    \node (zeroInst)  {};
    \node (firstInst)  [below of=zeroInst  , node distance = 0.8cm] {}; 
    \node (secondInst) [below of=firstInst , node distance = 0.8cm] {}; 
    %% \node (thirdInst)  [below of=secondInst, node distance = 0.8cm] {}; 
    %% \node (fourthInst) [below of=secondInst , node distance = 0.8cm] {}; 
    \node (fifthInst)  [below of=secondInst, node distance = 0.8cm] {}; 
    \node (sixthInst)  [below of=fifthInst , node distance = 0.8cm] {}; 

    %% \onslide<\grayBoxFr-\promPtrFr> 
    %%   {\instBackground{colorFETCH}{zeroInst}};
    %% \onslide<\satZeroFr>
    %%   {\instBackground{colorSAT  }{zeroInst}};
    \onslide<\grayBoxFr->
      {\instBackground{colorCOM  }{zeroInst}};

    \onslide<\grayBoxFr-\simRelThreeFr>
      {\instBackground{colorFETCH}{firstInst}};
    \onslide<\satFirstFr>
      {\instBackground{colorSAT  }{firstInst}};
    \onslide<\comFirstFr->
      {\instBackground{colorCOM  }{firstInst}};

    %% \onslide<\grayBoxFr->
    %%   {\instBackground{colorFETCH}{secondInst}};
    \onslide<\grayBoxFr->
      {\instBackground{colorSAT  }{secondInst}};
    \onslide<\comSecondFr->
      {\instBackground{colorCOM  }{secondInst}};

    %% \onslide<\grayBoxFr->
    %%   {\instBackground{colorFETCH}{thirdInst}};
    %% \onslide<\comThirdFr->
    %%   {\instBackground{colorCOM  }{thirdInst}};

    %% \onslide<\grayBoxFr-\simRelThreeFr,\satSecondFr->
    %%   {\instBackground{colorFETCH}{fourthInst}};
    %% \onslide<\grayBoxFr-\promFirstFr,\satFourthOneFr->
    %%   {\instBackground{colorSAT  }{fourthInst}};
    %% \onslide<\comFourthFr->
    %%   {\instBackground{colorCOM  }{fourthInst}};

    \onslide<\grayBoxFr->
      {\instBackground{colorFETCH}{fifthInst}};
    \onslide<\comFifthFr->
      {\instBackground{colorCOM  }{fifthInst}};

    \onslide<\promWriteFifthFr-\promSecondFr>{
      \draw[colorPROM, ultra thick, rounded corners=3pt] ($(fifthInst)  + (-1.1,0.35)$) rectangle ++(2.2,-0.7);
      \node[anchor=west] at ($(fifthInst) + (1.4, 0.0)$) {\Large \textcolor{colorPROM}{\promisedTXT}};
    }

    %% \onslide<-\satZeroFr>  {\ptrPromise{zeroInst}};
    \onslide<\promPtrFr-\comFirstFr>     {\ptrPromise{firstInst}};
    \onslide<\promFirstFr-\comSecondFr> {\ptrPromise{secondInst}};
    %% \onslide<\promSecondFr>              {\ptrPromise{thirdInst}};
    %% \onslide<\promSecondFr-\comFourthFr> {\ptrPromise{fourthInst}};
    \onslide<\promSecondFr>              {\ptrPromise{fifthInst}};
    \onslide<\promFifthFr>               {\ptrPromise{sixthInst}};

    \node at (zeroInst)   {\Large $\readInst{a}{x};$};
    \node at (firstInst)  {\Large $\readInst{b}{y};$};
    \node at (secondInst) {\Large $\readInst{c}{x};$};
    %% \node at (thirdInst)  {\Large $\writeInst{z}{1};$};
    %% \node at (fourthInst) {\Large $\readInst{d}{x};$};
    \node at (fifthInst)  {\Large $\writeInst{y}{1};$};

    \onslide<\firstFr>{
      \node at (sixthInst)  {\fontsize{50}{60}\selectfont \dots};
      \draw[-,ultra thick] ($(zeroInst) + (1.4,0.5)$) -- ($(sixthInst) + (1.4,-0.5)$);
      \draw[-,ultra thick] ($(zeroInst) + (1.6,0.5)$) -- ($(sixthInst) + (1.6,-0.5)$);
      \node at ($(secondInst) + (2.5,0.0)$) {\fontsize{50}{60}\selectfont \dots};
    }
    
    %% \onslide<\grayBoxFEFr-\grayBoxNEFr>{
    %%   \draw[<-,ultra thick] ($(zeroInst)  + (1.2, 0.0)$) -- +(0.7,0);
    %%   \node[anchor=west] at ($(zeroInst)  + (2.0, 0.0)$) {\Large Fully};
    %%   \node[anchor=west] at ($(zeroInst)  + (3.9, 0.0)$) {\Large executed by $\ARM$};
    %% }
    %% \onslide<\grayBoxPEFr-\grayBoxNEFr>{
    %%   %% \draw[<-,ultra thick] ($(fourthInst)  + (1.2, 0.0)$) -- +(0.7,0);
    %%   %% \node[anchor=west] at ($(fourthInst) + (2.0, 0.0)$) {\Large Partially Executed by $\ARM$};
    %%   \draw[<-,ultra thick] ($(secondInst)  + (1.2, 0.0)$) -- +(0.7,0);
    %%   \node[anchor=west] at ($(secondInst) + (2.0, 0.0)$) {\Large Partially executed by $\ARM$};
    %% }
    %% \onslide<\grayBoxNEFr>{
    %%   \draw[<-,ultra thick] ($(fifthInst)  + (1.2, 0.0)$) -- +(0.7,0);
    %%   \node[anchor=west] at ($(fifthInst) + (2.0, 0.0)$) {\Large Not};
    %%   \node[anchor=west] at ($(fifthInst) + (3.9, 0.0)$) {\Large executed by $\ARM$};
    %% }

    %% \onslide<\comThirdFr->{
    %%   \draw[colorPROM, ultra thick, rounded corners=3pt] ($(thirdInst)  + (-1.2,0.45)$) rectangle ++(2.4,-0.9);
    %%   \node[anchor=west] at ($(thirdInst) + (2.0, 0.0)$) {\Large \textcolor{colorPROM}{Promised by $\Promise$}};
    %% }
    
    \onslide<\comFifthFr>{
      \fill[colorBUG, rounded corners=3pt] ($(sixthInst) + (-3.0,-0.9)$) rectangle ++(11.6,0.9);
    }
    
    \onslide<\simRelOneFr>{
      \node [anchor=west] at ($(sixthInst) + (-3.0,-0.5)$) { \huge $\simrel$ --- инвариант};
    }
    \onslide<\simRelTwoFr->{
      \node [anchor=west] at ($(sixthInst) + (-3.0,-0.5)$)
        { \huge $\simrel = \simrel_{\Promise\;\text{ждёт}} \cup \simrel_{\Promise\;\text{выполняется}}$ };
    }
    \onslide<\simRelThreeFr-\satFirstFr,\promFirstFr-\befComFifthFr,\promWriteFifthFr,\promFifthFr>{
      \draw[ultra thick, rounded corners=3pt] ($(sixthInst) + (-1.6,0.0)$) rectangle ++(3.65,-0.9);
    }
    \onslide<\comFirstFr,\comSecondFr,\promSecondFr>{
      \draw[ultra thick, rounded corners=3pt] ($(sixthInst) + (2.7,0.0)$) rectangle ++(5.35,-0.9);
    }
    \end{tikzpicture}
  }

  %% \mycallout<\timeNoteFr>[left]{green}{($(fifthInst) + (2.9, 0.0)$)}{(-0.5cm, 1.0cm)}{{\bf Note:} we assign \\ a timestamp}
}

\againframe<3>{mainingred}

\fr{\huge \ARMpop\only<11->{ + метки времени}}{
  \setorder{{firstFr}, {storageAppearFr}, {comLeftFr}, {comRightFr},
            {leftOptionFr}, {propLeftFr}, {undoPropLeftFr},
            {rightOptionFr}, {propRightFr},
            {alertFr}, {uncommitFr},
            {comLeftTwoFr}, {comRightTwoFr},
            {leftOptionTwoFr}, {rightOptionTwoFr}}

  \cntrd{
    \begin{tikzpicture}
      \node (leftInst)  {};
      \node (rightInst) [right of=leftInst, node distance = 3.0cm] {};
      \node (instMiddle)  at ($.5*(leftInst) + .5*(rightInst)$) {};
      %% \node (leftBuffer)  at ($(leftInst)  + (-0.75,0)$) {};
      \node (leftBuffer)  at (leftInst) {};
      %% \node (rightBuffer) at ($(rightInst) + (    0,0)$) {};
      \node (rightBuffer) at (rightInst) {};
      \node (middleBuffer) at ($.5*(leftBuffer) + .5*(rightBuffer)$) {};
      
      \onslide<\storageAppearFr->
        {\storageTwoThreadsRelative{(leftBuffer)}{(rightBuffer)}{1}{1}{-1.5}}

      \onslide<\storageAppearFr->
        {\instBackground{colorFETCH}{leftInst}};
      \onslide<\comLeftFr-\alertFr,\comLeftTwoFr->
        {\instBackground{colorCOM  }{leftInst}};

      \onslide<\storageAppearFr->
        {\instBackground{colorFETCH}{rightInst}};
      \onslide<\comRightFr-\alertFr,\comRightTwoFr->
        {\instBackground{colorCOM  }{rightInst}};
        
      \onslide<\comLeftFr-\leftOptionFr,\undoPropLeftFr-\alertFr>
        { \node [anchor=east] at ($(leftBuffer)  + (0.0, -1.0)$) {\Large $\writeReq{x}{1}$}; }
      \onslide<\comRightFr-\rightOptionFr>
        { \node [anchor=west] at ($(rightBuffer) + (0.0, -1.0)$) {\Large $\writeReq{x}{2}$}; }

      \onslide<\comLeftTwoFr->
        { \node [anchor=east] at ($(leftBuffer)  + (0.0, -1.0)$) {\Large $\writeReq{x}{1} \,@\, \tstampWOsize{3}$}; }
      \onslide<\comRightTwoFr->
        { \node [anchor=west] at ($(rightBuffer) + (0.0, -1.0)$) {\Large $\writeReq{x}{2} \,@\, \tstampWOsize{8}$}; }

      \onslide<\propLeftFr>
        { \node [anchor=east] at ($(middleBuffer) + (0.0, -2.0)$) {\Large $\writeReq{x}{1}$}; }
      \onslide<\propRightFr>
        { \node [anchor=west] at ($(middleBuffer) + (0.0, -2.0)$) {\Large $\writeReq{x}{2}$}; }
        
      \onslide<\leftOptionFr,\leftOptionTwoFr>
        { \draw[->,colorPROP,ultra thick] ($(leftBuffer)  + (-0.8,-1.5)$) to[out=-90,in=180] ($(middleBuffer) + (-0.5,-2.0)$); }
      \onslide<\rightOptionFr>
        { \draw[->,colorPROP,ultra thick] ($(rightBuffer) + ( 0.8,-1.5)$) to[out=-90,in=  0] ($(middleBuffer) + ( 0.5,-2.0)$); }
      \onslide<\rightOptionTwoFr>
        { \draw[->,colorNPROP,ultra thick] ($(rightBuffer) + ( 0.8,-1.5)$) to[out=-90,in=  0] ($(middleBuffer) + ( 0.5,-2.0)$); }

      \node at (leftInst)  {\Large $\writeInst{x}{1};$};
      \node at (rightInst) {\Large $\writeInst{x}{2};$};
      \draw[-,ultra thick] ($(instMiddle) + (-0.1,0.5)$) -- ($(instMiddle) + (-0.1,-0.5)$);
      \draw[-,ultra thick] ($(instMiddle) + ( 0.1,0.5)$) -- ($(instMiddle) + ( 0.1,-0.5)$);

      \onslide<\leftOptionTwoFr>{
        \node[fill=green!20, rounded corners, draw,rectangle]
          at ($(middleBuffer) + (0, -3.5)$) { \LARGE {\bf Можно} из левого}; }

      \onslide<\rightOptionTwoFr>{
        \node[fill=red!20, rounded corners, draw,rectangle]
          at ($(middleBuffer) + (0, -3.5)$) { \LARGE {\bf Нельзя} из правого! }; }
    \end{tikzpicture}
  
    %% \vspace{1cm}
    %% \onslide<\leftOptionTwoFr> { \LARGE Possible to propagate from left }
    %% \onslide<\rightOptionTwoFr>{ \LARGE {\bf Im}possible to propagate from right! }
  }
  \dimalert<\alertFr>{Будем определять порядок заранее!}
}

\fr{\LARGE Структура д-ва из $\Promise$ в \ARMpop}{
  \Large
  \iteN{
    \item Вводим промежуточную $\ARMt$
    \vfill
    \item Доказываем $\ARMt \approx \ARMpop$
    \vfill
    \item Показываем запаздывающую симуляцию для $\Promise$ и $\ARMt$
  }
}

\againframe<11>{resultsFrame}

\begin{frame}<-2>[label=armAxFrame]{\huge \only<-2>{Исполнение в \ARMax}\only<3->{Обход \ARMax}}
  \setorder{{firstFr},{bigLBFr},{titleTraverseFr},{legendFr},
            {issueLeftFr},
            {coverReadRightFr},{issueRightFr},{coverWriteRightFr},
            {coverReadLeftFr},{coverWriteLeftFr}}
  %% , {initEventsFr}
  \prevFr{\befBigLBFr}{\bigLBFr}
  \prevFr{\befTitleTraverseFr}{\titleTraverseFr}
  \prevFr{\befIssueLeftFr}{\issueLeftFr}
  \prevFr{\befCoverReadRightFr}{\coverReadRightFr}
  \prevFr{\befIssueRightFr}{\issueRightFr}
  \prevFr{\befCoverWriteRightFr}{\coverWriteRightFr}
  \prevFr{\befCoverReadLeftFr}{\coverReadLeftFr}
  \prevFr{\befCoverWriteLeftFr}{\coverWriteLeftFr}


  \cntrd{
    \vspace{-1.5cm}
    \begin{tikzpicture}[every node/.style={transform shape}]
      \node (leftFirstInst)   {};
      \node (leftSecondInst)  [below of= leftFirstInst, node distance = 0.8cm] {};
      \node (rightFirstInst)  [right of= leftFirstInst, node distance = 3.0cm] {};
      \node (rightSecondInst) [below of=rightFirstInst, node distance = 0.8cm] {};

      \node (instMiddle)   at ($.5*(leftFirstInst) + .5*(rightFirstInst)$) {};
      \node (leftBuffer)   at (leftSecondInst)  {};
      \node (rightBuffer)  at (rightSecondInst) {};
      \node (middleBuffer) at ($.5*(leftBuffer) + .5*(rightBuffer)$) {};
      \node (finalValues)  at ($(middleBuffer) + (0, -4.5)$) {};
      
      %% \node (ptrLeft)  at ($.5*(rightFirstInst) + .5*(rightSecondInst) + (1.5, 0)$) {};
      %% \node (ptrRight) at ($(ptrLeft) + (1.0, 0)$) {};
      \node (ptrLeft)  at ($(instMiddle) + (0.0, -1.7)$) {};
      \node (ptrRight) at ($(ptrLeft) + (0.0, -1.5)$) {};

      %% \node (leftCurlyCenter)  at ($(ptrRight) + (1.0, 0)$) {};
      \node (leftCurlyCenter)  at ($(leftSecondInst) +  (-3.25, -4.0)$) {};
      \node (rightCurlyCenter) at ($(rightSecondInst) + ( 3.25, -4.0)$) {};
      
      \node (curlyTopShift) at (0, -1.5) {};
      \node (curlyBotShift) at (0,  1.5) {};
      \node (leftCurlyTop)  at ($(leftCurlyCenter)  + (curlyTopShift)$) {};
      \node (rightCurlyTop) at ($(rightCurlyCenter) + (curlyTopShift)$) {};

      \node (leftCurlyBot)  at ($(leftCurlyCenter)  + (curlyBotShift)$) {};
      \node (rightCurlyBot) at ($(rightCurlyCenter) + (curlyBotShift)$) {};
      
      \node at (leftFirstInst)   {\Large $\readInst{a}{x};$};
      \node at (leftSecondInst)  {\Large $\writeInst{y}{1};$};
      \node at (rightFirstInst)  {\Large $\readInst{b}{y};$};
      \node at (rightSecondInst) {\Large $\writeInst{x}{1};$};

      \draw[-,thick] ($(instMiddle) + (-0.1,0.5)$) -- ($(instMiddle) + (-0.1,-0.5-0.8)$);
      \draw[-,thick] ($(instMiddle) + ( 0.1,0.5)$) -- ($(instMiddle) + ( 0.1,-0.5-0.8)$);
      
  \only<-\befTitleTraverseFr>{ \node (bigLBpos) at ($(ptrRight) + (0.0, -1.0)$)  {}; }
  \only<\titleTraverseFr->   { \node (bigLBpos) at ($(ptrRight) + (-2.0, -1.0)$) {}; }
      
  \uncover<\bigLBFr->{
  \node at (bigLBpos) {
    \begin{tikzpicture}[scale=2.0,every node/.style={transform shape}]
  \node (dGraphCenter) at ($(0.0, 0.0)$) {};

    \uncover<\coverReadRightFr-\befCoverWriteRightFr>{ \traverseSingleBorder{colorCOV}{d11} }
    \uncover<\coverWriteRightFr-\befCoverReadLeftFr>{ \traverseTwoVertBorder{colorCOV}{d11}{d12} }

    \uncover<\coverReadLeftFr-\befCoverWriteLeftFr>{
      \draw[pattern=custom north east lines, hatchcolor=colorCOV, thick, rounded corners]
        ($(d1) + (-0.45,0.4)$)   -- ($(d11) + (0.45,0.4)$)
        -- ($(d12) + (0.45,-0.4)$) -- ($(d12) + (-0.45,-0.4)$) -- ($(d11) + (-0.45,-0.4)$) -- ($(d1) + (-0.45,-0.4)$) -- cycle;
    }

    \uncover<\coverWriteLeftFr->{ \traverseTwoVertBorder{colorCOV}{d1}{d12} }

    \uncover<\issueLeftFr->{ \traverseSingleBorder{colorISS}{d2} }
    \uncover<\issueRightFr->{ \traverseSingleBorder{colorISS}{d12} }

    \node (d1)  at ($(dGraphCenter) + (-0.5, 0.5)$) {$\rlab{}{x}{1}$ };
    \node (d2)  at ($(d1)  + (0, -1)$)  {$\wlab{}{y}{1}$ };
    \node (d11) at ($(d1)  + (1,  0)$)  {$\rlab{}{y}{1}$ };
    \node (d12) at ($(d11) + (0, -1)$)  {$\wlab{}{x}{1}$ };

    %% \uncover<\initEventsFr->{
    %%   \node (di1) at ($(d1)  + (0, 1)$)  {$\wlab{}{x}{0}$ };
    %%   \node (di2) at ($(d11) + (0, 1)$)  {$\wlab{}{y}{0}$ };

    %%   \draw[po, ultra thick] (di1)  edge node {} (d1);  % {\small $\lPO$} (d1);
    %%   \draw[po, ultra thick] (di1)  edge node {} (d11); % {\small $\lPO$} (d11);
    %%   \draw[po, ultra thick] (di2)  edge node {} (d1);  % {\small $\lPO$} (d1);
    %%   \draw[po, ultra thick] (di2)  edge node {} (d11); % {\small $\lPO$} (d11);
    %% }

  \draw[po, ultra thick] (d1)  edge node  {} (d2);  % {\small $\lPO$} (d2);
  \draw[po, ultra thick] (d11) edge node {} (d12); % {\small $\lPO$} (d12);
  \draw[rf, ultra thick] (d2)  edge node {} (d11); % {\small $\lRF$} (d11);
  \draw[rf, ultra thick] (d12) edge node {} (d1);
    \end{tikzpicture}
    };
  }

  \only<\titleTraverseFr->{
    \uncover<\legendFr->{
      \node (covLegend) at ($(ptrRight) + (3.0, -0.2)$) {};
      \node (covEllL)   at ($(covLegend) + (-1.6,-0.5)$) {};
      \node (covEllR)   at ($(covLegend) + (1.6,0.5)$) {};

      \node (issLegend) at ($(ptrRight) + (3.0, -1.7)$) {};
      \node (issEllL)   at ($(issLegend) + (-1.6,-0.5)$) {};
      \node (issEllR)   at ($(issLegend) + (1.6,0.5)$) {};
  
      \draw[pattern=custom north east lines, hatchcolor=colorCOV, thick, rounded corners]
        (covEllL) rectangle (covEllR);
      \node at (covLegend) {\Large Покрытые};

      \draw[pattern=custom north east lines, hatchcolor=colorISS, thick, rounded corners]
        (issEllL) rectangle (issEllR);
      \node at (issLegend) {\Large Выпущенные};
    }
  }
    
    \end{tikzpicture}
  }
\end{frame}

\phrase{Как по графу построить исполнение \Promise? \\ \pause Нужен обход!}

\againframe<3->{armAxFrame}

\fr{\LARGE Структура д-ва из $\Promise$ в \ARMax}{
  \Large
  \iteN{
    \item Вводим обход \ARMax-исполнения
    \vfill
    \item Доказываем полноту обхода
    \vfill
    \item Показываем симуляцию для $\Promise$ и обхода 
  }
}

\section{Итоги}

\fri{\Large Теоретическая и практическая значимость работы}{
  \large
  \item Предложен операционный способ представления реалистичной семантики многопоточности с помощью меток времени и фронтов.
  \vfill
  \item Приводится метод доказательства корректности компиляции из ``обещающей'' в аксиоматические модели памяти,
  который может быть использован для последующих доказательств.
  \vfill
  \item Доказательство корректности эффективной компиляции для архитектуры ARM является
  необходимым аргументом в пользу ``обещающей'' модели памяти как новой модели памяти для C/C++.
}

\fr{\LARGE Публикации}{
  \scriptsize
  
  \begin{block}{\scriptsize ВАК}
    \begin{itemize}
      \item А. Подкопаев, О. Лахав, В. Вафеядис. \\
      О корректности компиляции подмножества обещающей модели памяти в аксиоматическую модель ARMv8.3 //
      НТВ СПбГПУ'17 \\
      \item А. Подкопаев, О. Лахав, В. Вафеядис. \\
      Обещающая компиляция в ARMv8.3 // Труды ИСП РАН'??
    \end{itemize}
  \end{block}

  \begin{block}{\scriptsize SCOPUS и Web of Science}
    \begin{itemize}
      \item A. Podkopaev, O. Lahav, V. Vafeiadis. \\
      Promising compilation to ARMv8 POP // ECOOP'17 \\
    \end{itemize}
  \end{block}

  \begin{block}{\scriptsize Остальные}
    \begin{itemize}
      \item А. Подкопаев, О. Лахав, В. Вафеядис. \\
        Обещающая компиляция в ARMv8 // PLC'17
      \item A. Podkopaev, I. Sergey, A. Nanevski \\
      Operational Aspects of C/C++ Concurrrency // arXiv'16; \\
    \end{itemize}
  \end{block}
}

%% \fr{\LARGE Публикации}{
%%   \footnotesize

%%   \begin{block}{ВАК}
%%     \begin{itemize}
%%       \item О корректности компиляции подмножества обещающей модели памяти в аксиоматическую модель \ARMax
%%         @ НТВ СПбГПУ'17 \#4; \\
%%         {\color{darkGreen}{*}}А. Подкопаев, О. Лахав, В. Вафеядис
%%       \item Обещающая компиляция в \ARMax{\color{darkGreen}{*}} @ Откр. конф. ИСП РАН'17
%%         %% (А. Подкопаев, О. Лахав, В. Вафеядис)
%%     \end{itemize}
%%   \end{block}

%%   \begin{block}{SCOPUS и Web of Science}
%%     \begin{itemize}
%%       \item Promising compilation to ARMv8 POP{\color{darkGreen}{*}} @ ECOOP'17 \\
%%          %% (A. Podkopaev, O. Lahav, V. Vafeiadis)
%%     \end{itemize}
%%   \end{block}

%%   \begin{block}{Остальные}
%%     \begin{itemize}
%%       \item Обещающая компиляция в ARMv8{\color{darkGreen}{*}} @ PLC'17
%%       \item Operational Aspects of C/C++ Concurrrency @ arXiv'16; \\
%%          A. Podkopaev, I. Sergey, A. Nanevski
%%     \end{itemize}
%%   \end{block}
%% }

\fri{\huge Планы на будущее}{
  \LARGE
  \item Компиляция из всей обещающей машины в $\ARMax$ и Power в Coq
  \vfill
  \item Логика для программ в обещающей машине
}

\againframe<13->{resultsFrame}

\begin{frame}[allowframebreaks]{Ссылки}
\bibliographystyle{apalike}
\scriptsize
\bibliography{main}
\end{frame}

%% \iffalse
%% \fi

%% \fri{\Huge Результаты данной работы}{
%% \LARGE
%%   \item Доказана корректность компиляции подмножества\tikzmark{subset} ``обещающей'' семантики \cite{Kang-al:POPL17}
%%         в модель ARMv8 \cite{Flur-al:POPL16}

%%   \mycallout<2>[left]{green}{(subset)}{(-0.5cm, 1.0cm)}{Расслабленные (relaxed) \\ чтения и записи,\\
%%                                                         высвобождающие (release) и \\
%%                                                         преобретающие (acquire) \\ барьеры памяти}

%%   \onslide<3->{
%%     \item Формализована модель ARMv8 \cite{Flur-al:POPL16} и доказаны вспомогательные утверждения про неё
%%   }
%% }

%% \phrase{Корректность компиляции...\\ \pause Что это значит?}

%% %% \fr{\LARGE Корректность компиляции (1/2)}{
%% %%   \LARGE

%% %%   $\onslide<3->{compile :} \tikzmark{proglang}S \onslide<3->{\rightarrow} \tikzmark{hardlang}\onslide<2->{T}$\\
%% %%   \onslide<4->{
%% %%   $\onslide<6->{\forall Prog \in S.}$ \\
%% %%   ${} \quad \onslide<5->{\semState{\onslide<6->{compile(\tikzmark{hardsem}Prog)}}{T}}$
%% %%   $\onslide<6->{\subseteq}$
%% %%   $\semState{\onslide<6->{Prog}\tikzmark{progsem}}{S}$.
%% %%   }

%% %%   \mycallout<1>{green}{(proglang)}{(0.0cm, 1.0cm)}{Исходный язык}
%% %%   \mycallout<2>{green}{(hardlang)}{(0.0cm, 1.0cm)}{Целевой язык}

%% %%   \mycallout<4>{green}{(progsem)}{( 1.0cm, 1.0cm)}{Модель Памяти исходного языка}
%% %%   \mycallout<5>{green}{(hardsem)}{(-1.0cm, 1.0cm)}{Модель Памяти целевого языка}
%% %% }

%% %% \phrase{И ``обещающая'', и ARMv8- семантики заданы {\bf операционно}:}
  
%% \fr{\huge Корректность компиляции из $S$ в $T$}{
%%   \LARGE
%%   $\onslide<2->{\forall \tikzmark{prog}Prog \in Syntax,}$ \\
%%   \onslide<3->{${} \;\; \{t_j\}_{j \in [1..k]}$ --} \tikzmark{hardlang}$T$\onslide<3->{-исполнение}
%%          \tikzmark{cprog}\only<2-7>{$compile(Prog)$}\only<8->{$Prog$}\onslide<3->{.} \\
%%   \onslide<4->{${} \quad \exists \{s_i\}_{i \in [1..n]}$ --} \tikzmark{proglang}$S$\onslide<4->{-исполнение $Prog$,} \\
%%   \onslide<5->{${} \quad s_n \tikzmark{stateeq}\simeq t_k.$}

%%   \mycallout<1>{green}{(proglang)}{(0.0cm, 1.0cm)}{Исходная модель}
%%   \mycallout<1>{green}{(hardlang)}{(0.0cm, -1.0cm)}{Целевая модель}
%%   \mycallout<2>{green}{(prog)}{(-2.0cm, -1.0cm)}{Исходная программа}
%%   \mycallout<2>{green}{(cprog)}{(0.0cm, 1.0cm)}{Результат компиляции}
%%   \mycallout<6>{green}{(stateeq)}{(-2.0cm, 1.0cm)}{e.g., финальное состояние \\ памяти совпадает}
%%   \mycallout<7>{yellow}{(cprog)}{(0.0cm, 1.0cm)}{Рассматриваем $compile = id$}
%% }

%% %% \phrase{А в случае \\ ``обещающей'' и ``ARM''- семантик?}

%% %% \fr{\huge Корректность компиляции \\ из $Promise$ в $ARM$}{
%% %%   \LARGE
%% %%   $\forall Prog \in Syntax,$ \\
%% %%   ${} \quad\{a_j\}_{j \in [1..k]}$ -- $ARM$-исполнение \only<2->{$Prog$.}\\
%% %%   \only<1>{${} \quad \quad com\tikzmark{cprog}pile(Prog)$. \\} %%   ${} \quad \exists \{p_i\}_{i \in [1..n]}$ -- $Promise$-исполнение $Prog$, \\
%% %%   ${} \quad p_n \tikzmark{stateeq}\simeq a_k.$
%% %% }

%% \phrase{Стандартная техника --- симуляция}

%% \fr{\huge Пример симуляции}{
%%   \cntrd{
%%   \begin{tikzpicture}
%%     \fill<-1>[gray!40!white , draw=black] (1,0) rectangle (2,1);
%%     \fill<2->[green!40!white, draw=black] (1,0) rectangle (2,1);

%%     \fill<-3>[gray!40!white , draw=black] (2,0) rectangle (3,1);
%%     \fill<4->[green!40!white, draw=black] (2,0) rectangle (3,1);

%%     \fill<-5>[gray!40!white , draw=black] (3,0) rectangle (4,1);
%%     \fill<6->[green!40!white, draw=black] (3,0) rectangle (4,1);

%%     \fill<-7>[gray!40!white , draw=black] (4,0) rectangle (5,1);
%%     \fill<8->[green!40!white, draw=black] (4,0) rectangle (5,1);

%%     \fill<- 9>[gray!40!white , draw=black] (5,0) rectangle (6,1);
%%     \fill<10->[green!40!white, draw=black] (5,0) rectangle (6,1);

%%     \node[opacity=.6] at (3.5,0.5) {\LARGE $Prog$};

%%     \draw<1,2>[->,ultra thick] (1,-0.4) -- (1,0);
%%     \node<1,2> at (1,-0.8) {\LARGE $S$};

%%     \draw<3,4>[->,ultra thick] (2,-0.4) -- (2,0);
%%     \node<3,4> at (2,-0.8) {\LARGE $S$};

%%     \draw<5,6>[->,ultra thick] (3,-0.4) -- (3,0);
%%     \node<5,6> at (3,-0.8) {\LARGE $S$};

%%     \draw<7,8>[->,ultra thick] (4,-0.4) -- (4,0);
%%     \node<7,8> at (4,-0.8) {\LARGE $S$};

%%     \draw<9,10>[->,ultra thick] (5,-0.4) -- (5,0);
%%     \node<9,10> at (5,-0.8) {\LARGE $S$};

%%     \draw<11>[->,ultra thick] (6,-0.4) -- (6,0);
%%     \node<11> at (6,-0.8) {\LARGE $S$};
%%   \end{tikzpicture}
%%   }

%%   Инвариант
%%   \onslide<2,4,6,8,10>{\textcolor{red}{не}}
%%   \only<2,4,6,8,10>{\textcolor{red}{выполняется}}\only<1,3,5,7,9,11>{\textcolor{darkGreen}{выполняется}}
  
%%   \vspace{1cm}
%%   \hrule
%%   \vspace{1cm}
  
%%   \begin{columns}

%%   \colu{.4\textwidth}{
%%     \begin{tikzpicture}
%%       \fill[green!40!white, draw=black] (0,0) rectangle (1,1);
%%       \node[right] at (1.2,0.5) {-- выполнено $T$};
%%     \end{tikzpicture}
%%   }

%%   \colu{.4\textwidth}{
%%     $S$ -- исходная модель\\
%%     $T$ -- целевая модель\\
%%   }

%%   \end{columns}
%% }

%% %% \fr{\huge Пример симуляции}{
%% %%   %% \cntrd{
%% %%   \begin{tikzpicture}
%% %%     \node at (3.7,2.2) {Состояние целевой машины};
%% %%     \draw[rounded corners] (-1,2.0) rectangle (6.2,4.2);
%% %%     \fill[gray!40!white, draw=black] (1,3) rectangle (2,4);
%% %%     \fill[gray!40!white, draw=black] (2,3) rectangle (3,4);
%% %%     \fill[gray!40!white, draw=black] (3,3) rectangle (4,4);
%% %%     \fill[gray!40!white, draw=black] (4,3) rectangle (5,4);
%% %%     \fill[gray!40!white, draw=black] (5,3) rectangle (6,4);

%% %%     \node at (-1,0.5) {Исходная машина};
%% %%     \fill[gray!40!white, draw=black] (1,0) rectangle (2,1);
%% %%     \fill[gray!40!white, draw=black] (2,0) rectangle (3,1);
%% %%     \fill[gray!40!white, draw=black] (3,0) rectangle (4,1);
%% %%     \fill[gray!40!white, draw=black] (4,0) rectangle (5,1);
%% %%     \fill[gray!40!white, draw=black] (5,0) rectangle (6,1);
    
%% %%     \node[opacity=.6] at (3.5,3.5) {\LARGE $compile(Prog)$};
%% %%     \node[opacity=.6] at (3.5,0.5) {\LARGE $Prog$};
%% %%     \node[left] at (4, -2) {Инвариант};
%% %%     %% \node[left] at (6.8, -2) {\textcolor{red}{не выполняется}};
%% %%     \node[left] at (6.3, -2) {\textcolor{darkGreen}{выполняется}};
%% %%   \end{tikzpicture}
%% %%   %% }
%% %% }

%% \fr{}{
%% \huge
%% ...но симуляция напрямую не применина в нашем случае\pause:
%%   \iteN{
%%     \LARGE
%%     \item ARM выполняет инструкции не по порядку;
%%     \item ``Обещающая'' семантика имеет больше явных ограничений.
%%   }
%% }

%% \fr{\huge План доказательства}{
%%   \LARGE
%%   \iteN{
%%     \pause
%%     \item \toGray{5}{6}{Вводим промежуточную семантику ARM+$\tau$}
%%     \pause
%%     \ite{
%%       \Large
%%       \item \toGray{5}{6}{обладает явными ограниченями, похожими на ``обещающую'' семантику;}
%%     }
%%     \pause
%%     \item \toGray{5}{6}{Доказываем бисимуляцию между ARM+$\tau$ и ARM;}
%%     \pause
%%     \item Показываем ``запаздывающую'' симуляцию ARM+$\tau$ ``обещающей'' семантикой.
%%   }
%% }

%% \fr{\huge ``Запаздывающая'' симуляция}{
%%   \cntrd{
%%   \begin{tikzpicture}
%%     \fill<-1>[gray!40!white , draw=black] (1,0) rectangle (2,1);
%%     \fill<2->[green!40!white, draw=black] (1,0) rectangle (2,1);

%%     \fill<-5,7-9>[gray!40!white , draw=black] (2,0) rectangle (3,1);
%%     \fill<6,10>[yellow!40!white, draw=black] (2,0) rectangle (3,1);
%%     \fill<11->[green!40!white, draw=black] (2,0) rectangle (3,1);

%%     \fill<-4>[gray!40!white , draw=black] (3,0) rectangle (4,1);
%%     \fill<5-8>[yellow!40!white, draw=black] (3,0) rectangle (4,1);
%%     \fill<9->[green!40!white, draw=black] (3,0) rectangle (4,1);

%%     \fill<-7>[gray!40!white , draw=black] (4,0) rectangle (5,1);
%%     \fill<8->[green!40!white, draw=black] (4,0) rectangle (5,1);

%%     \fill<-3>[gray!40!white , draw=black] (5,0) rectangle (6,1);
%%     \fill<4->[green!40!white, draw=black] (5,0) rectangle (6,1);

%%     \node[opacity=.6] at (3.5,0.5) {\LARGE $Prog$};

%%     \draw<1,2>[->,ultra thick] (1,-0.4) -- (1,0);
%%     \node<1,2> at (1,-0.8) {\LARGE $Promise$};

%%     \draw<3-11>[->,ultra thick] (2,-0.4) -- (2,0);
%%     \node<3-11> at (2,-0.8) {\LARGE $Promise$};

%%     \draw<12>[->,ultra thick] (3,-0.4) -- (3,0);
%%     \node<12> at (3,-0.8) {\LARGE $Promise$};

%%     \draw<13>[->,ultra thick] (4,-0.4) -- (4,0);
%%     \node<13> at (4,-0.8) {\LARGE $Promise$};

%%     \draw<14>[->,ultra thick] (5,-0.4) -- (5,0);
%%     \node<14> at (5,-0.8) {\LARGE $Promise$};

%%     \draw<15>[->,ultra thick] (6,-0.4) -- (6,0);
%%     \node<15> at (6,-0.8) {\LARGE $Promise$};
%%   \end{tikzpicture}
%%   }

%%   Инвариант: ``обещающая'' машина
%%   \only<2,11-14>{\textcolor{darkGreen}{исполняется}}\only<1,3-10,15->{\textcolor{darkGreen}{ждёт}}
  
%%   \vspace{0.5cm}
%%   \hrule
%%   \vspace{0.5cm}
  
%%   %% \begin{columns}

%%   %% \colu{.4\textwidth}{
%%     \begin{tikzpicture}
%%       \fill[green!40!white, draw=black] (0,0) rectangle (1,1);
%%       \node[right] at (1.2,0.5) {-- выполнено $ARMv8$};

%%       \fill[yellow!40!white, draw=black] (0,1.5) rectangle (1,2.5);
%%       \node[right] at (1.2,2.0) {-- частично выполнено $ARMv8$};
%%     \end{tikzpicture}
%%   %% }

%%   %% \colu{.4\textwidth}{
%%     $Promise$ -- ``обещающая'' модель\\
%%     %% $ARM$ -- целевая модель\\
%%   %% }

%%   %% \end{columns}
%% }

%% %% \fr{\huge ``Запаздывающая'' симуляция (1/3)}{
%% %%   \LARGE
%% %%   Базовое отношение на состояниях $\simrelBase \subset \ARMt_{state} \times \Promise_{state}:$ \\
%% %%   \light{$\simrelBase \triangleq$ $\invCf$ $\cap \invPrefix$ $\cap \invMem$ $\cap \invView$ $\cap \invState$.}
%% %% }

%% %% \fr{\huge ``Запаздывающая'' симуляция (2/3)}{
%% %%   \LARGE
%% %%   $\invPromUptoARM(\aT,\p) \triangleq$ \\
%% %%   ${} \quad \forall \tId, \invPromUptoARMtId(\tId,\aT,\p)$; \\
%% %%   $\simrel \triangleq \simrelBase \cap \invPromUptoARM$ \\
%% %%   \vfill
%% %%   \pause
%% %%   $\invPromUptoARMnot(\aT,\p) \triangleq$ \\
%% %%   ${} \quad \exists! \tId, \lnot \invPromUptoARMtId(\tId,\aT,\p)$; \\
%% %%   $\simrelPre \triangleq \simrelBase \cap \invPromUptoARMnot$
%% %% }

%% \fr{\huge ``Запаздывающая'' симуляция. Формально}{
%% \tiny

%% {\bf Лемма 1:}  \\
%% $\forall (\aT, \p) \in \simrelPre, \exists \p', \p \promStep \p', (\aT, \p') \in \simrelPre \cup \simrel$.

%% \vfill
%% {\bf Лемма 2:}  \\
%% $\forall (\aT, \p) \in \simrelPre, \exists n, \{\p_i\}_{i \in [0, n]}, \p_0 = \p, (\forall i < n, \p_i \promStep \p_{i+1})$,\\
%% ${} \quad (\forall i < n, (\aT, \p_i) \in \simrelPre), (\aT, \p_n) \in \simrel$.\\

%% \vfill
%% {\bf Лемма 3:}  \\
%% $\forall (\aT, \p) \in \simrel$,\\
%% ${} \quad (\forall \aT', \aT \armStepgen{\lnot \; \transenv{Write commit}} \aT' \Rightarrow (\aT', \p) \in \simrelPre \cup \simrel) \land$\\
%% ${} \quad (\forall\aT', \aT \armStepgen{\transenv{Write commit}} \aT' \Rightarrow \exists \p', \p \promStepgen{\transenv{Promise write}} \p',  (\aT', \p') \in \simrelPre \cup \simrel)$.

%% \vfill
%% {\bf Лемма 4:}  \\
%% $\forall (\aT, \p) \in \simrel, \aT', \aT \armStep \aT' \Rightarrow$ \\
%% ${} \quad {} \exists \p', \p \promStep^{*} \p', (\aT', \p') \in \simrel$.

%% \vfill
%% {\bf Теорема:} \\
%% $\begin{array}{l}
%%   \forall \Cfprom, %\Cfarm = compilation(\Cfprom), \\
%%   \{\aT_i\}_{i \in [0..n]}, \\
%%   \quad \ainit(compile(\Cfarm)) = \aT_0 \armStep \dots \armStep \aT_n, \finalStateA(\aT_n),\\
%%   \exists \{\p_i\}_{i \in [0..k]}, \\
%%   \quad \pinit(\Cfprom) = \p_0 \promStep \dots \promStep \p_k, \finalStateProm(\p_k), \\
%%   \sameMemory(\aT_n, \p_k). \\
%% \end{array}$
%% }

%% %% \begin{frame}[plain]
%% %%  \cntrd{
%% %%    \shadowimage[width=.9\linewidth]{paperScreenshot.pdf}
%% %%  }
%% %% \end{frame}


%% \fr{\huge Планы на будущее}{
%% \ite{
%%   \LARGE
%%   \item Поддержка остальных конструкций ``обещающей'' семантики \\
%%         {\normalsize \light{(Read-Modify-Writes, Release/Acquire accesses, SC fences)}}
%%   \vfill
%%   \item Механизация доказательства в Coq
%% }

%% \vfill
%% \pause
%% \cntrd{\huge \it \textcolor{red}{Спасибо!}}
%% }


%% \phrase{The first step is a {\bf formal semantics}}
%% \phraseL{
%%   {\Huge PL semantics should}
%%   \vfill
%%   \ite{
%%     \pitem allow \alert{efficient implementation}; \\
%%     \pause (x86, Power, ARM)
%%     \pitem validate \alert{compiler optimizations}; \\
%%     \pause (merging, rearranging, etc)
%%     \pitem allow \alert{high-level reasoning}. \\
%%     %% \pitem avoid \alert{``undefined behavior''}.
%%   }
%% }

%% \fr{}{
%% \begin{center}
%% \begin{tabular}{l | l | l | l}
%%              & Eff. Impl. & Comp. Opt. & H.-l. Reasoning \\ % & No UB \\
%%              \hline
%% Lamport's SC & \fail      & \fail  & \tick \\ % & \tick \\
%% C/C++11 MM   & \tick      & \tick  & \fail \\ % ~ (OOTA) \\ % & \fail \\
%% Java    MM   & \tick      & \fail  & \tick \pause \\ % & \tick \pause \\
%% \multicolumn{4}{l}{} \\
%% \multicolumn{4}{l}{{\bf A promising semantics}} \\
%% \cite{Kang-al:POPL17} & \tick & \tick & \tick \\ % \tick \\
%% \end{tabular}
%% \end{center}
%% }

%% \fr{}{

%% {\Large The Promise machine \cite{Kang-al:POPL17} is {\bf proved} to be {\bf compilable} to
%% x86 and Power.} \\

%% \pause
%% {\bf Sketch of the proof}:
%% \ite{
%%   \pitem x86 = SC + transformations, \\
%%          Power = ``StrongPower'' + transformations \cite{Lahav-Vafeiadis:FM16};
%%   \pitem the transformations are proved to be {\bf sound} in the Promise machine;
%%   \pitem For every program, SC behaviors $\subset$ ``StrongPower'' behaviors $\subset$
%%          behaviors of an axiomatic promise-free version of the Promise machine.
%% }

%% \pause

%% \only<6->{
%% \dimslide
%% \begin{textblock*}{\textwidth}(0.1\textwidth,0.4\textheight)
%% \begin{alertblock}{}
%%   \LARGE
%%   The proof scheme isn't applicable to {\bf ARM}
%%   \onslide<7->{ \\ and here is why:}
%% \end{alertblock}
%% \end{textblock*}
%% }

%% %% \pause but has {\bf not} been proved for {\bf ARM}\pause,
%% %% and the {\bf same proof} techique is {\bf not applicable}.
%% }

%% \fr{}{
%% \LARGE
%% $$\begin{array}{c}
%% [x] := 0; [y] := 0 \\
%% \begin{array}{l||l||l}
%% \inarr{
%%   a := [x]; \uncover<2->{\comment{1}} \tikzmark{comment} \\
%%   {} [x] := 1
%% } &
%% \inarr{
%%   b := [x]; \\
%%   {} [y] := b
%% } &
%% \inarr{
%%   c := [y]; \\
%%   {} [x] := c \\
%% }
%% \end{array}\end{array}$$

%% \mycallout<2>{green}{(comment)}{(-0.5cm, -1.0cm)}{Allowed by ARMv8 \cite{Flur-al:POPL16}}

%% \onslide<3>{
%% \begin{alertblock}{}
%% The behavior cannot be explained by transformations over a strong enough model.
%% \end{alertblock}
%% }
%% }

%% \fr{\LARGE Compilation scheme}{
%%   \huge
%%   \[\begin{array}{l | l}
%%     \Promise & \ARM \\
%%     \hline
%%     [x]_{\sf rlx} := a & [x] := a \\
%%     a := [x]_{\sf rlx} & a := [x] \\
%%     \hline
%%     \acqFence & \dmb \; \LD \\
%%     \relFence & \dmb \; \SY \\
%%   \end{array}\]
%% %% \pause

%% \only<2->{
%% \dimslide
%% \begin{textblock*}{\textwidth}(0.1\textwidth,0.4\textheight)
%% \begin{alertblock}{}
%%   \LARGE
%%   As the compilation scheme is bijection, we use {\bf one} syntax in
%%   examples.
%% \end{alertblock}
%% \end{textblock*}
%% }
%% }

%% \fr{\LARGE Example 1}{
%% \LARGE
%% $$\begin{array}{c}
%% \markExecuted{4}{5}{[x] := 0; [y] := 0} \\
%% \begin{array}{l||l}
%% \inarr{
%%   {} \markExecuted{5}{6}{[x] := 1}; \\
%%   {} \markExecuted{6}{7}{[y] := 1}
%% } &
%% \inarr{
%%   \markExecuted{7}{8}{a := [y]};   \uncover<2-3,15->{\comment{1}} \\
%%   \markExecuted{8}{9}{b := [x]} \; \uncover<3,16->  {\comment{0}}
%% }
%% \end{array}\end{array}$$

%%   \uncover<4->{
%%     \Large
%%     \cntrd{
%%     \begin{tikzpicture}[scale = 0.8, transform shape]
%%       \storageTwoThreads{1}{2}
%%       \node<6-9>[right] at (1, -0.1) {$[x] := 1$};
%%       \node<7-9>[right] at (1,  0.7) {$[y] := 1$};
%%       \node<8-12>[right] at (3.5, -0.1) {$a := [y]$};
%%       \node<9-13>[right] at (3.5,  0.7) {$b := [x]$};

%%       \node<10-16>[right] at (1,  0.7) {$[x] := 1$};
%%       \node<10>   [right] at (1, -0.1) {$[y] := 1$};

%%       \node<11>   [right] at (2.25, -1.4) {$[y] := 1$};
%%       \node<13-14>[right] at (2.25, -1.4) {$a := [y]$};
%%       \node<14-15>[right] at (2.25, -0.8) {$b := [x]$};

%%       \node<17>   [right] at (2.25, -1.4) {$[x] := 1$};

%%       \node<5-11>  at (2.25, -2.7) {$[x] := 0; [y] := 0$};
%%       \node<12-17> at (2.25, -2.7) {$[x] := 0; [y] := 1$};
%%       \node<18->   at (2.25, -2.7) {$[x] := 1; [y] := 1$};
%%     \end{tikzpicture}
%%     }
%%   }
%% }

%% \fr{\LARGE Example 2}{
%% \LARGE
%% $$\begin{array}{c}
%% \markExecuted{4}{5}{[x] := 0; [y] := 0} \\
%% \begin{array}{l||l}
%% \inarr{
%%   \markExecuted{5}{6}{[x] := 1}; \\
%%   \markExecuted{6}{7}{\tikzmark{dmbsy} \dmb ~\SY}; \\
%%   \markExecuted{7}{8}{[y] := 1}
%% } &
%% \inarr{
%%   \markExecuted{8}{9}{a := [y]}; \uncover<2-3>{\comment{1}} \\
%%   \dmb \tikzmark{dmbld} ~\LD; \\
%%   b := [x] \; \uncover<3>  {\nocomment{0}}
%% }
%% \end{array}\end{array}$$

%%   \uncover<4->{
%%     \Large
%%     \cntrd{
%%     \begin{tikzpicture}[scale = 0.8, transform shape]
%%       \storageTwoThreads{1.5}{2}
%%       \node<6->[right] at (1, -0.1) {$[x] := 1$};
%%       \node<7->[right] at (1,  0.5) {$\dmb ~\SY$};
%%       \node<8->[right] at (1,  1.1) {$[y] := 1$};
%%       %% \node<7-9>[right] at (1,  0.7) {$[y] := 1$};
%%       \node<9->[right] at (3.5, -0.1) {$a := [y]$};
%%       %% \node<9-13>[right] at (3.5,  0.7) {$b := [x]$};

%%       %% \node<10-16>[right] at (1,  0.7) {$[x] := 1$};
%%       %% \node<10>   [right] at (1, -0.1) {$[y] := 1$};

%%       %% \node<11>   [right] at (2.25, -1.4) {$[y] := 1$};
%%       %% \node<13-14>[right] at (2.25, -1.4) {$a := [y]$};
%%       %% \node<14-15>[right] at (2.25, -0.8) {$b := [x]$};

%%       %% \node<17>   [right] at (2.25, -1.4) {$[x] := 1$};

%%       \node<5->  at (2.25, -2.7) {$[x] := 0; [y] := 0$};
%%     \end{tikzpicture}
%%     }
%%   }

%%   \mycallout<10>{red}{(dmbsy)}{(-0.5cm, -1.0cm)}{Prevents reordering}
%%   \mycallout<11>{red}{(dmbld)}{(-0.5cm, -1.0cm)}{Forbids issuing until \\ $a := [y]$ is satisfied}
%% }

%% \fr{Example 3} {
%% % TODO: decrease slide numbers > 2 by 1.
%% \LARGE
%% $$\begin{array}{c}
%% \hspace{12pt} \markExecuted{4}{5}{[x] := 0; [y] := 0} \\
%% \begin{array}{l||l||l}
%% \inarr{
%%   \markExecuted{5}{6}{a := [x]}; \uncover<2,26->{\comment{1}} \tikzmark{comment} \\
%%   \markExecuted{6}{7}{[x] := 1}
%% } &
%% \inarr{
%%   \markExecuted{9}{10}{b := [x]}; \uncover<12->{\comment{1}} \\
%%   \markExecuted{12}{13}{[y] := b}
%% } &
%% \inarr{
%%   \markExecuted{18}{19}{c := [y]};  \uncover<21->{\comment{1}} \\
%%   \markExecuted{21}{22}{[x] := c}} \\
%% \end{array}\end{array}$$

%% \uncover<4->{
%%   \Large
%%   \begin{center}
%%   \begin{tikzpicture}[scale = 0.8, transform shape]
%%     \storageThreeThreads{1.2}{1.8}{1.2}
%%     \node<7-8>[right] at (1.0,  0.15) {$[x] := 1$};
%%     \node<6-7>[right] at (1.0, -0.4 ) {$a := [x]$};

%%     \node<10>[right] at (5.0, -0.125) {$b := [x]$};
%%     \node<13>[right] at (5.0, -0.125) {$[y] := \progval{1}$};

%%     \node<11>[right] at (3.0, -1.0 ) {$b := [x]$};
%%     \node<9-14>[right] at (3.0, -1.55) {$[x] := 1$};
%%     \node<8-15>[right] at (3.0, -2.1 ) {$a := [x]$};

%%     \node<14> [right] at (3.0, -1.0 ) {$[y] := \progval{1}$};
%%     \node<15> [right] at (3.0, -1.55) {$[y] := \progval{1}$};
%%     \node<15-26>[right] at (3.0, -1.0 ) {$[x] := 1$};
%%     \node<16> [right] at (3.0, -2.1 ) {$[y] := \progval{1}$};
%%     \node<16-24>[right] at (3.0, -1.55) {$a := [x]$};

%%     \node<17> [right] at (5.75, -3.0) {$[y] := \progval{1}$};
    
%%     \node<19>[right] at (8.5, -0.125) {$c := [y]$};
%%     \node<20>[right] at (5.75, -3.0 ) {$c := [y]$};

%%     \node<22>[right] at (8.5, -0.125) {$[x] := \progval{1}$};
%%     \node<23>[right] at (5.75, -3.0 ) {$[x] := \progval{1}$};

%%     \node<25>[right] at (5.75, -3.0) {$a := [x]$};
%%     \node<27>[right] at (5.75, -3.0) {$[x] := 1$};

%%     \node<5-17> at (5.75, -0.2 -1.2 - 1.8 - 1.2) {$[x] := 0; [y] := 0$};
%%     \node<18-23> at (5.75, -0.2 -1.2 - 1.8 - 1.2) {$[x] := 0; [y] := \progval{1}$};
%%     \node<24-27> at (5.75, -0.2 -1.2 - 1.8 - 1.2) {$[x] := \progval{1}; [y] := \progval{1}$};
%%     \node<28-> at (5.75, -0.2 -1.2 - 1.8 - 1.2) {$[x] := 1; [y] := \progval{1}$};
%%   \end{tikzpicture}
%%   \end{center}
%% }
%% }

%% \fr{Simulation}{
%%   \LARGE
%%   \iteN{
%%     \item  Introduce $SimInvariant : T_{State} \times S_{State}$;
%%     \pitem Show that \\
%%     $\forall t, t' \in T_{State}, s \in S_{State}.$ \\
%%     \pause
%%     $\quad correct(t), t \xrightarrow[T]{} t', SimInvariant(t, s)$, \\
%%     \pause
%%     $\exists s' \in S_{State}. s \xrightarrow[S]{}^{*} s', SimInvariant(t', s')$.
%%   }
%% }

%% \fri{\LARGE The ARM+$\tau$ machine}{ 
%%   \LARGE
%%   \item Add $\tau$-map component to the ARM state;
%%   \item Modify {\bf Write Commit} rule;
%%   \item Modify {\bf Propagate} rule.
%% }

%% \fr{\LARGE ARM+$\tau$ simulates ARM}{ 
%% \Large
%% $\forall \Cfarm, \{\s_i\}_{i \in [0..n]}$, \\ \pause
%% $\quad \sinit(\Cfarm) = \s_0 \armStepP \dots \armStepP \s_n, \finalStateP(\s_n)$,\\ \pause
%% $\exists \{\aT_i | \s_i = \armState{\aT_i} \}_{i \in [0..n]}$, \\
%% ${} \quad {} \aT_0 \armStep \dots \armStep \aT_n$.

%% \vfill

%% \pause
%% {\bf Sketch of the proof:}
%% \ite{
%%   \pitem Construct an order on writes from $\s_n$;
%%   \pitem Show $\s_i$ doesn't contradict the order for all $i$;
%%   \pitem Show the order may coincide with $\tau$s in $\{\aT_i\}_{i \in [0..n]}$.
%% }
%% }

\end{document}
