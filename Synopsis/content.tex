%% \newcommand{\ifext}[2]{\ifdefined\extflag{#1}\else{#2}\fi}
%% \newcommand{\tick}{\checkmark}%
%% \newcommand{\tickP}{\checkmark}%
%% \newcommand{\tickPP}{\checkmark}%
%% \newcommand{\xmark}{\text{\ding{55}}}
%% \newcommand{\fail}{\xmark}%
%% \newcommand\Cf{\mathit{Prog}}%\textsf{Prog}

\section*{Общая характеристика работы}

\newcommand{\actuality}{\underline{\textbf{\actualityTXT}}}
\newcommand{\progress}{\underline{\textbf{\progressTXT}}}
\newcommand{\aim}{\underline{{\textbf\aimTXT}}}
\newcommand{\tasks}{\underline{\textbf{\tasksTXT}}}
\newcommand{\novelty}{\underline{\textbf{\noveltyTXT}}}
\newcommand{\influence}{\underline{\textbf{\influenceTXT}}}
\newcommand{\methods}{\underline{\textbf{\methodsTXT}}}
\newcommand{\defpositions}{\underline{\textbf{\defpositionsTXT}}}
\newcommand{\reliability}{\underline{\textbf{\reliabilityTXT}}}
\newcommand{\probation}{\underline{\textbf{\probationTXT}}}
\newcommand{\contribution}{\underline{\textbf{\contributionTXT}}}
\newcommand{\publications}{\underline{\textbf{\publicationsTXT}}}


{\actuality}
Параллельные вычисления являются неотъемлимой частью современных
информационных систем. В настоящее время абсолютное большинство процессоров, используемых
в настольных компьютерах, серверах и смартфонах, являются многоядерными,
а большинство критичных к производительности программ --- многопоточными.
Многопоточность предоставляет программисту широкие возможности для ускорения вычислений,
потенциально позволяя уменьшить время исполнения, или увеличить производительность программы,
во столько же раз, сколько процессоров (ядер) находится в вычислительной системе.
Однако многопоточные программы сложны для разработки и поддержки --- их поведение
зависит не только от входных данных, но и от планировщика задач в процессоре, который
управляет исполнением потоков программы на процессорных ядрах.
Как следствие, многопоточные программы являются источниками специфических, трудно поддающихся
тестированию и ручной отладке ошибок, таких как \emph{состояние гонки}, \emph{взаимная блокировка} и др.
Эти ошибки могут воспроизводиться очень редко, например, в одном из 10 тысяч запусков
программы, или в 0.01\% случаев, однако даже это может быть критично ---
подобная ``редкая" \, ошибка в ядре операционной системы Linux, которая
используется как минимум в 20 миллионах серверов по всему миру, будет приводить
к 200 тысячам неправильных срабатываний каждый день.
Таким образом, необходимо иметь специализированные методы для анализа многопоточных программ.

Базовым и ключевым инструментом любого анализа является семантика языка,
на котором анализируемая программа написана. Семантики для языков и систем (процессоров)
с многопоточностью называют \emph{моделями памяти}.
\emph{Последовательная консистентность} --- это самая простая и естественная модель;
она подразумевает, что каждое поведение многопоточной программы может быть
получено некоторым поочередным исполнением команд из каждого потока на одном ядре
(процессоре). Однако эта модель не соответствует современным системам ---
в результате обработки оптимизирующими компиляторами и запуска на суперскалярных 
процессорах, многопоточная программа приобретает поведения, которые не могут
быть описаны подобным образом. Реалистичные модели памяти, позволяющие
упомянутые поведения, называются \emph{слабыми} (weak memory models).

На данный момент научное сообщество в тесном сотрудничестве с индустрией
разработало и продолжает совершенствовать множество слабых моделей памяти для
наиболее распространенных процессорных архитектур и языков программирования.
При этом процессорные и языковые модели существенно влияют на друг друга.
Модель процессора должна отражать поведения, наблюдаемые при запуске программ
на существующих процессорах, и давать пространство для развития
обратносовместимых архитектур. В то же время, языковая модель должна не только
предоставлять разумную и удобную абстракцию для программиста, не запрещать основные
компиляторные оптимизации, но и быть совместной с моделям целевых архитектур,
т.е. поддерживать эффективную трансляцию в низкоуровневый код без изменения
семантики программы.

Существующие модели памяти для наиболее популярных языков программирования обладают рядом недостатков.
Так, известно, что модель памяти Java (the Java Memory Model) некорректна по отношению
к базовым оптимизациям, а модель памяти C/С++ (the C/C++11 Memory Model) разрешает
поведения программ, в которых появляются т.н. \emph{значения из воздуха} (Out-Of-Thin-Air values),
что существенно усложняет семантику и рассуждения о программах.
Кроме того, модель памяти C/C++ определяет лишь многопоточную составляющую семантики, что, вместе со стилем
описания модели (аксиоматическим), оставляет открытым вопрос её интеграции с остальной частью семантики,
описанной в стандарте операционно.

Таким образом, для развития инструментов анализа многопоточных программ необходимо
разработать операционные подходы к заданию слабых моделей памяти.

\textbf{Comment.} Многоописания контекста. Почему x86, Power и ARM важны (с цифрами)?
Следить, если Promise станет стандартом для C++ и/или Java.

%% Обзор, введение в тему, обозначение места данной работы в
%% мировых исследованиях и~т.\:п., можно использовать ссылки на другие
%% работы~\cite{Gosele1999161} (если их нет, то в автореферате
%% автоматически пропадёт раздел <<Список литературы>>). Внимание! Ссылки
%% на другие работы в разделе общей характеристики работы можно
%% использовать только при использовании \verb!biblatex! (из-за технических
%% ограничений \verb!bibtex8!. Это связано с тем, что одна и та же
%% характеристика используются и в тексте диссертации, и в
%% автореферате. В последнем, согласно ГОСТ, должен присутствовать список
%% работ автора по теме диссертации, а \verb!bibtex8! не умеет выводить в одном
%% файле два списка литературы).

%% Для генерации содержимого титульного листа автореферата, диссертации и
%% презентации используются данные из файла \verb!common/data.tex!. Если,
%% например, вы меняете название диссертации, то оно автоматически
%% появится в итоговых файлах после очередного запуска \LaTeX. Согласно
%% ГОСТ 7.0.11-2011 <<5.1.1 Титульный лист является первой страницей
%% диссертации, служит источником информации, необходимой для обработки и
%% поиска документа.>> Наличие логотипа организации на титульном листе
%% упрощает обработку и поиск, для этого разметите логотип вашей
%% организации в папке images в формате PDF (лучше найти его в векторном
%% варианте, чтобы он хорошо смотрелся при печати) под именем
%% \verb!logo.pdf!. Настроить размер изображения с логотипом можно в
%% соответствующих местах файлов \verb!title.tex!  отдельно для
%% диссертации и автореферата. Если вам логотип не нужен, то просто
%% удалите файл с логотипом.

{\progress}
Научное сообщество в сотрудничестве с индустрией разработало модели
памяти как для языков программирования
\cite{Batty-al:POPL11, Manson-al:POPL05},
так и для основных процессорных архитектур: 
x86 \cite{Sewell-al:CACM10},
Power \cite{Sarkar-al:PLDI11,Alglave-al:TOPLAS14},
ARM \cite{Flur-al:POPL16}.
Кроме того, есть работы посвященные моделям памяти графических
процессоров \todo{(TODO: цитаты)}

%% \cite{Kang-al:POPL17}

%% Этот раздел должен быть отдельным структурным элементом по
%% ГОСТ, но он, как правило, включается в описание актуальности
%% темы. Нужен он отдельным структурынм элемементом или нет ---
%% смотрите другие диссертации вашего совета, скорее всего не нужен.

{\aim} данной работы является исследование применимости операционных
подходов для определения реалистичных моделей памяти и анализа
многопоточных программ на примере языков C/C++.

Для~достижения поставленной цели необходимо были поставлены следующие {\tasks}:
\begin{enumerate}
  \item Провести обзор существующих операционных моделей слабой памяти.
  \item Разработать операционный аналог модели памяти С/С++11.
  \item Формализовать модель памяти ARMv8 и доказать дополнительные свойства о ней, необходимые для
        доказательства корректности компиляции. 
  \item Доказать корректность компиляции из ``обещающей" \; семантики в модель памяти ARMv8.
\end{enumerate}

{\novelty}
\begin{enumerate}
  \item Впервые \ldots
  \item Впервые \ldots
  \item Было выполнено оригинальное исследование \ldots
\end{enumerate}

{\influence} \ldots

{\methods} \ldots

{\defpositions}
\begin{enumerate}
  %% \item Проведён обзор существующих операционных моделей слабой памяти.
  %%       \todo{(пока не сделано)}
  \item Предложена операционная модель памяти для подмножества стандарта C/С++11.
  \item Реализован компонентный метод описания интерпретаторов для модели C/C++11 в системе PLT/Redex.
  %% \item Разработан реляционный интерпретатор для модели C/C++11, позволяющий исправлять ошибки синхронизации
  %%       в программах.
  %%       \todo{(пока не сделано)}
  %% \item Реализован метод автоматического поиска сертификата в ``обещающей" \; модели памяти.
  %%       \todo{(пока не сделано)}
  \item Проведена формализация операционной модели памяти ARMv8 POP, доказаны вспомогательные утверждения про
        модель, полезные для проверки корректности компиляции.
  \item Доказана корректность компиляции для подмножества ``обещающей" \; семантики в модель памяти ARMv8 POP.
  \item Доказана корректность компиляции для ``обещающей" \; семантики в аксиоматическую модель памяти ARMv8.2.
        \todo{(пока не сделано)}
  %% \item Показано, что операционные модели x86-TSO и Power есть ограниченные версии модели ARMv8.
\end{enumerate}

\textcolor{blue}{План публикаций ВАК:}
\begin{enumerate}
  \item \emph{Operational Aspects of C/C++ Concurrency.}
  
    \textcolor{green}{Труды ИСП РАН.}

  \item \emph{Сертифицируемость шагов ``обещающей" \; семантики в контексте доказательства корректности компиляции
    в модель памяти ARMv8 POP.}

    В этой статье я собираюсь описать часть доказательства, которая не описана в моей статье на ECOOP'17.
    \textcolor{red}{Подготовить черновой вариант за июль.}
    \textcolor{green}{Вестник Политеха.}
    
    \textbf{Гранты}. Посмотреть, что есть с грантами СПбГУ.

  \item \emph{Корректность компиляции для подмножества ``обещающей" \; семантики в аксиоматическую модель памяти ARMv8.2.}
  
    Эту статью мы, видимо, не будем пытаться опубликовать к защите, но в текст диссертации поместим как результат.

    Это материал, который ещё нужно сделать, и то, что я планирую в расширенном варианте потом опубликовать
    на нормальной конференции типа ECOOP.
    \textcolor{green}{Вестник Политеха.}
    
  \item Кирпич нужно подготовить к концу марта.
  \item Новый вариант названия ``Слабые модели памяти для языка C/C++".
\end{enumerate}

{\reliability} полученных результатов обеспечивается \ldots \ Результаты находятся в соответствии с результатами, полученными другими авторами.

{\probation}
Основные результаты работы докладывались~на:
перечисление основных конференций, симпозиумов и~т.\:п.

{\contribution} Автор принимал активное участие \ldots

%\publications\ Основные результаты по теме диссертации изложены в ХХ печатных изданиях~\cite{Sokolov,Gaidaenko,Lermontov,Management},
%Х из которых изданы в журналах, рекомендованных ВАК~\cite{Sokolov,Gaidaenko}, 
%ХХ --- в тезисах докладов~\cite{Lermontov,Management}.

\ifnumequal{\value{bibliosel}}{0}{% Встроенная реализация с загрузкой файла через движок bibtex8
    \publications\ Основные результаты по теме диссертации изложены в XX печатных изданиях, 
    X из которых изданы в журналах, рекомендованных ВАК, 
    X "--- в тезисах докладов.%
}{% Реализация пакетом biblatex через движок biber
%Сделана отдельная секция, чтобы не отображались в списке цитированных материалов
    \begin{refsection}[vak,papers,conf]% Подсчет и нумерация авторских работ. Засчитываются только те, которые были прописаны внутри \nocite{}.
        %Чтобы сменить порядок разделов в сгрупированном списке литературы необходимо перетасовать следующие три строчки, а также команды в разделе \newcommand*{\insertbiblioauthorgrouped} в файле biblio/biblatex.tex
        \printbibliography[heading=countauthorvak, env=countauthorvak, keyword=biblioauthorvak, section=1]%
        \printbibliography[heading=countauthorconf, env=countauthorconf, keyword=biblioauthorconf, section=1]%
        \printbibliography[heading=countauthornotvak, env=countauthornotvak, keyword=biblioauthornotvak, section=1]%
        \printbibliography[heading=countauthor, env=countauthor, keyword=biblioauthor, section=1]%
        \nocite{%Порядок перечисления в этом блоке определяет порядок вывода в списке публикаций автора
                vakbib1,vakbib2,%
                confbib1,confbib2,%
                bib1,bib2,%
        }%
        \publications\ Основные результаты по теме диссертации изложены в \arabic{citeauthor} печатных изданиях, 
        \arabic{citeauthorvak} из которых изданы в журналах, рекомендованных ВАК, 
        \arabic{citeauthorconf} "--- в тезисах докладов.
    \end{refsection}
    \begin{refsection}[vak,papers,conf]%Блок, позволяющий отобрать из всех работ автора наиболее значимые, и только их вывести в автореферате, но считать в блоке выше общее число работ
        \printbibliography[heading=countauthorvak, env=countauthorvak, keyword=biblioauthorvak, section=2]%
        \printbibliography[heading=countauthornotvak, env=countauthornotvak, keyword=biblioauthornotvak, section=2]%
        \printbibliography[heading=countauthorconf, env=countauthorconf, keyword=biblioauthorconf, section=2]%
        \printbibliography[heading=countauthor, env=countauthor, keyword=biblioauthor, section=2]%
        \nocite{vakbib2}%vak
        \nocite{bib1}%notvak
        \nocite{confbib1}%conf
    \end{refsection}
}
При использовании пакета \verb!biblatex! для автоматического подсчёта
количества публикаций автора по теме диссертации, необходимо
их здесь перечислить с использованием команды \verb!\nocite!.
    

 % Характеристика работы по структуре во введении и в автореферате не отличается (ГОСТ Р 7.0.11, пункты 5.3.1 и 9.2.1), потому её загружаем из одного и того же внешнего файла, предварительно задав форму выделения некоторым параметрам

%Диссертационная работа была выполнена при поддержке грантов ...

%\underline{\textbf{Объем и структура работы.}} Диссертация состоит из~введения, четырех глав, заключения и~приложения. Полный объем диссертации \textbf{ХХХ}~страниц текста с~\textbf{ХХ}~рисунками и~5~таблицами. Список литературы содержит \textbf{ХХX}~наименование.

%\newpage
\section*{Содержание работы}
Во \underline{\textbf{введении}} обосновывается актуальность
исследований, выполненных в рамках данной диссертационной работы,
приводится краткий обзор научной литературы по изучаемой проблеме,
формулируется цель, ставятся задачи работы, излагается научная новизна
и практическая значимость представленного исследования.

\underline{\textbf{Первая глава}} посвящена обзору области исследования.
Рассматриваются требования к реалистичным моделями памяти
языков программирования, предъявляемые через призму наблюдаемых сценариев поведения многопоточных программ,
применяемых компиляторами оптимизаций, а также моделей памяти процессорных архитектур.
Описывается модель памяти C/C++11. Рассматриваются проблемы модели C/C++11, в том числе
проблема ,,значений из воздуха``. Приводится описание существующих решений к заданию
слабой модели памяти без ,,значений из воздуха``, в частности, ,,обещающей`` модели памяти.
На основе ранее выполненного обзора делаются следующие выводы.
\begin{itemize}
  \item Модель памяти промышленного языка программирования должна удовлетворять, как минимум, трём критериям.
    Во-первых, должна существовать корректная схема компиляции в модель целевой процессорной
    архитектуры.
    Во-вторых, основные компиляторные оптимизации должны быть корректны в рамках модели.
    В-третьих, у модели должна отсутствовать проблема ,,значений из воздуха``.
  \item При разработке новой модели памяти языка программирования нужно доказывать корректность эффективной компиляции
     в модели памяти целевых процессорных архитектур.
  \item Существующие модели памяти промышленных языков программирования не удовлетворяют всем приведённым выше
    критериям.
  \item Требуется разработать операционную модель памяти с синтаксисом модели C/C++11, которая
    не имеет проблемы ,,значений из воздуха``.
  %% \item ,,Обещающая`` модель памяти является перспективной альтернативой существующей модели памяти C/C++.
\end{itemize}

\underline{\textbf{Вторая глава}} посвящена предложенной в диссертации операционной  модели памяти для C/С++. Модель представлена в виде операционной семантики малого шага с помощью редукционных
контекстов. Далее для обозначения этой модели мы будем использовать название \OpCpp.

Основное отличие слабых моделей памяти от модели последовательной консистентности заключается в том, что
первые не гарантируют для локации в памяти единственность значения, которое может быть прочитано из локации
в каждый конкретный момент времени.
Так, например, следующая программа может завершиться с $a = 1, b = 0$, хотя, казалось бы, $a = 1$ гарантирует, что
в локацию $d$ уже записано новое значение $239$:
\[
\begin{array}{c}
[f] := 0; [d] := 0; \\
\begin{array}{l||l}
  {} [d] := 239; & a := [f]; \\
  {} [f] := 1;   & b := [d]; \\
\end{array}
\end{array}
\]
Как следствие, оперативная память (далее просто ,,память``) в рамках слабых моделей памяти не может быть представлена как функция из локации в значения.

Память в модели $\OpCpp$ представляется как множество \emph{сообщений}. Каждое сообщение содержит целевую локацию, записываемое значение и
\emph{метку времени} --- натуральное число, которое определяет полный порядок на сообщениях, относящихся к одной локации.
Последнее нужно для того, чтобы гарантировать последовательную консистентность для программ, оперирующих только над одной локацией --- эту гарантию
предоставляют большинство слабых моделей памяти, в том числе модель C/C++11.
При выполнении инструкции чтения из некоторой локации поток может недетерминировано выбрать сообщение, относящееся к локации, и произвести
чтение из него.

%% Представление памяти как множества сообщений позволяет симулировать исполнение приведенной выше программы, которое заканчивается результатом
%% $a = 1, b = 0$, следующим образом. После выполнения Сначала левый поток выполняет 

Недетерминированность чтения из локации ограничена гарантией, которую также предоставляет модель C/C++11:
после того, как поток прочитал или записал сообщение в локацию $x$ с меткой времени $t$, он (поток) больше не может читать из сообщений
с меткой времени, которая меньше $t$. Для реализации данного ограничения в модели $\OpCpp$ у каждого потока есть т.н. \emph{базовый фронт}
(current view, current viewfront) --- функция из локаций в метки времени, определяющая осведомленность потока о сообщениях в памяти.

Некоторые программы имеют такие слабые исполнения, разрешенные моделью C/C++11, которые не могут быть промоделированы только недетерминированной
памятью, а дополнительно требуют исполнения инструкций не по порядку. Например, следующая программа может завершиться с
результатом $a = 1, b = 1$ в модели C/C++11:
\[
\begin{array}{c}
[x] := 0; [y] := 0; \\
\begin{array}{l||l}
  {} a := [x]; & b := [y]; \\
  {} [y] := 1; & [x] := 1; \\
\end{array}
\end{array}
\]
Для представления таких исполнений в модели $\OpCpp$ у каждого потока есть \emph{буфер отложенных операций}. Так, модель
позволяет потоку в каждый момент отложить текущую операцию вместо её выполнения.
С помощью этого механизма модель $\OpCpp$ может исполнить приведенную выше программу и получить результат $a = 1, b = 1$ следующим образом.
Сначала левый поток откладывает чтение из локации $x$ и выполняет запись в $y$. После этого правый поток читает из вновь добавленного
сообщения и записывает $1$ в $x$. Теперь левый поток исполняет отложенное чтение и читает из сообщения, добавленного правым потоком, и получает
$a = 1, b = 1$.

Для поддержки высвобождающих (release) барьеров и записей, а также приобретающих (acquire) барьеров и чтений
модель $\OpCpp$ использует дополнительные фронты --- высвобождающий и приобретающий для каждого потока,
а также фронт сообщений для каждого сообщения в памяти. Для поддержки чтений с модификатором доступа
consume модель использует динамическую пометку зависимых от consume-чтения инструкций.

%% Рассмотрим следующую программу, в которой левый поток передает сообщение через локацию $d$ в правый поток:
%% \[
%% \begin{array}{c}
%% [f] := 0; [d] := 0; \\
%% \begin{array}{l||l}
%%   {} [d] := 239; & a := [f]; \\
%%   {} [f] := 1;   & b := [d]; \\
%% \end{array}
%% \end{array}
%% \]
%% Здесь локация $f$ используется как индекс того, что левый поток уже записал нужные данные в локацию $d$.
%% В рамках модели последовательной консистентности гарантируется, что если $a = 1$, т.е. правый поток
%% ,,увидел`` запись в $f$, то дальше он прочитает из записи в $d$, сделанной левым потоком, и в результате $b$ будет
%% равняться $239$. Это гарантируется тем, что 

Для предложенной модели был реализован интерпретатор на языке Racket с помощью библиотеки описания редукционных
семантик PLT/Redex. Код проекта доступен по адресу \url{github.com/anlun/OperationalSemanticsC11}.

Апробация предложенной семантики была выполнена на наборе из более чем 40 программ-тестов (litmus tests), взятых из  тематической литературы, а также 
на алгоритме RCU (Read-Copy-Update). Поведение модели $\OpCpp$ совпадает с поведением модели C/C++11 на большинстве этих тестов.
Отличие наблюдается на двух следующих категориях тестов. Первая категория --- это программы, которые имеют исполнения со
,,значениями из воздуха`` в рамках модели C/C++11. Для таких тестов модель $\OpCpp$ не выдает исполнения со ,,значениями из воздуха``,
что является её положительным свойством. Вторая категория --- это программы, в которых существуют антизависимости по управлению,
адресу или значению, ведущие к инструкциям записи. На таких программах $\OpCpp$ не способна получить все возможные в рамках C/C++11
исполнения, поскольку выполнение инструкций не по порядку в $\OpCpp$ реализовано синтаксическим способом. Этот недостаток модели не позволяет ей поддержать все необходимые компиляторные оптимизации.

Одновременно с моделью памяти $\OpCpp$ исследователями Kang, Hur, Lahav, Vafeiadis и Dreyer была представлена ,,обещающая`` модель
памяти, которая очень близка $\OpCpp$, но использует другой механизм для выполнения инструкций не по порядку, который позволяет
поддержать большее число компиляторных оптимизаций, чем модель $\OpCpp$.
Из-за данного преимущества диссертант принял решение продолжить свою исследовательскую работу в рамках ,,обещающей`` модели.

%% {
%% %% \setlength{\belowcaptionskip}{-20pt} 
%% \begin{figure}[h!]
%% \centering
%% {\scriptsize
%% %% \begin{tabular}{| l ||@{~}c@{~}|@{~}c@{~}|@{~}c@{~}|@{~}c@{~}|@{~}c@{~}|@{~}c@{~}|@{~}c@{~}|@{~}c@{~}|@{~}c@{~}||@{~}c@{~}|}
%% \begin{tabular}{| l ||@{~}c@{~}|@{~}c@{~}|@{~}c@{~}|@{~}c@{~}|@{~}c@{~}|@{~}c@{~}|@{~}c@{~}|@{~}c@{~}||@{~}c@{~}|}
%%   \hline
%%   \textbf{Название теста} & \textsf{VF} & \textsf{WF} & \textsf{SCF}
%%   & \textsf{NAF} & \textsf{PO} & \textsf{ARR} % & \textsf{VS}
%%   & \textsf{CR} & \textsf{JN} & \textbf{C11} \\
%% %
%% % \textbf{Test name} & History & Viewfronts & Write-fronts & SC-front
%% %           & NA-front & Postponed operations & Acquire Read Restrictions (& Value Stealing) & Consume & Joining threads w/ non-empty operation buffers & \textbf{Full Support} \\

%% \hline\hline
%% \multicolumn{10}{|c|}{Store Buffering (\textsf{SB})\ifext{, \S\ref{app:sb}}{}} \\
%% \hline
%% \textsf{rel+acq}   & \tick & &       & & & & & & \tick\\ 
%% \textsf{sc}        & \tick & & \tick & & & & & & \tick\\ 
%% \textsf{sc+rel}    & \tick & & \tick & & & & & & \tick\\ 
%% \textsf{sc+acq}    & \tick & & \tick & & & & & & \tick\\ 

%% \hline
%% \multicolumn{10}{|c|}{Load Buffering (\textsf{LB})\ifext{, \S\ref{app:lb}}{}} \\
%% \hline
%% \textsf{rlx}         & \tick & & & & \tick & & & & \tick\\ 
%% \textsf{rel+rlx}     & \tick & & & & \tick & & & & \tick\\ 
%% \textsf{acq+rlx}     & \tick & & & & \tick & & & & \fail\\ 
%% \textsf{rel+acq+rlx} & \tick & & & & \tick & \tick & & & \tick\\ 
%% \textsf{rlx+use}     & \tick & & & & \tick & & & & \tick\\ 
%% \textsf{rlx+let}     & \tick & & & & \tick & & & & \tick\\ 
%% \textsf{rlx+join}    & \tick & & & & \tick & & & \tick & \tickP\\ 
%% \textsf{rel+rlx+join} & \tick & & & & \tick & & & \tick & \tickP\\ 
%% \textsf{acq+rlx+join} & \tick & & & & \tick & & & \tick & \fail\\ 

%% \hline
%% \multicolumn{10}{|c|}{Message passing (\textsf{MP})\ifext{, \S\ref{app:mp}}{}} \\
%% \hline
%% \textsf{rlx+na}            & \tick &       & & \tick & & &       & & \tick\\ 
%% \textsf{rel+rlx+na}        & \tick &       & & \tick & & &       & & \tick\\ 
%% \textsf{rlx+acq+na}        & \tick &       & & \tick & & &       & & \tick\\ 
%% \textsf{rel+acq+na}        & \tick &       & & \tick & & \tick & & & \tick\\ 
%% \textsf{rel+acq+na+rlx(\_2)} & \tick & \tick & & \tick & & \tick & & & \tick\\ 
%% \textsf{con+na(\_2)}       & \tick &       & & \tick & & & \tick & & \tick\\ 
%% \textsf{cas+rel+acq+na}    & \tick &       & & \tick & & \tick & & & \tick\\ 
%% \textsf{cas+rel+rlx+na}    & \tick &       & & \tick & & &     & & \tick\\ 

%% \hline
%% \multicolumn{10}{|c|}{Coherence of Read-Read (\textsf{CoRR})\ifext{, \S\ref{app:corr}}{}} \\
%% \hline
%% \textsf{rlx}      & \tick & &       & & &  & & & \tick\\ 
%% \textsf{rel+acq}  & \tick & &       & & &  & & & \tick\\ 

%% \hline
%% \multicolumn{10}{|c|}{Independent Reads of Independent Writes (\textsf{IRIW})\ifext{, \S\ref{app:iriw}}{}} \\
%% \hline
%% \textsf{rlx}      & \tick & &       & & &  & & & \tick\\ 
%% \textsf{rel+acq}  & \tick & &       & & &  & & & \tick\\ 
%% \textsf{sc}       & \tick & & \tick & & &  & & & \tick\\ 

%% \hline
%% \multicolumn{10}{|c|}{Write-to-Read Causality (\textsf{WRC})\ifext{, \S\ref{app:wrc}}{}} \\
%% \hline
%% \textsf{rlx}      & \tick & &       & & & &  & & \tick\\ 
%% \textsf{rel+acq}  & \tick & &       & & & &  & & \tick\\ 
%% \textsf{cas+rel}  & \tick & &       & & & \tick & & & \tick\\ 
%% \textsf{cas+rlx}  & \tick & &       & & & &  & & \tick\\ 

%% \hline
%% \multicolumn{10}{|c|}{Out-of-Thin-Air (\textsf{OTA})\ifext{, \S\ref{app:ota}}{}} \\
%% \hline
%% \textsf{lb}       & \tick & &       & & \tick & & & & \fail\\ 
%% \textsf{if}       & \tick & &       & & \tick & & & & \fail\\ 

%% \hline
%% \multicolumn{10}{|c|}{Write Reorder (\textsf{WR})\ifext{, \S\ref{app:wr}}{}} \\
%% \hline
%% \textsf{rlx}      & \tick & &       & & \tick & & & & \tick\\ 
%% \textsf{rlx+rel}  & \tick & &       & & \tick & \tick & & & \tick\\ 
%% \textsf{rel}      & \tick & &       & & \tick & \tick & & & \tick\\ 

%% %% \hline
%% %% \multicolumn{11}{|c|}{Value Stealing (\textsf{VS})\ifext{, \S\ref{app:ss}}{}} \\
%% %% \hline
%% %% \textsf{rlx}      & \tick & &       & & \tick & \tick & \tick & & & \tick\\ 

%% \hline
%% \multicolumn{10}{|c|}{Speculative Execution (\textsf{SE})\ifext{, \S\ref{app:se}}{}} \\
%% \hline
%% \textsf{simple}      & \tick & &       & & \tick & & & & \tick\\ 
%% \textsf{prop}        & \tick & &       & & \tick & & & & \tick\\ 
%% \textsf{nested}      & \tick & &       & & \tick & & & & \tick\\ 

%%   \hline
%%   \multicolumn{10}{|c|}{Locks\ifext{, \S\ref{app:locks}}{}} \\
%%   \hline
%%   Dekker & \tick & & & \tick & & & & & \tick\\ 
%%   Cohen~\cite{Turon-al:OOPSLA14}  & \tick & & & \tick & & & & & \tick\\ 

%% %% LB & $\checkmark$ & & \\
%% %% \multicolumn{4}{l}{ A snippet code here
%% %% }\\
%% %% SB & $\checkmark$ & & \\

%% \hline

%% \end{tabular}
%% }
%% \caption{Litmus tests
%%   \ifext{(Appendix~\ref{sec:litmusTests})}{} and corresponding
%%   semantic aspects of our framework:
%%   {viewfronts}~(\textsf{VF},~\S\ref{sec:hist}),
%%   {write-fronts}~(\textsf{WF},~\S\ref{sec:wrf}),
%%   SC-fronts~(\textsf{SCF},~\S\ref{sec:sc}), {non-atomic
%%     fronts}~(\textsf{NAF},~\S\ref{sec:na}), {postponed
%%     operations}~(\textsf{PO},~\S\ref{sec:postponed-sem}),
%%   {acquire read restrictions}),
%%     %($\stGamma$)}~(\textsf{ARR},~\S\ref{sec:postponed-sem}),
%%   %% {value stealing}~(\textsf{VS},~\S\ref{sec:postponed-sem}),
%%   %% {if speculations}~(\textsf{IS}),
%%   {consume-reads}~(\textsf{CR}), {joining threads with non-empty
%%     operation buffers}~(\textsf{JN},~\S\ref{sec:join}).  The column
%%   \textbf{C11} indicates whether the behavior is coherent with the C11
%%   standard.  }
%% \label{fig:litmusTbl}
%% \end{figure}
%% }

В \underline{\textbf{третьей главе}}  описывается ,,обещающая`` модель памяти, а также связанные с ней  открытые  научные задачи.

,,Обещающая`` модель памяти является операционной моделью памяти для синтаксиса модели C/C++11.
Она использует те же базовые понятия, что и предложенная диссертантом модель $\OpCpp$,
метки времени и фронты, однако вместо механизма откладывания выполнения инструкций она,
в соответствии со своим названием, использует механизм \emph{обещаний}. Так, в каждый момент исполнения поток
,,обещающей`` модели памяти может совершить одно из двух действий: либо выполнить следующую инструкцию,
либо пообещать сделать запись в локацию. Последнее может быть выполнено вне зависимости от того, какая
инструкция является следующей.
Если поток выбирает пообещать сделать запись в локацию, то он добавляет соответствующее сообщение в память,
делая это сообщение видимым для других потоков. Далее в ходе исполнения поток должен будет выполнить соответствующую инструкцию записи, таким образом выполняя сделанное ранее обещание.

Для того, чтобы запретить ,,значения из воздуха``, после каждого шага исполнения каждый поток должен выполнить
т.н. \emph{сертификацию} --- предъявить, что он (поток) может быть локально исполнен таким образом, что
он выполнит все оставшиеся обещания. К сожалению, задача сертификации является алгоритмически неразрешимой, если
язык, для которого определена ,,обещающая`` модель, является полным по Тьюрингу. Как следствие, для
этой модели невозможно разработать интерпретатор, что является её недостатком по сравнению с представленной в
диссертации моделью $\OpCpp$.

Несмотря на этот недостаток, ,,обещающая`` модель обладает рядом существенных достоинств.
В частности, ,,обещающая`` модель не имеет проблемы ,,значений из воздуха``, что делает возможным для неё разработать
выразительную программную логику. Также для модели была доказана корректность существенного класса компиляторных
оптимизаций и корректность эффективной компиляции в модели памяти процессоров x86 и Power.

Открытой проблемой является доказательство корректности компиляции из ,,обещающей`` модели памяти в модели памяти архитектуры
ARM, которая, наравне с x86 и Power, является одной из наиболее распространённых процессорных архитектур на данный момент.
Решение этой задачи описано в последующих главах диссертации.

В \underline{\textbf{четвертой главе}} приводится описание операционной модели ARMv8 POP и представлено доказательство
корректности компиляции из существенного подмножества ,,обещающей`` модели памяти в эту модель.
В конце главы приводятся рассуждения о том, как нужно доработать доказательство, чтобы покрыть всю ,,обещающую`` модель.

Здесь и далее под корректностью компиляции мы понимаем следующее утверждение.\\
\textbf{Определение.} Для языков $L$ и $L'$ с моделями памяти $M$ и $M'$ соответственно схема компиляции $\textsf{compile} : L \rightarrow L'$
называется \emph{корректной}, если выполняется следующее условие:
\[ \forall Prog \in L. \; \sembr{\textsf{compile}(Prog)}_{M'} \subseteq \sembr{Prog}_{M}, \]
где $\sembr{Prog}_{M}$ --- множество результатов сценариев поведения программы $Prog$ в модели памяти $M$.
В доказательствах корректности компиляции в ARMv8 POP и ARMv8.3 под результатом сценария поведения мы будем
понимать финальное состояние памяти.

Рассмотренное подмножество ,,обещающей`` модели памяти ($\Promise$) состоит из расслабленных (relaxed, rlx) записей и чтений,
а также высвобождающих (release, rel) и приобретающих (acquire, acq) барьеров памяти. При этом подразумевается следующая
схема компиляции:
\[
  \begin{array}{c@{~~}@{~~}l@{~~}|@{~~}l@{~~}|@{~~}l@{~~}|@{~~}l}
    \textbf{Promise:}   & [x]_{\textsf{rlx}} := \; a & a := [x]_{\textsf{rlx}}  &  \acqFence & \relFence \\[2pt]
    \textbf{ARMv8 POP:} & \writeInst{x}{a}    & \readInst{a}{x}  &  \dmbLD & \dmbSY \\
  \end{array}
\]
Первый и второй столбцы подразумевают, что расслабленные операции записи и чтения из языка, на котором определена ,,обещающая`` модель,
переходят в обычные операции записи и чтения в терминах ARMv8-ассемблера, а приобретающий и высвобождающий барьеры --- в барьер
по чтению и в полный барьер. Такая схема компиляции считается эффективной и применяется в компиляторах GCC и LLVM.
Поскольку схема компиляции в данном случае является биекцией, то далее в этой главе мы предполагаем, что язык задания ,,обещающей`` и ARMv8 POP
моделей совпадает.

Основной результат главы %про корректность компиляции из ,,обещающей`` семантики в ARMv8 POP
сформулирован следующим образом. \\
\emph{Теорема 3.1.} Для любой программы $\Cf$ на языке задания модели и её сценария поведения в модели ARMv8 POP существует
такой сценарий поведения $\Cf$ в ,,обещающей`` модели, что финальное состояние памяти в сценариях поведения совпадает.

В рамках доказательства теоремы нужно построить сценарий поведения в ,,обещающей`` модели по сценарию поведения в модели ARMv8 POP.
Поскольку обе модели заданы операционным способом, то существуют две абстрактные машины, которые представляют данные модели.
Обычно для решения задачи построения исполнения одной машины по исполнению другой машины используют технику симуляции,
которая является специализированной формой индукции. В рамках данной техники предполагается ввести  отношение симуляции,
которое связывает состояния машин, и доказать  две следующие леммы:  отношение связывает начальные состояния машин (база индукции);
для любого шага симулируемой машины существует ноль или более шагов симулирующей машины, таких что
новые состояния машин опять связаны отношением симуляции (индукционный переход).

%% Модель памяти ARMv8 POP --- операционная модель памяти для архитектуры ARMv8.0, предложенная в 2016 г.
%% Данная модель определена в терминах, достаточно близких к физической реализации архитектуры ARM.
%% Абстрактная машина, которая реализует модель ARMv8 POP, состоит из двух компонент: подсистем памяти и управления.

%% Подсистема памяти является иерархической структурой \emph{буферов}, каждый из которых является списком запросов к подсистеме.
%% Запросом, при этом, может быть чтение из локации, запись в локацию или барьер памяти.
%% Идейно подсистема памяти похожа на иерархическую систему кэшей, используемую в современных процессорах.
%% Так, подсистема управления может послать запрос в подсистему хранения. В таком случае запрос сначала попадает в буфер, который локален
%% для соответствующего потока, потом он может быть передан следующему в иерархии буферу, который является общий для некоторого
%% набора потоков, и так далее, пока запрос не попадёт в основную память или, если это запрос на чтение, не будет
%% удовлетворён из сообщения.

%% Подсистема управления выполняет программу каждого потока. Шагом исполнения программы потока является полное или частичное исполнение
%% некоторой инструкции из потока. Например, исполнение инструкции чтения выполняется в три шага --- отправка запроса в подсистему памяти,
%% получение ответа от подсистемы памяти и завершение исполнения инструкции чтения. При этом между упомянутыми шагами исполнения могут
%% исполняться другие инструкции, поскольку подсистема управления может исполнять инструкции не по порядку и спекулятивно.

В доказательстве индукционного перехода сложным является то, что между моделями существует два существенных различия.
Во-первых, ,,обещающая`` модель может исполнять не по порядку только инструкции записи, что соответствует имеющемуся в ней механизму обещаний,
тогда как модель ARMv8 POP может исполнять инструкции в несколько шагов, не по порядку и спекулятивно.
Во-вторых, в ,,обещающей`` модели в тот момент, когда сообщение попадает в память, этому сообщению присваивается некоторая
метка времени, которая служит его порядковым номером в множестве сообщений, относящихся к той же локации. В модели ARMv8 POP
меток времени нет и порядок на сообщениях к одной локации определяется не сразу после того, как
сообщения попадут в её подсистему памяти и станут видимыми для других потоков.

Для того, чтобы обойти первое различие, мы использовали технику ,,запаздывающей`` симуляции. В рамках данной техники отношение
симуляции представляется как объединение двух взаимоисключающих отношений, например, $A$ и $B$. Далее индукционный
переход формулируется следующим образом. Если состояние симулируемой машины $x$ связано с состояние симулирующей машины $y$ отношением $A$,
т.е. выполняется $(x,y) \in A$, то для любого состояния $x'$, в которое может перейти симулируемая машина, выполняется $(x', y) \in A \cup B$.
С другой стороны, если $(x, y) \in B$, то существует состояние $y'$, в которое может перейти симулирующая машина, что $(x, y') \in A \cup B$.
При условии, что не существует такой бесконечной цепочки $\{y'_i\}_{i \in \mathbb{N}}$, что $y'_i$ переходит $y'_{i+1}$ и $(x, y'_i) \in B$ для всех $i$,
то из нового варианта индукционного перехода следует изначальное утверждение симуляции.

В доказательстве теоремы 3.1 отношение $A$ символизирует, что ,,обещающая`` машина ждёт, пока ARM-машина выполнит действие, которое ,,обещающая``
машина может повторить, тогда как отношение $B$ означает, что ,,обещающая`` машина может симулировать несколько действий, уже выполненных ARM-машиной.

Для того, чтобы обойти второе различие между ,,обещающей`` и ARMv8 POP моделями, в доказательстве вводится ограниченная версия ARM-машины,
которая добавляет метки времени к сообщениям записи в подсистеме памяти, тем самым определяя порядок на сообщениях к одной локации
раньше, чем это делает обычная ARMv8 POP модель. Это изменение также добавляет дополнительные ограничения на сценарии поведения ARM-машины.
Тем не менее, мы доказываем, что новая модель эквивалентна изначальной, что позволяет свести доказательство теоремы 3.1
к доказательству корректности компиляции из ,,обещающей`` в модифицированную ARMv8 POP модели.

В \underline{\textbf{пятой главе}} обсуждается аксиоматическая модель памяти ARMv8.3.
Приводятся рассуждения о том, почему метод доказательства корректности компиляции из
,,обещающей`` модели памяти, использованный её авторами для аксиоматических моделей архитектур x86 и Power,
не подходит для модели ARMv8.3. Далее приводится доказательство корректности компиляции из ,,обещающей`` модели
памяти в подмножество модели ARMv8.3. Доказательство основано на построении операционной семантики обхода аксиоматических сценариев
поведения программ в модели ARMv8.3. В конце главы приводятся рассуждения о применимости использованного подхода для других
аксиоматических моделей.

В рамках аксиоматической (или декларативной) модели памяти сценарий поведения программы представляется как граф, в котором вершинами
являются \emph{события} (операции над памятью), а ребрами --- различные отношения на событиях, такие как программный
порядок, отношение ,,читает из`` и др. При этом граф считается согласованным с моделью, если выполняются \emph{аксиомы} модели,
которые обычно формулируются как наличие некоторого полного порядка на подмножестве событий или отсутствие  в графе путей определённого
типа.

При доказательстве корректности компиляции из ,,обещающей`` модели в аксиоматическую модель техника симуляции напрямую
неприменима, поскольку сценарий поведения в аксиоматической модели не является последовательностью шагов исполнения некоторой абстрактной машины.
Поэтому в существующем доказательстве корректности компиляции в 
модели x86 и Power авторы использовали другой метод. Этот метод состоит из двух частей.
Во-первых, доказывается, что модели x86 и Power могут быть представлены как набор программных оптимизаций поверх
более простых моделей. Эти оптимизации являются доказано корректными в рамках ,,обещающей`` модели, из чего следует,
что доказательство корректности компиляции может быть сведено к доказательству для более простых моделей.
Далее показывается, что эти более простые модели могут быть симулированы аксиоматическим аналогом
,,обещающей`` модели без механизма обещаний.

Нам не удалось применить такой подход для модели ARMv8.3, поскольку эта модель не представима как набор тех же
оптимизаций над упрощенной моделью, не требующих механизма обещаний. Поэтому мы разработали альтернативный подход,
который заключается в построении операционной семантики обхода аксиоматических построений, которая может быть
симулирована ,,обещающей`` моделью напрямую.

\emph{Обходом} мы называем последовательность переходов между \emph{конфигурациями} обхода. Конфигурацией исполнения $G$ является
упорядоченная пара подмножеств вершин сценария исполнения, $\tup{C, \IssuedSet}$.
Первое подмножество, $C$, называется \emph{множеством покрытых событий}, а второе, $\IssuedSet$ --- \emph{множеством выпущенных событий};
элементы этих множеств называются \emph{покрытыми} и \emph{выпущенными} соответственно.

Конфигурация обхода называется \emph{корректной}, если выполняются следующие условия.
\begin{itemize}
  \item Множество покрытых событий префикс-замкнуто относительно отношения программного порядка.
  \item Множество выпущенных событий содержит только события записи.
  \item Если событие записи покрыто, то оно является также выпущенным.
\end{itemize}
При доказательстве симуляции ,,обещающей`` моделью обхода покрытые события будут соответствовать инструкциям, выполненным
,,обещающей`` машиной, а выпущенные события --- сообщениям в памяти ,,обещающей`` машины.

Шаги обхода задаются следующим образом:
\begin{mathpar}
\inferrule*{
    a \in \nextset(G, C) \cap \coverable(G, C, \IssuedSet) 
}{
    G \vdash 
    \tup{C, \IssuedSet} \travConfigStep \tup{C \cup \{a\}, \IssuedSet}
} \and
\inferrule*{
    w \in \issuable(G, C, \IssuedSet) \setminus \IssuedSet 
    }{
    G \vdash
    \tup{C, \IssuedSet} \travConfigStep \tup{C, \IssuedSet \cup \{w\}}
}
\end{mathpar}
Здесь первое правило соответствует покрытию события $a$, а второе --- созданию события $w$; $\nextset(G, C)$ --- обозначает множество событий,
непосредственно следующих в отношении программного порядка за покрытыми; $\coverable(G, C, I)$ и $\issuable(G, C, I)$ --- это
события, покрываемые и выпускаемые в текущей конфигурации, которые определены в соответствии с требованиями ,,обещающей`` модели
на исполнение инструкции и обещание сообщения соответственно.

Далее мы доказываем теорему 5.1 о полноте обхода. \\
\emph{Теорема 5.1.} Для любого корректного сценария поведения $G$ в модели ARMv8.3 существует обход
$\tup{W^{\textrm{init}}, W^{\textrm{init}}} \travConfigStep^{*} \tup{E, W}$,
где $W^{\textrm{init}}$ --- множество инициализирующих записей сценария $G$, $E$ --- все события сценария $G$,
$W$ --- все события записи сценария $G$.

Используя теорему 5.1 для построения операционного исполнения программы в модели ARMv8.3,
мы доказываем, что ,,обещающая`` модель может симулировать сценарий поведения модели ARMv8.3.

%% Модель ARMv8.3 является аксиоматической, что не позволяет напрямую использовать технику симуляции для доказательства
%% корректности компиляции из ,,обещающей`` модели. 

%% В \underline{\textbf{главе}}
%% Обсуждается структура доказательства корректности компиляции. Доказательство основано на симуляции ,,обещающей`` моделью
%% памяти операционной семантики обхода исполнения программ в модели ARMv8.3.

В \underline{\textbf{заключении}} приведены основные результаты работы, которые заключаются в следующем:
%% Согласно ГОСТ Р 7.0.11-2011:
%% 5.3.3 В заключении диссертации излагают итоги выполненного исследования, рекомендации, перспективы дальнейшей разработки темы.
%% 9.2.3 В заключении автореферата диссертации излагают итоги данного исследования, рекомендации и перспективы дальнейшей разработки темы.
\begin{enumerate}
  \item Разработан операционный аналог модели памяти C/C++11.
    Данный аналог показывает такие же сценарии поведения, что и модель C/С++11, на большинстве
    тестов, приведенных в литературе, но не обладает сценариями поведения со значениями из воздуха.
    В отличие от ``обещающей'' семантики, предлагаемый аналог является запускаемым, что упрощает
    разработку средств анализа программ для него. Негативным отличием от ``обещающей'' семантики является то,
    предлагаемый аналог накладывает синтаксические ограничения на поведения программ.
  \item Доказана корректность компиляции из существенного подмножества ``обещающей'' семантики в операционную модель
    памяти ARMv8 POP.
  \item Доказана корректность компиляции из существенного подмножества ``обещающей'' семантики в
    аксиоматическую модель памяти ARMv8.3.
\end{enumerate}

В рамках \textbf{рекомендации по применению результатов работы} в индустрии и научных исследованиях указывается,
что модель памяти промышленного языка программирования должна быть лишена сценариев поведения, имеющих значения
из воздуха, а также либо быть представленной в операционной форме, либо иметь эквивалетный ей операционный аналог.
Последнее позволяет реализовать интерпретатор модели и выполнять отладку программ в рамках модели.

Также были определены \textbf{перспективы дальнейшей разработки тематики}, основным из которых является
разработка обобщенной аксиоматической модели памяти для процессорных архитектур, которая будет
определена для синтаксиса модели C/C++11 и будет строгим надмножеством существующих моделей памяти
x86, Power и ARM, а также для которой будет применим предложенный метод доказательства корректности компиляции
из ``обещающей'' модели памяти. Это позволит свести дальнейшие доказательства корректности компиляции из
``обещающей'' модели к доказательству корректности компиляции в обобщенную аксиоматическую модель, что
сводится к рассуждениям об ацикличности и вложенности путей на графах.



%%  картинку можно добавить так:
%% \begin{figure}[ht] 
%%   \center
%%   \includegraphics [scale=0.27] {latex}
%%   \caption{Подпись к картинке.} 
%%   \label{img:latex}
%% \end{figure}

%% Формулы в строку без номера добавляются так:
%% \[ 
%%   \lambda_{T_s} = K_x\frac{d{x}}{d{T_s}}, \qquad
%%   \lambda_{q_s} = K_x\frac{d{x}}{d{q_s}},
%% \]



%\newpage
%% При использовании пакета \verb!biblatex! список публикаций автора по теме
%% диссертации формируется в разделе <<\publications>>\ файла
%% \verb!../common/characteristic.tex!  при помощи команды \verb!\nocite! 

%% \ifdefmacro{\microtypesetup}{\microtypesetup{protrusion=false}}{} % не рекомендуется применять пакет микротипографики к автоматически генерируемому списку литературы
%% \ifnumequal{\value{bibliosel}}{0}{% Встроенная реализация с загрузкой файла через движок bibtex8
%%   \renewcommand{\bibname}{\large \authorbibtitle}
%%   \nocite{*}
%%   \insertbiblioauthor           % Подключаем Bib-базы
%%   %\insertbiblioother   % !!! bibtex не умеет работать с несколькими библиографиями !!!
%% }{% Реализация пакетом biblatex через движок biber
%%   %% \insertbiblioauthor           % Вывод всех работ автора
%%  \insertbiblioauthorgrouped    % Вывод всех работ автора, сгруппированных по источникам
%% %  \insertbiblioauthorimportant  % Вывод наиболее значимых работ автора (определяется в файле characteristic во второй section)
%%   \insertbiblioother            % Вывод списка литературы, на которую ссылались в тексте автореферата
%% }
%% \ifdefmacro{\microtypesetup}{\microtypesetup{protrusion=true}}{}

\newcounter{firstbib}

\section*{\LARGE Публикации автора по теме диссертации}

Ниже приведён перечень публикаций, где были представлены основные результаты данной  диссертационной работы. \\

\renewcommand{\bibsection}{\noindent \textbf{\refname}}

\renewcommand{\refname}{Статьи из \vakJournals}
\begin{thebibliography}{99}
\bibitem{Podkopaev-al:NTV17} Подкопаев, А. В. О корректности компиляции подмножества обещающей модели памяти в аксиоматическую модель ARMv8.3 / А.В. Подкопаев, О. Лахав, В. Вафеядис // Научно-технические ведомости Санкт-Петербургского государственного политехнического университета информатика, телекоммуникации, управление. ---~2017. ---~Т.~4, \textnumero~??. ---~C.~??--??.
\bibitem{Podkopaev-al:ISPRAS17} Подкопаев, А. В. Обещающая компиляция в ARMv8.3 / А.В. Подкопаев, О. Лахав, В. Вафеядис // Труды ИСП РАН. ---~2017. ---~Т.~?, \textnumero~??. ---~C.~??--??.
\setcounter{firstbib}{\value{enumiv}}
\end{thebibliography}

\renewcommand{\refname}{Статьи в изданиях, входящих в базы цитирования Web of Science и SCOPUS}
\begin{thebibliography}{99}
\setcounter{enumiv}{\value{firstbib}}
\bibitem{Podkopaev-al:ECOOP17} Podkopaev, A. Promising compilation to ARMv8 POP / A. Podkopaev, O. Lahav, V. Vafeiadis // 31st European Conference on Object-Oriented Programming (ECOOP 17), Leibniz International Proceedings in Informatics (LIPIcs).  ---~2017. ---~P.~22:1--22:28.
\setcounter{firstbib}{\value{enumiv}}
\end{thebibliography}

\renewcommand{\refname}{Статьи в других изданиях}
\begin{thebibliography}{99}
\setcounter{enumiv}{\value{firstbib}}
\bibitem{Podkopaev-al:PLC17} Подкопаев, А. В. Обещающая компиляция в ARMv8 / А.В. Подкопаев, О. Лахав, В. Вафеядис // Языки программирования и компиляторы. Труды конференции. Ростов-на-Дону, Россия. ---~2017. ---~C.~223--226.
\bibitem{Podkopaev-al:CoRR16} Podkopaev, A. Operational Aspects of {C/C++} Concurrency / A. Podkopaev, I. Sergey, A. Nanevski
  [Электронный ресурс]. --- URL: \url{http://arxiv.org/abs/1606.01400} (дата обращения: 14.11.2017).
\end{thebibliography}
