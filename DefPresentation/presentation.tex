\ifdefined\ishandout
  %% \documentclass[handout,xcolor=dvipsnames]{beamer}
  \documentclass[xcolor=dvipsnames]{beamer}
\else
  \documentclass[xcolor=dvipsnames]{beamer}
\fi

\mode<presentation>
{
%% \usetheme{Frankfurt}
%% \useoutertheme{split}
	%% \usetheme{CambridgeUS}
	%% \usetheme{Hannover}
  \usetheme{Singapore}
	\usecolortheme{rose}
\beamertemplatenavigationsymbolsempty
\setbeamertemplate{mini frames}{}
}
\usepackage{amsmath,stmaryrd}
\usepackage{amsfonts}
\usepackage{amssymb}
\usepackage{pifont}
\usepackage{cmap}
\usepackage{listings}
\usepackage{lmodern}
\usepackage{xparse}
\usepackage{color}
\usepackage{minted}
\usepackage{graphicx}
\usepackage{tikz}
\usetikzlibrary{positioning}
\usetikzlibrary{fadings,decorations.pathmorphing,decorations.pathreplacing}
\usetikzlibrary{shapes}
\usetikzlibrary{shapes.multipart,matrix,calc}
\usetikzlibrary{arrows}
\usetikzlibrary{patterns}
\usetikzlibrary{shapes.callouts} 
\usetikzlibrary{shadows,calc}
\usepackage{wrapfig}
\usepackage{stmaryrd}
\usepackage{hyperref}
\usepackage{mathtools}
\usepackage[absolute,overlay]{textpos}
\usepackage{xspace}
\usepackage[nomessages]{fp}% http://ctan.org/pkg/fp
\usepackage{wasysym}

\usepackage[labelformat=empty]{caption}
\usepackage{fontspec}
%% \usepackage{polyglossia}
%% \setdefaultlanguage{russian}
\usepackage{ebproof}

\setmainfont[Ligatures=TeX]{CMU Serif}
\setsansfont[Ligatures=TeX]{CMU Sans Serif}
%% \setmonofont{Nimbus Mono PS}

\tikzset{
    invisible/.style={opacity=0,text opacity=0},
    visible on/.style={alt=#1{}{invisible}},
    alt/.code args={<#1>#2#3}{%
      \alt<#1>{\pgfkeysalso{#2}}{\pgfkeysalso{#3}} % \pgfkeysalso doesn't change the path
    },
}

\NewDocumentCommand{\mycallout}{r<> O{center} O{opacity=1.0,text opacity=1} m m m +m}{%
\tikz[remember picture, overlay]\node[align=#2, fill=#4!20, %text width=3cm,
#3,visible on=<#1>, rounded corners,
draw,rectangle callout,anchor=pointer,callout relative pointer={#6}
]
at #5 {#7};
}

\newcommand{\tikzmark}[1]{\tikz[overlay,remember picture,baseline=-0.5ex] \node (#1) {};}
\newcommand{\prevFr}[2]{\FPeval{#1}{clip(#2-1)}}
\newcommand{\nextFr}[2]{\FPeval{#1}{clip(#2+1)}}
\newcommand{\toGray}[3]{\only<-#1>{#3} \only<#2->{\textcolor{gray!30}{#3}}}

\newcommand{\toShadow}[3]{
  \only<#1>{#3}\only<#2>{\textcolor{colorSHADE}{#3}}
}

\definecolor{myGray}{RGB}{50,50,50}

\newcommand{\itab}[1]{\hspace{0em}\rlap{#1}}
\newcommand{\tab}[1]{\hspace{.2\textwidth}\rlap{#1}}

\newcommand{\inarr}[1]{\begin{array}{@{}l@{}}#1\end{array}}
\newcommand{\inarrII}[2]{\begin{array}{@{}l@{~~}||@{~~}l@{}}\inarr{#1}&\inarr{#2}\end{array}}
\newcommand{\inarrIII}[3]{\begin{array}{@{}l@{~~}||@{~~}l@{~~}||@{~~}l@{}}\inarr{#1}&\inarr{#2}&\inarr{#3}\end{array}}
\newcommand{\inarrIV}[4]{\begin{array}{@{}l@{~~}||@{~~}l@{~~}||@{~~}l@{~~}||@{~~}l@{}}\inarr{#1}&\inarr{#2}&\inarr{#3}&\inarr{#4}\end{array}}
\newcommand{\inarrV}[5]{\begin{array}{@{}l@{~~}||@{~~}l@{~~}||@{~~}l@{~~}||@{~~}l@{~~}||@{~~}l@{}}\inarr{#1}&\inarr{#2}&\inarr{#3}&\inarr{#4}&\inarr{#5}\end{array}}
\newcommand{\inarrc}[1]{\begin{array}{@{}c@{}}#1\end{array}}
\newcommand{\inarrIIId}[3]{\begin{array}{@{}l@{~}||@{~}l@{~}||@{~}l@{}}\inarr{#1}&\inarr{#2}&\inarr{#3}\end{array}}

\newcommand{\myblock}[1]{\begin{block}{} #1 \end{block}}


\definecolor{CommentColor}{rgb}{0.0,0.4,0.1}

\newcommand{\commenta}[1]{\color{CommentColor}{\texttt{/*}\textit{ #1 }\texttt{*/}}}
\newcommand{\comment}[1]{{\color{CommentColor}~\texttt{/\!\!/}\,{#1}}}
\newcommand{\nocomment}[1]{{\color{red}~~\texttt{/\!\!/}\,{#1}}}
\newcommand{\progval}[1]{{\color{CommentColor} #1}}

\newcommand\semState[2]{\llbracket #1 \rrbracket _{#2}}

\newcommand{\storageThreeThreads}[3]{
    \draw (0,0.5) rectangle (3,1) node [pos=.5] {Thread 1};
    \draw (3.5,0.5) rectangle (6.5 ,1) node [pos=.5] {Thread 2};
    \draw (7.0,0.5) rectangle (10.0,1) node [pos=.5] {Thread 3};
    \draw (1,0.5) -- (1, 0.5 -#1) -- (5.0, 0.5 -#1) -- (5.0,0.5);
    \draw (3.0, 0.5 -#1) -- (3.0, 0.5 -#1 - #2) -- (8.5, 0.5 -#1 - #2) -- (8.5, 0.5);
    \draw (5.75, 0.5 -#1 - #2) -- (5.75, 0.5 -#1 - #2 - #3);

    \node at (5.75, 0.3 -#1 - #2 - #3) {The main memory};
    \draw (4.75, 0.1 -#1 - #2 - #3) -- (6.75, 0.1 -#1 - #2 - #3);
}

\newcommand{\storageTwoThreads}[2]{
    \draw (0,#1) rectangle (2,#1 + 0.5) node [pos=.5] {Thread 1};
    \draw (2.5,#1) rectangle (4.5,#1 + 0.5) node [pos=.5] {Thread 2};
    \draw (1,#1) -- (1,-0.5) -- (3.5,-0.5) -- (3.5,#1);
    \draw (2.25, -0.5) -- (2.25, -#2);

    \node at (2.25, -#2 - 0.2) {The main memory};
    \draw (1.25, -#2 - 0.4) -- (3.25, -#2 - 0.4);
}

\newcommand{\executed}[1]{\colorbox{gray}{\ensuremath{#1}}}
\newcommand{\markExecuted}[3]{
  \only<-#1>{#3}
  \only<#2->{\executed{#3}}
}
\newcommand{\light}[1]{\textcolor{gray!30}{#1}}

\makeatletter
\def\ft@overlay{}

\addtobeamertemplate{footline}{}%
{%
  \lineskiplimit0pt
  \begin{tikzpicture}[remember picture,overlay]%
  \ft@overlay
  \end{tikzpicture}%
  \gdef\ft@overlay{}%
}

\newcommand<>{\addtooverlay}[1]{%
  \only#2{%
  \expandafter\gdef\expandafter\ft@overlay\expandafter{\ft@overlay #1}%
  }%
}

\def\shadowshift{3pt,-3pt}
\def\shadowradius{6pt}

\colorlet{innercolor}{black!60}
\colorlet{outercolor}{blue!05}

% this draws a shadow under a rectangle node
\newcommand\drawshadow[1]{
    \begin{pgfonlayer}{shadow}
        \shade[outercolor,inner color=innercolor,outer color=outercolor] ($(#1.south west)+(\shadowshift)+(\shadowradius/2,\shadowradius/2)$) circle (\shadowradius);
        \shade[outercolor,inner color=innercolor,outer color=outercolor] ($(#1.north west)+(\shadowshift)+(\shadowradius/2,-\shadowradius/2)$) circle (\shadowradius);
        \shade[outercolor,inner color=innercolor,outer color=outercolor] ($(#1.south east)+(\shadowshift)+(-\shadowradius/2,\shadowradius/2)$) circle (\shadowradius);
        \shade[outercolor,inner color=innercolor,outer color=outercolor] ($(#1.north east)+(\shadowshift)+(-\shadowradius/2,-\shadowradius/2)$) circle (\shadowradius);
        \shade[top color=innercolor,bottom color=outercolor] ($(#1.south west)+(\shadowshift)+(\shadowradius/2,-\shadowradius/2)$) rectangle ($(#1.south east)+(\shadowshift)+(-\shadowradius/2,\shadowradius/2)$);
        \shade[left color=innercolor,right color=outercolor] ($(#1.south east)+(\shadowshift)+(-\shadowradius/2,\shadowradius/2)$) rectangle ($(#1.north east)+(\shadowshift)+(\shadowradius/2,-\shadowradius/2)$);
        \shade[bottom color=innercolor,top color=outercolor] ($(#1.north west)+(\shadowshift)+(\shadowradius/2,-\shadowradius/2)$) rectangle ($(#1.north east)+(\shadowshift)+(-\shadowradius/2,\shadowradius/2)$);
        \shade[outercolor,right color=innercolor,left color=outercolor] ($(#1.south west)+(\shadowshift)+(-\shadowradius/2,\shadowradius/2)$) rectangle ($(#1.north west)+(\shadowshift)+(\shadowradius/2,-\shadowradius/2)$);
        %% \filldraw ($(#1.south west)+(\shadowshift)+(\shadowradius/2,\shadowradius/2)$) rectangle ($(#1.north east)+(\shadowshift)-(\shadowradius/2,\shadowradius/2)$);
    \end{pgfonlayer}
}

% create a shadow layer, so that we don't need to worry about overdrawing other things
\pgfdeclarelayer{shadow} 
\pgfsetlayers{shadow,main}

\newsavebox\mybox
\newlength\mylen

%% \newcommand\shadowimage[2][]{%
%% \begin{tikzpicture}
%% \node[anchor=south west,inner sep=0] (image) at (0,0) {\colorbox{white}{\includegraphics[#1]{#2}}};
%% \drawshadow{image}
%% \end{tikzpicture}}

\newcommand\shadowimage[2][]{%
\setbox0=\hbox{\includegraphics[#1]{#2}}
\setlength\mylen{\wd0}
\ifnum\mylen<\ht0
\setlength\mylen{\ht0}
\fi
\divide \mylen by 120
\def\shadowshift{\mylen,-\mylen}
\def\shadowradius{\the\dimexpr\mylen+\mylen+\mylen\relax}
\begin{tikzpicture}
  \node[anchor=south west,inner sep=0] (image) at (0,0) {\colorbox{white}{\includegraphics[#1]{#2}}};
  \begin{pgfonlayer}{shadow}
     \drawshadow{image}
  \end{pgfonlayer}
\end{tikzpicture}}

\makeatother

% SET OVERLAY NUMBER ORDER
\usepackage{etoolbox}
\usepackage{xstring}
\newcounter{mycounter}

% #1 is of the form <name>[=-|n], e.g. foo=- or foo=10 or foo
% Stores in #2 the overlay specification for <name> s.t. it can be appended to the content of <name>
\newcommand*{\getNewOverlayContent}[2]{%
    \getItemSpec{#1}{itemSpec}%
    \IfStrEq{\itemSpec}{-}{%
        \csedef{#2}{\arabic{beamerpauses}-}%
    }{%
        \IfStrEq{\itemSpec}{}{%
            \csedef{#2}{\arabic{beamerpauses}}%
        }{%
            \IfInteger{\itemSpec}{%
%               \mycounter=\
                \setcounter{mycounter}{\arabic{beamerpauses}}%
                \addtocounter{mycounter}{\itemSpec}%
                \addtocounter{mycounter}{-1}%
                \csedef{#2}{\arabic{beamerpauses}-\arabic{mycounter}}%
            }{%
                \PackageError{setorder}{Argument has illegal format}{Argument was #1}%
            }%
        }%
    }%
%   input: #1, itemspec:\itemSpec, beamervalue: \arabic{beamerpauses}, content: \csuse{#2} \\
}
% #1 is of the form 'foo=1' or 'foo=-' or 'foo'. 
% #2 Is the name of the macro which should hold the result
% This macro stores the part infront '=' (the name) in #2.
\newcommand*{\getItemName}[2]{% 
    \IfSubStr{#1}{=}{%
        \StrBefore{#1}{=}[\tmp]%
        \csdef{#2}{\tmp}%
    }{%
        \csdef{#2}{#1}% 
    }%
}
% #1 is of the form 'foo=1' or 'foo=-' or 'foo'. 
% #2 Is the name of the macro which should hold the result
% This macro stores the part behind '=' (the overlay spec) in #2. The stored part is empty iff there is no '=' in #1
\newcommand*{\getItemSpec}[2]{%
    \StrBehind{#1}{=}[\tmp]%
    \csdef{#2}{\tmp}%
}
% #2 is the name where content should be appended. 
% It has been ensured previously that #2 is a defined macro
% #1 is the content to append
% Depending on whether #2 is empty or not a (,) is added 
% before appending #1
\newcommand*{\appendToOverlaySpecification}[2]{%
    \IfStrEq{\csexpandonce{#2}}{}{%
        % #1 i.e. <name> is empty
        \cseappto{#2}{\csname#1\endcsname}%
    }{%
        \cseappto{#2}{,\csname#1\endcsname}%
    }%
}

\newcommand*{\setorderItem}[1]{%
    \getNewOverlayContent{#1}{overlaycontent}%
    \getItemName{#1}{cmdname}%
    \appendToOverlaySpecification{overlaycontent}{\cmdname}%
}

\newcommand*{\setorderList}[1]{%
    \forcsvlist{\setorderItem}{#1}%
    \stepcounter{beamerpauses}%
}
\newcommand*{\setorder}[1]{%    
    \clearNamesListofLists{#1}%
    \forcsvlist{\setorderList}{#1}%
    %% \createBef{#1}%
}

%% \newcommand*{\createBefItem}[1]{%
%%     \getItemName{#1}{cmdname}%
%%     \getItemName{bef#1}{cmdnameBef}%
%%     \csdef{\cmdnameBef}{}% 
%%     \prevFr{\cmdnameBef}{\cmdname}%
%% }

%% \newcommand*{\createBefList}[1]{%
%%     \forcsvlist{\createBefItem}{#1}%
%% }

%% \newcommand*{\createBef}[1]{%
%%     \forcsvlist{\createBefList}{#1}%
%% }

% takes a list of lists of the form: {foo=1, bla},{gar=-} and then defines empty macros for each name
\newcommand*{\clearNamesListofLists}[1]{%
    \forcsvlist{    \clearNamesList}{#1}%
}
\newcommand*{\clearNamesList}[1]{%
    \forcsvlist{    \clearName}{#1}%
}
\newcommand*{\clearName}[1]{%
    \getItemName{#1}{cmdname}%
    \csdef{\cmdname}{}% 
}

\lstdefinelanguage
   [sparc]{Assembler}     % add a "x64" dialect of Assembler
   [x86masm]{Assembler} % based on the "x86masm" dialect
   % with these extra keywords
   {morekeywords={ldw, stw, LD, ADD, SUB, ST, OP, BR, BLTZ, BGTZ, HALT, CMOZ, BL}} % etc.
\lstset{language={[sparc]Assembler}}

\newcommand{\complScheme}{\mathsf{compl}}
\newcommand{\lwsync}{\mathsf{lwsync}}
\newcommand{\hwsync}{\mathsf{hwsync}}
\newcommand{\opt}{\mathsf{opt}}
\newcommand{\correctOpt}{\mathsf{CorrectOpt}}
\newcommand{\MM}{\mathcal{M}}
\newcommand{\LL}{\mathsf{L}}
\newcommand{\LowLL}{\mathsf{L'}}
\newcommand{\LowMM}{\mathcal{M'}}
\newcommand{\PP}{\mathsf{P}}

\newcommand{\valuecom}[1]{\textcolor{green!60!black}{// #1}}
\newcommand{\valuefail}[1]{\textcolor{red!60!black}{// #1}}
\newcolumntype{L}{>{$}l<{$}} % math-mode version of "l" column type

\declaretheorem[name=Теорема,style=default,numberwithin=section]{thm}
\declaretheorem[name=Лемма,style=default,numberwithin=section]{lem}
%% \declaretheorem[name=Определение,style=default,sibling=thm]{mydefinition}
\declaretheorem[name=Определение,style=definition,numbered=no]{mydefinition}

% enviorments
\theoremstyle{plain}
%% \newtheorem{notation}[theorem]{Notation}
\newtheorem{notation}{Notation}
\newtheorem*{notation*}{Notation}

\newtheorem{definition}{Определение}
\newtheorem*{definition*}{Определение}
\newtheorem{proposition}{Утверждение}
\newtheorem*{proposition*}{Утверждение}
\newtheorem{theorem}{Теорема}
\newtheorem*{theorem*}{Теорема}
\newtheorem{lemma}{Лемма}
\newtheorem*{lemma*}{Лемма}

\crefformat{section}{#2\S{}#1#3}
\Crefname{section}{Section}{Section}
\Crefformat{section}{Section #2#1#3}
\Crefname{figure}{\text{Figure}}{\text{Figures}}
\crefname{corollary}{\text{Corollary}}{\text{corollaries}}
\Crefname{corollary}{\text{Corollary}}{\text{Corollaries}}
\crefname{lemma}{\text{Lemma}}{\text{Lemmas}}
\Crefname{lemma}{\text{Lemma}}{\text{Lemmas}}
\crefname{proposition}{\text{Prop.}}{\text{Propositions}}
\Crefname{proposition}{\text{Proposition}}{\text{Propositions}}
\crefname{definition}{\text{Def.}}{\text{Definitions}}
\Crefname{definition}{\text{Definition}}{\text{Definitions}}
\crefname{notation}{\text{Notation}}{\text{Notations}}
\Crefname{notation}{\text{Notation}}{\text{Notations}}
\crefname{theorem}{\text{Theorem}}{\text{Theorems}}
\Crefname{theorem}{\text{Theorem}}{\text{Theorems}}


\newenvironment{claim}[1]{\par\noindent\underline{Claim:}\space#1}{}
\newenvironment{claimproof}[1]{\par\noindent\underline{Proof:}\space#1}
{\leavevmode\unskip\penalty9999 \hbox{}\nobreak\hfill\quad\hbox{$\square$}}

\newcommand{\textdom}[1]{\mathsf{#1}}
\newcommand{\textcode}[1]{\texorpdfstring{\texttt{#1}}{#1}}
\newcommand{\kw}[1]{\textbf{\textcode{#1}}}
\newcommand{\skipc}{\kw{skip}}
\newcommand{\ite}[3]{\kw{if}\;#1\:\kw{then}\;#2\;\\ \kw{else}\;#3\;\kw{fi}}
\newcommand{\iteml}[3]{
  \kw{if} \; #1\\
  \begin{array}[t]{@{}l@{}l}
    \kw{then}& \begin{array}[t]{l} #2 \end{array} \\
    \kw{else}& \begin{array}[t]{l} #3 \end{array} \\
  \end{array}\\
  \kw{fi}
}
\newcommand{\itne}[2]{\kw{if}\;#1\:\kw{then}\\ \quad\;{#2}}
\newcommand{\while}[2]{\kw{while}\;#1\;\kw{do}\;#2}
\newcommand{\ALT}{\;\;|\;\;}

\newcommand{\ie}{\emph{i.e.,} }
\newcommand{\eg}{\emph{e.g.,} }
\newcommand{\etal}{\emph{et~al.}}
\newcommand{\wrt}{w.r.t.~}

\newcommand{\inarrC}[1]{\begin{array}{@{}c@{}}#1\end{array}}
\newcommand{\inpar}[1]{\left(\begin{array}{@{}l@{}}#1\end{array}\right)}
\newcommand{\inset}[1]{\left\{\begin{array}{@{}l@{}}#1\end{array}\right\}}
\newcommand{\inarr}[1]{\begin{array}{@{}l@{}}#1\end{array}}
\newcommand{\inarrII}[2]{\begin{array}{@{}l@{~~}||@{~~}l@{}}\inarr{#1}&\inarr{#2}\end{array}}
\newcommand{\inarrIII}[3]{\begin{array}{@{}l@{~~}||@{~~}l@{~~}||@{~~}l@{}}\inarr{#1}&\inarr{#2}&\inarr{#3}\end{array}}
\newcommand{\inarrIV}[4]{\begin{array}{@{}l@{~~}||@{~~}l@{~~}||@{~~}l@{~~}||@{~~}l@{}}\inarr{#1}&\inarr{#2}&\inarr{#3}&\inarr{#4}\end{array}}
\newcommand{\inarrV}[5]{\begin{array}{@{}l@{~~}||@{~~}l@{~~}||@{~~}l@{~~}||@{~~}l@{~~}||@{~~}l@{}}\inarr{#1}&\inarr{#2}&\inarr{#3}&\inarr{#4}&\inarr{#5}\end{array}}

%% \renewcommand{\comment}[1]{\color{teal}{~~\texttt{/\!\!/}\textit{#1}}}


\newcommand{\set}[1]{\{{#1}\}}
\newcommand{\sem}[1]{\llbracket #1 \rrbracket}
\newcommand{\pfn}{\rightharpoonup}
\newcommand{\suchthat}{\; | \;}
\newcommand{\N}{{\mathbb{N}}}
\newcommand{\dom}[1]{\textit{dom}{({#1})}}
\newcommand{\codom}[1]{\textit{codom}{({#1})}}
\newcommand{\before}[2]{{#1}_{#2}^\uparrow}
\newcommand{\after}[2]{{#1}_{#2}^\downarrow}
%\newcommand{\fv}[1]{fv{[{#1}]}}
\newcommand{\tup}[1]{{\langle{#1}\rangle}}
\newcommand{\nin}{\not\in}
\newcommand{\suq}{\subseteq}
\newcommand{\sqsuq}{\sqsubseteq}
\newcommand{\sqsu}{\sqsubset}
\newcommand{\sqslq}{\sqsupseteq}
\newcommand{\size}[1]{|{#1}|}
\newcommand{\block}[1]{\langle {#1}\rangle}
\newcommand{\true}{\top}
\newcommand{\maketil}[1]{{#1}\ldots{#1}}
\newcommand{\til}{\maketil{,}}
\newcommand{\cuptil}{\maketil{\cup}}
\newcommand{\uplustil}{\maketil{\uplus}}
\renewcommand*{\mathellipsis}{\mathinner{{\ldotp}{\ldotp}{\ldotp}}}
\newcommand{\rst}[1]{|_{#1}}
\newcommand{\imm}[1]{{#1}{\rst{\text{imm}}}}
\renewcommand{\succ}[2]{\text{succ}_{#1}(#2)}
\newcommand{\aite}[3]{(#1?#2,#3)}
%\newcommand{\defeq}{\mathrel{\stackrel{\mathsf{def}}{=}}}
\newcommand{\defeq}{\triangleq}
\newcommand{\powerset}[1]{\mathcal{P}({#1})}
\newcommand{\finpowerset}[1]{\mathcal{P}_{<\omega}({#1})}
\renewcommand{\implies}{\Rightarrow}


\colorlet{colorPO}{gray!60!black}
\colorlet{colorRF}{green!60!black}
\colorlet{colorMO}{orange}
\colorlet{colorFR}{purple}
\colorlet{colorECO}{red!80!black}
\colorlet{colorSYN}{green!40!black}
\colorlet{colorHB}{blue}
\colorlet{colorPPO}{magenta}
\colorlet{colorPB}{olive}
\colorlet{colorSBRF}{olive}
\colorlet{colorRMW}{olive!70!black}
\colorlet{colorRSEQ}{blue}
\colorlet{colorSC}{violet}
\colorlet{colorPSC}{violet}
\colorlet{colorREL}{olive}
\colorlet{colorCONFLICT}{olive}
\colorlet{colorRACE}{olive}
\colorlet{colorWB}{orange!70!black}
\colorlet{colorPSC}{violet}
\colorlet{colorSCB}{violet}
\colorlet{colorDEPS}{violet}

\tikzset{
   every path/.style={>=stealth},
   sb/.style={->,color=colorSB,thin,shorten >=-0.5mm,shorten <=-0.5mm},
   sw/.style={->,color=colorSYN,shorten >=-0.5mm,shorten <=-0.5mm},
   rf/.style={->,color=colorRF,dashed,,shorten >=-0.5mm,shorten <=-0.5mm},
   hb/.style={->,color=colorHB,thick,shorten >=-0.5mm,shorten <=-0.5mm},
   mo/.style={->,color=colorMO,dotted,very thick,shorten >=-0.5mm,shorten <=-0.5mm},
   no/.style={->,dotted,thick,shorten >=-0.5mm,shorten <=-0.5mm},
   fr/.style={->,color=colorFR,dotted,thick,shorten >=-0.5mm,shorten <=-0.5mm},
   deps/.style={->,color=colorDEPS,dotted,thick,shorten >=-0.5mm,shorten <=-0.5mm},
   rmw/.style={->,color=colorRMW,thick,shorten >=-0.5mm,shorten <=-0.5mm},
}

%% Orders
\newcommand{\na}{\mathtt{na}}
\newcommand{\pln}{\mathtt{pln}}
\newcommand{\rlx}{\mathtt{rlx}}
\newcommand{\rel}{{\mathtt{rel}}}
\newcommand{\acq}{{\mathtt{acq}}}
\newcommand{\con}{{\mathtt{con}}}
\newcommand{\acqrel}{{\mathtt{acqrel}}}
\newcommand{\sco}{{\mathtt{sc}}}
\newcommand{\sto}{{\mathtt{st}}}
\newcommand{\full}{{\mathtt{sy}}}
\newcommand{\ld}{{\mathtt{ld}}}
\newcommand{\isb}{{\mathtt{isb}}}

%% Event labels

\newcommand{\evlab}[4]{{#1}({#3},{#4})}
\newcommand{\evflab}[1]{{\flab}^{#1}}
\newcommand{\evulab}[4]{{\ulab}^{#1}({#2},{#3},{#4})}
\newcommand{\rlab}[3]{{\lR}^{#1}({#2},{#3})}
\newcommand{\erlab}[4]{{\lR}^{#1}({#2},{#3},{#4})}
\newcommand{\wlab}[3]{{\lW}^{#1}({#2},{#3})}
\newcommand{\flab}[1]{{\lF}(#1)}
%\newcommand{\ulab}[4]{{\lU}^{#1}({#2},{#3},{#4})}

\newcommand{\lE}{{\mathtt{E}}}
\newcommand{\lC}{{\mathtt{C}}}
%\newcommand{\lP}{{\mathit{P}}}
\newcommand{\lM}{{\mathtt{M}}}
\newcommand{\lS}{{\mathtt{S}}}
\newcommand{\lR}{{\mathtt{R}}}
\newcommand{\lW}{{\mathtt{W}}}
\newcommand{\lA}{{\mathtt{A}}}
\newcommand{\lQ}{{\mathtt{Q}}}
\newcommand{\lL}{{\mathtt{L}}}
\newcommand{\lU}{{\mathtt{RMW}}}
\newcommand{\lF}{{\mathtt{F}}}
\newcommand{\lRES}{{\mathtt{Res}}}
\newcommand{\lAT}{{\mathtt{At}}}
\newcommand{\lATR}{{\mathtt{AtR}}}

\newcommand{\lLAB}{{\mathtt{lab}}}
\newcommand{\lTID}{{\mathtt{tid}}}
\newcommand{\lTYP}{{\mathtt{typ}}}
\newcommand{\lLOC}{{\mathtt{loc}}}
\newcommand{\lMOD}{{\mathtt{mod}}}
\newcommand{\lVALR}{{\mathtt{val_r}}}
\newcommand{\lVALW}{{\mathtt{val_w}}}
\newcommand{\lELAB}{{\mathtt{elab}}}
\newcommand{\lAVALS}{{\mathtt{atvals}}}

%% Relations

\newcommand{\lX}{\mathtt{X}}
\newcommand{\lPO}{{\color{colorPO}\mathtt{po}}}
\newcommand{\lRF}{{\color{colorRF} \mathtt{rf}}}
\newcommand{\lRMW}{{\color{colorRMW} \mathtt{rmw}}}
\newcommand{\lMO}{{\color{colorMO} \mathtt{mo}}}
\newcommand{\lMOx}{{\color{colorMO} \mathtt{mo}}_x}
\newcommand{\lMOy}{{\color{colorMO} \mathtt{mo}}_y}
\newcommand{\lCO}{{\color{colorMO} \mathtt{co}}}
\newcommand{\lCOx}{{\color{colorMO} \mathtt{co}}_x}
\newcommand{\lCOy}{{\color{colorMO} \mathtt{co}}_y}
\newcommand{\lFR}{{\color{colorFR} \mathtt{rb}}}
\newcommand{\lFRx}{{\color{colorFR} \mathtt{rb}}_x}
\newcommand{\lFRy}{{\color{colorFR} \mathtt{rb}}_y}
\newcommand{\lECO}{{\color{colorECO} \mathtt{eco}}}
\newcommand{\lSBRF}{{\color{colorSBRF} \mathtt{sbrf}}}
\newcommand{\lRSEQ}{{\color{colorRSEQ}\mathtt{rseq}}}
\newcommand{\lSW}{{\color{colorSYN}\mathtt{sw}}}
\newcommand{\lASW}{{\color{colorSYN}\mathtt{asw}}}
\newcommand{\lSO}{{\color{colorSYN}\mathtt{so}}}
\newcommand{\lHB}{{\color{colorHB}\mathtt{hb}}}
%\newcommand{\lWB}{{\color{colorWB} \mathtt{wb}}}
\newcommand{\lDOB}{{\mathtt{dob}}}
\newcommand{\lBOB}{{\mathtt{bob}}}
\newcommand{\lAOB}{{\mathtt{aob}}}
\newcommand{\lOBS}{{\mathtt{obs}}}
\newcommand{\lEORD}{{\mathtt{eord}}}
\newcommand{\lTORD}{{\mathtt{tord}}}
\newcommand{\lSC}{{\color{colorSC}{\mathtt{sc}}}}

\newcommand{\lSCB}{{\color{colorSCB} \mathtt{scb}}}
\newcommand{\lPSC}{{\color{colorPSC} \mathtt{psc}}}
\newcommand{\lPSCB}{\lPSC_{\rm base}}
\newcommand{\lPSCF}{\lPSC_\lF}
\newcommand{\lCONFLICT}{{\color{colorCONFLICT} \mathtt{conflict}}}
\newcommand{\lRACE}{{\color{colorRACE} \mathtt{race}}}
\newcommand{\lNARACE}{{\color{colorRACE} \mathtt{na-race}}}

\newcommand{\lmakeW}[1]{\mathtt{w}#1}
\newcommand{\lWFR}{\lmakeW{\lFR}}
\newcommand{\lWECO}{\lmakeW{\lECO}}
\newcommand{\lWSCB}{\lmakeW{\lSCB}}
\newcommand{\lWPSCB}{\lmakeW{\lPSCB}}
\newcommand{\lWPSCF}{\lmakeW{\lPSCF}}
\newcommand{\lWPSC}{\lmakeW{\lPSC}}
\newcommand{\lWB}{\lmakeW{\lMO}}

\newcommand{\lDEPS}{{{\color{colorDEPS}\mathtt{deps}}}}
\newcommand{\lCTRL}{{{\color{colorDEPS}\mathtt{ctrl}}}}
\newcommand{\lCTRLISYNC}{{{\color{colorDEPS}\mathtt{ctrl_{isync}}}}}
\newcommand{\lDATA}{{{\color{colorDEPS}\mathtt{data}}}}
\newcommand{\lADDR}{{{\color{colorDEPS}\mathtt{addr}}}}

\newcommand{\lmakeE}[1]{#1\mathtt{e}}
\newcommand{\lRFE}{\lmakeE{\lRF}}
\newcommand{\lCOE}{\lmakeE{\lCO}}
\newcommand{\lFRE}{\lmakeE{\lFR}}
\newcommand{\lMOE}{\lmakeE{\lMO}}
\newcommand{\lmakeI}[1]{#1\mathtt{i}}
\newcommand{\lRFI}{\lmakeI{\lRF}}
\newcommand{\lCOI}{\lmakeI{\lCO}}
\newcommand{\lFRI}{\lmakeI{\lFR}}

\newcommand{\Tid}{\mathsf{Tid}}
\newcommand{\Loc}{\mathsf{Loc}}
\newcommand{\Val}{\mathsf{Val}}
\newcommand{\Lab}{\mathsf{Lab}}
\newcommand{\Mod}{\mathsf{Mod}}
\newcommand{\Modr}{\mathsf{Mod}_{\lR}}
\newcommand{\Modw}{\mathsf{Mod}_{\lW}}
\newcommand{\Modf}{\mathsf{Mod}_{\lF}}
\newcommand{\Modrmw}{\mathsf{Mod}_{\lU}}

\newcommand{\reorder}[1]{\mathsf{Reorder}({#1})}
%\newcommand{\deorder}[1]{\mathsf{Deorder}({#1})}
\newcommand{\remove}[1]{\mathsf{RemoveWR}({#1})}
\newcommand{\reorderWR}[1]{\mathsf{ReorderWR}({#1})}

\newcommand{\travConfigStep}{\rightarrow_{\rm TC}}
\newcommand{\ARM}{\ensuremath{\mathsf{ARM}}\xspace}
\newcommand{\correctTmap}{{\rm \bf correct\text{-}tmap}}
\newcommand{\rmwCovering}{{\rm \bf rmw\text{-}covering}}
\newcommand{\invWsc}{{\rm \bf write\text{-}sc\text{-}covering}}
\newcommand{\invViewRel}{\mathcal{I}_{\rm \bf view\text{-}rel}}
\newcommand{\setView}{{\rm set\text{-}view}}
\newcommand{\domView}{{\rm dom\text{-}view}}
\newcommand{\scSet}{{\rm sc\text{-}set}}
\newcommand{\scSetF}{{\rm sc\text{-}set\text{-}fun}}
\newcommand{\scSynchSet}{{\rm sc\text{-}synch}}
\newcommand{\acqSynchSet}{{\rm acq\text{-}synch}}
\newcommand{\relSynchFSet}{{\rm rel\text{-}synchf}}
\newcommand{\acqSet}{{\rm acq\text{-}set}}
\newcommand{\acqSetF}{{\rm acq\text{-}set\text{-}fun}}
\newcommand{\curSet}{{\rm cur\text{-}set}}
\newcommand{\curSetF}{{\rm cur\text{-}set\text{-}fun}}
\newcommand{\relSetFF}{{\rm rel\text{-}setf\text{-}fun}}
\newcommand{\msgSetF}{{\rm msg\text{-}set\text{-}fun}}
\newcommand{\msgSet}{{\rm msg\text{-}set}}
\newcommand{\relFunSet}{{\rm rel\text{-}setf}}
\newcommand{\lessView}{{\sf \bf less\text{-}view}}
\newcommand{\TSf}{\ensuremath{\mathcal{TS}}\xspace}
\newcommand{\TS}{\ensuremath{\mathit{TS}}\xspace}
\newcommand{\View}{\ensuremath{\mathcal{V}}\xspace}
\newcommand{\SCview}{\ensuremath{\mathcal{S}}\xspace}
\newcommand{\letdef}[2]{\kw{let} \; #1 \defeq #2 \; \kw{in}}
\newcommand{\nextset}{{\sf Next}}
\newcommand{\coverable}{{\sf Coverable}}
\newcommand{\promisable}{{\sf Promisable}}
%% \newcommand{\addToMemory}[2]{#1 \buildrel{\text{A}}\over\hookleftarrow #2}
\newcommand{\addToMemory}[2]{#1 \cup \{#2\}}

\newcommand{\stepa}{\leadsto_{\ARM}}
\newcommand{\stepp}{\promStepgen{}}
%% \leadsto_{\Promise}}
\newcommand{\stepptid}{\leadsto_{\Promise \; tid}}
\newcommand{\lISB}{\mathtt{F^{\isb}}}
\newcommand{\lDMB}{\mathtt{DMB}}
\newcommand{\lDMBSY}{\flab{\SY}}
\newcommand{\lDMBLD}{\flab{\LD}}
\newcommand{\lDMBST}{\flab{\ST}}
\newcommand{\SY}{\mathtt{sy}}
\newcommand{\LD}{\mathtt{ld}}
\newcommand{\ST}{\mathtt{st}}
\newcommand{\removeCR}[1]{\mathsf{RemoveCR}({#1})}
%% \spnewtheorem{defn}{Definition}{\bfseries}{}
%% \crefname{defn}{Definition}{Definitions}
\newcommand{\Timestamp}{\mathtt{Time}}
\newcommand{\TimestampMap}{\mathit{TimeMap}}

\newcommand{\Prog}{\ensuremath{Prog}\xspace}
\newcommand{\ProgARM}{\ensuremath{Prog_\ARM}\xspace}

% Promise definitions
\newcommand{\msg}[4]{\tup{#1:#2@#3,#4}}
\newcommand{\Promise}{\ensuremath{\mathsf{Promise}}\xspace}
%% \newcommand\PromSet{\mathit{promises}}
\newcommand\PromSet{\mathtt{P}}
\newcommand\PromState{\mathit{st}}
\newcommand{\V}{V}
\newcommand{\view}{view}
%% \newcommand{\viewCur}{view_{\sf cur}}
%% \newcommand{\viewAcq}{view_{\sf acq}}
%% \newcommand{\viewRel}{view_{\sf rel}}
\newcommand{\viewCur}{{\sf cur}}
\newcommand{\viewAcq}{{\sf acq}}
\newcommand{\viewRel}{{\sf rel}}
\newcommand{\viewfRel}{{\sf rel}}
\newcommand{\loc}{\ell}
\newcommand{\val}{v}
%% \newcommand{\tstamp}{\tau}
\newcommand{\tstampRange}[2]{(#1,#2]}
\newcommand{\simRel}{\mathcal{I}}
\newcommand{\invMemOne}{\mathcal{I}_{\rm \bf mem1}}
\newcommand{\invMemTwo}{\mathcal{I}_{\rm \bf mem2}}
\newcommand{\invMemThree}{\mathcal{I}_{\rm \bf mem3}}
\newcommand{\invView}{\mathcal{I}_{\rm \bf view}}
\newcommand{\invState}{\mathcal{I}_{\rm \bf state}}
\newcommand{\pstate}{\sigma}
\newcommand{\labelF}{{\rm label}}
\newcommand{\ltsReq}{{\rm lts\text{-}req}}
\newcommand{\maxF}{{\rm max}}
\newcommand{\compileF}{{\rm compile}}
\newcommand{\compileMod}{{\rm mod\text{-}compile}}
\newcommand{\compileFenceMod}{{\rm fence\text{-}mod\text{-}compile}}
%% \newcommand{\writeLbl}[3]{W \; (#1, #2, #3)}
\newcommand{\writeLbl}[2]{W \; (#1, #2)}
\newcommand{\writeBlankLbl}{W \; \_}
\newcommand{\fenceLbl}[1]{F \; #1}
%% \newcommand{\readLbl}[3]{R \; (#1, #2, #3)}
\newcommand{\readLbl}[2]{R \; (#1, #2)}
\newcommand{\readBlankLbl}{R \; \_}
\newcommand{\casLbl}[5]{U \; (#1, #2, #3, #4, #5)}
\newcommand{\svaleq}{\asymp}
\newcommand{\strans}[1]{\xrightarrow{#1}}

%% \newcommand\readWriteTau{\textsf{access-tmap}}
%% \newcommand\readsView{\textsf{reads-view}}

%\newcommand{\RC}{\ensuremath{\mathsf{RC11}}\xspace}
%\newcommand{\WRC}{\ensuremath{\mathsf{WRC11}}\xspace}

\newcounter{mylabelcounter}

\makeatletter
\newcommand{\labelAxiom}[2]{%
\hfill{\normalfont\textsc{(#1)}}\refstepcounter{mylabelcounter}
\immediate\write\@auxout{%
  %% \string\newlabel{#2}{{1}{\thepage}{{#1}}{mylabelcounter.\number\value{mylabelcounter}}{}}
  \string\newlabel{#2}{{\unexpanded{\normalfont\textsc{#1}}}{\thepage}{{\unexpanded{\normalfont\textsc{#1}}}}{mylabelcounter.\number\value{mylabelcounter}}{}}
}%
}
\makeatother

\newcommand{\squishlist}[1][$\bullet$]{%[$\tinybullet$]{
 \begin{list}{#1}
  { \setlength{\itemsep}{0pt}
     \setlength{\parsep}{0pt}
     \setlength{\topsep}{1pt}
     \setlength{\partopsep}{0pt}
     \setlength{\leftmargin}{1.2em}
     \setlength{\labelwidth}{0.5em}
     \setlength{\labelsep}{0.4em} } }
\newcommand{\squishend}{
  \end{list}  }

\newcommand{\app}[1]{{\color{blue}\textbf{ANTON: #1}}}

\newcommand{\optarrow}{\rightsquigarrow}

%% Old defs.tex content

%% \newcommand{\excl}[1]{{\color{purple} #1}}
\newcommand{\ifext}[2]{\ifdefined\extflag{#1}\else{#2}\fi}

\newcommand{\Reads}{{\sf Reads}}
%% \newcommand{\Reads}{\{ \angled{\tId, \cpath, \loc} \mid \tId, \cpath, \loc \}}

\newcommand{\excl}[1]{}
\newcommand\tT{\mathbf{t}}

\newcommand{\ListOf}[1]{\mathit{List}\;{#1}}
\newcommand{\Label}{\mathit{Label}}
\newcommand{\lab}{\mathit{lab}}
\newcommand{\Pset}{\mathbb{P}}
\newcommand{\dep}{\mathit{dep}}
\newcommand{\regs}{\mathrm{regs}}

\newcommand\viewf{\textbf{\sf viewf}}
\newcommand{\nthf}{\mathrm{nth}}

\newcommand{\deltaToView}{\delta\textup{\sf -to-view}}
\newcommand{\deltaHmap}{\textup{\sf comb-time}}
\newcommand{\invDeltaDefOne}{\inv_{\delta\textup{\sf -con-1}}}
\newcommand{\invDeltaDefTwo}{\inv_{\delta\textup{\sf -con-2}}}
\newcommand{\invDeltaDefThree}{\inv_{\delta\textup{\sf -con-3}}}
\newcommand{\invDeltaDefFour}{\inv_{\delta\textup{\sf -con-4}}}

\newcommand{\tIdState}{\textup{\sf thread-state}}
\newcommand{\lastCommittedWrite}{\textup{\sf last-write-com}}

\newcommand\certifiableTid{\mathsf{certifiable}_{\tId}}
\newcommand\certifiable{\mathsf{certifiable}}

\newcommand\lastInstr[1]{#1.\textup{\sf last}}

\newcommand\length{\textup{\sf length}}
\newcommand\prefix{\textup{\sf prefix}}

\newcommand{\ExclReadType}{\mathit{er}}
\newcommand{\ExclWriteType}{\mathit{ew}}
\newcommand{\ComWriteState}{\mathit{im}}

\newcommand\InMemory{im}
\newcommand{\NotInMemory}{\textup{\sf no-mem}}
\newcommand{\IssuedToMemory}{\textup{\sf mem}}
\newcommand{\ExclIssuedToMemory}{\textup{\sf excl-mem}}

\newcommand{\readInst }[3][\ExclReadType]{#2 \;:=_{\excl{#1}}\;[#3]}
\newcommand{\fenceInst}[1]{\fence{#1}}
\newcommand{\dmbSY}{\fenceInst{\SY}}
\newcommand{\dmbLD}{\fenceInst{\LD}}

\newcommand{\readInstParam}[3]{#2 \;:=_{#1}\;[#3]}
\newcommand{\writeInstParam}[3]{[#2] \;:=_{#1}\;#3}
\newcommand{\repeatReadInst}[2]{\kw{repeat} \;[#2]_{#1}\; \kw{end}}
\newcommand{\casInstParam}[6]{#3\;:=\;\kw{cas}_{#1,#2}(#4, #5, #6)}

\newcommand\ifGoto{\textsf{if}-\textsf{goto}}
\newcommand{\ifGotoInst}[2]{\textsf{if} \; #1 \; \textsf{goto} \; #2}
\newcommand{\writeInst}[2]{[#1]\;:=\;#2}
\newcommand{\writeExclInst}[3]{\textup{\sf atomic-write}(#1, [#2], #3)}
\newcommand{\assignInst}[2]{#1\;:=\;#2}
\newcommand{\repeatInst}[1]{\kw{repeat} \;#1\; \kw{end}}

\newcommand{\restrict}[2]{#1{\restriction_{#2}}}

%% \newcommand{\writeRI}[3][, \ExclType]{#2:#3\excl{, #1}}
\newcommand{\writeRI}[3][]{#2:#3\excl{#1}}

%% \newcommand\ARM{\mathrm{ARM}}
\newcommand\ARMt{\mathrm{ARM}{+}\tau}
%% \newcommand\Promise{\mathrm{Promise}}

\newcommand{\tstamp}[1]{\mbox{\small\color{brown!60!black}\bf{#1}}}

\newcommand{\OrdPrevRequest}{\textup{\sf prev-Ord-req}}

\newcommand{\uniqueTimeLoc}{\textup{\sf uniq-time-loc}}
\newcommand{\ordPrevRequest}{\textup{\sf ord-prev-req}}
\newcommand{\coherentThread}{\textup{\sf coherent-thread}}
\newcommand{\timeRangeCondition}{\textup{\sf time-range}}
\newcommand{\sameMemory}{\textup{\sf same-memory}}
\newcommand{\instToLbl}{\textup{\sf inst-to-lbl}}
\newcommand{\cmdsToLbls}{\textup{\sf cmds-to-lbls}}
\newcommand{\cmdsToLblsAux}{\textup{\sf cmds-to-lbls-aux}}

\newcommand{\instToVrtx}{\textup{\sf inst-to-vertex}}
\newcommand{\cmdsToVrtxs}{\textup{\sf cmds-to-vertices}}
\newcommand{\cmdsToVrtxsAux}{\textup{\sf cmds-to-vertices-aux}}

\newcommand{\prevInstrCommitted}{\textup{\sf prev-instr-committed}}
\newcommand{\prevReadsCommitted}{\textup{\sf prev-reads-committed}}
\newcommand{\prevFencesCommitted}{\textup{\sf prev-fences-committed}}
\newcommand{\prevBrCommitted}{\textup{\sf prev-branches-committed}}
\newcommand{\prevBrFencesCommitted}{\textup{\sf prev-branches-and-fences-committed}}
\newcommand{\prevCmdDetermined}{\textup{\sf prev-fully-determined}}
\newcommand{\prevNoRestart}{\textup{\sf no-prev-restartable-reads-from-loc}}
\newcommand{\prevExclCommitted}{\textup{\sf prev-excl-to-loc-committed}}
\newcommand{\noFollowingWcom}{\textup{\sf no-following-com-writes-to-loc}}

%% \newcommand\reorderableRel{\leftrightarrow} 
%% \newcommand\notReorderableRel{\not \leftrightarrow} 
\newcommand\reorderableRel[2]{#1 \hookrightarrow #2} 
\newcommand\notReorderableRel[2]{#1 \not \hookrightarrow #2} 
%% \newcommand\ReorderingFunction{is\_reorderable}
%% \newcommand{\checkReorderings}[1]{\textup{\sf check-reorderings}(#1)}
\newcommand{\checkReorderings}[1]{#1 \setminus {\reorderableRel{}{}}}

\newcommand{\deleteUpdReads}{\textup{\sf delete-upd-reads}}
\newcommand{\acceptRequest}{\textup{\sf accept-request}}
\newcommand{\acceptExclWrite}{\textup{\sf accept-excl-write}}

\newcommand{\readsBetweenCommitted}{\textup{\sf reads-in-between-committed}}
\newcommand{\noWritesBetween}{\textup{\sf no-writes-to-loc-in-between}}
\newcommand{\noDiffReadsBetween}{\textup{\sf no-different-write-reads-in-between}}
\newcommand{\samePropagated}{\textup{\sf propagated-to-same-threads}}
\newcommand{\fullyPropagated}{\textup{\sf fully-propagated}}
\newcommand{\getNewTapeCell}{\textup{\sf get-new-tapecell}}
\newcommand{\prevReadFromOther}{\textup{\sf prev-read-from-other-write}}
%\newcommand{\nextPath}{\textup{\sf next-path}}
\newcommand{\tapeUpdRestart}{\textup{\sf tape-upd-restart}}
\newcommand{\doesntPreventExcl}{\textup{\sf doesnt-prevent-excl}}
\newcommand{\tapeUpdWcom}{\textup{\sf tape-upd-Wcom}}
\newcommand{\tapeUpdRsat}{\textup{\sf tape-upd-Rsat}}
\newcommand{\tapeUpdIf}{\textup{\sf tape-upd-IfGoto}}
\newcommand{\noExclInBetween}{\textup{\sf no-excl-in-between}}
\newcommand{\getLoc}{\textup{\sf get-loc}}

%% PROBLEMS WITH: \C, \next, \a
  \newcommand\armStepWriteCommit{\armStepgen{\transenv{Write commit} \; \tId \; \cpath \; x \; \stval \; \tau}}
  \newcommand\armStepWriteCommitLoc{\armStepgen{\transenv{Write commit} \; \tId \; \cpath \; \loc \; \stval \; \tau}}
  \newcommand\armStepWriteCommitPrime{\armStepgen{\transenv{Write commit} \; \tId' \; \cpath' \; y \; \stval' \; \tau}}
  \newcommand\armStepWriteCommitP{\armStepPgen{\transenv{Write commit} \; \tId \; \cpath \; x \; \stval}}
  \newcommand\armStepWriteCommitPLoc{\armStepPgen{\transenv{Write commit} \; \tId \; \cpath \; \loc \; \stval}}
  \newcommand\armStepWriteCommitPrimeP{\armStepPgen{\transenv{Write commit} \; \tId' \; \cpath' \; y \; \stval'}}


  \newcommand\promStepBranch{\promStepgen{\transenv{Branch commit} \; \tId}}
  \newcommand\promTStepBranch{\promTStepgen{\transenv{Branch commit}}}

  \newcommand\promStepAcquire{\promStepgen{\transenv{Acquire fence commit} \; \tId}}
  \newcommand\promTStepAcquire{\promTStepgen{\transenv{Acquire fence commit}}}
  \newcommand\promStepRelease{\promStepgen{\transenv{Release fence commit} \; \tId}}
  \newcommand\promTStepRelease{\promTStepgen{\transenv{Release fence commit}}}

  \newcommand\promStepRead{\promStepgen{\transenv{Read from memory} \; \tId \; \writeEvt{x}{\stval}{\tau}{\R}}}
  \newcommand\promTStepRead{\promTStepgen{\transenv{Read from memory} \; \writeEvt{x}{\stval}{\tau}{\R}}}
  \newcommand\promTStepReadLoc{\promTStepgen{\transenv{Read from memory} \; \writeEvt{\loc}{\stval}{\tau}{\R}}}

  \newcommand\promStepPromise{\promStepgen{\transenv{Promise write} \; \tId \; \writeEvt{x}{\stval}{\tau}{\R}}}
  \newcommand\promStepPromiseRPrime{\promStepgen{\transenv{Promise write} \; \tId \; \writeEvt{x}{\stval}{\tau}{\R'}}}
  \newcommand\promTStepPromise{\promTStepgen{\transenv{Promise write} \; \writeEvt{x}{\stval}{\tau}{\R}}}
  \newcommand\promTStepPromiseLoc{\promTStepgen{\transenv{Promise write} \; \writeEvt{\loc}{\stval}{\tau}{\R}}}

  \newcommand\promStepFulfill{\promStepgen{\transenv{Fulfill promise} \; \tId \; \writeEvt{x}{\stval}{\tau}{\R}}}
  \newcommand\promTStepFulfill{\promTStepgen{\transenv{Fulfill promise} \; \writeEvt{x}{\stval}{\tau}{\R}}}
  \newcommand\promTStepFulfillLoc{\promTStepgen{\transenv{Fulfill promise} \; \writeEvt{\loc}{\stval}{\tau}{\R}}}

  \newcommand\promStepAssign{\promStepgen{\transenv{Local variable assignment} \; \tId}}
  \newcommand\promTStepAssign{\promTStepgen{\transenv{Local variable assignment}}}

  \newcommand\promStepNop{\promStepgen{\transenv{Execution of $\nop$} \; \tId}}
  \newcommand\promTStepNop{\promTStepgen{\transenv{Execution of $\nop$}}}

  \newcommand{\tapeToCertificate}{\mathsf{tape}\textsf{-}\mathsf{to}\textsf{-}\mathsf{certificate}}
  %% \newcommand{\Timestamp}{\mathit{Time}}

  \newcommand{\graybox}[1]{$\colorbox{gray!20}{$#1\!$}$}

  %\newcommand\Rarm{\R_{\ARM}}
  \newcommand\Rarm{\mathsf{view}_{\ARM}}
  \newcommand\transenv[1]{\textcolor{darkgray}{\textup{\textsf{\textbf{\mathversion{bold}#1}}}}}

  \newcommand\promMessage{\mathit{msg}}
  \newcommand\promMessageSet{\mathit{Msg}}

  \newcommand\FtypeARM{\mathit{fmod}_{\ARM}}
  %% \newcommand\FtypeProm{\mathit{ftype}_{\Promise}}
  \newcommand\FtypeProm{\mathit{fmod}}

  \newcommand\RtypeProm{\mathit{rtype}_{\Promise}}
  \newcommand\WtypeProm{\mathit{wtype}_{\Promise}}

  %% \newcommand\V{\mathit{V}}
  \newcommand\R{\mathit{view}}
  %% \newcommand\View{\mathit{View}}
  \newcommand\Rsc{\R_{\mathrm{sc}}}
  \newcommand\Rcur{\R_{\mathrm{cur}}}
  \newcommand\Racq{\R_{\mathrm{acq}}}
  \newcommand\Rrel{\R_{\mathrm{rel}}}
  \newcommand\Rna{\R_{\mathrm{na}}}
  \newcommand\Rwrite{\R_{\mathrm{write}}}
  
  \newcommand\lessUpToDelta[3]{#2 <_{#1} #3}

  \newcommand\StmtARM{S}
  \newcommand\StmtProm{S_{\Promise}}


  %% \newcommand\C{C}
  \newcommand\Cf{\mathit{Prog}}%\textsf{Prog}
  \newcommand\Carm{{cmds}}
  \newcommand\CARM{{Cmds}}
  \newcommand\Cfarm{\Cf}
  \newcommand\Cprom{{cmds}}
  \newcommand\Cfprom{\Cf}
  %% \newcommand\Carm{\textsf{C}_{\ARM}}
  %% \newcommand\Cfarm{\Cf_{\ARM}}
  %% \newcommand\Cprom{\textsf{C}_{\Promise}}
  %% \newcommand\Cfprom{\Cf_{\Promise}}
  
  %% \newcommand\PromSet{\mathit{promises}}
  %% \newcommand\PromState{\mathit{st}}

  \newcommand\scAcqRead[2]{#1 := [#2]_{\sf LDAR}}
  \newcommand\scRelWrite[2]{[#1]_{\sf STLR} := #2}

  \newcommand\acqFence{\fence{\sf acquire}}
  \newcommand\relFence{\fence{\sf release}}
  \newcommand\scFence{\fence{\sf sc}}
  \newcommand\fence[1]{\mathsf{fence}({#1})}

  \newcommand\syFence{\fence{\sf sy}}
  \newcommand\ldFence{\fence{\sf ld}}
  
  \newcommand\reqInfoRead[1]{{\sf rd} \; #1}
  \newcommand\reqInfoWrite[2]{{\sf wr} \; #1:#2}
  \newcommand\reqInfoFence{{\sf dmb}}

  \newcommand\stRequest[3]{\angled{#1, #2, #3}}
  \newcommand\stRequestWrite[4]{\angled{#1, #2, \reqInfoWrite{#3}{#4}}}
  \newcommand\stRequestRead[3]{\angled{#1, #2, \reqInfoRead{#3}}}
  \newcommand\stRequestFence[2]{\angled{#1, #2, \reqInfoFence}}

  \newcommand\moTau{\textsf{tedges}}
  \newcommand\invTT{\textsf{inv}}

  % \newcommand\opstau{op\_\loc\_\tau}
  \newcommand\opstau{\textsf{com-writes-time}}
  \newcommand\readsSatisfiedR{\textsf{sat-reads-view}}
  \newcommand\readsCommittedR{\textsf{com-reads-view}}

  \newcommand\hmap{\mathit{H}}
  \newcommand\tmap{\mathit{H}_{\tau}}
  \newcommand\rmap{\mathit{H}_{\mathsf{view}}}
  \newcommand\omap{\mathit{H}_{\le}}

  \newcommand\StateARM{\mathsf{State}_{\ARM}}
  \newcommand\StateARMtau{\mathsf{State}_{\ARMt}}
  \newcommand\StateProm{\mathsf{State}_{\Promise}}
  \newcommand\TStateProm{\mathsf{TState}_{\Promise}}

  \newcommand\armStep{\xrightarrow[\ARMt]{}}
  \newcommand\armStepl{\xrightarrow[\ARMt]{}}
  \newcommand\armStepgen[1]{\xrightarrow[\ARMt]{#1}}
  \newcommand\armStepP{\xrightarrow[\ARM]{}}
  \newcommand\armStepPgen[1]{\xrightarrow[\ARM]{#1}}

  \newcommand\armStepPrSat{\armStepPgen{\transenv{Read satisfy} \; \tId \; \cpath \; \tId' \; \cpath' \; x \; \stval}}
  \newcommand\armStepPrSatLoc{\armStepPgen{\transenv{Read satisfy} \; \tId \; \cpath \; \tId' \; \cpath' \; \loc \; \stval}}
  \newcommand\armStepRSat{\armStepgen{\transenv{Read satisfy} \; \tId \; \cpath \; \tId' \; \cpath' \; x \; \stval}}
  \newcommand\armStepPrSatFail{\armStepPgen{\transenv{Read satisfy (fail)} \; \tId \; \cpath \; \tId' \; \cpath' \; x \; \stval}}
  \newcommand\armStepPrSatFailLoc{\armStepPgen{\transenv{Read satisfy (fail)} \; \tId \; \cpath \; \tId' \; \cpath' \; \loc \; \stval}}
  \newcommand\armStepRSatFail{\armStepgen{\transenv{Read satisfy (fail)} \; \tId \; \cpath \; \tId' \; \cpath' \; x \; \stval}}
  \newcommand\armStepPrInFlightSat{\armStepPgen{\transenv{Read satisfy from in-flight write} \; \tId \; \cpath \; \cpath' \; x \; \stval}}
  \newcommand\armStepPrInFlightSatLoc{\armStepPgen{\transenv{Read satisfy from in-flight write} \; \tId \; \cpath \; \cpath' \; \loc \; \stval}}
  \newcommand\armStepRInFlightSat{\armStepgen{\transenv{Read satisfy from in-flight write} \; \tId \; \cpath \; \cpath' \; x \; \stval}}

  \newcommand\promStep{\xrightarrow[\Promise]{}}
  \newcommand\promStepl{\xrightarrow[\Promise]{}}
  \newcommand\promStepgen[1]{\xrightarrow[\Promise]{#1}}
  \newcommand\promTStepgen[1]{\xrightarrow[\Promise \; \tId]{#1}}
  \newcommand\promTStep{\xrightarrow[\Promise \; \tId]{}}

  \newcommand\semState[2]{|[#1|] ^{#2}}
  \newcommand\semf[2]{|[#1|] ^{#2}}
  \newcommand\semfcom[2]{|[#1|] ^{#2}_\mathsf{com}}
  \newcommand\sembr[1]{|[#1|]}
  \newcommand\PreExecutions{{\rm PreExecs}}

  \newcommand\textIf{\text{\underline{if}} \;}
  \newcommand\textElif{\text{\underline{elif}} \;}
  \newcommand\textElse{\text{\underline{else}} \;}
  \newcommand\textThen{\text{\underline{then}} \;}
  \newcommand\textLet{\text{\underline{let}} \;}
  \newcommand\textIn{\text{\underline{in}}}

  %% \newcommand\nextPathCom[3]{nextPath_{com}(#1, #2, #3)}
  \newcommand\nextPathCom[3]{\mathsf{next}\textsf{-}\mathsf{path}(#1, #2, #3)}
  \newcommand\nextPathProm{\mathsf{next}\textsf{-}\mathsf{path}_{\Promise}}

  \newcommand\comShift{\phantom{{}_\mathsf{com}}}

  \newcommand\extendExpr{\expr\_extend}

  \newcommand\regst{\mathsf{regf}}
  \newcommand\regstcom{\mathsf{regf}_\mathsf{com}}
  \newcommand\regf{\mathit{regf}}
  
  \newcommand\Mpop{\mathit{M}_{\mathrm{POP}}}
  %% \newcommand\Mprom{\mathit{M}_{\mathrm{Promise}}}
  \newcommand\Mprom{\mathit{M}}
  \newcommand{\Mcomp}[3]{\angled{#1, #2, #3}}


  \newcommand\Request{\textit{req}}
  \newcommand\RequestSet{\mathit{ReqSet}}
  \newcommand\RequestInfo{\mathit{reqinfo}}
  \newcommand\RequestInfoSet{\mathit{ReqInfoSet}}
  \newcommand\Evt{\mathit{Evt}}
  \newcommand\Ord{\mathit{Ord}}
  \newcommand\Prop{\mathit{Prop}}
  \newcommand\ExclMap{\mathit{Excl}}

  \newcommand\IssuingOrder{\mathit{iord}}
  \newcommand\IssuingOrderf{\mathit{iordf}}

  \newcommand\Issued[1]{\mathsf{requested} \; #1}
  \newcommand\Satisfied{\mathsf{sat}}
  \newcommand\SatisfiedInFlight{\mathsf{inflight}}
  \newcommand\Committed{\mathsf{com}}
  \newcommand\Plain{\mathsf{pln}}
  \newcommand\Exclusive{\mathsf{excl}}
  \newcommand\None{\mathsf{none}}
  \newcommand\Any{\mathsf{any}}

  %% \newcommand\LD{\mathsf{LD}}
  %% \newcommand\SY{\mathsf{SY}}
  %% \newcommand\ST{\mathsf{ST}}

  \newcommand\tId{\mathit{tid}}
  %% \newcommand\Tid{\mathit{Tid}}

  \newcommand\Tape{\mathit{Tape}}
  \newcommand\tape{\mathit{tape}}
  \newcommand\tapef{\mathit{tapef}}
  \newcommand\TapeCell{\mathit{TapeCell}}
  \newcommand\tapeCell{\mathit{tapecell}}
  \newcommand\Taken{\mathsf{taken}}
  \newcommand\Ignored{\mathsf{ignored}}

  \newcommand\cpath{\mathit{path}}
  \newcommand\cpathSY{\cpath^{\SY}}
  \newcommand\cpathLD{\cpath^{\LD}}
  \newcommand\cpathLDSY{\cpath^{\LD\SY}}
  \newcommand\Path{Path}
  
  \newcommand\tapeRead[1]{\textsf{R} \; #1}
  \newcommand\tapeFence[2]{\textsf{F} \; #1 \; #2}
  \newcommand\tapeWrite[1]{\textsf{W} \; #1}
  \newcommand\tapeIfGoto[2]{\textsf{If} \; #1 \; #2}
  \newcommand\tapeNop{\textsf{Nop}}
  \newcommand\tapeAssign{\textsf{Assign}}

  \newcommand\tapeSatisfied[2]{\Satisfied \; #1 \; #2}
  \newcommand\satisfiedState{\mathit{sat\text{-}state}}

  \newcommand\tapePending[2]{\textsf{pending} \; #1 \; #2}
  \newcommand\tapeWriteCommitted[3]{\Committed \; #1 \; #2 \; #3}

  \newcommand\Fstate{\mathit{st}_{\textup{\rm fence}}}
  \newcommand\Rstate{\mathit{st}_{\textup{\rm read}}}
  \newcommand\Wstate{\mathit{st}_{\textup{\rm write}}}
  \newcommand\IfState{\mathit{st}_{\textup{\rm ifgoto}}}

  \newcommand\locvar{\iota}
  %% \newcommand\loc{\ell}
  %% \newcommand\Loc{Loc}
  \newcommand\Reg{\mathit{Reg}}
  \newcommand\reg{\mathit{reg}}
  %% \newcommand\val{\mathit{val}}
  \newcommand\stval{\mathit{val}}
  \newcommand\Stval{\mathit{Val}}
  \newcommand\expr{\mathit{expr}}
  \newcommand\Expr{\mathit{Expr}}
  \newcommand\z{\mathit{k}}

  \newcommand\dmb{\textsf{dmb}}
  \newcommand\nop{\textsf{nop}}

  \newcommand\ImmediateEdge{ImmediateEdge}
   
  \newcommand\armState[1]{\ARM_{state}(#1)}
  \newcommand\angled[1]{\langle #1 \rangle}
  %% \newcommand\TSfprom{TSf}
  %% \newcommand\TSfprom{TSf_{\Promise}}
  %% \newcommand\TS{TS_{\Promise}}
  %% \newcommand\TSfprom{\textsf{tsf}}
  %% \newcommand\TS{\textsf{ts}}
  %% \newcommand\TSfprom{\mathit{tsf}}
  %% \newcommand\TS{\mathit{ts}}

  %% \newcommand\M{M}
  \newcommand\e{e}
  \newcommand\w{w}
  %% \newcommand\dom[1]{\mathsf{dom}(#1)}
  \newcommand\taumapping{\mathsf{map}_{\tau}}

  \newcommand\Nop{Nop}
  \newcommand\Write[1]{Write \; #1}
  \newcommand\WritePending[1]{\Write(\Pending \; #1)}
  \newcommand\Read[1]{Read \; #1}
  \newcommand\ReadIssued[1]{Read \; (\Issued \; #1)}
  \newcommand\ReadSatisfied[4]{Read \; (\Satisfied \; #1 \; #2 \; #3 \; #4)}
  %% \newcommand\next[2]{next(#1, #2)}
  \newcommand\nextPath[2]{\mathsf{next}\textsf{-}\mathsf{path}(#1, #2)}
  \newcommand\lastIndex{\mathsf{last}\textsf{-}\mathsf{index}}
  \newcommand\last[1]{last(#1)}
  \newcommand\lastSY{\mathsf{last}\SY}
  \newcommand\lastCF{\mathsf{lastCF}}
  \newcommand\lastLD{\mathsf{last}\LD}
  \newcommand\lastLDSY{\mathsf{last}\LD\SY}
  \newcommand\Fence[2]{Fence \; #1 \; #2}
  \newcommand\FenceSY[1]{\Fence{#1}{SY}}
  \newcommand\FenceLD[1]{\Fence{#1}{LD}}
  \newcommand\IfGoto[2]{IfGoto \; #1 \; #2}
  \newcommand\IfGotoK[1]{\IfGoto{#1}{k}}
  \newcommand\tick{\ding{51}} % ✓
  \newcommand{\tickP}{\tick}
  \newcommand{\tickPP}{\tick}

  \newcommand\fail{\ding{55}} % ✗
  \newcommand\Certificate{certificate}
  \newcommand\CommandState{instrPlan}
  %% \newcommand\State{State}
  \newcommand\writeEvt[4]{\angled{#1:#2@#3,#4}}
  %% \newcommand\writeEvt[4]{#1:#2@#3,#4}
  \newcommand\simrel{\mathcal{I}}
  \newcommand\simrelPre{\mathcal{I}_{\textup{\rm pre}}}
  \newcommand\simrelBase{\mathcal{I}_{\textup{\rm base}}}
  \newcommand\inv{\mathcal{I}}
  \newcommand\s{\mathbf{s}}
  \newcommand\sfst{\mathbf{s}_0}
  \newcommand\ssnd{\mathbf{s}_1}
  \newcommand\aT{\mathbf{a}}
  \newcommand\afst{\mathbf{a}_0}
  \newcommand\asnd{\mathbf{a}_1}
  \newcommand\p{\mathbf{p}}
  \newcommand\ptid{\mathbf{p}\textsf{-}\tId}
%  \newcommand\ordPlusMO{S_{\Ord \cup mo}}
  \newcommand\ordPlusMO{\textsf{S-Ord-mo}}

  \newcommand\finalStateP{\mathsf{Final}^{\ARM}}
  \newcommand\finalStateA{\mathsf{Final}^{\ARMt}}
  \newcommand\finalStateProm{\mathsf{Final}^{\Promise}}

  \newcommand\sinit{\mathbf{s}^{\rm init}}
  \newcommand\ainit{\mathbf{a}^{\rm init}}
  \newcommand\pinit{\mathbf{p}^{\rm init}}

  \newcommand\invI[1]{\inv_{\textup{\rm #1}}}
  \newcommand\invARM[1]{\inv^\ARM_{\textup{\rm #1}}}

  \newcommand\invTidWriteComCERT{\inv^{\tId}_{\mathsf{w-cert}}}
  \newcommand\invStateCERT{\invI{state-cert}}
  \newcommand\invViewDeltaCERT{\invI{view-cert}}
  \newcommand\invViewWriteCERT{\invI{view-write-cert}}
  \newcommand\invViewRelCERT{\invI{write-rel-cert}}
  \newcommand\invViewReadCERT{\invI{view-read-cert}}
  \newcommand\invWriteTimestampCERT{\invI{write-time-cert}}
  %% \newcommand\invViewCERT{\invI{view-cert}}
  \newcommand\invMemZeroCERT{\invI{mem-1-tid-cert}}
  \newcommand\invMemOneCERT{\invI{mem-1-com-cert}}
  \newcommand\invMemTwoCERT{\invI{mem-2-cert}}
  \newcommand\simrelBaseCERT{\mathcal{I}_{\textup{\rm base-cert}}}

  \newcommand\invCf{\invI{prg}}
  \newcommand\invTId{\invI{tid}}
  \newcommand\invPrefix{\invI{prefix}}
  \newcommand\invSPrefix{\invI{strong prefix}}
  %% \newcommand\invMemOne{\invI{mem1}}
  %% \newcommand\invMemTwo{\invI{mem2}}
  %% \newcommand\invMemThree{\invI{mem3}}
  %% \newcommand\invView{\invI{view}}
  %% \newcommand\invState{\invI{state}}
  \newcommand\invReach{\invI{reach}}
  \newcommand\invComWrite{\invI{com-SY}}
  \newcommand\invCert{\invI{cert}} %\Certificate}}
  \newcommand\invCertTid{\inv_{cert \; \tId}} %\Certificate}}
  \newcommand\simrelTid{\inv_{exec \; \tId}} %\Certificate}}
  \newcommand\invApartialOrderOrd{\inv^{ARM}_{\Ord \; \text{is a partial order}}}
  \newcommand\invAuniqWrite{\invARM{unique write}}
  \newcommand\invAtransClosedOrd{\invARM{\Ord = transitive\_closure(\Ord)}}
  \newcommand\invAimmediateEdge{\invARM{Immediate edge}}
  \newcommand\invAimmediatePath{\invARM{Immediate path}}
  \newcommand\invAmaxPath{\invARM{max \; \cpath}}
  \newcommand\invPromUptoARM{\invI{Promise is up to ARM}}
  \newcommand\invPromUptoARMtId{\inv^{\tId}_{\textup{Promise is up to ARM}}}
  \newcommand\invPromUptoARMnot{\invI{Promise isn't up to ARM}}


  \newcommand\correctStateA{\inv^{\ARMt}_{\textup{\rm correct}}}
  \newcommand\correctStateP{\invARM{correct}}
  \newcommand\invATapeCf{\invARM{tape-Prg}}
  \newcommand\invATapeCfState{\invARM{{tape-Prg-State}}}
  \newcommand\invAReadWrite{\invARM{{Read-Write}}}
  \newcommand\invAReadRead{\invARM{{Read-Read}}}
  \newcommand\invAWriteWriteRead{\invARM{{Write-Write-Read}}}
  \newcommand\invAReadCommittedWrite{\invARM{{Read-Write-Committed}}}
  \newcommand\invAview{\invARM{{View}}}
  \newcommand\invAviewWrite{\invARM{{View-Write}}}
  \newcommand\invAviewRead{\invARM{{View-Read}}}
  \newcommand\invAWriteView{\invARM{{Write-View}}}
  \newcommand\invAnextCommitted{\invARM{{Next-Committed}}}
  \newcommand\invACf{\invARM{Prg}}
  \newcommand\invAtId{\invARM{tid}}
  \newcommand\invAldRead{\invARM{LD-Read}}
  \newcommand\invAtypePreservation{\invARM{{tape-Type}}}
  \newcommand\invAcommittedPreservation{\invARM{{Committed-Preserve}}}
  \newcommand\invAstatePreservation{\invARM{{State-Preserve}}}
  \newcommand\invAstateCom{\invARM{S-Scom}}
  \newcommand\invAcomFences{\invARM{{Committed-Fences}}}
  \newcommand\invAordMOacyclic{\invARM{{Ord-mo-acyclic}}}
  \newcommand\invAordProp{\invARM{{Ord-Prop}}}
  \newcommand\invApropOrd{\invARM{{Prop-Ord}}}
  \newcommand\invAord{\invARM{{Ord-acyclic}}}
  \newcommand\invAevtTape{\invARM{{Evt-tape}}}
  \newcommand\invAtapeEvt{\invARM{{tape-Evt}}}
  \newcommand\invAtapeOrd{\invARM{{tape-Ord}}}
  \newcommand\invAReadWriteOne{\invARM{{Read-Write-1}}}
  \newcommand\invAReadWriteTwo{\invARM{{Read-Write-2}}}

  \newcommand\invPMP{\inv^{\Promise}_{\textsf{M-P}}}
  \newcommand\invPmessageView{\inv^{\Promise}_{\textup{\rm Message-View}}}

  \newcommand\armStepFetch{\overset{fetch \; \tId \; \cpath}{\armStepl}}
  \newcommand\armStepFetchPrime{\overset{fetch \; \tId' \; \cpath'}{\armStepl}}
  \newcommand\armStepProp{\overset{e \rightsquigarrow \tId}{\armStepl}}
  \newcommand\armStepPropPrime{\overset{e' \rightsquigarrow \tId'}{\armStepl}}
  \newcommand\armStepWritePending{\overset{\dashrightarrow \stRequestWrite{\tId}{ \cpath}{ x}{\stval}}{\armStepl}}
  \newcommand\armStepWritePendingPrime{\overset{\dashrightarrow \stRequestWrite{\tId'}{ \cpath'}{ y}{\stval'}}{\armStepl}}

  \newcommand\armStepReadRequest{\overset{\dashrightarrow \stRequestRead{\tId}{ \cpath}{ x}}{\armStepl}}
  \newcommand\armStepReadRequestPrime{\overset{\dashrightarrow \stRequestRead{\tId'}{ \cpath'}{ y}}{\armStepl}}
  \newcommand\armStepCondBranch{\overset{\dashrightarrow \tId, \cpath, \text{choose branch}}{\armStepl}}
  \newcommand\armStepCondBranchPrime{\overset{\dashrightarrow \tId', \cpath', \text{choose branch}}{\armStepl}}
  \newcommand\armStepFenceCommit{\overset{\dashrightarrow \stRequestFence{\tId}{ \cpath}}{\armStepl}}
  \newcommand\armStepFenceCommitPrime{\overset{\dashrightarrow \stRequestFence{\tId'}{ \cpath'}}{\armStepl}}
  \newcommand\armStepReadSatisfy{\overset{\dashrightarrow \stRequestRead{\tId}{ \cpath}{ x}, \stRequestWrite{\tId'}{ \cpath'}{ x}{\stval}}{\armStepl}}
  \newcommand\armStepReadSatisfyPrime{\overset{\dashrightarrow \stRequestRead{\tId'}{ \cpath'}{ y}, \stRequestWrite{\tId''}{ \cpath''}{ y}{\stval'}}{\armStepl}}
  \newcommand\armStepReadSatisfyFail{\overset{\not \dashrightarrow \stRequestRead{\tId}{ \cpath}{ x}, \stRequestWrite{\tId'}{ \cpath'}{ x}{\stval}}{\armStepl}}
  \newcommand\armStepReadSatisfyFailPrime{\overset{\not \dashrightarrow \stRequestRead{\tId'}{ \cpath'}{ y}, \stRequestWrite{\tId''}{ \cpath''}{ y}{\stval'}}{\armStepl}}
  \newcommand\armStepReadSatisfyInF{\overset{\dashrightarrow \stRequestRead{\tId}{ \cpath}{ x}, \stRequestWrite{\tId}{ \cpath'}{ x}{\stval}}{\armStepl}}
  \newcommand\armStepReadSatisfyInFPrime{\overset{\dashrightarrow \stRequestRead{\tId'}{ \cpath'}{ y}, \stRequestWrite{\tId'}{ \cpath''}{ y}{\stval'}}{\armStepl}}
  \newcommand\armStepReadCommit{\overset{\dashrightarrow \tId, \cpath, \text{read commit}}{\armStepl}}
  \newcommand\armStepReadCommitPrime{\overset{\dashrightarrow \tId', \cpath', \text{read commit}}{\armStepl}}

  \newcommand\armStepFetchP{\overset{\text{fetch} \; \tId \; \cpath}{\armStepP}}
  \newcommand\armStepFetchPrimeP{\overset{\text{fetch} \; \tId' \; \cpath'}{\armStepP}}
  \newcommand\armStepPropP{\overset{e \rightsquigarrow \tId}{\armStepP}}
  \newcommand\armStepPropPrimeP{\overset{e' \rightsquigarrow \tId'}{\armStepP}}
  \newcommand\armStepWritePendingP{\overset{\dashrightarrow \stRequestWrite{\tId}{ \cpath}{ x}{\stval}}{\armStepP}}
  \newcommand\armStepWritePendingPrimeP{\overset{\dashrightarrow \stRequestWrite{\tId'}{ \cpath'}{ y}{\stval'}}{\armStepP}}
  \newcommand\armStepReadRequestP{\overset{\dashrightarrow \stRequestRead{\tId}{ \cpath}{ x}}{\armStepP}}
  \newcommand\armStepReadRequestPrimeP{\overset{\dashrightarrow \stRequestRead{\tId'}{ \cpath'}{ y}}{\armStepP}}
  \newcommand\armStepCondBranchP{\overset{\dashrightarrow \tId, \cpath, \text{choose branch}}{\armStepP}}
  \newcommand\armStepCondBranchPrimeP{\overset{\dashrightarrow \tId', \cpath', \text{choose branch}}{\armStepP}}
  \newcommand\armStepFenceCommitP{\overset{\dashrightarrow \stRequestFence{\tId}{ \cpath}}{\armStepP}}
  \newcommand\armStepFenceCommitPrimeP{\overset{\dashrightarrow \stRequestFence{\tId'}{ \cpath'}}{\armStepP}}
  \newcommand\armStepReadSatisfyP{\overset{\dashrightarrow \stRequestRead{\tId}{ \cpath}{ x}, \stRequestWrite{\tId'}{ \cpath'}{ x}{\stval}}{\armStepP}}
  \newcommand\armStepReadSatisfyPrimeP{\overset{\dashrightarrow \stRequestRead{\tId'}{ \cpath'}{ y}, \stRequestWrite{\tId''}{ \cpath''}{ y}{\stval'}}{\armStepP}}
  \newcommand\armStepReadSatisfyFailP{\overset{\not \dashrightarrow \stRequestRead{\tId}{ \cpath}{ x}, \stRequestWrite{\tId'}{ \cpath'}{ x}{\stval}}{\armStepP}}
  \newcommand\armStepReadSatisfyFailPrimeP{\overset{\not \dashrightarrow \stRequestRead{\tId'}{ \cpath'}{ y}, \stRequestWrite{\tId''}{ \cpath''}{ y}{\stval'}}{\armStepP}}
  \newcommand\armStepReadSatisfyInFP{\overset{\dashrightarrow \stRequestRead{\tId}{ \cpath}{ x}, \stRequestWrite{\tId}{ \cpath'}{ x}{\stval}}{\armStepP}}
  \newcommand\armStepReadSatisfyInFPrimeP{\overset{\dashrightarrow \stRequestRead{\tId'}{ \cpath'}{ y}, \stRequestWrite{\tId'}{ \cpath''}{ y}{\stval'}}{\armStepP}}
  \newcommand\armStepReadCommitP{\overset{\dashrightarrow \tId, \cpath, \text{read commit}}{\armStepP}}
  \newcommand\armStepReadCommitPrimeP{\overset{\dashrightarrow \tId', \cpath', \text{read commit}}{\armStepP}}


\newcommand{\event}[3]{#1#2#3}
\tikzset{
   every path/.style={>=stealth},
   po/.style={->,color=brown,,shorten >=-0.5mm,shorten <=-0.5mm},
   rf/.style={->,color=green!60!black,dashed,,shorten >=-0.5mm,shorten <=-0.5mm},
   fr/.style={->,color=red,thick,shorten >=-0.5mm,shorten <=-0.5mm},
   mo/.style={->,color=orange!60!red,dotted,thick,shorten >=-0.5mm,shorten <=-0.5mm},
   no/.style={->,dotted,thick,shorten >=-0.5mm,shorten <=-0.5mm},
   deps/.style={->,color=violet,dotted,thick,shorten >=-0.5mm,shorten <=-0.5mm},
}

\newcommand{\IssuedSet}{I}
\newcommand{\issuable}{{\sf Issuable}}

%% \newcommand{\comment}[1]{\color{teal}{~~\texttt{/\!\!/}\textit{#1}}}

\newcommand{\lEID}{\lE_{tid}}
\newcommand{\TransSet}{{\rm Transitions}}
\newcommand{\LabelSet}{{\rm Labels}}

\newcommand{\pseudoCompileF}{{\rm pseudo\text{-}compile}}
\newcommand{\pseudoCompileReq}{{\rm pseudo\text{-}compile\text{-}req}}
\newcommand{\compileReq}{{\rm compile\text{-}req}}

\newcommand{\scRel}{{\rm sc\text{-}rel}}
\newcommand{\scRelF}{{\rm sc\text{-}rel\text{-}fun}}
\newcommand{\acqRel}{{\rm acq\text{-}rel}}
\newcommand{\acqRelF}{{\rm acq\text{-}rel\text{-}fun}}
\newcommand{\curRel}{{\rm cur\text{-}rel}}
\newcommand{\curRelF}{{\rm cur\text{-}rel\text{-}fun}}
\newcommand{\relRelFF}{{\rm rel\text{-}rel\text{-}fun}}
\newcommand{\relRel}{{\rm rel\text{-}rel}}
\newcommand{\msgRelF}{{\rm msg\text{-}rel\text{-}fun}}
\newcommand{\msgRel}{{\rm msg\text{-}rel}}

\newcommand{\step}{\longrightarrow}
\newcommand{\astep}[1]{\xrightarrow{#1}}
\newcommand{\bstep}[1]{\xRightarrow{#1}}
\newcommand{\cstep}[2]{\xrightarrow[{#1}]{{#2}}}
\newcommand{\dstep}[2]{\dhxrightarrow[{#1}]{{#2}}}
\newcommand{\Pfin}{\txtsub{P}{final}}
\newcommand{\fin}{\text{final}}
\newcommand{\rulename}[1]{{\textsc{({#1})}}}
\newcommand{\onestep}[2]{{\textsc{({#1}}{({#2})}\textsc{)}}}

\newcommand{\mrel}{R}
\newcommand{\lmrel}{\texttt{view}}
\newcommand{\msc}{S}

\newcommand{\I}{{\cal I}}

\newcommand{\tmin}{\in}
\newcommand{\tmle}{\leqslant}
\newcommand{\tcom}{{\cal V}}
%\newcommand{\tcom}{V}
\newcommand{\gsco}{{\cal S}}
%\newcommand{\gsco}{F}
%\newcommand{\mem}{{\cal M}}
\newcommand{\mem}{M}
\newcommand{\scmap}{S}
\newcommand{\msgid}{i}
\newcommand{\gts}{\mathcal{T\!S}}
\newcommand{\lts}{\mathit{TS}}
\newcommand{\tconf}{{\mathbf{TC}}}
\newcommand{\mconf}{{\mathbf{MS}}}
\newcommand{\lstate}{\texttt{st}}
\newcommand{\lview}{\texttt{view}}
\newcommand{\lprmem}{\texttt{prm}}
\newcommand{\lmem}{\txtsub{\mem}{p}}
%\newcommand{\lprom}{{\cal P}}
\newcommand{\lprom}{P}
\newcommand{\limitedmem}{{\cal L}}
\newcommand{\ogts}{\txtsub{\gts}{fut}}
\newcommand{\ogsco}{\txtsub{\gsco}{future}}
\newcommand{\olts}{\txtsub{\lts}{fut}}
\newcommand{\omem}{\txtsub{\mem}{future}}
\newcommand{\omsc}{\txtsub{\msc}{future}}
\newcommand{\lr}{{\text{r}}}
\newcommand{\lw}{{\text{w}}}
\newcommand{\cur}{cur}
%% \newcommand{\acq}{acq}
%% \newcommand{\rel}{rel}
%% \newcommand{\lsc}{sc}
%% \newcommand{\lscm}{s\!f}

\newcommand{\suc}{{succ}}
\newcommand{\lsuc}{\texttt{succ}}

\newcommand{\locprm}{P}
\newcommand{\llocprm}{\texttt{promise}}
\newcommand{\SUBS}{\mathit{sub}}
\newcommand{\lSUBS}{\texttt{sub}}

\newcommand{\relo}{{\texttt{rel}}}
\newcommand{\acqo}{{\texttt{acq}}}
%% \newcommand{\sco}{{\texttt{sc}}}
%% \newcommand{\na}{\texttt{na}}
%% \newcommand{\pln}{\texttt{pln}}
%% \newcommand{\atm}{\texttt{atm}}
\newcommand{\ra}{\texttt{ra}}
%% \newcommand{\nf}{\texttt{nf}}
%% \newcommand{\rlx}{\texttt{rlx}}
%% %\newcommand{\unord}{\texttt{uno}}
%% \newcommand{\relacqo}{{\texttt{relacq}}}

%% \newcommand{\Typ}{{\textdom{Typ}}}
%% \newcommand{\Lab}{{\textdom{Lab}}}
%% \newcommand{\Loc}{{\textdom{Loc}}}
%% \newcommand{\Val}{{\textdom{Val}}}
\newcommand{\AVal}{{\textdom{G}}}
%% \newcommand{\Tid}{{\textdom{Tid}}}
%% \newcommand{\Ord}{{\textdom{Ord}}}
%% \newcommand{\Reg}{{\textdom{Reg}}}
%% \newcommand{\Time}{{\textdom{Time}}}
%% \newcommand{\Timemap}{{\textdom{Timemap}}}
%% \newcommand{\Id}{{\textdom{Id}}}

\newcommand{\writeInstVolatile}[2]{[#1]\;:=_{\rm volatile}\;#2}
\newcommand{\readInstVolatile}[2]{#1 \;:=_{\rm volatile}\;[#2]}

\newcommand{\nullPtr}{\kw{null}}
\newcommand{\funcSt}[1]{{\rm #1}}

\newcommand{\stPhi}{\phi} 
\newcommand{\stAlpha}{\alpha}
\newcommand{\stGamma}{\gamma} 

\newcommand{\stPostOp}{\beta}
\newcommand{\stObservedWrites}{\omega}
% Shortcuts
\newcommand{\arrayBlock}[1]{\begin{array}{c}#1\end{array}}

\newcommand{\vName}{\mathsf{x}}


%% \selectlanguage{russian}
%% \newtranslation[to=russian]{Section}{Раздел}
%% \AtBeginSection{\frame{\sectionpage}}
\AtBeginSection[]
{
  \begin{frame}<beamer>
    \large
    \LARGE
    %% \frametitle{Outline for section \thesection}
    \tableofcontents[currentsection]
  \end{frame}
}

\addtobeamertemplate{navigation symbols}{}{%
    \usebeamerfont{footline}%
    \usebeamercolor[fg]{footline}%
    \hspace{1em}%
    {\bf \huge \insertframenumber}
}

\definecolor{darkGreen}{RGB}{0,127,0}
\definecolor{darkYellow}{RGB}{127,127,0}
\definecolor{darkRed}{RGB}{127,0,0}
\definecolor{darkPurple}{RGB}{127,0,127}
\colorlet{colorFETCH}{gray!40!white}
\colorlet{colorSAT}{yellow!40!white}
\colorlet{colorCOM}{green!40!white}
\colorlet{colorBUG}{red!20!white}
\colorlet{colorPROP}{green!80!black}
\colorlet{colorNPROP}{red!80!black}
\colorlet{colorVAL}{green!80!black}
\colorlet{colorPROM}{darkGreen}
\colorlet{colorPOSTPONE}{magenta!80!black}
\colorlet{colorSHADE}{gray!70}
\colorlet{colorCOV}{magenta!20}
\colorlet{colorISS}{green!30!white}

\colorlet{colorPO}{gray!60!black}
\colorlet{colorRF}{green!60!black}
\colorlet{colorMO}{orange}
\colorlet{colorFR}{purple}
\colorlet{colorECO}{red!80!black}
\colorlet{colorSYN}{green!40!black}
\colorlet{colorHB}{blue}
\colorlet{colorPPO}{magenta}
\colorlet{colorPB}{olive}
\colorlet{colorSBRF}{olive}
\colorlet{colorRMW}{olive!70!black}
\colorlet{colorRSEQ}{blue}
\colorlet{colorSC}{violet}
\colorlet{colorPSC}{violet}
\colorlet{colorREL}{olive}
\colorlet{colorCONFLICT}{olive}
\colorlet{colorRACE}{olive}
\colorlet{colorWB}{orange!70!black}
\colorlet{colorPSC}{violet}
\colorlet{colorSCB}{violet}
\colorlet{colorDEPS}{violet}

\tikzset{
   every path/.style={>=stealth},
   po/.style={->,color=brown,,shorten >=-0.5mm,shorten <=-0.5mm},
   rf/.style={->,color=green!60!black,dashed,,shorten >=-0.5mm,shorten <=-0.5mm},
   fr/.style={->,color=red,thick,shorten >=-0.5mm,shorten <=-0.5mm},
   mo/.style={->,color=orange!60!red,dotted,thick,shorten >=-0.5mm,shorten <=-0.5mm},
   no/.style={->,dotted,thick,shorten >=-0.5mm,shorten <=-0.5mm},
   deps/.style={->,color=violet,dotted,thick,shorten >=-0.5mm,shorten <=-0.5mm},
}

\newcommand{\lX}{\mathtt{X}}
\newcommand{\lPO}{{\color{colorPO}\mathtt{po}}}
\newcommand{\lRF}{{\color{colorRF} \mathtt{rf}}}
\newcommand{\lRMW}{{\color{colorRMW} \mathtt{rmw}}}
\newcommand{\lMO}{{\color{colorMO} \mathtt{mo}}}
\newcommand{\lMOx}{{\color{colorMO} \mathtt{mo}}_x}
\newcommand{\lMOy}{{\color{colorMO} \mathtt{mo}}_y}
\newcommand{\lCO}{{\color{colorMO} \mathtt{co}}}
\newcommand{\lCOx}{{\color{colorMO} \mathtt{co}}_x}
\newcommand{\lCOy}{{\color{colorMO} \mathtt{co}}_y}
\newcommand{\lFR}{{\color{colorFR} \mathtt{fr}}}
\newcommand{\lFRx}{{\color{colorFR} \mathtt{fr}}_x}
\newcommand{\lFRy}{{\color{colorFR} \mathtt{fr}}_y}
\newcommand{\lECO}{{\color{colorECO} \mathtt{eco}}}
\newcommand{\lSBRF}{{\color{colorSBRF} \mathtt{sbrf}}}
\newcommand{\lRSEQ}{{\color{colorRSEQ}\mathtt{rseq}}}
\newcommand{\lSW}{{\color{colorSYN}\mathtt{sw}}}
\newcommand{\lHB}{{\color{colorHB}\mathtt{hb}}}
%\newcommand{\lWB}{{\color{colorWB} \mathtt{wb}}}
\newcommand{\lDOB}{{\mathtt{dob}}}
\newcommand{\lBOB}{{\mathtt{bob}}}
\newcommand{\lAOB}{{\mathtt{aob}}}
\newcommand{\lOBS}{{\mathtt{obs}}}
\newcommand{\lEORD}{{\mathtt{eord}}}
\newcommand{\lTORD}{{\mathtt{tord}}}
\newcommand{\lSC}{{\mathtt{sc}}}

\tikzset{
    ncbar angle/.initial=90,
    ncbar/.style={
        to path=(\tikztostart)
        -- ($(\tikztostart)!#1!\pgfkeysvalueof{/tikz/ncbar angle}:(\tikztotarget)$)
        -- ($(\tikztotarget)!($(\tikztostart)!#1!\pgfkeysvalueof{/tikz/ncbar angle}:(\tikztotarget)$)!\pgfkeysvalueof{/tikz/ncbar angle}:(\tikztostart)$)
        -- (\tikztotarget)
    },
    ncbar/.default=0.5cm,
}

\tikzset{square left brace/.style={ncbar=0.3cm}}
\tikzset{square right brace/.style={ncbar=-0.3cm}}

\tikzset{round left paren/.style={ncbar=0.5cm,out=120,in=-120}}
\tikzset{round right paren/.style={ncbar=0.5cm,out=60,in=-60}}

%% Custom TikZ patterns

% defining the new dimensions and parameters
\newlength{\hatchspread}
\newlength{\hatchthickness}
\newlength{\hatchshift}
\newcommand{\hatchcolor}{}
% declaring the keys in tikz
\tikzset{hatchspread/.code={\setlength{\hatchspread}{#1}},
         hatchthickness/.code={\setlength{\hatchthickness}{#1}},
         hatchshift/.code={\setlength{\hatchshift}{#1}},% must be >= 0
         hatchcolor/.code={\renewcommand{\hatchcolor}{#1}}}
% setting the default values
\tikzset{hatchspread=10pt,
         hatchthickness=4pt,
         hatchshift=0pt,% must be >= 0
         hatchcolor=black}
% declaring the pattern
\pgfdeclarepatternformonly[\hatchspread,\hatchthickness,\hatchshift,\hatchcolor]% variables
   {custom north west lines}% name
   {\pgfqpoint{\dimexpr-2\hatchthickness}{\dimexpr-2\hatchthickness}}% lower left corner
   {\pgfqpoint{\dimexpr\hatchspread+2\hatchthickness}{\dimexpr\hatchspread+2\hatchthickness}}% upper right corner
   {\pgfqpoint{\dimexpr\hatchspread}{\dimexpr\hatchspread}}% tile size
   {% shape description
    \pgfsetlinewidth{\hatchthickness}
    \pgfpathmoveto{\pgfqpoint{0pt}{\dimexpr\hatchspread+\hatchshift}}
    \pgfpathlineto{\pgfqpoint{\dimexpr\hatchspread+0.15pt+\hatchshift}{-0.15pt}}
    \ifdim \hatchshift > 0pt
      \pgfpathmoveto{\pgfqpoint{0pt}{\hatchshift}}
      \pgfpathlineto{\pgfqpoint{\dimexpr0.15pt+\hatchshift}{-0.15pt}}
    \fi
    \pgfsetstrokecolor{\hatchcolor}
%    \pgfsetdash{{1pt}{1pt}}{0pt}% dashing cannot work correctly in all situation this way
    \pgfusepath{stroke}
   }

\pgfdeclarepatternformonly[\hatchspread,\hatchthickness,\hatchshift,\hatchcolor]% variables
   {custom north east lines}% name
   {\pgfqpoint{\dimexpr-2\hatchthickness}{\dimexpr-2\hatchthickness}}% lower left corner
   {\pgfqpoint{\dimexpr\hatchspread+2\hatchthickness}{\dimexpr\hatchspread+2\hatchthickness}}% upper right corner
   {\pgfqpoint{\dimexpr\hatchspread}{\dimexpr\hatchspread}}% tile size
   {% shape description
    \pgfsetlinewidth{\hatchthickness}
    \pgfpathmoveto{\pgfqpoint{\dimexpr\hatchshift-0.15pt}{-0.15pt}}
    \pgfpathlineto{\pgfqpoint{\dimexpr\hatchspread+0.15pt}{\dimexpr\hatchspread-\hatchshift+0.15pt}}
    \ifdim \hatchshift > 0pt
      \pgfpathmoveto{\pgfqpoint{-0.15pt}{\dimexpr\hatchspread-\hatchshift-0.15pt}}
      \pgfpathlineto{\pgfqpoint{\dimexpr\hatchshift+0.15pt}{\dimexpr\hatchspread+0.15pt}}
    \fi
    \pgfsetstrokecolor{\hatchcolor}
%    \pgfsetdash{{1pt}{1pt}}{0pt}% dashing cannot work correctly in all situation this way
    \pgfusepath{stroke}
   }

\begin{document}

\title
{Операционные методы в приложении к слабым моделям памяти}
\author
[А.В. Подкопаев]
{\textbf{Подкопаев Антон Викторович}}
\institute{
Специальность: 05.13.11
``Математическое и программное обеспечение вычислительных машин,
комплексов и компьютерных сетей'' \\
\vspace{1.0cm}
\begin{tabular}{r l}
Научный руководитель: & д.т.н., проф. Кознов Д.В. (СПбГУ) \\
\\
Официальные оппоненты: & д.т.н., проф. Гергель В.П. (ННГУ) \\
                       & к.т.н., доц. Лукашин А.А. (СПбГПУ) \\
\\
Ведущая организация: & ИСП РАН
\end{tabular}

\vspace{1.0cm}
Санкт-Петербург
%% Семинар ИСП РАН
}
\date[17-05-18]{17.05.2018}

\newcommand{\fr} [2]{\begin{frame}{#1} #2 \end{frame}}
\newcommand{\frN}[2]{\begin{frame}[noframenumbering]{#1} #2 \end{frame}}
\newcommand{\ite}[1]{\begin{itemize} #1 \end{itemize}}
\newcommand{\iteN}[1]{\begin{enumerate} #1 \end{enumerate}}
\newcommand{\colu}[2]{\begin{column}{#1} #2 \end{column}}
\newcommand{\ex}{\textbf{Ex: }}
\newcommand{\exN}[1]{\textbf{Ex #1: }}
\newcommand{\df}{\textbf{Def: }}
\newcommand{\pitem}{\pause \item}
\newcommand{\lstasm}[1]{\lstinputlisting[language={[sparc]Assembler}]{codes/#1}}
\newcommand{\lstinlineasm}[1]{\lstinline[language={[sparc]Assembler}]{#1}}
\newcommand{\fri}[2]{\fr{#1}{\ite{#2}}}
\newcommand{\fre}[2]{\fr{#1}{\iteN{#2}}}
\newcommand{\frc}[2]{\fr{#1}{\begin{columns} #2 \end{columns}}}
\newcommand{\colui}[2]{\colu{#1}{\ite{#2}}}
\newcommand{\phrase}[1]{
  \fr{}{
  \begin{center}
    \Huge #1
  \end{center}
  }
}
\newcommand{\phraseL}[1]{
  \fr{}{
  \begin{center}
    \Large #1
  \end{center}
  }
}
\newcommand{\alertcolor}{red!60!black}
\setbeamercolor{alerted text}{fg=\alertcolor}
\newcommand\tick{{\color{green!50!black} \ding{51}}}
\newcommand\fail{{\color{red!50!black} \ding{55}}}

\newcommand{\relo}{{\texttt{rel}}}
\newcommand{\acqo}{{\texttt{acq}}}
\newcommand{\sco}{{\texttt{sc}}}
\newcommand{\na}{\texttt{na}}
\newcommand{\pln}{\texttt{pln}}
\newcommand{\atm}{\texttt{atm}}
\newcommand{\ra}{\texttt{ra}}
\newcommand{\rlx}{\texttt{rlx}}
%\newcommand{\unord}{\texttt{uno}}
\newcommand{\relacqo}{{\texttt{relacq}}}
\newcommand{\rlxmsg}[3]{\tup{#1\mathbin{:}#2\text{\smaller@}#3}}
\newcommand{\ts}[1]{\color{blue!60!green}{\mbox{\smaller#1}}}

\newcommand{\rlab}{\texttt{R}}
\newcommand{\wlab}{\texttt{W}}
\newcommand{\ulab}{\texttt{U}}
\newcommand{\slab}{\texttt{S}}
\newcommand{\tlab}{\texttt{T}}
\newcommand{\valw}{\mathit{val}_{\sf w}}
\newcommand{\flab}{\texttt{F}}

\newcommand{\dimslide}[1]{
   \addtooverlay<#1>{%
     \draw[fill=black,opacity=0.70] 
     (current page.north east) rectangle (current page.south west);
   }
}
\NewDocumentCommand{\dimalert}{r<> m}{%
\only<#1>{
\dimslide{#1}
\begin{textblock*}{\textwidth}(0.1\textwidth,0.4\textheight)
\begin{alertblock}{}
  \LARGE #2
\end{alertblock}
\end{textblock*}
}
}
\newcommand{\cntrd}[1]{ \begin{center} #1 \end{center} }

\newcommand{\instBackground}[2]{
    \fill[#1, rounded corners=3pt] ($(#2)  + (-1.1,0.35)$) rectangle ++(2.2,-0.7);
}
\newcommand{\parLines}[1]{
      \draw[-,ultra thick] ($(#1) + (-0.1,0.5)$) -- ($(#1) + (-0.1,-0.5-0.8)$);
      \draw[-,ultra thick] ($(#1) + ( 0.1,0.5)$) -- ($(#1) + ( 0.1,-0.5-0.8)$);
}

\newcommand{\indepCommentLeft}[2]{
      \draw[-,red,ultra thick] ($(#1) + (0.1, 0)$)
        to[out=0,in=0] ($(#2) + (0.1, 0)$);
      \node[anchor=west] at ($(#1) + (0.3, -0.4)$)  {\Large \textcolor{red}{Independent}};
}

\newcommand{\indepCommentRight}[2]{
      \draw[-,red,ultra thick] ($(#1) + (0.1, 0)$)
        to[out=180,in=180] ($(#2) + (0.1, 0)$);
      \node[anchor=east] at ($(#1) + (-0.1, -0.4)$)  {\Large \textcolor{red}{Independent}};
}

\newcommand{\promMsg}[3]{\angled{#1:#2@\tstampWOsize{#3}}}

\newcommand{\promisedTXT}{Обещана} % Promised
\newcommand{\postponedTXT}{Отложена} % Postponed
\newcommand{\lviewTXT}{Л.фронт} % LView
\newcommand{\rviewTXT}{П.фронт} % RView

\newcommand{\ptrPromise}[1]{
    \draw[->,ultra thick] ($(#1)  + (-3.0,0.39)$) -- +(1.9, 0.0);
    \node at ($(#1) + (-2.1,0.62)$) {\large \Promise};
}
\newcommand{\ptrPromiseLeft}[1]{
    \draw[->,ultra thick] ($(#1)  + (-1.2,0.39)$) -- +(1.6, 0.0);
}
\newcommand{\ptrPromiseRight}[1]{
  \ptrPromiseLeft{#1}
    %% \draw[->,ultra thick] ($(#1)  + (3.0,0.39)$) -- +(-1.9, 0.0);
}
\newcommand{\postponedBorder}[1]{
    \draw[colorPOSTPONE, ultra thick] ($(#1)  + (-1.1,0.3)$) rectangle ++(2.2,-0.6);
}
\newcommand{\postponedCommentLeft}[1]{
    \node[anchor=east] at ($(#1) - (1.4, 0.0)$) {\Large \textcolor{colorPOSTPONE}{\postponedTXT}};
}
\newcommand{\promisedBorder}[1]{
    \draw[colorPROM, ultra thick, rounded corners=3pt] ($(#1)  + (-1.1,0.35)$) rectangle ++(2.2,-0.7);
}
\newcommand{\promisedCommentDown}[1]{
    \node at ($(#1) + (0, -0.6)$) {\Large \textcolor{colorPROM}{\promisedTXT}};
}
\newcommand{\promisedCommentLeft}[1]{
    \node[anchor=east] at ($(#1) - (1.4, 0.0)$) {\Large \textcolor{colorPROM}{\promisedTXT}};
}
\newcommand{\promisedCommentRight}[1]{
    \node[anchor=west] at ($(#1) + (1.4, 0.0)$) {\Large \textcolor{colorPROM}{\promisedTXT}};
}

\newcommand{\traverseSingleBorder}[2]{
    \draw[pattern=custom north east lines, hatchcolor=#1, thick, rounded corners] ($(#2)  + (-0.35,-0.3)$) rectangle ++(0.7,0.6);
}

\newcommand{\traverseTwoVertBorder}[3]{
    \draw[pattern=custom north east lines, hatchcolor=#1, thick, rounded corners] ($(#2)  + (-0.45,0.4)$) rectangle ($(#3)  + (0.45,-0.4)$);
}

{\setbeamertemplate{footline}{}
\setbeamertemplate{headline}{}
\frame[plain,noframenumbering]{\titlepage}}
%% \begin{frame}[plain,noframenumbering]
%%   \titlepage
%% \end{frame}

%% \fri{\huge Что нужно для быстрого ПО?}{
%% \LARGE
%% \pause
%% \item \only<-4>{Хороший Алгоритм} \only<5->{\textcolor{gray!30}{Хороший Алгоритм}}
%% \vfill
%% \pause
%% \item \only<-5>{Эффективный}\only<6>{{\bf ?Эффективный?}}\only<7->{{\bf Оптимизирующий}} Компилятор
%% \vfill
%% \pause
%% \item \only<-7>{Производительный}\only<8>{{\bf ?Производительный?}}\only<9->{{\bf Оптимизирующий}} Процессор
%% }

%% \section{Контекст, мотивация, задачи}
\phrase{
  {\bf Модель памяти} \\ (memory model, MM) --- семантика многопоточной системы\\
}

\phrase{
  Последовательная консистентность (SC) \cite{Lamport:TC79}
  \vfill
  %% \pause
  \emph{семантика системы --- попеременное исполнение потоков}
}

\newcommand{\loadBufferValues}[2]{
      \node (memoryValues) at ($(middleBuffer) + (2.0, -3.2)$) {};

      \onslide<\storageAppearFr->{
        %% \storageTwoThreadsRelative{(leftBuffer)}{(rightBuffer)}{1.7}{0.5}{-3.0}
        \draw[draw,fill=yellow!20] ($(memoryValues) + (-1.9,0.8)$) rectangle ($(memoryValues) + (1.9,-0.3)$) ;
        \node[anchor=north] at ($(memoryValues) + (0,0.85)$) {\Large Память};
      }

      \node at ($(memoryValues) + (-0.05, -0.1)$) {\Large $;$}; 
      \node<-\befPropWriteRightFr>[anchor=east] at (memoryValues) {\Large $\writeReq{x}{0}$}; 
      \node<\propWriteRightFr->[anchor=east] at (memoryValues) {\Large $\writeReq{x}{\textcolor{colorVAL}{1}}$}; 
      \node<-\befPropWriteLeftFr>[anchor=west] at (memoryValues) {\Large $\writeReq{y}{0}$}; 
      \node<\propWriteLeftFr->[anchor=west] at (memoryValues) {\Large $\writeReq{y}{\textcolor{colorVAL}{1}}$}; 

      \node (values) at ($(memoryValues) + (-4.0, 0.0)$) {};
      
      \onslide<\firstFr->
        { \draw[draw,fill=yellow!20] ($(values) + (-1.9,  0.8)$) rectangle
                                     ($(values) + ( 1.9, -0.3)$) ;
          \node[anchor=north] at ($(values) + (0,  0.85)$) {\Large Регистры}; }
      \node at ($(values) + (-0.05, -0.1)$) {\Large $;$}; 
      \node<-\befPropReadLeftFr>[anchor=east] at (values) {\Large $a = \bot$}; 
      \node<\propReadLeftFr->[anchor=east] at (values) {\Large $a = \textcolor{colorVAL}{#1}$}; 
      \node<-\befPropReadRightFr>[anchor=west] at (values) {\Large $b = \bot$}; 
      \node<\propReadRightFr->[anchor=west] at (values) {\Large $b = \textcolor{colorVAL}{#2}$}; 
}

\fr{\huge Исполнение в SC}{
  \setorder{{firstFr, storageAppearFr,pointerAppearFr},
            {propReadLeftFr},
            {propReadRightFr},
            {comRightFr, propWriteRightFr},
            {comLeftFr, propWriteLeftFr},
            {noWeakFr}}
  \prevFr{\befPropWriteLeftFr}{\propWriteLeftFr}
  \prevFr{\befPropReadLeftFr}{\propReadLeftFr}
  \prevFr{\befPropWriteRightFr}{\propWriteRightFr}
  \prevFr{\befPropReadRightFr}{\propReadRightFr}

  \cntrd{
    \begin{tikzpicture}
      \node (leftFirstInst)   {};
      \node (leftSecondInst)  [below of= leftFirstInst, node distance = 0.8cm] {};
      \node (leftThirdInst)   [below of= leftSecondInst, node distance = 0.8cm] {};
      \node (rightFirstInst)  [right of=  leftFirstInst, node distance = 3.0cm] {};
      \node (rightSecondInst) [below of= rightFirstInst, node distance = 0.8cm] {};
      \node (rightThirdInst)  [below of= rightSecondInst, node distance = 0.8cm] {};

      \node (instMiddle)   at ($.5*(leftFirstInst) + .5*(rightFirstInst)$) {};
      \node (leftBuffer)   at (leftSecondInst)  {};
      \node (rightBuffer)  at (rightSecondInst) {};
      \node (middleBuffer) at ($.5*(leftBuffer) + .5*(rightBuffer)$) {};
      \node (finalValues)  at ($(middleBuffer) + (0, -4.5)$) {};

      \onslide<\pointerAppearFr-\befPropReadLeftFr>
        { \ptrPromiseLeft{leftFirstInst} } 
      \onslide<\propReadLeftFr-\befPropWriteLeftFr>
        { \ptrPromiseLeft{leftSecondInst} } 
      \onslide<\propWriteLeftFr->
        { \ptrPromiseLeft{leftThirdInst} } 

      \onslide<\pointerAppearFr-\befPropReadRightFr>
        { \ptrPromiseRight{rightFirstInst} }
      \onslide<\propReadRightFr-\befPropWriteRightFr>
        { \ptrPromiseRight{rightSecondInst} } 
      \onslide<\propWriteRightFr->
        { \ptrPromiseRight{rightThirdInst} } 

      \loadBufferValues{0}{0}

      \node at (leftFirstInst)   {\Large $\readInst{a}{x};$};
      \node at (leftSecondInst)  {\Large $\writeInst{y}{1};$};
      \node at (rightFirstInst)  {\Large $\readInst{b}{y};$};
      \node at (rightSecondInst) {\Large $\writeInst{x}{1};$};

      \draw[-,ultra thick] ($(instMiddle) + (-0.1,0.5)$) -- ($(instMiddle) + (-0.1,-0.5-0.8)$);
      \draw[-,ultra thick] ($(instMiddle) + ( 0.1,0.5)$) -- ($(instMiddle) + ( 0.1,-0.5-0.8)$);
      
      \onslide<\noWeakFr>
        { \draw[fill=red!30,rounded corners=3pt]
            ($(middleBuffer) + (-5.4, -1.0)$) rectangle
            ($(middleBuffer) + ( 5.4, -2.0)$);
          \node at ($(middleBuffer) + (0, -1.5)$)
          { \huge Невозможно получить $a = b = 1$ }; }
    \end{tikzpicture}
  }
}

\phrase{
  \LARGE
  Не-SC исполнения --- {\bf слабые}
  \vfill
  %% \pause
  {\bf Слабые} MM разрешают \\ слабые исполнения 
  \vfill
  %% \pause
  Реальные системы имеют слабые MM \\
  %% \pause
  {\Large (x86, Power, ARM, C++, Java)}
}

\phrase{
  Почему системы имеют слабые MM?
}

\fr{\huge Процессоры и компиляторы \newline {\bf оптимизируют} программы}{
  \setorder{{firstFr, listFr}, {correctFr, oneThreadFr}, {weakFr}}
  \LARGE
  
  \uncover<\listFr->{
    \ite{
    \large
      \item переупорядочивание инструкций \tikzmark{reorder}
      \item кэш
      \item буфера
      \item удаление чтения после записи 
      \item спекулятивное исполнение \tikzmark{fakeelim}
      %% \item удаление антизависимости 
      \item {\Huge \dots} \tikzmark{dots}
    }
  }

  \vfill
  
  \onslide<\weakFr->{
    Приводит к слабым исполнениям
  }
  
  \onslide<\correctFr>{
    \tikz[remember picture, overlay]{
      \draw [decorate,decoration={brace,amplitude=10pt},xshift=-4pt,yshift=0pt,thick]
            let \p1 = (reorder) in
            let \p2 = (dots) in
            let \p3 = (reorder) in
            (\x3, \y1) -- (\x3, \y2)
            node [anchor=west,black,midway,xshift=0.4cm] {\Large Корректны};
    }
  }

  \onslide<\oneThreadFr->{
    \tikz[remember picture, overlay]{
      \draw [decorate,decoration={brace,amplitude=10pt},xshift=-4pt,yshift=0pt,thick]
            let \p1 = (reorder) in
            let \p2 = (dots) in
            let \p3 = (reorder) in
            (\x3, \y1) -- (\x3, \y2)
            node [anchor=west,black,midway,xshift=0.4cm] (test) {\Large Корректны};
      \node [anchor=west,below of=test, node distance=0.45cm,xshift=0.1cm] {\Large для {\bf одного}};
      \node [anchor=west,below of=test, node distance=0.9cm,xshift=-0.5cm] {\Large потока};
    }
  }
}

%% \fr{}{
%% \LARGE
%% $$\begin{array}{c}
%% \uncover<7->{\hspace{12pt} [x] := 0; [y] := 0} \\
%% \begin{array}{l||l}
%% \only<-10,13->{\inarr{
%%   \tikzmark{reorder1} {} [x] := 1;  \\
%%   \tikzmark{reorder2} a := [y]
%% }}
%% \only<11-12>{\inarr{
%%   a := [y]; \\
%%   {} [x] := 1
%% }}
%% \uncover<7->{& \inarr{
%%   {} [y] := 1; \\
%%   b := [x]
%% }}
%% \end{array}\end{array}$$

%% \mycallout<2>{green}{(reorder1)}{(-0.5cm, -0.5cm)}{{\bf Компилятор:} \\ Независимые обращения. \\ Можно переупорядочить.}
%% \mycallout<3>{blue}{(reorder2)}{(-0.5cm, 1.0cm)}{{\bf Процессор:} \\ Независимые обращения. \\ Можно выполнить не по порядку.}

%% \dimalert<5>{Всегда ли корректны такие преобразования?}

%% \only<8-11>{\cntrd{\textcolor{red}{a = b = 0}}}

%% \dimalert<9>{А если переупорядочить?}

%% \only<12->{\cntrd{\textcolor{darkGreen}{a = b = 0}}}
%% \only<14->{\cntrd{\bf \alert{Такое поведение наблюдается для GCC + x86!}}}

%%     \tikz[remember picture, overlay]{ \draw<10,13->[<->,ultra thick] (reorder2) to[out=178,in=182] (reorder1); }
%% }

%% \fr{}{
%%   \huge
%%   Подобные поведения называются {\bf слабыми}\pause,
%%   а семантики --- {\bf слабыми моделями памяти}
%% }

%% \fr{}{

%%   Модели памяти процессоров
%%   \uncover<2->{
%%     \ite{
%%     \item \toGray{7}{8}{x86, \cite{Owens-al:TPHOL09}} \tikzmark{x86}
%%     \item \toGray{7}{8}{Power, \cite{Alglave-al:TOPLAS14}} \tikzmark{Power}
%%     \item ARM,\tikzmark{ARM} \cite{Flur-al:POPL16}
%%     \item ...
%%     }
%%   }
%%   \vfill

%%   Модели памяти ЯП
%%   \uncover<3->{
%%     \ite{
%%     %% \item \toGray{4}{5}{Последовательная консистентность, \cite{Lamport:TC79}}
%%     \item \toGray{4}{5}{C/C++11, \cite{Batty-al:POPL11}}
%%     \item \toGray{4}{5}{Java\tikzmark{jmm}, \cite{Manson-al:POPL05}}
%%     \uncover<5->{ \item Обещающая\tikzmark{Promise} семантика, \cite{Kang-al:POPL17} }
%%     }

%%     \mycallout<4>{red}{(jmm)}{(-2.0cm, 0.5cm)}{Имеют ряд существенных недостатков}
%%   }

%%   %% \uncover<6>{%
%%     \tikz[remember picture, overlay]{
%%       \draw<6-8>[-,ultra thick]
%%         let \p1 = (x86) in
%%         let \p2 = (Power) in
%%         (\x2, \y1) -- (\x2 ++ 0.5cm, \y1) --
%%         (\x2 ++ 0.5cm, \y2) -- (\x2, \y2);

%%         \node<7-8>[above right = -0.8cm and 1.0cm of x86, align=left, fill=green!20, rounded corners, draw,rectangle] (e) 
%%           {Корректность компиляции \\ показана в \cite{Kang-al:POPL17}};
%%         \node<8>[below right = 0.8cm and 1.0cm of x86, align=left, fill=red!20, rounded corners, draw,rectangle] (f) 
%%           {Та же схема доказательства \\ не подходит для ARM!};
%%         \draw<9>[->,red,ultra thick] (Promise) to[out=30,in=-30] (ARM);
%%     }
%%   %% }
%% }
\phrase{
  Модель памяти для ЯП \\
  должна соответствовать \\ \textbf{3} критериям
}

\begin{frame}<6>{\huge 1. Эффективная компиляция}
  \setorder{{firstFr},{programFr}, {scmmFr}, {powermmFr}, {compiledFr}, {noteffFr}}
  %% \LARGE
  %% {\bf Ex.} SC MM requires fences

  \cntrd{
    \begin{tikzpicture}
      \node (leftFirstInst)   {};
      \node (leftSecondInst)  [below of= leftFirstInst, node distance = 0.8cm] {};
      \node (leftThirdInst)   [below of= leftSecondInst, node distance = 0.8cm] {};
      \node (rightFirstInst)  [right of=  leftFirstInst, node distance = 3.0cm] {};
      \node (rightSecondInst) [below of= rightFirstInst, node distance = 0.8cm] {};
      \node (rightThirdInst)  [below of= rightSecondInst, node distance = 0.8cm] {};

      \node (instMiddle)   at ($.5*(leftFirstInst) + .5*(rightFirstInst)$) {};
      
      \uncover<\programFr->{
        \node at (leftFirstInst)   {\Large $\readInst{a}{x};$};
        \node at (leftSecondInst)  {\Large $\writeInst{y}{1};$};
        \node at (rightFirstInst)  {\Large $\readInst{b}{y};$};
        \node at (rightSecondInst) {\Large $\writeInst{x}{1};$};

        \draw[-,ultra thick] ($(instMiddle) + (-0.1,0.5)$) -- ($(instMiddle) + (-0.1,-0.5-0.8)$);
        \draw[-,ultra thick] ($(instMiddle) + ( 0.1,0.5)$) -- ($(instMiddle) + ( 0.1,-0.5-0.8)$);
      }
      
      \uncover<\scmmFr->{
        \node[anchor=west] at
          ($.5*(leftFirstInst) + .5*(leftSecondInst) + (-7.0, 0.0)$) {\large Исходная модель (SC MM)};
      }

      \node (leftFirstInstC)   [below of= leftFirstInst, node distance = 4cm] {};
      \node (leftSecondInstC)  [below of= leftFirstInstC, node distance = 0.8cm] {};
      \node (leftThirdInstC)   [below of= leftSecondInstC, node distance = 0.8cm] {};
      \node (rightFirstInstC)  [right of=  leftFirstInstC, node distance = 3.0cm] {};
      \node (rightSecondInstC) [below of= rightFirstInstC, node distance = 0.8cm] {};
      \node (rightThirdInstC)  [below of= rightSecondInstC, node distance = 0.8cm] {};

      \node (instMiddleC)   at ($.5*(leftFirstInstC) + .5*(rightFirstInstC)$) {};
      
      \uncover<\compiledFr->{
        \node at (leftFirstInstC)   {\Large $\readInst{a}{x};$};
        \node at (leftSecondInstC)  {\Large ${\rm lwsync};$};
        \node at (leftThirdInstC)  {\Large $\writeInst{y}{1};$};
        \node at (rightFirstInstC)  {\Large $\readInst{b}{y};$};
        \node at (rightSecondInstC) {\Large ${\rm lwsync};$};
        \node at (rightThirdInstC) {\Large $\writeInst{x}{1};$};

        \draw[-,ultra thick] ($(instMiddleC) + (-0.1,0.5)$) -- ($(instMiddleC) + (-0.1,-0.5-1.6)$);
        \draw[-,ultra thick] ($(instMiddleC) + ( 0.1,0.5)$) -- ($(instMiddleC) + ( 0.1,-0.5-1.6)$);
      }

      \node<\powermmFr->[anchor=west] at ($(leftSecondInstC) + (-7.0, 0.0)$) {\large Целевая модель (Power MM)};

      \draw<\compiledFr->[->,thick] ($(instMiddle) + (0.0,-1.5)$) -- ($(instMiddleC) + (0.0,1.0)$);
      \node<\noteffFr->[anchor=east] at ($.5*(instMiddle) + .5*(instMiddleC) + (0.0,-0.2)$)
        {\Large \bf \textcolor{darkRed}{Не эффективно!}};
    \end{tikzpicture}
  }
\end{frame}

\begin{frame}<8>{\huge 2. Компиляторные оптимизации}
  \setorder{{firstFr},{programFr}, {scmmFr}, {powermmFr, compiledFr}, {behFr}}

  \cntrd{
    \begin{tikzpicture}
      \node (leftFirstInst)   {};
      \node (leftSecondInst)  [below of= leftFirstInst, node distance = 0.8cm] {};
      \node (leftThirdInst)   [below of= leftSecondInst, node distance = 0.8cm] {};
      \node (rightFirstInst)  [right of=  leftFirstInst, node distance = 3.0cm] {};
      \node (rightSecondInst) [below of= rightFirstInst, node distance = 0.8cm] {};
      \node (rightThirdInst)  [below of= rightSecondInst, node distance = 0.8cm] {};

      \node (instMiddle)   at ($.5*(leftFirstInst) + .5*(rightFirstInst)$) {};
      
      \uncover<\programFr->{
        \node at (leftFirstInst)   {\Large $\readInst{a}{x};$};
        \node at (leftSecondInst)  {\Large $\writeInst{y}{1};$};
        \node at (rightFirstInst)  {\Large $\readInst{b}{y};$};
        \node at (rightSecondInst) {\Large $\writeInst{x}{1};$};

        \draw[-,ultra thick] ($(instMiddle) + (-0.1,0.5)$) -- ($(instMiddle) + (-0.1,-0.5-0.8)$);
        \draw[-,ultra thick] ($(instMiddle) + ( 0.1,0.5)$) -- ($(instMiddle) + ( 0.1,-0.5-0.8)$);
      }
      
      \uncover<\scmmFr->{
        \node[anchor=west] at
          ($.5*(leftFirstInst) + .5*(leftSecondInst) + (-7.0, 0.0)$) {\large Исходная};
      }

      \node (leftFirstInstC)   [below of= leftFirstInst, node distance = 4cm] {};
      \node (leftSecondInstC)  [below of= leftFirstInstC, node distance = 0.8cm] {};
      \node (leftThirdInstC)   [below of= leftSecondInstC, node distance = 0.8cm] {};
      \node (rightFirstInstC)  [right of=  leftFirstInstC, node distance = 3.0cm] {};
      \node (rightSecondInstC) [below of= rightFirstInstC, node distance = 0.8cm] {};
      \node (rightThirdInstC)  [below of= rightSecondInstC, node distance = 0.8cm] {};

      \node (instMiddleC)   at ($.5*(leftFirstInstC) + .5*(rightFirstInstC)$) {};
      
      \uncover<\compiledFr->{
        \node at (leftFirstInstC)   {\Large $\writeInst{y}{1};$};
        \node at (leftSecondInstC)  {\Large $\readInst{a}{x};$};
        \node at (rightFirstInstC)  {\Large $\readInst{b}{y};$};
        \node at (rightSecondInstC) {\Large $\writeInst{x}{1};$};

        \draw[-,ultra thick] ($(instMiddleC) + (-0.1,0.5)$) -- ($(instMiddleC) + (-0.1,-0.5-0.8)$);
        \draw[-,ultra thick] ($(instMiddleC) + ( 0.1,0.5)$) -- ($(instMiddleC) + ( 0.1,-0.5-0.8)$);
      }

      \uncover<\powermmFr->{
        \node[anchor=west] at
          ($.5*(leftFirstInstC) + .5*(leftSecondInstC) + (-7.0, 0.0)$) {\large Оптимизированная};
      }
      
      \uncover<\behFr->{
        \draw [thick] ($(leftFirstInst) + (-0.9, 0.5)$)
          to [square right brace] ($(leftSecondInst) + (-0.9, -0.5)$);
        \draw [thick] ($(leftFirstInst) + (-0.9-0.15, 0.5)$)
          to [square right brace] ($(leftSecondInst) + (-0.9-0.15, -0.5)$);
        \draw [thick] ($(rightFirstInst) + (0.9, 0.5)$)
          to [square left brace] ($(rightSecondInst) + (0.9, -0.5)$);
        \draw [thick] ($(rightFirstInst) + (0.9+0.15, 0.5)$)
          to [square left brace] ($(rightSecondInst) + (0.9+0.15, -0.5)$);

        \draw [thick] ($(leftFirstInstC) + (-0.9, 0.5)$)
          to [square right brace] ($(leftSecondInstC) + (-0.9, -0.5)$);
        \draw [thick] ($(leftFirstInstC) + (-0.9-0.15, 0.5)$)
          to [square right brace] ($(leftSecondInstC) + (-0.9-0.15, -0.5)$);
        \draw [thick] ($(rightFirstInstC) + (0.9, 0.5)$)
          to [square left brace] ($(rightSecondInstC) + (0.9, -0.5)$);
        \draw [thick] ($(rightFirstInstC) + (0.9+0.15, 0.5)$)
          to [square left brace] ($(rightSecondInstC) + (0.9+0.15, -0.5)$);
      }

      \node<\behFr->[rotate=90] at ($.5*(instMiddle) + .5*(instMiddleC) + (0.0,-0.5)$) {\Huge $\subseteq$};
      %% \node<\noteffFr->[anchor=east] at ($.5*(instMiddle) + .5*(instMiddleC) + (0.0,-0.2)$)
      %%   {\Large \bf \textcolor{darkRed}{Not efficient}};
    \end{tikzpicture}
  }
\end{frame}

\begin{frame}<2->{\huge 3. Отсутствие ``значений из воздуха''}
  \setorder{{firstFr}, {programFr}, {ootaFr}}

  \cntrd{
    \begin{tikzpicture}
      \node (leftFirstInst)   {};
      \node (leftSecondInst)  [below of= leftFirstInst, node distance = 0.8cm] {};
      \node (leftThirdInst)   [below of= leftSecondInst, node distance = 0.8cm] {};
      \node (rightFirstInst)  [right of=  leftFirstInst, node distance = 3.0cm] {};
      \node (rightSecondInst) [below of= rightFirstInst, node distance = 0.8cm] {};
      \node (rightThirdInst)  [below of= rightSecondInst, node distance = 0.8cm] {};

      \node (instMiddle)   at ($.5*(leftFirstInst) + .5*(rightFirstInst)$) {};
      
      \uncover<\programFr->{
        \node at (leftFirstInst)   {\Large $\readInst{a}{x};$};
        \node at (leftSecondInst)  {\Large $\writeInst{y}{a};$};
        \node at (rightFirstInst)  {\Large $\readInst{b}{y};$};
        \node at (rightSecondInst) {\Large $\writeInst{x}{b};$};

        \draw[-,ultra thick] ($(instMiddle) + (-0.1,0.5)$) -- ($(instMiddle) + (-0.1,-0.5-0.8)$);
        \draw[-,ultra thick] ($(instMiddle) + ( 0.1,0.5)$) -- ($(instMiddle) + ( 0.1,-0.5-0.8)$);
      }

      \onslide<\ootaFr>
        { \draw[fill=red!30,rounded corners=3pt]
            ($(instMiddle) + (-5.4, -3.0)$) rectangle
            ($(instMiddle) + ( 5.4, -4.0)$);
          \node at ($(instMiddle) + (0, -3.5)$)
          { \huge C/C++11 разрешает $a = b = 8$ }; }
    \end{tikzpicture}
  }
\end{frame}

\begin{frame}<6-7>{\huge MM для языков программирования}
  \setorder{{firstFr},{reqFr},{appearFr},
            {notionEffImplFr, effImplFr},
            {notionComplOptFr, complOptFr},
            {notionHighLevelFr, ootaFr, highLevelFr},
            {ootaCppFr},
            {notionUndefBehFr}, {undefBehFr},
            {promiseFr},{complCorFr}}

  %% \large

  \uncover<\appearFr->{
  \begin{tabular}{@{}l l | l | l | l}
    \uncover<\effImplFr->{& {\bf ЭК}}   \uncover<\complOptFr->{& {\bf КО}}
    \uncover<\highLevelFr->{& Нет {\bf ЗВ}} \uncover<\undefBehFr->{& No {\bf UB}} \\
    %% \tabitem
    SC MM, \cite{Lamport:TC79}
      \uncover<\effImplFr->{& \fail}   \uncover<\complOptFr->{& \fail}
      \uncover<\highLevelFr->{& \tick} \uncover<\undefBehFr->{& \tick} \\
    %% \tabitem
    Java MM\tikzmark{jmm}, \cite{Manson-al:POPL05} \tikzmark{java}
      \uncover<\effImplFr->{& \tick}   \uncover<\complOptFr->{& \fail}
      \uncover<\highLevelFr->{& \tick} \uncover<\undefBehFr->{& \tick} \\
    %% \tabitem
    C/C++11 MM, \cite{Batty-al:POPL11} \tikzmark{cpp}
      \uncover<\effImplFr->{& \tick}   \uncover<\complOptFr->{& \tick$^{*}$}
      \uncover<\highLevelFr->{& \fail\tikzmark{ootaCppMark}} \uncover<\undefBehFr->{& \fail} \\
    \\
    \uncover<\promiseFr->{%
      %% \tabitem
      Proposed solution \cite{Kang-al:POPL17}\tikzmark{Promise}
      \uncover<\effImplFr->{& \tick\tikzmark{effImplMark}}   \uncover<\complOptFr->{& \tick}
      \uncover<\highLevelFr->{& \tick} \uncover<\undefBehFr->{& \tick} \\
        $\quad ~~~\; \Promise$ MM, for C/C++ and Java
    } \\
  \end{tabular}
  }
  %% \vfill
  
  \uncover<\reqFr->{
  Требования:
  \ite{
    \item возможность {\bf Э}ффективной {\bf К}омпиляции \\ (x86, Power, ARM)
    \item разрешение {\bf К}омпиляторных {\bf О}птимизаций
    \item отсутствие {\bf З}начений из {\bf В}оздуха
    %% \uncover<\notionUndefBehFr-> {\item avoid {\bf U}ndefined {\bf B}ehavior}
  }
  }


  \addtooverlay<\ootaCppFr>{%
    \draw[ultra thick, rounded corners=3pt] ($(ootaCppMark) + (-0.35,-0.25)$) rectangle ++(0.5,0.55);
  }
  %% \addtooverlay<\complCorFr>{%
  %%   \draw[colorPROM, ultra thick, rounded corners=3pt] ($(effImplMark) + (-0.45,-0.3)$) rectangle ++(0.6,0.6);
  %% }

 %%  \mycallout<\ootaFr>{yellow}{(oota)}{(-1.0cm, -1.0cm)}{
 %%    $\comment{Out\text{-}Of\text{-}Thin\text{-}Air}$ \\
 %%    $\begin{array}{l || l}
 %%    a := [x]; \comment{8} & b := [y]; \comment{8} \\
 %% {} [y] := a; & [x] := b; \\
 %%    \end{array}$
 %%  }
\end{frame}

\phrase{C/C++11 MM --- {\bf аксиоматическая} MM}

\fr{\huge Исполнения в C/C++11 MM}{
  \setorder{{firstFr}, {lbHighlightFr}, {bigLBFr}, {axiomsFr}, {removeValueFr}, {randomValueFr}}
  \prevFr{\befBigLBFr}{\bigLBFr}
  \prevFr{\befRemoveValueFr}{\removeValueFr}
  \prevFr{\befRandomValueFr}{\randomValueFr}

  \cntrd{
    \vspace{-1.5cm}
    \begin{tikzpicture}[every node/.style={transform shape}]
      \node (leftFirstInst)   {};
      \node (leftSecondInst)  [below of= leftFirstInst, node distance = 0.8cm] {};
      \node (rightFirstInst)  [right of= leftFirstInst, node distance = 3.0cm] {};
      \node (rightSecondInst) [below of=rightFirstInst, node distance = 0.8cm] {};

      \node (instMiddle)   at ($.5*(leftFirstInst) + .5*(rightFirstInst)$) {};
      \node (leftBuffer)   at (leftSecondInst)  {};
      \node (rightBuffer)  at (rightSecondInst) {};
      \node (middleBuffer) at ($.5*(leftBuffer) + .5*(rightBuffer)$) {};
      \node (finalValues)  at ($(middleBuffer) + (0, -4.5)$) {};
      
      %% \node (ptrLeft)  at ($.5*(rightFirstInst) + .5*(rightSecondInst) + (1.5, 0)$) {};
      %% \node (ptrRight) at ($(ptrLeft) + (1.0, 0)$) {};
      \node (ptrLeft)  at ($(instMiddle) + (0.0, -1.7)$) {};
      \node (ptrRight) at ($(ptrLeft) + (0.0, -1.5)$) {};

      %% \node (leftCurlyCenter)  at ($(ptrRight) + (1.0, 0)$) {};
      \node (leftCurlyCenter)  at ($(leftSecondInst) +  (-3.25, -4.0)$) {};
      \node (rightCurlyCenter) at ($(rightSecondInst) + ( 3.25, -4.0)$) {};
      
      \node (curlyTopShift) at (0, -1.5) {};
      \node (curlyBotShift) at (0,  1.5) {};
      \node (leftCurlyTop)  at ($(leftCurlyCenter)  + (curlyTopShift)$) {};
      \node (rightCurlyTop) at ($(rightCurlyCenter) + (curlyTopShift)$) {};

      \node (leftCurlyBot)  at ($(leftCurlyCenter)  + (curlyBotShift)$) {};
      \node (rightCurlyBot) at ($(rightCurlyCenter) + (curlyBotShift)$) {};
      
      \node at (leftFirstInst)   {\Large $\readInst{a}{x};$};
      \node<-\befRemoveValueFr> at (leftSecondInst)  {\Large $\writeInst{y}{1};$};
      \node<\removeValueFr->    at (leftSecondInst)  {\Large $\writeInst{y}{a};$};
      \node at (rightFirstInst)  {\Large $\readInst{b}{y};$};
      \node<-\befRemoveValueFr> at (rightSecondInst) {\Large $\writeInst{x}{1};$};
      \node<\removeValueFr->    at (rightSecondInst) {\Large $\writeInst{x}{b};$};

      \draw[-,thick] ($(instMiddle) + (-0.1,0.5)$) -- ($(instMiddle) + (-0.1,-0.5-0.8)$);
      \draw[-,thick] ($(instMiddle) + ( 0.1,0.5)$) -- ($(instMiddle) + ( 0.1,-0.5-0.8)$);
      
  \uncover<-\befBigLBFr>{
      \draw[->, ultra thick] (ptrLeft) -- (ptrRight);
      \draw [decorate,decoration={brace,amplitude=10pt,raise=4pt},yshift=0pt,thick]
        (leftCurlyTop) -- (leftCurlyBot) node [black,midway,xshift=-0.8cm] {};
      \draw [decorate,decoration={brace,amplitude=10pt,mirror,raise=4pt},yshift=0pt,thick]
        (rightCurlyTop) -- (rightCurlyBot) node [black,midway,xshift=0.8cm] {};

  \node (aGraphCenter) at ($(leftCurlyCenter) + (1.0, 0.0)$) {};
  \node (a1)  at ($(aGraphCenter) + (-0.5, 0.5)$) {$\rlab{}{x}{0}$ };
  \node (a2)  at ($(a1)  + (0, -1)$)  {$\wlab{}{y}{1}$ };
  \node (a11) at ($(a1)  + (1,  0)$)  {$\rlab{}{y}{0}$ };
  \node (a12) at ($(a11) + (0, -1)$)  {$\wlab{}{x}{1}$ };
  \draw[po] (a1)  edge  (a2);
  \draw[po] (a11) edge (a12);
  \draw[fr] (a11)  edge node[below] {\small $\lFR$} (a2);
  \draw[fr] (a1)  edge node[below] {} (a12);
  %% \draw[fr,bend right=30] (12) edge node[below] {\small $\lFR$} (1);

  \node at ($(aGraphCenter) + (1.1, -0.3)$) {\Huge ,};

  \node (bGraphCenter) at ($(aGraphCenter) + (2.5, 0.0)$) {};
  \node (b1)  at ($(bGraphCenter) + (-0.5, 0.5)$) {$\rlab{}{x}{0}$ };
  \node (b2)  at ($(b1)  + (0, -1)$)  {$\wlab{}{y}{1}$ };
  \node (b11) at ($(b1)  + (1,  0)$)  {$\rlab{}{y}{1}$ };
  \node (b12) at ($(b11) + (0, -1)$)  {$\wlab{}{x}{1}$ };
  \draw[po] (b1)  edge  (b2);
  \draw[po] (b11) edge (b12);
  \draw[fr] (b1)  edge node[below] {\small $\lFR$} (b12);
  \draw[rf] (b2)  edge node[above] {\small $\lRF$} (b11);

  \node at ($(bGraphCenter) + (1.1, -0.3)$) {\Huge ,};

  \node (cGraphCenter) at ($(bGraphCenter) + (2.5, 0.0)$) {};
  \node (c1)  at ($(cGraphCenter) + (-0.5, 0.5)$) {$\rlab{}{x}{1}$ };
  \node (c2)  at ($(c1)  + (0, -1)$)  {$\wlab{}{y}{1}$ };
  \node (c11) at ($(c1)  + (1,  0)$)  {$\rlab{}{y}{0}$ };
  \node (c12) at ($(c11) + (0, -1)$)  {$\wlab{}{x}{1}$ };
  \draw[po] (c1)  edge  (c2);
  \draw[po] (c11) edge (c12);
  \draw[fr] (c11) edge node[below] {\small $\lFR$} (c2);
  \draw[rf] (c12) edge node[above] {\small $\lRF$} (c1);

  \node at ($(cGraphCenter) + (1.1, -0.3)$) {\Huge ,};
  

  \node (dGraphCenter) at ($(cGraphCenter) + (2.5, 0.0)$) {};
  \draw<\lbHighlightFr>[colorPROM, ultra thick, rounded corners=3pt] ($(dGraphCenter) + (-1.0,-1.0)$) rectangle ++(2.0,2.0);

  \node (d1)  at ($(dGraphCenter) + (-0.5, 0.5)$) {$\rlab{}{x}{1}$ };
  \node (d2)  at ($(d1)  + (0, -1)$)  {$\wlab{}{y}{1}$ };
  \node (d11) at ($(d1)  + (1,  0)$)  {$\rlab{}{y}{1}$ };
  \node (d12) at ($(d11) + (0, -1)$)  {$\wlab{}{x}{1}$ };
  \draw[po] (d1)  edge (d2);
  \draw[po] (d11) edge (d12);
  \draw[rf] (d2)  edge node[below] {\small $\lRF$} (d11);
  \draw[rf] (d12) edge node[below] {} (d1);
  }
  

  \uncover<\bigLBFr->{
  \node at ($(ptrRight) + (1.9, -1.0)$) {
    \begin{tikzpicture}[scale=2.0,every node/.style={transform shape}]
  \node (dGraphCenter) at ($(0.0, 0.0)$) {};
  
  \uncover<-\befRandomValueFr>{
    \node (d1)  at ($(dGraphCenter) + (-0.5, 0.5)$) {$\rlab{}{x}{1}$ };
    \node (d2)  at ($(d1)  + (0, -1)$)  {$\wlab{}{y}{1}$ };
    \node (d11) at ($(d1)  + (1,  0)$)  {$\rlab{}{y}{1}$ };
    \node (d12) at ($(d11) + (0, -1)$)  {$\wlab{}{x}{1}$ };
  }

  \uncover<\randomValueFr->{
    \node (d1)  at ($(dGraphCenter) + (-0.5, 0.5)$) {$\rlab{}{x}{8}$ };
    \node (d2)  at ($(d1)  + (0, -1)$)  {$\wlab{}{y}{8}$ };
    \node (d11) at ($(d1)  + (1,  0)$)  {$\rlab{}{y}{8}$ };
    \node (d12) at ($(d11) + (0, -1)$)  {$\wlab{}{x}{8}$ };
  }
  \draw[po, ultra thick] (d1)  edge node[left]  {\small $\lPO$} (d2);
  \draw[po, ultra thick] (d11) edge node[right] {\small $\lPO$} (d12);
  \draw[rf, ultra thick] (d2)  edge node[below] {\small $\lRF$} (d11);
  \draw[rf, ultra thick] (d12) edge node[below] {} (d1);
    \end{tikzpicture}
    };
  }
    
  \uncover<\axiomsFr->{
  \node[anchor=west] at ($(ptrRight) + (-5.5,  0.0)$)
    {\LARGE Аксиомы:};
  \node[anchor=west] at ($(ptrRight) + (-5.5, -1.0)$)
    {\LARGE 1. $\lPO|_{\mathtt{loc}} \cup \lRF$};
  \node[anchor=west] at ($(ptrRight) + (-4.8, -2.0)$)
    {\LARGE ациклично};
  \node[anchor=west] at ($(ptrRight) + (-5.5, -3.0)$)
    {\LARGE \dots};
  }

    \end{tikzpicture}
  }
}

\begin{frame}{\huge Постановка задачи}
   \setorder{{opCppFr},{promiseFr},{armPOPSectionFr},{armAxSectionFr}}
\begin{itemize}
  \Large

  \uncover<\opCppFr->{
  \item Разработать операционную модель памяти С/С++11, свободную от проблемы ``значений из воздуха''.
  }
  \vfill

    \only<\promiseFr>{
      \begin{block}{Обещающая семантика \cite{Kang-al:POPL17}}
        %% \cntrd{  }
        %% \uncover<\promisePropertiesFr->{
          \begin{itemize}
            \item Операционная
            \item Доказана корректность компиляции в MM x86 и Power
            \item Разрешает оптимизации
            \item Не имеет ``значений из воздуха''
            %% \uncover<\promisePerspectiveFr->{
              \item Может стать частью стандартов C/C++ и Java
            %% }
            %% \uncover<\promiseARMHoleFr->{
              %% \vfill 
              \item \alert{Нет д-ва корректности компиляции в MM ARM}!
            %% }
          \end{itemize}
        
      \end{block}
    }

  \uncover<\armPOPSectionFr->{
  \item Доказать корректность эффективной схемы компиляции из существенного подмножества обещающей модели (ККОМ) в операционную модель памяти ARMv8 POP.
  }

  \vfill
  \uncover<\armAxSectionFr->{
  \item Доказать ККОМ в аксиоматическую модель памяти ARMv8.3.
  }
\end{itemize}
\end{frame}

%% \fr{\huge Постановка задачи}{
%% \Large
%% \begin{block}{Проблема}
%%   Не существует MM промышленного ЯП, удовлетворяющей основным критериям
%% \end{block}
%% \pause
%% \begin{block}{Цель}
%%   Разработать операционный аналог C/C++11 MM без значений из воздуха, \\ удовлетворяющий основным критериям
%% \end{block}
%% }

%% \section{Описание результатов}

\begin{frame}<8>[label=resultsFrame]{\huge Результаты}
  \setorder{{firstFr},{opCppFr},
    {promiseDiscussionFr},{promisePropertiesFr},{promisePerspectiveFr},{promiseARMHoleFr},
    {compilationFr},{compilationAxFr},
    {opCppSectionFr},{armPOPSectionFr},{armAxSectionFr},{fullResultFr},{interpreterLinkFr},{thanksFr}}
  \prevFr{\befCompilationFr}{\compilationFr}
  \nextFr{\nextPromisePropertiesFr}{\promisePropertiesFr}

  \prevFr{\befOpCppSectionFr}{\opCppSectionFr}
  \nextFr{\nextArmAxSectionFr}{\armAxSectionFr}

  \large
  \begin{itemize}
    \uncover<\opCppFr->{\item
      \toShadow
        {-\befOpCppSectionFr,\opCppSectionFr,\nextArmAxSectionFr-}
        {\armPOPSectionFr,\armAxSectionFr}
      {Предложена операционная модель памяти C/С++11, для этой модели реализован интерпретатор.}
    } \\
    \only<\interpreterLinkFr->{ {\scriptsize [Интерпретатор: \url{github.com/anlun/OperationalSemanticsC11}]} }
    \vfill
    
    \only<\promiseDiscussionFr-\promiseARMHoleFr>{
      \begin{block}{Конкурирующая работа}
        \cntrd{ Обещающая семантика \cite{Kang-al:POPL17} }
        \uncover<\promisePropertiesFr->{
          \begin{itemize}
            \item Операционная
            \item Доказана корректность компиляции в MM x86 и Power
            \item Разрешает оптимизации
            \item Не имеет значений из воздуха
            \uncover<\promisePerspectiveFr->{
              \item Может стать частью стандартов C/C++ и Java
            }
            \uncover<\promiseARMHoleFr->{
              \vfill 
              \item \alert{Нет д-ва корректности компиляции в MM ARM}!
            }
          \end{itemize}
        }
      \end{block}
    }
    
    \uncover<\compilationFr->{
      \item 
        \toShadow
          {-\befOpCppSectionFr,\armPOPSectionFr,\nextArmAxSectionFr-}
          {\opCppSectionFr,\armAxSectionFr}
        {Доказана корректность компиляции из существенного подмножества\tikzmark{subset} обещающей модели (ККОМ) в операционную модель памяти \ARMpop.}
    }
    \vfill
    \uncover<\compilationAxFr->{
      %% \item
      %%   \toShadow
      %%     {-\befOpCppSectionFr,\armAxSectionFr,\nextArmAxSectionFr-}
      %%     {\opCppSectionFr,\armPOPSectionFr}
      %%   {Разработан метод доказательства ККОМ в аксиоматические модели}
      %% \vfill
      \item
        \toShadow
          {-\befOpCppSectionFr,\armAxSectionFr,\nextArmAxSectionFr-}
          {\opCppSectionFr,\armPOPSectionFr}
        {Доказана ККОМ в аксиоматическую модель памяти $\ARMax$.}
    }
  \end{itemize}

  %% \mycallout<2>[left]{green}{(subset)}{(-0.5cm, 1.0cm)}{Расслабленные (relaxed) \\ чтения и записи,\\
  %%                                                       высвобождающие (release) и \\
  %%                                                       приобретающие (acquire) \\ барьеры памяти}

  \uncover<\thanksFr->{\cntrd{\huge \textcolor{red}{Спасибо!}}}
\end{frame}

\againframe<9>{resultsFrame}

\fri{\Large Операционная модель \OpCpp}{
  \large
  \item Недетерминированная память \\
        (метки времени, фронты)
  \vfill
  \item Отложенные операции
  \vfill
  \hrule
  \vfill
  \item Поддержаны основные конструкции C/C++11 MM\\
    (relaxed, release, acquire, SC, CAS, consume)
    %% \textcolor{gray!50!white}{(relaxed, release, acquire, SC, fences, CAS)}
  \vfill
  \item Интерпретатор
  \begin{itemize}
    \large
    \item Реализован на Racket с помощью PLT/Redex
    \item \url{github.com/anlun/OperationalSemanticsC11}
  \end{itemize}
  \vfill
  \item Апробирована на 41 тесте
  \begin{itemize}
    \large
    \item Совпадает с C/C++11 MM на 36 тестах
    \item \tick: 2 теста --- ``значения из воздуха''
    \item \fail: 3 теста --- синтаксическое ограничение
  \end{itemize}
}

\againframe<10>{resultsFrame}

\fr{\Huge Корректность компиляции}{
  \LARGE

  $\onslide<3->{compile :} \tikzmark{proglang}S \onslide<3->{\rightarrow} \tikzmark{hardlang}\onslide<2->{T}$\\
  \onslide<4->{
  $\onslide<6->{\forall Prog \in S.} \\$
  $\onslide<5->{\quad \semState{\onslide<6->{compile(\tikzmark{hardsem}Prog)}}{T}}$
  $\onslide<6->{\subseteq}$
  $\semState{\onslide<6->{Prog}\tikzmark{progsem}}{S}$.
  }

  \mycallout<1>{green}{(proglang)}{(0.0cm, 1.0cm)}{Исходный язык}
  \mycallout<2>{green}{(hardlang)}{(0.0cm, 1.0cm)}{Целевой язык}

  \mycallout<4>{green}{(progsem)}{(0.0cm, 1.0cm)}{Исходная MM}
  \mycallout<5>{green}{(hardsem)}{(0.0cm, 1.0cm)}{Целевая MM}
}


\fr{\large Структура д-ва корректности компиляции из обещающей модели $\Promise$ в модель \ARMpop}{
  %% \Large
  \iteN{
    \item Вводится инструментированная ограниченная версия модели $\ARMpop$, т.н. $\ARMt$.
    \vfill
    \item Доказывается эквивалентность $\ARMt$ и $\ARMpop$.
    \[\begin{array}{l}
      \forall P. \; \semState{P}{\ARMt} = \semState{P}{\ARMpop}.
    \end{array}\]
    \vfill
    \item Доказывается симуляция $\ARMt$ обещающей моделью. \\
    \vspace{.3cm}
    { \footnotesize
    \textbf{Теорема 1.} $S({\sf init}_{\ARMt}, {\sf init}_{\Promise}) $. \\
    \textbf{Теорема 2.} $\forall arm, arm \to arm', promise, S(arm, promise).$ \\
    $\quad \exists promise', promise \to promise' \land S(arm', promise').$
    }
  }
}

\againframe<11>{resultsFrame}

\fr{\large Структура д-ва корректности компиляции из обещающей модели в аксиоматическую модель \ARMax}{
  \iteN{
    \item Вводится операционная семантика обхода \ARMax-исполнения $G$. \\
    \vspace{.5cm}
    { \footnotesize
    \begin{minipage}[t]{0.5\textwidth}
    \begin{prooftree}
      \Hypo{a \in \nextset(G, C) \cap \coverable(G, C, I)}
      \Infer1{G \vdash \tup{C, I} \travConfigStep \tup{C \cup \{a\}, I}}
    \end{prooftree}
    \end{minipage}
    \begin{minipage}[t]{0.4\textwidth}
    \begin{prooftree}
      \Hypo{w \in \issuable(\tikzmark{issuableMk}G, C, I) \setminus I}
      \Infer1{G \vdash \tup{C, \IssuedSet} \travConfigStep \tup{C, \IssuedSet \cup \{w\}}}
    \end{prooftree}
    \end{minipage}
    }
    \vspace{.3cm}
    \item Доказывается полнота обхода.
    { \footnotesize
    \[\forall P, G. \; G \in \semState{P}{\ARMax} \Rightarrow G \vdash \tup{G.W_{init}, G.W_{init}} \to^{*} \tup{G.E, G.W}.\]
    }
    \item Показывается, что обещающая модель симулирует обход \ARMax-исполнения. \\
    \vspace{.3cm}
    { \footnotesize
    \textbf{Теорема 1.} $S_{G}(\tup{G.W_{init}, G.W_{init}}, {\sf init}_{\Promise}) $. \\
    \textbf{Теорема 2.} $\forall G \vdash \tup{C, I} \to \tup{C', I'}, promise, S_{G}(\tup{C, I}, promise).$ \\
    $\quad \exists promise', promise \to promise' \land S_{G}(\tup{C', I'}, promise').$
    }
  }
}

%% \section{Итоги}

\fri{\Large Новизна результатов}{
\large
  \item Предложенная модель памяти для C/C++11 является запускаемой, что отличает её от обещающей модели.
  \vfill
  \item Доказательство ККОМ в модель $\ARMax$ не опирается на специфические св-ва целевой модели, что
        отличает его от существующих доказательств для моделей x86 и Power.
  \vfill
  \item Доказательства ККОМ в модели $\ARMpop$ и $\ARMax$ являются первыми результатами о компиляции для данных моделей.
}

\fri{\Large Теоретическая и практическая значимость работы}{
  \large
  \item Предложен операционный способ представления реалистичной семантики многопоточности с помощью меток времени и фронтов.
  \vfill
  \item Приводится метод доказательства корректности компиляции из обещающей в аксиоматические модели памяти,
  который может быть использован для последующих доказательств.
  \vfill
  \item Доказательство корректности эффективной компиляции для архитектуры ARM является
  необходимым аргументом в пользу обещающей модели памяти как новой модели памяти для C/C++ и Java.
}

\fr{\LARGE Публикации}{
  \scriptsize
  
  \begin{block}{\scriptsize ВАК}
    \begin{itemize}
      \item Подкопаев, А. В.
      О корректности компиляции подмножества обещающей модели памяти в аксиоматическую модель ARMv8.3 /
      А.В. Подкопаев, О. Лахав, В. Вафеядис. // НТВ СПбГПУ ИТУ.---2017.---Т.10,№4.---С. 51-69.
      \item Подкопаев, А. В. Обещающая компиляция в ARMv8.3 /
      А. Подкопаев, О. Лахав, В. Вафеядис // Труды ИСП РАН.---2017.---Т.29,№5.---С. 149-164.
    \end{itemize}
  \end{block}

  \begin{block}{\scriptsize SCOPUS и Web of Science}
    \begin{itemize}
      \item Podkopaev, A. Promising compilation to ARMv8 POP / \\
      A. Podkopaev, O. Lahav, V. Vafeiadis //
      ECOOP 17, LIPIcs.---2017.---P. 22:1--22:28.
    \end{itemize}
  \end{block}

  \begin{block}{\scriptsize РИНЦ}
    \begin{itemize}
      \item Подкопаев, А. В. Обещающая компиляция в ARMv8 / \\
      А.В. Подкопаев, О. Лахав, В. Вафеядис //
      Языки программирования и компиляторы. Труды конференции.---2017.---C. 223--226. \\
      \item Podkopaev, A.
      Operational Aspects of C/C++ Concurrency / \\
      A. Podkopaev, I. Sergey, A. Nanevski [Электронный ресурс].---URL:
      \url{http://arxiv.org/abs/1606.01400}.
    \end{itemize}
  \end{block}
}

\againframe<13->{resultsFrame}

\begin{frame}[allowframebreaks]{Ссылки}
\bibliographystyle{apalike}
\scriptsize
\bibliography{main}
\end{frame}

\phrase{Дополнительные слайды}
\appendix
%%% Оформление заголовков приложений ближе к ГОСТ:
\setlength{\midchapskip}{20pt}
%% \renewcommand*{\afterchapternum}{\par\nobreak\vskip \midchapskip}
\renewcommand\thechapter{\Asbuk{chapter}} % Чтобы приложения русскими буквами нумеровались
   % Предварительные настройки для правильного подключения Приложений
%% \begin{figure*}[t]

\newcommand{\litmusTestStart}[3]{
\begin{minipage}[t]{0.2\linewidth}
\textbf{#1} \\
Fully Supported: #2 \\
Requires: #3\\
\end{minipage}
}
\newcommand{\litmusTestEnd}{
\vspace{.2cm}
\hrule
\vspace{.2cm}
}

\newcommand{\tickP}{\checkmark}%
\newcommand{\tickPP}{\checkmark}%

\newcommand{\sbTemplate}[6]{
\begin{minipage}[t]{0.3\linewidth}
\vspace{-.2cm}
\begin{equation*}
\begin{tabular}{c}
  $\writeInstParam{#1}{x}{0}; \writeInstParam{#2}{y}{0};$ \\
\begin{tabular}{L || L}
  \writeInstParam{#3}{x}{1}; & \writeInstParam{#5}{y}{1}; \\
  \readInstParam{#4}{a}{y} & \readInstParam{#6}{b}{x} \\
\end{tabular}
\end{tabular}
\end{equation*}
\end{minipage}
}

\newcommand{\lbTemplate}[6]{
\begin{minipage}[t]{0.3\linewidth}
\vspace{-.2cm}
\begin{equation*}
\begin{tabular}{c}
  $\writeInstParam{#1}{x}{0}; \writeInstParam{#2}{y}{0};$ \\
\begin{tabular}{L || L}
  \readInstParam{#3}{a}{y} & \readInstParam{#5}{b}{x} \\
  \writeInstParam{#4}{x}{1}; & \writeInstParam{#6}{y}{1}; \\
\end{tabular}
\end{tabular}
\end{equation*}
\end{minipage}
}

\chapter{Каталог тестов для модели C/C++11}
\label{sec:litmusTests}

\section{Store Buffering (SB)}
\label{app:sb}

\litmusTestStart{SB\_rel+acq}{\tick}{History + Viewfronts}
%
\begin{minipage}[t]{0.3\linewidth}
Possible outcomes:\\
\[\begin{array}{l}
a = 0 \land b = 0\\
a = 0 \land b = 1\\
a = 1 \land b = 0\\
a = 1 \land b = 1\\
\end{array}\]
\end{minipage}
%
\sbTemplate{\rel}{\rel}{\rel}{\acq}{\rel}{\acq}
\litmusTestEnd


\litmusTestStart{SB\_sc}{\tick}{SC + History + Viewfronts}
\begin{minipage}[t]{0.3\linewidth}
Forbidden outcomes:\\
\[\begin{array}{l}
a = 0 \land b = 0\\
\end{array}\]
\end{minipage}
%
\sbTemplate{\sco}{\sco}{\sco}{\sco}{\sco}{\sco}
\litmusTestEnd

\litmusTestStart{SB\_sc+rel}{\tick}{SC + History + Viewfronts}
\begin{minipage}[t]{0.3\linewidth}
Possible outcomes:\\
\[\begin{array}{l}
a = 0 \land b = 0\\
a = 0 \land b = 1\\
a = 1 \land b = 0\\
a = 1 \land b = 1\\
\end{array}\]
\end{minipage}
%
\sbTemplate{\sco}{\sco}{\rel}{\sco}{\sco}{\sco}
\litmusTestEnd

\litmusTestStart{SB\_sc+acq}{\tick}{SC + History + Viewfronts}
\begin{minipage}[t]{0.3\linewidth}
Possible outcomes:\\
\[\begin{array}{l}
a = 0 \land b = 0\\
a = 0 \land b = 1\\
a = 1 \land b = 0\\
a = 1 \land b = 1\\
\end{array}\]
\end{minipage}
%
\sbTemplate{\sco}{\sco}{\sco}{\acq}{\sco}{\sco}
\litmusTestEnd

%\newpage

\section{Load Buffering (LB)}
\label{app:lb}

\litmusTestStart{LB\_rlx}{\tick}{Postponed Reads + History + Viewfronts}
\begin{minipage}[t]{0.3\linewidth}
Possible outcomes:\\
\[\begin{array}{l}
a = 0 \land b = 0\\
a = 0 \land b = 1\\
a = 1 \land b = 0\\
a = 1 \land b = 1\\
\end{array}\]
\end{minipage}
%
\lbTemplate{\rlx}{\rlx}{\rlx}{\rlx}{\rlx}{\rlx}
\litmusTestEnd

\litmusTestStart{LB\_rel+rlx}{\tick}{Postponed Reads + History + Viewfronts}
\begin{minipage}[t]{0.3\linewidth}
Possible outcomes:\\
\[\begin{array}{l}
a = 0 \land b = 0\\
a = 0 \land b = 1\\
a = 1 \land b = 0\\
a = 1 \land b = 1\\
\end{array}\]
\end{minipage}
%
\lbTemplate{\rlx}{\rlx}{\rlx}{\rel}{\rlx}{\rel}
\litmusTestEnd

\litmusTestStart{LB\_acq+rlx}{\fail}{Postponed Reads + History + Viewfronts}
\begin{minipage}[t]{0.3\linewidth}
Possible outcomes:\\
\[\begin{array}{l}
a = 0 \land b = 0\\
a = 0 \land b = 1\\
a = 1 \land b = 0\\
a = 1 \land b = 1\\
\end{array}\]
\end{minipage}
%
\lbTemplate{\rlx}{\rlx}{\acq}{\rlx}{\acq}{\rlx}

Our semantics doesn't allow the $a = 1 \land b = 1$ outcome for the program.
It doesn't allow reordering of an acquire read with a subsequent write.
The known sound compilation schemes of acquire read to major platforms (x86, ARM, Power) don't
allow the behavior either. 

\litmusTestEnd

\litmusTestStart{LB\_rel+acq+rlx}{\tick}{Postponed Reads + History + Viewfronts}
\begin{minipage}[t]{0.3\linewidth}
Forbidden outcomes:\\
\[\begin{array}{l}
a = 1 \land b = 1\\
\end{array}\]
\end{minipage}
%
\lbTemplate{\rlx}{\rlx}{\acq}{\rlx}{\rlx}{\rel}
\litmusTestEnd

\litmusTestStart{LB\_rlx+use}{\tick}{Postponed Reads + History + Viewfronts}
\begin{minipage}[t]{0.3\linewidth}
Allowed outcome:\\
\lstinline{  r1 = 1 |$\land$| r2 = 1}\\
\end{minipage}
%
\begin{minipage}[t]{0.3\linewidth}
\vspace{-.2cm}
  \begin{tabular}{l@{\ \ \ }l}
    \begin{minipage}[l]{4.3cm} \small
\begin{lstlisting}
  |$[$|x|$]_{rlx}$| := 0; |$[$|y|$]_{rlx}$| := 0;
\end{lstlisting}
\vspace{-.2cm}
\begin{tabular}{l||l}
\begin{lstlisting}
r1 = |$[$|y|$]_{rlx}$|;
|$[$|z1|$]_{rlx}$| := r1;
|$[$|x|$]_{rlx}$| := 1
\end{lstlisting}
\hspace{.6cm}
&
\begin{lstlisting}
r2 = |$[$|x|$]_{rlx}$|;
|$[$|z2|$]_{rlx}$| := r2;
|$[$|y|$]_{rlx}$| := 1
\end{lstlisting}
\end{tabular}
    \end{minipage}
&
  \end{tabular}
\end{minipage}

%% Comment:
%% Unfortunately, our semantics isn't able to reorder writes,
%% thus, we can't store 1s to z1 and z2.
\litmusTestEnd

\litmusTestStart{LB\_rlx+let}{\tick}{Postponed Reads + History + Viewfronts}
\begin{minipage}[t]{0.3\linewidth}
Allowed outcome:\\
\lstinline{  r1 = 1 |$\land$| r'1 = 2 |$\land$| r2 = 1 |$\land$| r'2 = 2}\\
\end{minipage}
%
\begin{minipage}[t]{0.3\linewidth}
\vspace{-.2cm}
  \begin{tabular}{l@{\ \ \ }l}
    \begin{minipage}[l]{4.3cm} \small
\begin{lstlisting}
  |$[$|x|$]_{rlx}$| := 0; |$[$|y|$]_{rlx}$| := 0;
\end{lstlisting}
\vspace{-.2cm}
\begin{tabular}{l||l}
\begin{lstlisting}
r1 = |$[$|y|$]_{rlx}$|;
r'1 = r1 + 1;
|$[$|x|$]_{rlx}$| := 1
\end{lstlisting}
\hspace{.6cm}
&
\begin{lstlisting}
r2 = |$[$|x|$]_{rlx}$|;
r'2 = r2 + 1;
|$[$|y|$]_{rlx}$| := 1
\end{lstlisting}
\end{tabular}
    \end{minipage}
&
  \end{tabular}
\end{minipage}
\litmusTestEnd

\litmusTestStart{LB\_rlx+join}{\tickPP}{Postponed Reads + History + Viewfronts + JN}
\begin{minipage}[t]{0.2\linewidth}
Allowed outcomes:\\
\lstinline{  r1 = 1 |$\land$| r2 = 1}\\
\end{minipage}
%
\begin{minipage}[t]{0.4\linewidth}
\vspace{-.2cm}
  \begin{tabular}{l@{\ \ \ }l}
    \begin{minipage}[l]{4.3cm} \small
\begin{lstlisting}
            |$[$|x|$]_{rlx}$| := 0; |$[$|y|$]_{rlx}$| := 0;
\end{lstlisting}
\vspace{-.2cm}
\begin{tabular}{l||l||l||l}
\begin{lstlisting}
r1 = |$[$|y|$]_{rlx}$|;
|$[$|z1|$]_{rlx}$| := r1
\end{lstlisting}
\hspace{.6cm}
&
\begin{lstlisting}
  0
\end{lstlisting}
\hspace{.6cm}
&
\begin{lstlisting}
r2 = |$[$|x|$]_{rlx}$|;
|$[$|z2|$]_{rlx}$| := r2
\end{lstlisting}
\hspace{.6cm}
&
\begin{lstlisting}
  0
\end{lstlisting}
\end{tabular}

\vspace{-1pt}
\begin{tabular}{l||l}
  \begin{lstlisting}
            |$[$|x|$]_{rlx}$| := 1
  \end{lstlisting}
%% \hspace{.708cm}
\hspace{2.52em}
&
  \begin{lstlisting}
      |$[$|y|$]_{rlx}$| := 1
  \end{lstlisting}
\end{tabular}
    \end{minipage}
&
  \end{tabular}
\end{minipage}

%% Unfortunately, our semantics isn't complete for this snippet,
%% because it can't postpone write, or anything across a join point.
\litmusTestEnd

\litmusTestStart{LB\_rel+rlx+join}{\tickPP}{Postponed Reads + History + Viewfronts + JN}
\begin{minipage}[t]{0.2\linewidth}
Allowed outcomes:\\
\lstinline{  r1 = 1 |$\land$| r2 = 1}\\
\end{minipage}
%
\begin{minipage}[t]{0.4\linewidth}
\vspace{-.2cm}
  \begin{tabular}{l@{\ \ \ }l}
    \begin{minipage}[l]{4.3cm} \small
\begin{lstlisting}
            |$[$|x|$]_{rlx}$| := 0; |$[$|y|$]_{rlx}$| := 0;
\end{lstlisting}
\vspace{-.2cm}
\begin{tabular}{l||l||l||l}
\begin{lstlisting}
r1 = |$[$|y|$]_{rlx}$|;
|$[$|z1|$]_{rlx}$| := r1
\end{lstlisting}
\hspace{.6cm}
&
\begin{lstlisting}
  0
\end{lstlisting}
\hspace{.6cm}
&
\begin{lstlisting}
r2 = |$[$|x|$]_{rlx}$|;
|$[$|z2|$]_{rlx}$| := r2
\end{lstlisting}
\hspace{.6cm}
&
\begin{lstlisting}
  0
\end{lstlisting}
\end{tabular}

\vspace{-1pt}
\begin{tabular}{l||l}
  \begin{lstlisting}
            |$[$|x|$]_{rel}$| := 1
  \end{lstlisting}
%% \hspace{.708cm}
\hspace{2.52em}
&
  \begin{lstlisting}
      |$[$|y|$]_{rel}$| := 1
  \end{lstlisting}
\end{tabular}
    \end{minipage}
&
  \end{tabular}
\end{minipage}
\litmusTestEnd

\litmusTestStart{LB\_acq+rlx+join}{\fail}{Postponed Reads + History + Viewfronts + JN}
\begin{minipage}[t]{0.2\linewidth}
Allowed outcomes:\\
\lstinline{  r1 = 1 |$\land$| r2 = 1}\\
\end{minipage}
%
\begin{minipage}[t]{0.4\linewidth}
\vspace{-.2cm}
  \begin{tabular}{l@{\ \ \ }l}
    \begin{minipage}[l]{4.3cm} \small
\begin{lstlisting}
            |$[$|x|$]_{rlx}$| := 0; |$[$|y|$]_{rlx}$| := 0;
\end{lstlisting}
\vspace{-.2cm}
\begin{tabular}{l||l||l||l}
\begin{lstlisting}
r1 = |$[$|y|$]_{acq}$|;
|$[$|z1|$]_{rlx}$| := r1
\end{lstlisting}
\hspace{.6cm}
&
\begin{lstlisting}
  0
\end{lstlisting}
\hspace{.6cm}
&
\begin{lstlisting}
r2 = |$[$|x|$]_{acq}$|;
|$[$|z2|$]_{rlx}$| := r2
\end{lstlisting}
\hspace{.6cm}
&
\begin{lstlisting}
  0
\end{lstlisting}
\end{tabular}

\vspace{-1pt}
\begin{tabular}{l||l}
  \begin{lstlisting}
            |$[$|x|$]_{rlx}$| := 1
  \end{lstlisting}
%% \hspace{.708cm}
\hspace{2.52em}
&
  \begin{lstlisting}
      |$[$|y|$]_{rlx}$| := 1
  \end{lstlisting}
\end{tabular}
    \end{minipage}
&
  \end{tabular}
\end{minipage}
\litmusTestEnd

\section{Message Passing (MP)}
\label{app:mp}

\litmusTestStart{MP\_rlx+na}{\tick}{NA + History + Viewfronts}
\begin{minipage}[t]{0.3\linewidth}
Possible outcomes:\\
\lstinline{  r1 = 0}\\
\lstinline{  r1 = 5}\\
\lstinline{  stuck}\\
\end{minipage}
%
\begin{minipage}[t]{0.3\linewidth}
\vspace{-.2cm}
  \begin{tabular}{l@{\ \ \ }l}
    \begin{minipage}[l]{4.3cm} \small
\begin{lstlisting}
        |$[$|f|$]_{rlx}$| := 0; |$[$|d|$]_{na}$| := 0;
\end{lstlisting}
\vspace{-.2cm}
\begin{tabular}{l||l}
\begin{lstlisting}
|$[$|d|$]_{na}$| := 5;
|$[$|f|$]_{rlx}$| := 1
\end{lstlisting}
\hspace{.6cm}
&
\begin{lstlisting}
repeat |$[$|f|$]_{rlx}$| end;
r1 = |$[$|d|$]_{na}$|
\end{lstlisting}
\end{tabular}
    \end{minipage}
&
  \end{tabular}
\end{minipage}
\litmusTestEnd

\litmusTestStart{MP\_rel+rlx+na}{\tick}{NA + History + Viewfronts}
\begin{minipage}[t]{0.3\linewidth}
Possible outcomes:\\
\lstinline{  r1 = 0}\\
\lstinline{  r1 = 5}\\
\lstinline{  stuck}\\
\end{minipage}
%
\begin{minipage}[t]{0.3\linewidth}
\vspace{-.2cm}
  \begin{tabular}{l@{\ \ \ }l}
    \begin{minipage}[l]{4.3cm} \small
\begin{lstlisting}
        |$[$|f|$]_{rlx}$| := 0; |$[$|d|$]_{na}$| := 0;
\end{lstlisting}
\vspace{-.2cm}
\begin{tabular}{l||l}
\begin{lstlisting}
|$[$|d|$]_{na}$| := 5;
|$[$|f|$]_{rel}$| := 1
\end{lstlisting}
\hspace{.6cm}
&
\begin{lstlisting}
repeat |$[$|f|$]_{rlx}$| end;
r1 = |$[$|d|$]_{na}$|
\end{lstlisting}
\end{tabular}
    \end{minipage}
&
  \end{tabular}
\end{minipage}
\litmusTestEnd

\litmusTestStart{MP\_rlx+acq+na}{\tick}{NA + History + Viewfronts}
\begin{minipage}[t]{0.3\linewidth}
Possible outcomes:\\
\lstinline{  r1 = 0}\\
\lstinline{  r1 = 5}\\
\lstinline{  stuck}\\
\end{minipage}
%
\begin{minipage}[t]{0.4\linewidth}
\vspace{-.2cm}
  \begin{tabular}{l@{\ \ \ }l}
    \begin{minipage}[l]{4.3cm} \small
\begin{lstlisting}
        |$[$|f|$]_{rlx}$| := 0; |$[$|d|$]_{na}$| := 0;
\end{lstlisting}
\vspace{-.2cm}
\begin{tabular}{l||l}
\begin{lstlisting}
|$[$|d|$]_{na}$| := 5;
|$[$|f|$]_{rlx}$| := 1
\end{lstlisting}
\hspace{.6cm}
&
\begin{lstlisting}
repeat |$[$|f|$]_{acq}$| end;
r1 = |$[$|d|$]_{na}$|
\end{lstlisting}
\end{tabular}
    \end{minipage}
&
  \end{tabular}
\end{minipage}
\litmusTestEnd

\litmusTestStart{MP\_rel+acq+na}{\tick}{NA + History + Viewfronts}
\begin{minipage}[t]{0.3\linewidth}
Possible outcomes:\\
\lstinline{  r1 = 5}\\
\end{minipage}
%
\begin{minipage}[t]{0.3\linewidth}
\vspace{-.2cm}
  \begin{tabular}{l@{\ \ \ }l}
    \begin{minipage}[l]{4.3cm} \small
\begin{lstlisting}
        |$[$|f|$]_{rel}$| := 0; |$[$|d|$]_{na}$| := 0;
\end{lstlisting}
\vspace{-.2cm}
\begin{tabular}{l||l}
\begin{lstlisting}
|$[$|d|$]_{na}$| := 5;
|$[$|f|$]_{rel}$| := 1
\end{lstlisting}
\hspace{.6cm}
&
\begin{lstlisting}
repeat |$[$|f|$]_{acq}$| end;
r1 = |$[$|d|$]_{na}$|
\end{lstlisting}
\end{tabular}
    \end{minipage}
&
  \end{tabular}
\end{minipage}
\litmusTestEnd

\litmusTestStart{MP\_rel+acq+na+rlx}{\tick}{Write-fronts + NA + History + Viewfronts}
\begin{minipage}[t]{0.3\linewidth}
Possible outcomes:\\
\lstinline{  r1 = 5}\\
\end{minipage}
%
\begin{minipage}[t]{0.3\linewidth}
\vspace{-.2cm}
  \begin{tabular}{l@{\ \ \ }l}
    \begin{minipage}[l]{4.3cm} \small
\begin{lstlisting}
        |$[$|f|$]_{rel}$| := 0; |$[$|d|$]_{na}$| := 0;
\end{lstlisting}
\vspace{-.2cm}
\begin{tabular}{l||l}
\begin{lstlisting}
|$[$|d|$]_{na}$| := 5;
|$[$|f|$]_{rel}$| := 1;
|$[$|f|$]_{rlx}$| := 2
\end{lstlisting}
\hspace{.6cm}
&
\begin{lstlisting}
repeat |$[$|f|$]_{acq}$| == 2 end;
r1 = |$[$|d|$]_{na}$|
\end{lstlisting}
\end{tabular}
    \end{minipage}
&
  \end{tabular}
\end{minipage}
\litmusTestEnd

\litmusTestStart{MP\_rel+acq+na+rlx\_2}{\tick}{Write-fronts + NA + History + Viewfronts}
\begin{minipage}[t]{0.3\linewidth}
Possible outcomes:\\
\lstinline{  r1 = 5 /\ r2 = <0, 1>}\\
\end{minipage}
%
\begin{minipage}[t]{0.3\linewidth}
\vspace{-.2cm}
  \begin{tabular}{l@{\ \ \ }l}
    \begin{minipage}[l]{4.3cm} \small
\begin{lstlisting}
|$[$|f|$]_{na}$| := 0; |$[$|d|$]_{na}$| := 0; |$[$|x|$]_{na}$| := 0;
\end{lstlisting}
\vspace{-.2cm}
\begin{tabular}{l||l}
\begin{lstlisting}
|$[$|d|$]_{na}$| := 5;
|$[$|f|$]_{rel}$| := 1;
|$[$|x|$]_{rel}$| := 1;
|$[$|f|$]_{rlx}$| := 2
\end{lstlisting}
\hspace{.6cm}
&
\begin{lstlisting}
repeat |$[$|f|$]_{acq}$| == 2 end;
r1 := |$[$|d|$]_{na}$|;
r2 := |$[$|x|$]_{rlx}$|
\end{lstlisting}
\end{tabular}
    \end{minipage}
&
  \end{tabular}
\end{minipage}
\litmusTestEnd

\litmusTestStart{MP\_con+na}{\tick}{Consume + NA + History + Viewfronts}
\begin{minipage}[t]{0.3\linewidth}
Possible outcomes:\\
\lstinline{  r1 = 0}\\
\lstinline{  r1 = 5}\\
\end{minipage}
%
\begin{minipage}[t]{0.3\linewidth}
\vspace{-.2cm}
  \begin{tabular}{l@{\ \ \ }l}
    \begin{minipage}[l]{4.3cm} \small
\begin{lstlisting}
        |$[$|f|$]_{con}$| := null; |$[$|d|$]_{na}$| := 0;
\end{lstlisting}
\vspace{-.2cm}
\begin{tabular}{l||l}
\begin{lstlisting}
|$[$|d|$]_{na}$| := 5;
|$[$|f|$]_{rel}$| := d
\end{lstlisting}
\hspace{.6cm}
&
\begin{lstlisting}
r0 := |$[$|f|$]_{con}$|;
if r0 != null
then r1 = |$[$|r0|$]_{na}$|
else r1 = 0
fi
\end{lstlisting}
\end{tabular}
    \end{minipage}
&
  \end{tabular}
\end{minipage}
\litmusTestEnd

\litmusTestStart{MP\_con+na\_2}{\tick}{Consume + NA + History + Viewfronts}
\begin{minipage}[t]{0.3\linewidth}
Possible outcomes:\\
\lstinline{  r2 = 0 /\ r3 = <0, 1>}\\
\lstinline{  r2 = 5 /\ r3 = <0, 1>}\\
\end{minipage}
%
\begin{minipage}[t]{0.3\linewidth}
\vspace{-.2cm}
  \begin{tabular}{l@{\ \ \ }l}
    \begin{minipage}[l]{4.3cm} \small
\begin{lstlisting}
|$[$|p|$]_{na}$| := null; |$[$|d|$]_{na}$| := 0; |$[$|x|$]_{na}$| := 0;
\end{lstlisting}
\vspace{-.2cm}
\begin{tabular}{l||l}
\begin{lstlisting}
|$[$|x|$]_{rlx}$| := 1;
|$[$|d|$]_{na}$| := 1;
|$[$|p|$]_{rel}$| := d
\end{lstlisting}
\hspace{.6cm}
&
\begin{lstlisting}
r1 = |$[$|p|$]_{con}$|;
if   r1 != null
then r2 = |$[$|r1|$]_{na}$|;
     r3 = |$[$|x|$]_{rlx}$|
else r2 = 0; r3 = 0
fi
\end{lstlisting}
\end{tabular}
    \end{minipage}
&
  \end{tabular}
\end{minipage}
\litmusTestEnd

\litmusTestStart{MP\_cas+rel+acq+na from \cite{Vafeiadis-Narayan:OOPSLA13}}{\tick}{NA + History + Viewfronts}
\begin{minipage}[t]{0.2\linewidth}
Impossible outcomes:\\
\lstinline{  stuck}\\
\end{minipage}
%
\begin{minipage}[t]{0.4\linewidth}
\vspace{-.2cm}
  \begin{tabular}{l@{\ \ \ }l}
    \begin{minipage}[l]{4.3cm} \small
\begin{lstlisting}
        |$[$|f|$]_{rlx}$| := 1; |$[$|d|$]_{na}$| := 0;
\end{lstlisting}
\vspace{-.2cm}
\begin{tabular}{l||l@{\ \ \ \ }||l}
\begin{lstlisting}
|$[$|d|$]_{na}$| := 5;
|$[$|f|$]_{rel}$| := 0
\end{lstlisting}
\hspace{.6cm}
&
\begin{lstlisting}
r1 = cas|$_{acq,rlx}$|(f, 0, 1);
if r1 == 0
then |$[$|d|$]_{rlx}$| := 6
else 0
fi
\end{lstlisting}
\hspace{.6cm}
&
\begin{lstlisting}
r2 = cas|$_{acq,rlx}$|(f, 0, 1);
if r2 == 0
then |$[$|d|$]_{rlx}$| := 7
else 0
fi
\end{lstlisting}
\end{tabular}
    \end{minipage}
&
  \end{tabular}
\end{minipage}
\litmusTestEnd

\litmusTestStart{MP\_cas+rel+rlx+na}{\tick}{NA + History + Viewfronts}
\begin{minipage}[t]{0.2\linewidth}
Possible outcomes:\\
\lstinline{  stuck}\\
\end{minipage}
%
\begin{minipage}[t]{0.4\linewidth}
\vspace{-.2cm}
  \begin{tabular}{l@{\ \ \ }l}
    \begin{minipage}[l]{4.3cm} \small
\begin{lstlisting}
        |$[$|f|$]_{rlx}$| := 1; |$[$|d|$]_{na}$| := 0;
\end{lstlisting}
\vspace{-.2cm}
\begin{tabular}{l||l||l}
\begin{lstlisting}
|$[$|d|$]_{na}$| := 5
|$[$|f|$]_{rel}$| := 0;
\end{lstlisting}
\hspace{.6cm}
&
\begin{lstlisting}
r1 = cas|$_{rlx,rlx}$|(f, 0, 1);
if r1 == 0
then |$[$|d|$]_{rlx}$| := 6
else 0
fi
\end{lstlisting}
\hspace{.6cm}
&
\begin{lstlisting}
r2 = cas|$_{rlx,rlx}$|(f, 0, 1);
if r2 == 0
then |$[$|d|$]_{rlx}$| := 7
else 0
fi
\end{lstlisting}
\end{tabular}
    \end{minipage}
&
  \end{tabular}
\end{minipage}
\litmusTestEnd

\section{Coherence of Read-Read (CoRR)}
\label{app:corr}

\litmusTestStart{CoRR\_rlx}{\tick}{History + Viewfronts}
\begin{minipage}[t]{0.3\linewidth}
Impossible outcomes:\\
\lstinline{  r1 = 1 |$\land$| r2 = 2 |$\land$| r3 = 2 |$\land$| r4 = 1}\\
\lstinline{  r1 = 2 |$\land$| r2 = 1 |$\land$| r3 = 1 |$\land$| r4 = 2}\\
\end{minipage}
%
\begin{minipage}[t]{0.3\linewidth}
\vspace{-.2cm}
  \begin{tabular}{l@{\ \ \ }l}
    \begin{minipage}[l]{4.3cm} \small
\begin{lstlisting}
                    |$[$|x|$]_{rlx}$| := 0;
\end{lstlisting}
\vspace{-.2cm}
\begin{tabular}{l||l||l||l}
\begin{lstlisting}
|$[$|x|$]_{rlx}$| := 1
\end{lstlisting}
\hspace{.6cm}
&
\begin{lstlisting}
|$[$|x|$]_{rlx}$| := 2
\end{lstlisting}
\hspace{.6cm}
&
\begin{lstlisting}
r1 = |$[$|x|$]_{rlx}$|;
r2 = |$[$|x|$]_{rlx}$|
\end{lstlisting}
\hspace{.6cm}
&
\begin{lstlisting}
r3 = |$[$|x|$]_{rlx}$|;
r4 = |$[$|x|$]_{rlx}$|
\end{lstlisting}
\end{tabular}
    \end{minipage}
&
  \end{tabular}
\end{minipage}
\litmusTestEnd

\litmusTestStart{CoRR\_rel+acq}{\tick}{History + Viewfronts}
\begin{minipage}[t]{0.3\linewidth}
Impossible outcomes:\\
\lstinline{  r1 = 1 |$\land$| r2 = 2 |$\land$| r3 = 2 |$\land$| r4 = 1}\\
\lstinline{  r1 = 2 |$\land$| r2 = 1 |$\land$| r3 = 1 |$\land$| r4 = 2}\\
\end{minipage}
%
\begin{minipage}[t]{0.3\linewidth}
\vspace{-.2cm}
  \begin{tabular}{l@{\ \ \ }l}
    \begin{minipage}[l]{4.3cm} \small
\begin{lstlisting}
                   |$[$|x|$]_{rel}$| := 0;
\end{lstlisting}
\vspace{-.2cm}
\begin{tabular}{l||l||l||l}
\begin{lstlisting}
|$[$|x|$]_{rel}$| := 1
\end{lstlisting}
\hspace{.6cm}
&
\begin{lstlisting}
|$[$|x|$]_{rel}$| := 2
\end{lstlisting}
\hspace{.6cm}
&
\begin{lstlisting}
r1 = |$[$|x|$]_{acq}$|;
r2 = |$[$|x|$]_{acq}$|
\end{lstlisting}
\hspace{.6cm}
&
\begin{lstlisting}
r3 = |$[$|x|$]_{acq}$|;
r4 = |$[$|x|$]_{acq}$|
\end{lstlisting}
\end{tabular}
    \end{minipage}
&
  \end{tabular}
\end{minipage}
\litmusTestEnd

\section{Independent Reads of Independent Writes (IRIW)}
\label{app:iriw}

\litmusTestStart{IRIW\_rlx}{\tick}{History + Viewfronts}
\begin{minipage}[t]{0.3\linewidth}
Possible outcomes:\\
\lstinline{  r1 = <0, 1>; r2 = <0, 1>;}\\
\lstinline{  r3 = <0, 1>; r4 = <0, 1>}\\
\end{minipage}
%
\begin{minipage}[t]{0.5\linewidth}
\vspace{-.2cm}
  \begin{tabular}{l@{\ \ \ }l}
    \begin{minipage}[l]{4.3cm} \small
\begin{lstlisting}
               |$[$|x|$]_{rlx}$| := 0; |$[$|y|$]_{rlx}$| := 0;
\end{lstlisting}
\vspace{-.2cm}
\begin{tabular}{l||l||l||l}
\begin{lstlisting}
|$[$|x|$]_{rlx}$| := 1
\end{lstlisting}
\hspace{.6cm}
&
\begin{lstlisting}
|$[$|y|$]_{rlx}$| := 1
\end{lstlisting}
\hspace{.6cm}
&
\begin{lstlisting}
r1 = |$[$|x|$]_{rlx}$|;
r2 = |$[$|y|$]_{rlx}$|
\end{lstlisting}
\hspace{.6cm}
&
\begin{lstlisting}
r3 = |$[$|y|$]_{rlx}$|;
r4 = |$[$|x|$]_{rlx}$|
\end{lstlisting}
\end{tabular}
    \end{minipage}
&
  \end{tabular}
\end{minipage}

Comment:
It is possible to get
\lstinline{r1 = 1; r2 = 0; r3 = 1; r4 = 0}
\litmusTestEnd

\litmusTestStart{IRIW\_rel+acq}{\tick}{History + Viewfronts}
\begin{minipage}[t]{0.3\linewidth}
Possible outcomes:\\
\lstinline{  r1 = <0, 1>; r2 = <0, 1>;}\\
\lstinline{  r3 = <0, 1>; r4 = <0, 1>}\\
\end{minipage}
%
\begin{minipage}[t]{0.5\linewidth}
\vspace{-.2cm}
  \begin{tabular}{l@{\ \ \ }l}
    \begin{minipage}[l]{4.3cm} \small
\begin{lstlisting}
               |$[$|x|$]_{rel}$| := 0; |$[$|y|$]_{rel}$| := 0;
\end{lstlisting}
\vspace{-.2cm}
\begin{tabular}{l||l||l||l}
\begin{lstlisting}
|$[$|x|$]_{rel}$| := 1
\end{lstlisting}
\hspace{.6cm}
&
\begin{lstlisting}
|$[$|y|$]_{rel}$| := 1
\end{lstlisting}
\hspace{.6cm}
&
\begin{lstlisting}
r1 = |$[$|x|$]_{acq}$|;
r2 = |$[$|y|$]_{acq}$|
\end{lstlisting}
\hspace{.6cm}
&
\begin{lstlisting}
r3 = |$[$|y|$]_{acq}$|;
r4 = |$[$|x|$]_{acq}$|
\end{lstlisting}
\end{tabular}
    \end{minipage}
&
  \end{tabular}
\end{minipage}

Comment:
It is possible to get
\lstinline{r1 = 1; r2 = 0; r3 = 1; r4 = 0}
\litmusTestEnd

\litmusTestStart{IRIW\_sc}{\tick}{SC + History + Viewfronts}
\begin{minipage}[t]{0.3\linewidth}
Forbidden outcomes:\\
\lstinline{  r1 = 1 |$\land$| r2 = 0 |$\land$| r3 = 1 |$\land$| r4 = 0}\\
\end{minipage}
%
\begin{minipage}[t]{0.5\linewidth}
\vspace{-.2cm}
  \begin{tabular}{l@{\ \ \ }l}
    \begin{minipage}[l]{4.3cm} \small
\begin{lstlisting}
               |$[$|x|$]_{sc}$| := 0; |$[$|y|$]_{sc}$| := 0;
\end{lstlisting}
\vspace{-.2cm}
\begin{tabular}{l||l||l||l}
\begin{lstlisting}
|$[$|x|$]_{sc}$| := 1
\end{lstlisting}
\hspace{.6cm}
&
\begin{lstlisting}
|$[$|y|$]_{sc}$| := 1
\end{lstlisting}
\hspace{.6cm}
&
\begin{lstlisting}
r1 = |$[$|x|$]_{sc}$|;
r2 = |$[$|y|$]_{sc}$|
\end{lstlisting}
\hspace{.6cm}
&
\begin{lstlisting}
r3 = |$[$|y|$]_{sc}$|;
r4 = |$[$|x|$]_{sc}$|
\end{lstlisting}
\end{tabular}
    \end{minipage}
&
  \end{tabular}
\end{minipage}
\litmusTestEnd

\section{Write-to-Read Causality (WRC)}
\label{app:wrc}

\litmusTestStart{WRC\_rel+acq}{\tick}{History + Viewfronts}
\begin{minipage}[t]{0.3\linewidth}
Forbidden outcomes:\\
\lstinline{  r2 = 1 |$\land$| r3 = 0}\\
\end{minipage}
%
\begin{minipage}[t]{0.3\linewidth}
\vspace{-.2cm}
  \begin{tabular}{l@{\ \ \ }l}
    \begin{minipage}[l]{4.3cm} \small
\begin{lstlisting}
        |$[$|x|$]_{rel}$| := 0; |$[$|y|$]_{rel}$| := 0;
\end{lstlisting}
\vspace{-.2cm}
\begin{tabular}{l||l||l}
\begin{lstlisting}
|$[$|x|$]_{rel}$| := 1
\end{lstlisting}
\hspace{.6cm}
&
\begin{lstlisting}
r1 = |$[$|x|$]_{acq}$|;
|$[$|y|$]_{rel}$| := r1
\end{lstlisting}
\hspace{.6cm}
&
\begin{lstlisting}
r2 = |$[$|y|$]_{acq}$|;
r3 = |$[$|x|$]_{acq}$|
\end{lstlisting}
\end{tabular}
    \end{minipage}
&
  \end{tabular}
\end{minipage}
\litmusTestEnd

\litmusTestStart{WRC\_rlx}{\tick}{History + Viewfronts}
\begin{minipage}[t]{0.3\linewidth}
Possible outcomes:\\
\lstinline{  r2 = 0 |$\land$| r3 = 0}\\
\lstinline{  r2 = 0 |$\land$| r3 = 1}\\
\lstinline{  r2 = 1 |$\land$| r3 = 0}\\
\lstinline{  r2 = 1 |$\land$| r3 = 1}\\
\end{minipage}
%
\begin{minipage}[t]{0.3\linewidth}
\vspace{-.2cm}
  \begin{tabular}{l@{\ \ \ }l}
    \begin{minipage}[l]{4.3cm} \small
\begin{lstlisting}
        |$[$|x|$]_{rlx}$| := 0; |$[$|y|$]_{rlx}$| := 0;
\end{lstlisting}
\vspace{-.2cm}
\begin{tabular}{l||l||l}
\begin{lstlisting}
|$[$|x|$]_{rlx}$| := 1
\end{lstlisting}
\hspace{.6cm}
&
\begin{lstlisting}
r1 = |$[$|x|$]_{rlx}$|;
|$[$|y|$]_{rlx}$| := r1
\end{lstlisting}
\hspace{.6cm}
&
\begin{lstlisting}
r2 = |$[$|y|$]_{rlx}$|;
r3 = |$[$|x|$]_{rlx}$|
\end{lstlisting}
\end{tabular}
    \end{minipage}
&
  \end{tabular}
\end{minipage}

%% Comment:
%% It is possible to get
\lstinline{r2 = 1; r3 = 0}
\litmusTestEnd

\litmusTestStart{WRC\_cas+rel}{\tick}{History + Viewfronts}
\begin{minipage}[t]{0.3\linewidth}
Impossible outcomes:\\
\lstinline{  r2 = 2 |$\land$| r3 = 0}\\
\end{minipage}
%
\begin{minipage}[t]{0.3\linewidth}
\vspace{-.2cm}
  \begin{tabular}{l@{\ \ \ }l}
    \begin{minipage}[l]{4.3cm} \small
\begin{lstlisting}
        |$[$|x|$]_{rel}$| := 0; |$[$|y|$]_{rel}$| := 0;
\end{lstlisting}
\vspace{-.2cm}
\begin{tabular}{l||l||l}
\begin{lstlisting}
|$[$|x|$]_{rel}$| := 1;
|$[$|y|$]_{rel}$| := 1
\end{lstlisting}
\hspace{.6cm}
&
\begin{lstlisting}
cas|$_{rel,acq}$|(y, 1, 2)
\end{lstlisting}
\hspace{.6cm}
&
\begin{lstlisting}
r1 = |$[$|y|$]_{rel}$|;
r2 = |$[$|x|$]_{rel}$|
\end{lstlisting}
\end{tabular}
    \end{minipage}
&
  \end{tabular}
\end{minipage}
\vspace{.2cm}
\hrule
\vspace{.2cm}

\litmusTestStart{WRC\_cas+rlx}{\tick}{History + Viewfronts}
\begin{minipage}[t]{0.3\linewidth}
Impossible outcomes:\\
\lstinline{  r2 = 2 |$\land$| r3 = 0}\\
\end{minipage}
%
\begin{minipage}[t]{0.3\linewidth}
\vspace{-.2cm}
  \begin{tabular}{l@{\ \ \ }l}
    \begin{minipage}[l]{4.3cm} \small
\begin{lstlisting}
        |$[$|x|$]_{rlx}$| := 0; |$[$|y|$]_{rlx}$| := 0;
\end{lstlisting}
\vspace{-.2cm}
\begin{tabular}{l||l||l}
\begin{lstlisting}
|$[$|x|$]_{rlx}$| := 1;
|$[$|y|$]_{rel}$| := 1
\end{lstlisting}
\hspace{.6cm}
&
\begin{lstlisting}
cas|$_{rlx,rlx}$|(y, 1, 2)
\end{lstlisting}
\hspace{.6cm}
&
\begin{lstlisting}
r1 = |$[$|y|$]_{rlx}$|;
r2 = |$[$|x|$]_{rlx}$|
\end{lstlisting}
\end{tabular}
    \end{minipage}
&
  \end{tabular}
\end{minipage}
\litmusTestEnd


%\newpage

\section{Out-of-Thin-Air reads}
\label{app:ota}

In our semantics it is not possible to get out-of-thin-air results,
unlike the C11 standard. But such reads are considered to be an
undesirable behavior by most of the standard's
clients~\cite{Batty-al:ESOP15}.

\litmusTestStart{OTA\_lb}{\fail}{Postponed reads + History + Viewfronts}
\begin{minipage}[t]{0.3\linewidth}
Possible outcomes:\\
\lstinline{  r1 = 0 |$\land$| r2 = 0}\\
\end{minipage}
%
\begin{minipage}[t]{0.3\linewidth}
\vspace{-.2cm}
  \begin{tabular}{l@{\ \ \ }l}
    \begin{minipage}[l]{4.3cm} \small
\begin{lstlisting}
  |$[$|x|$]_{rlx}$| := 0; |$[$|y|$]_{rlx}$| := 0;
\end{lstlisting}
\vspace{-.2cm}
\begin{tabular}{l||l}
\begin{lstlisting}
r1 = |$[$|y|$]_{rlx}$|;
|$[$|x|$]_{rlx}$| := r1
\end{lstlisting}
\hspace{.6cm}
&
\begin{lstlisting}
r2 = |$[$|x|$]_{rlx}$|;
|$[$|y|$]_{rlx}$| := r2
\end{lstlisting}
\end{tabular}
    \end{minipage}
&
  \end{tabular}
\end{minipage}

Comment: According to the C11 standard \cite{C:11,CPP:11},
\lstinline{r1} and \lstinline{r2} can get arbitrary values.
\litmusTestEnd

\litmusTestStart{OTA\_if}{\fail}{Postponed reads + History + Viewfronts}
\begin{minipage}[t]{0.4\linewidth}
Possible outcomes:\\
\lstinline{  r1 = 0 |$\land$| r2 = 0}\\
\end{minipage}
%
\begin{minipage}[t]{0.4\linewidth}
\vspace{-.2cm}
  \begin{tabular}{l@{\ \ \ }l}
    \begin{minipage}[l]{4.3cm} \small
\begin{lstlisting}
  |$[$|x|$]_{rlx}$| := 0; |$[$|y|$]_{rlx}$| := 0;
\end{lstlisting}
\vspace{-.2cm}
\begin{tabular}{l||l}
\begin{lstlisting}
r1 = |$[$|y|$]_{rlx}$|;
if r1
then |$[$|x|$]_{rlx}$| := 1
else r1 = 0 
fi
\end{lstlisting}
\hspace{.6cm}
&
\begin{lstlisting}
r2 = |$[$|x|$]_{rlx}$|;
if r2
then |$[$|y|$]_{rlx}$| := 1
else r2 = 0
fi
\end{lstlisting}
\end{tabular}
    \end{minipage}
&
  \end{tabular}
\end{minipage}

Comment: According to the C11 standard \cite{C:11,CPP:11},
\lstinline{r1} and \lstinline{r2} can be 1s at the end of execution.
\litmusTestEnd

\section{Write Reorder (WR), or 2+2W from \cite{Lahav-al:POPL16}}
\label{app:wr}

\litmusTestStart{WR\_rlx}{\tick}{History + Viewfronts + Operational Buffers}
\begin{minipage}[t]{0.3\linewidth}
Possible outcomes:\\
\lstinline{  r1 = 1 |$\land$| r2 = 2}\\
\lstinline{  r1 = 2 |$\land$| r2 = 1}\\
\lstinline{  r1 = 2 |$\land$| r2 = 2}\\
\end{minipage}
%
\begin{minipage}[t]{0.3\linewidth}
\vspace{-.2cm}
  \begin{tabular}{l@{\ \ \ }l}
    \begin{minipage}[l]{4.3cm} \small
\begin{lstlisting}
  |$[$|x|$]_{rlx}$| := 0; |$[$|y|$]_{rlx}$| := 0;
\end{lstlisting}
\vspace{-.2cm}
\begin{tabular}{l||l}
\begin{lstlisting}
|$[$|x|$]_{rlx}$| := 1;
|$[$|y|$]_{rlx}$| := 2
\end{lstlisting}
\hspace{.6cm}
&
\begin{lstlisting}
|$[$|y|$]_{rlx}$| := 1;
|$[$|x|$]_{rlx}$| := 2
\end{lstlisting}
\end{tabular}
\begin{lstlisting}
  r1 = |$[$|x|$]_{rlx}$|; r2 = |$[$|y|$]_{rlx}$|
\end{lstlisting}
    \end{minipage}
&
  \end{tabular}
\end{minipage}
\litmusTestEnd

\litmusTestStart{WR\_rlx+rel}{\tick}{History + Viewfronts + Operational Buffers}
\begin{minipage}[t]{0.3\linewidth}
Possible outcomes:\\
\lstinline{  r1 = 1 |$\land$| r2 = 2}\\
\lstinline{  r1 = 2 |$\land$| r2 = 1}\\
\lstinline{  r1 = 2 |$\land$| r2 = 2}\\
\end{minipage}
%
\begin{minipage}[t]{0.3\linewidth}
\vspace{-.2cm}
  \begin{tabular}{l@{\ \ \ }l}
    \begin{minipage}[l]{4.3cm} \small
\begin{lstlisting}
  |$[$|x|$]_{rlx}$| := 0; |$[$|y|$]_{rlx}$| := 0;
\end{lstlisting}
\vspace{-.2cm}
\begin{tabular}{l||l}
\begin{lstlisting}
|$[$|x|$]_{rlx}$| := 1;
|$[$|y|$]_{rel}$| := 2
\end{lstlisting}
\hspace{.6cm}
&
\begin{lstlisting}
|$[$|y|$]_{rlx}$| := 1;
|$[$|x|$]_{rel}$| := 2
\end{lstlisting}
\end{tabular}
\begin{lstlisting}
  r1 = |$[$|x|$]_{rlx}$|; r2 = |$[$|y|$]_{rlx}$|
\end{lstlisting}
    \end{minipage}
&
  \end{tabular}
\end{minipage}
\litmusTestEnd

\litmusTestStart{WR\_rel}{\tick}{History + Viewfronts + Operational Buffers}
\begin{minipage}[t]{0.3\linewidth}
Possible outcomes:\\
\lstinline{  r1 = 1 |$\land$| r2 = 2}\\
\lstinline{  r1 = 2 |$\land$| r2 = 1}\\
\lstinline{  r1 = 2 |$\land$| r2 = 2}\\
\end{minipage}
%
\begin{minipage}[t]{0.3\linewidth}
\vspace{-.2cm}
  \begin{tabular}{l@{\ \ \ }l}
    \begin{minipage}[l]{4.3cm} \small
\begin{lstlisting}
  |$[$|x|$]_{rel}$| := 0; |$[$|y|$]_{rel}$| := 0;
\end{lstlisting}
\vspace{-.2cm}
\begin{tabular}{l||l}
\begin{lstlisting}
|$[$|x|$]_{rel}$| := 1;
|$[$|y|$]_{rel}$| := 2
\end{lstlisting}
\hspace{.6cm}
&
\begin{lstlisting}
|$[$|y|$]_{rel}$| := 1;
|$[$|x|$]_{rel}$| := 2
\end{lstlisting}
\end{tabular}
\begin{lstlisting}
  r1 = |$[$|x|$]_{acq}$|; r2 = |$[$|y|$]_{acq}$|
\end{lstlisting}
    \end{minipage}
&
  \end{tabular}
\end{minipage}

\vspace{10pt}

%% This is an example of the program from \cite{Lahav-al:POPL16}.
%% The weak behavior (\lstinline{r1 = r2 = 1}) is allowed in C11,
%% but never observable under any known sound compilation scheme
%% of the C11 release writes.
%% Our semantics doesn't allow such behaviour as well as SRA
%% \cite{Lahav-al:POPL16}. 

\litmusTestEnd

%% \section{Value Stealing}
%% \label{app:ss}

%% \litmusTestStart{VS\_rlx}{\tick}{}
%% \begin{minipage}[t]{0.3\linewidth}
%% Possible outcomes:\\
%% \lstinline{  r1 = 0}\\
%% \lstinline{  r1 = 1}\\
%% \end{minipage}
%% %
%% \begin{minipage}[t]{0.3\linewidth}
%% \vspace{-.2cm}
%%   \begin{tabular}{l@{\ \ \ }l}
%%     \begin{minipage}[l]{4.3cm} \small
%% \begin{lstlisting}
%%         |$[$|x|$]_{rlx}$| := 0; |$[$|y|$]_{rlx}$| := 0;
%% \end{lstlisting}
%% \vspace{-.2cm}
%% \begin{tabular}{l||l||l}
%% \begin{lstlisting}
%% r1 = |$[$|x|$]_{rlx}$|;
%% |$[$|x|$]_{rlx}$| := 1
%% \end{lstlisting}
%% \hspace{.6cm}
%% &
%% \begin{lstlisting}
%% r2 = |$[$|x|$]_{rlx}$|;
%% |$[$|y|$]_{rlx}$| := r2
%% \end{lstlisting}
%% \hspace{.6cm}
%% &
%% \begin{lstlisting}
%% r3 = |$[$|y|$]_{rlx}$|;
%% |$[$|x|$]_{rlx}$| := r3
%% \end{lstlisting}
%% \end{tabular}
%%     \end{minipage}
%% &
%%   \end{tabular}
%% \end{minipage}

%% \litmusTestEnd

\section{Speculative Execution}
\label{app:se}

\litmusTestStart{SE\_simple}{\tick}{}
\begin{minipage}[t]{0.3\linewidth}
Possible outcomes:\\
\lstinline{  r0 = 0}\\
\lstinline{  r0 = 1}\\
\end{minipage}
%
\begin{minipage}[t]{0.4\linewidth}
\vspace{-.2cm}
  \begin{tabular}{l@{\ \ \ }l}
    \begin{minipage}[l]{4.3cm} \small
%
\begin{lstlisting}
|$[$|x|$]_{rlx}$| := 0; |$[$|y|$]_{rlx}$| := 0; |$[$|z|$]_{rlx}$| := 0;
\end{lstlisting}
\vspace{-.2cm}
\begin{tabular}{l||l}
\begin{lstlisting}
r1 = |$[$|x|$]_{rlx}$|;
if r1
then |$[$|z|$]_{rlx}$| := 1;
     |$[$|y|$]_{rlx}$| := 1
else |$[$|y|$]_{rlx}$| := 1
fi
\end{lstlisting}
\hspace{.6cm}
&
\begin{lstlisting}
r2 = |$[$|y|$]_{rlx}$|;
if r2
then |$[$|x|$]_{rlx}$| := 1
else 0 
fi
\end{lstlisting}
\end{tabular}
    \end{minipage}
&
  \end{tabular}

\begin{lstlisting}
                  r0 = |$[$|z|$]_{rlx}$|
\end{lstlisting}
\end{minipage}

\litmusTestEnd

\litmusTestStart{SE\_prop}{\tick}{}
\begin{minipage}[t]{0.3\linewidth}
Possible outcomes:\\
\lstinline{  r0 = 0}\\
\lstinline{  r0 = 1}\\
\end{minipage}
%
\begin{minipage}[t]{0.4\linewidth}
\vspace{-.2cm}
  \begin{tabular}{l@{\ \ \ }l}
    \begin{minipage}[l]{4.3cm} \small
%
\begin{lstlisting}
  |$[$|x|$]_{rlx}$| := 0; |$[$|y|$]_{rlx}$| := 0; |$[$|z|$]_{rlx}$| := 0;
\end{lstlisting}
\vspace{-.2cm}
\begin{tabular}{l||l}
\begin{lstlisting}
r1 = |$[$|x|$]_{rlx}$|;
if r1
then |$[$|z|$]_{rlx}$| := 1;
     r1 = |$[$|z|$]_{rlx}$|;
     |$[$|y|$]_{rlx}$| := r1
else |$[$|y|$]_{rlx}$| := 1
fi
\end{lstlisting}
\hspace{.6cm}
&
\begin{lstlisting}
r2 = |$[$|y|$]_{rlx}$|;
if r2
then |$[$|x|$]_{rlx}$| := 1
else 0 
fi
\end{lstlisting}
\end{tabular}
    \end{minipage}
&
  \end{tabular}

\begin{lstlisting}
                  r0 = |$[$|z|$]_{rlx}$|
\end{lstlisting}
\end{minipage}

\litmusTestEnd

\litmusTestStart{SE\_nested}{\tick}{}
\begin{minipage}[t]{0.3\linewidth}
Possible outcomes:\\
\lstinline{  r0 = 0}\\
\lstinline{  r0 = 1}\\
\end{minipage}
%
\begin{minipage}[t]{0.4\linewidth}
\vspace{-.2cm}
  \begin{tabular}{l@{\ \ \ }l}
    \begin{minipage}[l]{4.3cm} \small
%
\begin{lstlisting}
|$[$|x|$]_{rlx}$| := 0; |$[$|y|$]_{rlx}$| := 0; |$[$|z|$]_{rlx}$| := 0; |$[$|f|$]_{rlx}$| := 0;
\end{lstlisting}
\vspace{-.2cm}
\begin{tabular}{l||l}
\begin{lstlisting}
r1 = |$[$|x|$]_{rlx}$|;
if r1
then r2 = |$[$|f|$]_{rlx}$|;
     if r2
     then |$[$|z|$]_{rlx}$| := 1;
          |$[$|y|$]_{rlx}$| := 1
     else |$[$|y|$]_{rlx}$| := 1
     fi
else |$[$|y|$]_{rlx}$| := 1
fi
\end{lstlisting}
\hspace{.6cm}
&
\begin{lstlisting}
r3 = |$[$|y|$]_{rlx}$|;
if r3
then |$[$|f|$]_{rlx}$| := 1;
     |$[$|x|$]_{rlx}$| := 1
else 0 
fi
\end{lstlisting}
\end{tabular}
    \end{minipage}
&
  \end{tabular}

\begin{lstlisting}
                  r0 = |$[$|z|$]_{rlx}$|
\end{lstlisting}
\end{minipage}

\litmusTestEnd


\section{Locks}
\label{app:locks}

\begin{minipage}[t]{0.4\linewidth}
\textbf{Dekker's lock}\\\\
Possible outcomes:\\
\lstinline{  stuck}\\
Requires: RA + na\\
Fully Supported: $\tick$\\
\end{minipage}
%
\begin{minipage}[t]{0.4\linewidth}
\vspace{-.2cm}
  \begin{tabular}{l@{\ \ \ }l}
    \begin{minipage}[l]{4.3cm} \small
\begin{lstlisting}
  |$[$|x|$]_{rel}$| := 0; |$[$|y|$]_{rel}$| := 0; |$[$|d|$]_{na}$| := 0;
\end{lstlisting}
\vspace{-.2cm}
\begin{tabular}{l||l}
\begin{lstlisting}
|$[$|x|$]_{rel}$| := 1;
r1 = |$[$|y|$]_{acq}$|
if r1 == 0
then |$[$|d|$]_{na}$| := 5
else 0 
fi
\end{lstlisting}
\hspace{.6cm}
&
\begin{lstlisting}
|$[$|y|$]_{rel}$| := 1;
r2 = |$[$|x|$]_{acq}$|
if r2 == 0
then |$[$|d|$]_{na}$| := 6
else 0 
fi
\end{lstlisting}
\end{tabular}
    \end{minipage}
&
  \end{tabular}
\end{minipage}
\litmusTestEnd

\begin{minipage}[t]{0.4\linewidth}
\textbf{Cohen's lock}\\\\
Impossible outcomes (according to \cite{Turon-al:OOPSLA14}):\\
\lstinline{  stuck}\\
Requires: RA + na\\
Fully Supported: $\tick$\\
\end{minipage}
%
\begin{minipage}[t]{0.4\linewidth}
\vspace{-.2cm}
  \begin{tabular}{l@{\ \ \ }l}
    \begin{minipage}[l]{4.3cm} \small
\begin{lstlisting}
  |$[$|x|$]_{rel}$| := 0; |$[$|y|$]_{rel}$| := 0; |$[$|d|$]_{na}$| := 0;
\end{lstlisting}
\vspace{-.2cm}
\begin{tabular}{l||l}
\begin{lstlisting}
|$[$|x|$]_{rel}$| := choice 1 2;
repeat |$[$|y|$]_{acq}$| end;
r1 = |$[$|x|$]_{acq}$|
r2 = |$[$|y|$]_{acq}$|
if r1 == r2
then |$[$|d|$]_{na}$| := 5
else 0 
fi
\end{lstlisting}
\hspace{.6cm}
&
\begin{lstlisting}
|$[$|y|$]_{rel}$| := choice 1 2;
repeat |$[$|x|$]_{acq}$| end;
r3 = |$[$|x|$]_{acq}$|
r4 = |$[$|y|$]_{acq}$|
if r3 != r4
then |$[$|d|$]_{na}$| := 6
else 0 
fi
\end{lstlisting}
\end{tabular}
    \end{minipage}
&
  \end{tabular}
\end{minipage}
\litmusTestEnd
%% \end{figure*}

\chapter{Правила переходов и вспомогательные функции машины ARMv8 POP}
\label{sec:armpopTrans}

{\small

\inference{
  \tape \triangleq \tapef(\tId) \quad \tape(\cpath) = \bot \quad
  \lastInstr{\cpath} < size(\Carm) \\
  \exists \cpath'. (\tape(\cpath') \not = \bot \lor \cpath' = []) \land \cpath \in \nextPathCom{\cpath'}{\Carm}{\tape} \\
  \tape' \triangleq \tape[\cpath \mapsto \getNewTapeCell(\Carm[\lastInstr{\cpath}])]
}{
  \Cfarm[\tId \mapsto \Carm] \vdash
    \angled{\Mpop, \IssuingOrderf, \tapef}
    \armStepPgen{\transenv{Fetch instruction} \; \tId \; \cpath}
    \angled{\Mpop, \IssuingOrderf, \tapef[\tId \mapsto \tape']}
}

\vspace{.5cm}

\inference{
  \e \in \Evt \quad \lnot\Prop(\tId,\e) \quad \forall \e' <_{\Ord} \e. \; \Prop(\tId,\e') \quad
  \Prop' \triangleq \Prop \cup \{ (\tId, \e) \} \\
  \Ord' \triangleq (\Ord \cup \{ (\e, \e') \mid \Prop(\tId,\e') \land \lnot \Prop(\e.\tId,\e'),
  \notReorderableRel{\e}{\e'}, \lnot (\e' <_{\Ord} \e) \})^{+}
}{
  \Cfarm \vdash
    \angled{\Mcomp{\Evt}{\Ord}{\Prop}, \IssuingOrderf, \tapef}
    \armStepPgen{\transenv{Propagate} \; \e \; \tId}
    \angled{\Mcomp{\Evt}{\Ord'}{\Prop'}, \IssuingOrderf, \tapef}
}

\vspace{.5cm}

\inference{
  \tape \triangleq \tapef(\tId) \quad \tape(\cpath) = \tapeIfGoto{\None}{\z} \quad
  \stval \triangleq |[\expr|] _{com} \in \mathbb{Z} \\
  \prevBrCommitted(\cpath, \tape) \\
  (\IfState, \tape') \triangleq \tapeUpdIf(\stval, \z, \cpath, \tape) \\
  \Mpop' \triangleq \deleteUpdReads(\tId, \tape', \Mpop) \\
}{
  \Cfarm \vdash
    \angled{\Mpop, \IssuingOrderf, \tapef}
    \armStepPgen{\transenv{Branch commit} \; \tId \; \cpath}
    \angled{\Mpop', \IssuingOrderf, \tapef'}
}

\vspace{.5cm}

\inference{
  \tapef(\tId, \cpath) = \tapeFence{\None}{\LD} \quad
  \tapef' \triangleq \tapef[(\tId, \cpath) \mapsto \tapeFence{\Committed}{\LD}] \\
  \prevReadsCommitted(\cpath, \tapef(\tId)) \quad
  \prevFencesCommitted(\cpath, \tape(\tId)) \\
  \prevBrCommitted(\cpath, \tapef(\tId)) \\
}{
  \Cfarm \vdash
    \angled{\Mpop, \IssuingOrderf, \tapef}
    \armStepPgen{\transenv{Fence commit} \; \LD \; \tId \; \cpath}
    \angled{\Mpop, \IssuingOrderf, \tapef'}
}

\vspace{.5cm}

\inference{
  \tapef(\tId, \cpath) = \tapeFence{\None}{\SY} \quad
  \tapef' \triangleq \tapef[(\tId, \cpath) \mapsto \tapeFence{\Committed}{\SY}] \\
  \prevInstrCommitted(\cpath, \tapef(\tId)) \\
  %% \tape' \triangleq \tape[\cpath \mapsto \tapeFence{\Committed}{\Ftype}] \quad
  \Mpop' \triangleq \acceptRequest(\stRequestFence{\tId}{ \cpath}, \Mpop)
}{
  \Cfarm \vdash
    \angled{\Mpop, \IssuingOrderf, \tapef}
    \armStepPgen{\transenv{Fence commit} \; \SY \; \tId \; \cpath}
    \angled{\Mpop', \IssuingOrderf, \tapef'}
}

\vspace{.5cm}

\inference{
  \tapef(\tId, \cpath) = \tapeWrite{\None} \quad
  \Cfarm(\tId)[\lastInstr{\cpath}] = ``[\expr_0] := \expr_1" \\
  |[\expr_0|] = \loc \quad |[\expr_1|] = \stval \quad
  \tapef' = \tapef[(\tId, \cpath) \mapsto \tapeWrite{(\tapePending{\loc}{\stval})}]
}{
  \Cfarm \vdash
    \angled{\Mpop, \IssuingOrderf, \tapef}
    \armStepPgen{\transenv{Write pending} \; \tId \; \cpath \; \loc \; \stval}
    \angled{\Mpop, \IssuingOrderf, \tapef'}
}

\vspace{.5cm}

\inference{
  \tape \triangleq \tapef(\tId) \quad \tape(\cpath) = \tapeWrite{(\tapePending{\loc}{\stval})} \quad \Carm \triangleq \Cfarm(\tId) \\
  \Carm[\lastInstr{\cpath}] = ``[\expr_0] := \expr_1" \quad |[\expr_0|] _{com} = \loc \quad |[\expr_1|] _{com} = \stval \\
  \prevFencesCommitted(\cpath, \tape) \quad
  \prevBrCommitted(\cpath, \tape) \\
  \prevCmdDetermined(\cpath, \tape) \quad
  \prevNoRestart(\loc, \cpath, \tape)\\
  im \triangleq \noFollowingWcom(\loc, \cpath, \tape) \\
  \tape' \triangleq \tapeUpdWcom(im, \loc, \stval, \Carm, \tId, \cpath, \tape) \\
  \tapef' = \tapef[\tId \mapsto \tape'] \quad
  \Mpop'' \triangleq \deleteUpdReads(\tId, \tape', \Mpop) \\
  \Mpop' \triangleq \textIf im \; \textThen \acceptRequest(\stRequestWrite{\tId}{ \cpath}{\loc}{\stval}, \Mpop'') \; \textElse \Mpop'' 
}{
  \angled{\Mpop, \IssuingOrderf, \tapef, \Cfarm}
  \armStepWriteCommitPLoc
  \angled{\Mpop', \IssuingOrderf, \tapef', \Cfarm}
}

%% \inference{
%%   \tape \triangleq \tapef(\tId) \quad \tape(\cpath) = \tapeWrite{(\tapePending{x}{\stval})} \quad \Carm \triangleq \Cfarm(\tId) \\
%%   \Carm[\lastInstr{\cpath}] = ``[\expr_0] := \expr_1" \quad |[\expr_0|] _{com} = x \quad |[\expr_1|] _{com} = \stval \\
%%   \prevFencesCommitted(\cpath, \tape) \quad
%%   \prevBrCommitted(\cpath, \tape) \\
%%   \prevCmdDetermined(\cpath, \tape) \quad
%%   \prevNoRestart(x, \cpath, \tape)\\
%%   im \triangleq \noFollowingWcom(x, \cpath, \tape) \\
%%   \tape' \triangleq \tapeUpdWcom(im, x, \stval, \Carm, \tId, \cpath, \tape) \\
%%   \tapef' = \tapef[\tId \mapsto \tape'] \quad
%%   \Mpop'' \triangleq \deleteUpdReads(\tId, \tape', \Mpop) \\
%%   \Mpop' \triangleq \textIf im \; \textThen \acceptRequest(\stRequestWrite{\tId}{ \cpath}{ x}{\stval}, \Mpop'') \; \textElse \Mpop'' 
%% }{
%%   \angled{\Mpop, \IssuingOrderf, \tapef, \Cfarm}
%%   \armStepWriteCommitP
%%   \angled{\Mpop', \IssuingOrderf, \tapef', \Cfarm}
%% }

\vspace{.5cm}

\inference{
  \tape \triangleq \tapef(\tId) \quad \tape(\cpath) = \tapeRead{\None} \\
   \Cfarm(\tId)[\lastInstr{\cpath}] = ``\reg = [\expr]" \quad |[\expr|] = \loc \quad
  \e \triangleq \stRequestRead{\tId}{\cpath}{\loc} \\
  \tapef' = \tapef[(\tId, \cpath) \mapsto \tapeRead{(\Issued{\loc})}] \quad \prevFencesCommitted(\cpath, \tape) \\
  \IssuingOrder' \triangleq append(\e, \IssuingOrderf(\tId)) \quad
  \IssuingOrderf' = \IssuingOrderf[\tId \mapsto \IssuingOrder'] \quad
  \Mpop' \triangleq \acceptRequest(\e, \Mpop)
}{
  \Cfarm \vdash
    \angled{\Mpop, \IssuingOrderf, \tapef}
    \armStepPgen{\transenv{Read issue} \; \tId \; \cpath \; \loc}
    \angled{\Mpop', \IssuingOrderf', \tapef'}
}

\vspace{.5cm}

\inference{
  \tape \triangleq \tapef(\tId) \quad
  \tape(\cpath) = \tapeRead{(\Issued{\loc})} \quad
  \e \triangleq \stRequestRead{\tId}{ \cpath}{\loc} \quad \e' \triangleq \stRequestWrite{\tId'}{ \cpath'}{\loc}{\stval} \\
     \angled{\Evt, \Ord, \Prop} \triangleq \Mpop \quad
  \{\e, \e'\} \subseteq \Evt \quad \e' <_{\Ord} \e \quad
  \samePropagated(\e, \e', \Prop) \\
  \forall \e^{*}, \e' <_{\Ord} \e^{*} <_{\Ord} \e, get\loc(\e^{*}) \not = \loc \land \fullyPropagated(\e^{*}, \Prop) \\
  \lnot \prevReadFromOther(\loc, \e', \tId, \cpath, \IssuingOrderf(\tId)) \\
  \tape' \triangleq \tapeUpdRsat(\Plain, \loc, \stval, \Carm, \tId, \cpath, \tId', \cpath', \tape) \\
  \tapef' = \tapef[\tId \mapsto \tape'] \quad \Mpop' \triangleq \deleteUpdReads(\tId, \tape', \Mpop) \\
}{
  \Cfarm \vdash
    \angled{\Mpop, \IssuingOrderf, \tapef}
    \armStepPrSatLoc
    \angled{\Mpop', \IssuingOrderf, \tapef'}
}

\vspace{.5cm}

\inference{
  \tape \triangleq \tapef(\tId) \quad
  \tape(\cpath) = \tapeRead{(\Issued{\loc})} \quad
  \e \triangleq \stRequestRead{\tId}{ \cpath}{\loc} \quad \e' \triangleq \stRequestWrite{\tId'}{ \cpath'}{\loc}{\stval} \\
     \angled{\Evt, \Ord, \Prop} \triangleq \Mpop \quad
  \{\e, \e'\} \subseteq \Evt \quad \e' <_{\Ord} \e \quad
  \samePropagated(\e, \e', \Prop) \\
  \forall \e^{*}, \e' <_{\Ord} \e^{*} <_{\Ord} \e, get\loc(\e^{*}) \not = \loc \land \fullyPropagated(\e^{*}, \Prop) \\
  \prevReadFromOther(\loc, \e', \tId, \cpath, \IssuingOrderf(\tId)) \\
  \tape' \triangleq \tape[\cpath \mapsto \tapeRead{\None}] \quad \tapef' = \tapef[\tId \mapsto \tape'] \\
  \Mpop' \triangleq \deleteUpdReads(\tId, \tape', \Mpop) \\
}{
  \Cfarm \vdash
    \angled{\Mpop, \IssuingOrderf, \tapef}
    \armStepPrSatFailLoc
    \angled{\Mpop', \IssuingOrderf, \tapef'}
}

\vspace{.5cm}

\inference{
  \tape \triangleq \tapef(\tId) \quad
  \tape(\cpath) = \tapeRead{\None} \quad \tape(\cpath') = \tapeWrite{(\tapePending{\loc}{\stval})} \\
  \cpath' < \cpath \quad \Carm \triangleq \Cfarm(\tId) \quad
    \Carm[\lastInstr{\cpath}] = ``\reg = [\expr]" \quad |[\expr|]^{\cpath} = \loc \\
  \noWritesBetween(\loc, \Carm, \cpath', \cpath, \tape) \\
  \noDiffReadsBetween(\loc, \tId, \cpath', \cpath, \tape) \\
  \tape' \triangleq \tapeUpdRsat(\SatisfiedInFlight, \loc, \stval, \Carm, \tId, \cpath, \tId, \cpath', \tape) \\
  \tapef' = \tapef[\tId \mapsto \tape'] \quad \Mpop \triangleq \deleteUpdReads(\tId, \tape', \Mpop) \\
}{
  \Cfarm \vdash
    \angled{\Mpop, \IssuingOrderf, \tapef}
    \armStepPrInFlightSatLoc
    \angled{\Mpop', \IssuingOrderf, \tapef'}
}

\vspace{.5cm}

\inference{
  \tape \triangleq \tapef(\tId) \quad
  \tape(\cpath) = \tapeRead{(\tapeSatisfied{\satisfiedState}{\stRequestWrite{\tId'}{ \cpath'}{\loc}{\stval}})} \\
  \satisfiedState \not = \Committed \quad \Carm \triangleq \Cfarm(\tId) \quad
  \prevCmdDetermined(\cpath, \tape) \\
  \prevFencesCommitted(\cpath, \tape) \quad
  \prevBrCommitted(\cpath, \tape) \\
  \cpath'' = \mathsf{max}\{ \cpath^{*} < \cpath | \Carm[\lastInstr{\cpath^{*}}] = ``[\expr_0] := \expr_1", |[\expr_0|] ^{\cpath^{*}} _{com} = \loc \} \\
  \textIf (\tId', \cpath') = (\tId, \cpath'') \; \textThen \tape(\cpath') \; \text{is fully determined}
                                                   \; \textElse \tape(\cpath') \; \text{is committed} \\
  \readsBetweenCommitted(\cpath'', \cpath, \tape, \Carm) \\
  \tape' \triangleq \tape[\cpath \mapsto \tapeRead{(\tapeSatisfied{\Committed}{\stRequestWrite{\tId'}{ \cpath'}{\loc}{\stval}})}] \quad
  \tapef' = \tapef[\tId \mapsto \tape'] \\
}{
  \Cfarm \vdash
    \angled{\Mpop, \IssuingOrderf, \tapef}
    \armStepPgen{\transenv{Read commit} \; \tId \; \cpath}
    \angled{\Mpop, \IssuingOrderf, \tapef'}
}

\[
\begin{array}{l}
\prevInstrCommitted(\cpath, \tape) \triangleq \forall \cpath' < \cpath, \tape(\cpath') \; \text{is committed}. \\
\prevReadsCommitted(\cpath, \tape) \triangleq
  \forall \cpath' < \cpath, \tape(\cpath') = \tapeRead{\_} => \tape(\cpath') \; \text{is committed}. \\
\prevFencesCommitted(\cpath, \tape) \triangleq
  \forall \cpath' < \cpath, \tape(\cpath') = \tapeFence{\Fstate}{\_} => \Fstate = \Committed. \\
\prevBrCommitted(\cpath, \tape) \triangleq
  \forall \cpath' < \cpath, \tape(\cpath') = \tapeIfGoto{\IfState}{\_} => \IfState \not = \None. \\
\prevCmdDetermined(\cpath, \tape) \triangleq
  \forall \cpath' < \cpath, \tape(\cpath') \; \text{has a fully determined address}. \\
\prevNoRestart(\loc, \cpath, \tape) \triangleq \\
\quad \forall \cpath' < \cpath, \reg, \expr, \tape(\cpath') = \tapeRead{\Rstate},
      \Carm[\lastInstr{\cpath'}] = ``\reg = [\expr]", |[\expr|] = \loc => \\
\quad \quad \Rstate = \tapeSatisfied{\_}{\_} \; \land \tape(\cpath') \; \text{can't be restarted}. \\
\\
\noFollowingWcom(\loc, \cpath, \tape) \triangleq
  \nexists \cpath' > \cpath, \; \tape(\cpath') = \tapeWrite{(\tapeWriteCommitted{\_}{\loc}{\_})}. \\
%% \\
%% \checkReorderings{S} \triangleq S \setminus \{ (\e_0, \e_1) \mid \reorderableRel{\e_0}{\e_1} \}. \\
\end{array}
\]

\[
\begin{array}{l}
\tapeUpdIf(\stval, \z, \cpath, \tape) \triangleq \\
\quad \text{let} \; \IfState \triangleq
      \textIf \stval \not = 0 \; \textThen \Taken \; \textElse \Ignored \; \text{in} \\
\quad \text{let} \; \cpath_{drop} \triangleq
      \textIf \stval \not = 0 \;
      \textThen \mathsf{append}(\lastInstr{\cpath} + 1, \cpath) \;
      \textElse \mathsf{append}(\lastInstr{\cpath} + k, \cpath) \; \text{in} \\
\quad (\IfState, \lambda \; \cpath' -> \\
\quad \quad \textIf \mathsf{prefix}(\cpath_{drop}, \cpath') \; \textThen \bot \\
\quad \quad \textElif \cpath' = \cpath \; \textThen \tapeIfGoto{\IfState}{\z} \\
\quad \quad \textElse \tape(\cpath')). \\
\end{array}
\]

\[
\begin{array}{l}
\readsBetweenCommitted(\cpath'', \cpath, \tape, \Carm) \triangleq \\
\quad \forall \cpath^{*} > \cpath'', \cpath^{*} < \cpath, \\
\quad \quad \Carm[\lastInstr{\cpath^{*}}] = ``\reg = [\expr]", |[\expr|] ^{\cpath^{*}} _{com} = x =>
            \tape(\cpath^{*}) \; \text{is committed}. \\
\\
\noWritesBetween(\loc, \Carm, \cpath', \cpath, \tape) \triangleq \\
\quad \not \exists \cpath^{*} > \cpath', \cpath^{*} < \cpath, \Carm[\lastInstr{\cpath^{*}}] = ``[\expr_0] := \expr_1",
      |[\expr_0|]^{\cpath^{*}} = \loc.\\
\\
\noDiffReadsBetween(\loc, \tId', \cpath', \cpath, \tape) \triangleq \\
\quad \not \exists \tId'' \not = \tId', \cpath'' \not = \cpath', \cpath^{*} > \cpath', \cpath^{*} < \cpath,
      \tape(\cpath^{*}) = \tapeRead{(\tapeSatisfied{\_}{\stRequestWrite{\tId''}{ \cpath''}{\loc}{\_}})}. \\
\end{array}
\]

\[
\begin{array}{l}
\deleteUpdReads(\tId, \tape, \angled{\Evt, \Ord, \Prop}) \triangleq \\
\quad \textLet \textit{to-delete} \triangleq
\{ \e \in \Evt \mid \e.\tId = \tId, \tape'(\e.\cpath) \not = \tapeRead{(\Issued{\_})} \} \; \textIn \\
\quad \textLet \Evt' \triangleq \Evt \setminus \textit{to-delete} \; \textIn \\
\quad \textLet \Prop' = \Prop \setminus (\mathbb{N} \times \textit{to-delete}) \; \textIn \\
\quad \textLet \Ord'  = (\checkReorderings{\Ord \setminus (\Evt \times \textit{to-delete} \cup \textit{to-delete} \times \Evt)})^{+}
 \; \textIn \\
\quad \angled{\Evt', \Ord', \Prop'}. \\
\\
\acceptRequest(\e, \angled{\Evt, \Ord, \Prop}) \triangleq \\
\quad \textLet \Evt' = \Evt \cup \{ \e \} \; \textIn \\
\quad \textLet \Prop' = \Prop[\tId \mapsto \Prop(\tId) \cup \{ \e \}]  \; \textIn \\
\quad \textLet \Ord'  = (\Ord \cup \{(\e', \e) | \e' \in \Prop(\tId), \notReorderableRel{\e'}{\e} \})^{+} \; \textIn \\
\quad \angled{\Evt', \Ord', \Prop'}.
\end{array}
\]

\[
\begin{array}{l}
\samePropagated(\e, \e', \Prop) \triangleq \;
\{ \tId \mid \Prop(\tId, e) \} = \{ \tId \mid \Prop(\tId, e') \} \\
\fullyPropagated(\e, \Prop) \triangleq \;
\forall \tId, \Prop(\tId, e) \lor \nexists e'. \Prop(\tId, e'). \\
get\loc(\e) \triangleq
\; \text{match} \; \e \; \text{with}
\;   \stRequestRead{\_}{ \_}{\loc}
\mid \stRequestWrite{\_}{ \_}{\loc}{\_} -> \loc
\mid \_ -> \bot
\; \text{end}. \\
\end{array}
\]

\[
\begin{array}{l}
\prevReadFromOther(\loc, \e', \tId, \cpath, \IssuingOrder) \triangleq \\
\quad \exists \cpath^{*} < \cpath, \\
\quad \quad last\_index(\stRequestRead{\tId}{ \cpath^{*}}{\loc}, \IssuingOrder) > last\_index(\stRequestRead{\tId}{ \cpath}{\loc}, \IssuingOrder), \\
\quad \quad \tape(\cpath^{*}) = \tapeRead{(\tapeSatisfied{\_}{\Request})}, \Request \not = \e'). \\
\\
\getNewTapeCell(\StmtARM) \triangleq \\
\quad \text{match} \; \StmtARM \; \text{with} \\
\quad | ``\readInst{\reg}{\expr}" -> \tapeRead{\None} \\
\quad | ``\writeInst{\expr_0}{\expr_1}" -> \tapeWrite{\None} \\
\quad | ``\fenceInst{\Ftype}" -> \tapeFence{\None}{\Ftype} \\
\quad | ``\ifGotoInst{\expr}{\z}" -> \tapeIfGoto{\None}{\z} \\
\quad | ``\reg = \expr" -> \tapeAssign \\
\quad | ``\nop" -> \tapeNop \\
\quad \text{end}. \\
\\
\e <_{\Ord} \e' \triangleq (\e, \e') \in \Ord. \\
\reorderableRel{\stRequest{\tId}{ \cpath}{ \RequestInfo}}{\stRequest{\tId'}{ \cpath'}{ \RequestInfo'}} \triangleq \\
\quad \textIf \SY \in \{\RequestInfo,\RequestInfo'\} \; \textThen false \\
\quad \textElif \RequestInfo.\loc = \RequestInfo'.\loc \not = \bot \; \textThen false \\
\quad \textElse true. \\
%% \quad \textIf \{\RequestInfo,\RequestInfo'\} \subseteq \{\LD, \SY\} \; \textThen false \\
%% \quad \textElif \SY \in \{\RequestInfo,\RequestInfo'\} \; \textThen false \\
%% \quad \textElif \exists \loc, \stval. \{\RequestInfo,\RequestInfo'\} \subseteq \{\loc, \loc:\stval\} \; \textThen false \\
%% \quad \textElif \RequestInfo' = \LD \land \tId = \tId' \land \exists \loc, \RequestInfo = \loc \; \textThen false \\
%% \quad \textElif \RequestInfo  = \LD \land \tId = \tId' \; \text{then}  \; false \\
%% \quad \textElse true. \\
\end{array}
\]

\[
\begin{array}{l}
\nextPathCom{\cpath}{\Carm}{\tape} \triangleq filter(\lambda \cpath' \rightarrow \lastInstr{\cpath'} < size(\Carm),  \\
\quad \textIf \cpath = [] \; \textThen \{ [0] \} \\
\quad \textElif \tape(\cpath) = \bot \; \textThen \emptyset \\
\quad \textElif \exists k, \tape(\cpath) = \tapeIfGoto{\None}{k} \; \textThen
      \{ snoc(\cpath, \lastInstr{\cpath} + 1), snoc(\cpath, \lastInstr{\cpath} + k) \} \\
\quad \textElif \exists k, \tape(\cpath) = \tapeIfGoto{\Taken}{k} \; \textThen
      \{ snoc(\cpath, \lastInstr{\cpath} + k) \} \\
\quad \textElse \{ snoc(\cpath, \lastInstr{\cpath} + 1) \}). \\
\end{array}
\]

\[
\begin{array}{l}
\tapeUpdRestart(\Carm, \tId, \tape) \triangleq \\
\quad fixpoint(\lambda \tape' -> \\
\qquad \lambda \cpath -> \\
\qquad \quad \textIf \Carm[\lastInstr{\cpath}] = ``\reg = [\expr]" \land \semf{\expr}{\cpath} = \bot \\
\qquad \qquad \textThen \tapeRead{\None} \\
\qquad \quad \textElif \exists \cpath'',
                  \tape'(\cpath) = \tapeRead{(\tapeSatisfied{\SatisfiedInFlight}{\stRequestWrite{\tId}{ \cpath''}{ \_}{\_}})} \land
                  \tape'(\cpath'') = \tapeWrite{\None} \\
\qquad \qquad \textThen \tapeRead{\None} \\
\qquad \quad \textElif \Carm[\lastInstr{\cpath}] = ``[\expr_0] = \expr_1" \land
                  (\semf{\expr_0}{\cpath} = \bot \lor \semf{\expr_1}{\cpath} = \bot) \\
\qquad \qquad \textThen \tapeWrite{\None} \\
\qquad \quad \textElse \tape'(\cpath))(\tape). \\
\end{array}
\]

\[
\begin{array}{l}
\tapeUpdWcom(im, \loc, \stval, \Carm, \tId, \cpath, \tape) \triangleq \\
\quad \tapeUpdRestart(\Carm, \tId, \\
\qquad \lambda \cpath' -> \\
\qquad \quad \textIf   \cpath' = \cpath \; \textThen \tapeWrite{(\tapeWriteCommitted{im}{\loc}{\stval})} \\
\qquad \quad \textElif \cpath' < \cpath \; \textThen \tape(\cpath') \\
\qquad \quad \textElif \tape(\cpath') = \tapeRead{(\tapeSatisfied{\SatisfiedInFlight}{\stRequestWrite{\tId}{ \cpath}{\loc}{\stval}})}
            \\
\qquad \qquad \textThen \tape(\cpath') = \tapeRead{(\tapeSatisfied{\Plain}{\stRequestWrite{\tId}{ \cpath}{\loc}{\stval}})} \\
\qquad \quad \textElif \tape(\cpath') = \tapeRead{(\Issued{\loc})} \; \textThen \tapeRead{\None} \\
\qquad \quad \textElif \exists \cpath'' < \cpath,
                      \tape(\cpath') =\tapeRead{(\tapeSatisfied{\SatisfiedInFlight}{\stRequestWrite{\tId}{ \cpath''}{\loc}{\_}})}
                      \; \textThen \tapeRead{\None} \\
\qquad \quad \textElif \exists \tId'', \cpath'', \lnot (\tId'' = \tId \land \cpath'' \ge \cpath) \land
                      \tape(\cpath') =\tapeRead{(\tapeSatisfied{\Plain}{\stRequestWrite{\tId''}{ \cpath''}{\loc}{\_}})}
                      \\
\qquad \qquad \textThen \tapeRead{\None} \\
\qquad \quad \textElse \tape(\cpath')). \\
\\
\tapeUpdRsat(\satisfiedState, \loc, \stval, \Carm, \tId, \cpath, \tId', \cpath', \tape) \triangleq \\
\quad \tapeUpdRestart(\Carm, \tId, \\
\qquad \lambda \cpath'' -> \\
\qquad \quad \textIf   \cpath'' = \cpath \;
                  \textThen \tapeRead{(\tapeSatisfied{\satisfiedState}{\stRequestWrite{\tId'}{ \cpath'}{\loc}{\stval}})} \\
\qquad \quad \textElif \cpath'' < \cpath \; \textThen \tape(\cpath'') \\
\qquad \quad \textElif \exists \cpath^{*} < \cpath,
                  \tape(\cpath'') = \tapeRead{(\tapeSatisfied{\SatisfiedInFlight}{\stRequestWrite{\tId}{ \cpath^{*}}{\loc}{\_}})}
                  \; \textThen \tapeRead{\None} \\
\qquad \quad \textElif \exists \tId^{*}, \cpath^{*},
                  \lnot (\tId^{*} = \tId  \land \cpath^{*} > \cpath ) \land
                  \lnot (\tId^{*} = \tId' \land \cpath^{*} = \cpath') \land \\
\qquad \qquad \tape(\cpath'') = \tapeRead{(\tapeSatisfied{\Plain}{\stRequestWrite{\tId^{*}}{ \cpath^{*}}{\loc}{\_}})}
                  \; \textThen \tapeRead{\None} \\
\qquad \quad \textElse \tape(\cpath'')).
\end{array}
\]

}


\chapter{Доказательство лемм \ref{lem-snd} и \ref{lem-fst} о симуляции модели ARMv8 POP}
\label{sec:appendix-pop-proofs}

\noindent
\textbf{Лемма \ref{lem-snd}.}
$\forall (\aT, \p) \in \simrelPre. \;
\exists \p'. \Cfprom \vdash \p \promStep \p' \land (\aT, \p') \in \simrelPre \cup \simrel$.
\begin{proof}
Зафиксируем $\aT, \p$.
Поскольку $(\aT, \p) \in \simrelPre$, то $\invPromUptoARMnot(\aT, \p)$ выполняется.
Как следствие, существует единственный поток с идентификатором $\tId$, плёнка $\tape = \aT.\tapef(\tId)$ и
путь $\cpath = \p.\TSfprom(\tId).\cpath$ такие, что экземпляр $\tape(\cpath)$ завершён.
Далее в доказательстве мы конструируем состояние обещающей машины $\p'$ такое, что 
оно удовлетворяет утверждению теоремы.

Введём обозначения:
\[\begin{array}{l l l}
  \Carm & \triangleq & \Cfarm(\tId); \\
  \angled{\Rcur, \Racq, \Rrel} & \triangleq & \p.\TSfprom(\tId).\V; \\
  \Rcur', \Racq', \Rrel' & \triangleq & \p'.\TSfprom(\tId).\V;\\
  \cpath' & \triangleq & \p'.\TSfprom(\tId).\cpath;\\
  \PromState & \triangleq & \p.\TSfprom(\tId).\PromState;\\
  \PromState' & \triangleq & \p'.\TSfprom(\tId).\PromState;\\
  \PromSet' & \triangleq & \p'.\TSfprom(\tId).\PromSet;\\
  \PromSet & \triangleq & \p.\TSfprom(\tId).\PromSet;\\
  \hmap   & \triangleq & \aT.\hmap.
\end{array}\]
Мы покажем, что обещающая машина в состоянии $\p$ может сделать переход, связанный с $\tape(\cpath)$, в состояние $\p'$,
и при этом будет выполняться $(\aT, \p') \in \simrelPre \cup \simrel$.
Далее мы проведём разбор вариантов состояния экземпляра $\tape(\cpath)$.

Очевидно, что для любого нового состояния $\p'$, такого что $\p \promStep \p'$, выполняется 
$(\aT, \p') \in \invReach \cap \invMemThree$.
Выполнение $\invPrefix(\aT, \p')$ следует из того, что
  $\cpath' \in \nextPathCom{\cpath}{\Carm}{\tape}$, $\invAnextCommitted(\aT)$, и исполнение экземпляра $\tape(\cpath)$ завершено.
Поскольку переход $\p \promStep \p'$ будет использован для того, чтобы ``нагнать'' исполнение машины \ARMt,
то он не является переходом \transenv{Завершения записи}. Как следствие, $\p'.\Mprom = \p.\Mprom$, и 
утверждение $\invMemTwo(\aT, \p')$ следует из $\invMemTwo(\aT, \p)$.
Кроме того, очевидно, что будет выполняться
либо $\invPromUptoARM(\aT, \p')$, либо $\invPromUptoARMnot(\aT, \p')$ в зависимости от
завершённости экземпляра $\tape(\cpath')$.

%% \textbf{Note}:
%% %% ($CorrectType$): \\
%% $\aT.\Cfarm = \p.\Cfprom$ and the type of the 
%%  the type of the step $\p \promStep \p'$ corresponds to $\tape(\cpath)$.
%% $\invATapeCf(\aT)$,

Рассмотрим варианты состояния экземпляра $\tape(\cpath)$.
\begin{itemize}
  \item $\tape(\cpath) = \tapeNop$, $\tape(\cpath) = \tapeAssign$ или $\tape(\cpath) = \tapeIfGoto{\IfState}{\z}$. \\
    Во всех трёх случаях переход обещающей машины $\p \promStep \p'$ будет внутренним, т.е. $\epsilon$-переходом.
    %% By the transition definition, $\cpath' = \nextPathProm(\cpath, 1) = \cpath ++ [\lastInstr{\cpath} + 1]$.
    Поскольку $\epsilon$-переход не меняет состояние памяти, то утверждение $\invMemOne(\aT, \p')$ непосредственно
    следует из $\invMemOne(\aT, \p)$.
    Аналогично выполняются $\invView (\aT, \p')$ (поскольку $\p'.\TSfprom(\tId).\V = \p.\TSfprom(\tId).\V$),
    $\invState(\aT, \p')$ (поскольку $\p'.\TSfprom(\tId).\PromState = \p.\TSfprom(\tId).\PromState$ и
    $\regstcom(\Carm(\tId), \tape, \cpath') = \regstcom(\Carm(\tId), \tape, \cpath')$) и 
    $\invComWrite(\aT, \p')$ (поскольку экземпляр $\tape(\cpath)$ не является ни записью, ни барьером).

  \item $\tape(\cpath) = \tapeFence{\Committed}{\LD}$. \\
    Обещающая машина делает переход, соответствующий приобретающему барьеру памяти.
    Как следствие, новый базовый фронт $\Rcur'$ равен старому приобретающему фронту $\Racq$.
    Утверждения $\invMemOne(\aT, \p')$ и $\invState(\aT, \p')$ выполняются, поскольку
    переход не меняет состояния локальных переменных и памяти.

    Проверим, что выполняется $\invView (\aT, \p')$. Для этого нам нужно показать, что следующее утверждение верно:
      \[\begin{array}{l}
        \forall \tId', \tape' = \aT.\tapef(\tId'), \cpath'' = \p.\TSfprom(\tId').\cpath, \\
        \quad (\p'.\TSfprom(\tId').\Racq \le \bigsqcup \readsSatisfiedR(\cpath'', \tape', \hmap) \\
        \quad \quad \sqcup \bigsqcup \opstau(\tId', \cpath'', \tape', \hmap)) \land \\
        \quad (\p'.\TSfprom(\tId').\Rcur \le \bigsqcup \readsSatisfiedR(\lastLD(\tape', \cpath''), \tape', \hmap) \\
        \quad \quad \sqcup \bigsqcup \opstau(\tId', \cpath'', \tape', \hmap)) \land \\
        \quad (\p'.\TSfprom(\tId').\Rrel \le \bigsqcup \readsSatisfiedR(\lastLDSY(\tape', \cpath''), \tape', \hmap) \\
        \quad \quad \sqcup \bigsqcup \opstau(\tId', \lastSY(\tape', \cpath''), \tape', \hmap))).
      \end{array}\]
    Зафиксируем $\tId', \tape', \cpath''$. Если $\tId' \not = \tId$, то утверждение следует из $\invView(\aT, \p)$ и 
    $\p'.\TSfprom(\tId').\V = \p.\TSfprom(\tId').\V$.
    Пусть $\tId' = \tId$. Как следствие, $\tape' = \tape$ и $\cpath'' = \cpath'$.
    Также верно, что
    \begin{itemize}
      \item $\p'.\TSfprom(\tId).\Racq = \p.\TSfprom(\tId).\Racq = \Racq$,
      \item $\p'.\TSfprom(\tId).\Rcur = \p.\TSfprom(\tId).\Racq = \Racq$,
      \item $\p'.\TSfprom(\tId).\Rrel = \p.\TSfprom(\tId).\Rrel = \Rrel$.
    \end{itemize}
    Тогда утверждение в упрощённой форме выглядит так:
     \[\begin{array}{l}
       \quad (\Racq \le \bigsqcup \readsSatisfiedR(\cpath', \tape, \hmap) \\
       \quad \quad \sqcup \bigsqcup \opstau(\tId, \cpath', \tape, \hmap)) \land \\
       \quad (\Racq \le \bigsqcup \readsSatisfiedR(\lastLD(\tape, \cpath'), \tape, \hmap) \\
       \quad \quad \sqcup \bigsqcup \opstau(\tId, \cpath', \tape, \hmap)) \land \\
       \quad (\Rrel \le \bigsqcup \readsSatisfiedR(\lastLDSY(\tape, \cpath'), \tape, \hmap) \\
       \quad \quad \sqcup \bigsqcup \opstau(\tId, \lastSY(\tape, \cpath'), \tape', \hmap))).
     \end{array}\]
     Первый и третий конъюнкты следует из того, что выполняется $\invView(\aT, \p)$ и $\cpath' > \cpath$.
     Второй конъюнкт следует из определения $\lastLD(\tape, \cpath')$ и $\invView(\aT, \p)$.

     Утверждение $\invComWrite(\aT, \p')$ выполняется, поскольку экземпляр $\tape(\cpath)$ не является ни записью,
     ни $\SY$-барьером.

   \item $\tape(\cpath) = \tapeFence{\Committed}{\SY}$. \\
    Обещающая машина делает переход, соответствующий высвобождающему барьеру памяти.
    Как следствие, новый высвобождающий фронт $\Rrel'$ равен старому базовому фронту $\Rcur$.
    Утверждения $\invMemOne(\aT, \p')$ и $\invState(\aT, \p')$ выполняются, поскольку
    переход не меняет состояния локальных переменных и памяти.
    То, что утверждение $\invView(\aT, \p')$ выполняется, может быть показано аналогичными выкладками,
    что и в случае $\tape(\cpath) = \tapeFence{\Committed}{\LD}$.
     
    Из верности утверждения $\invComWrite(\aT, \p)$ следует, что
    не существует незавершённого экземпляра записи с путём $\cpath^{write} > \cpath$.
    Это означает, что выполняется $\invComWrite(\aT, \p')$.

    \item  $\tape(\cpath) = \tapeRead{(\tapeSatisfied{\Committed}{\writeEvt{\tId''}{ \cpath''}{ x}{\stval}})}$. \\
      We know that for all $\cpath^{*} < \cpath$, $\tape(\cpath)$ is committed. It means that
      the corresponding write $(\tId'', \cpath'')$ is either committed as it's a write from other thread,
      or it's committed as $\cpath'' < \cpath$.
      \[\begin{array}{l}
      \aT.\tapef(\tId'', \cpath'') = \tapeWrite{(\tapeWriteCommitted{\_}{x}{\stval})}; \\
      (\tau, \_, \R') \triangleq \aT.\hmap(\tId'', \cpath'') \not = (\bot, \_ , \bot). \\
      \end{array}\]
      
      We know that $\exists \expr, \Carm[\lastInstr{\cpath}] = ``\readInst{\reg}{\expr}"$,
      $\semf{\expr}{\p.\TSfprom(\tId).\PromState} = \semfcom{\expr}{} = x$ by $\invState(\aT, \p)$ and $\invATapeCfState(\aT)$.
      It means that the Promise machine can read from the same location as the ARM+$\tau$ machine did.

   Let's prove that $\writeEvt{x}{\stval}{\tau}{\R} \in \p.\Mprom \setminus \p.\TSfprom(\tId).\PromSet$, \ie,
   the Promise thread can read from the same write. By $\invMemOne(\aT, \p)$ we know that
   \[\begin{array}{l}
     \exists \R \le \R',
     (\cpath'' < \p.\TSfprom(\tId'').\cpath => \writeEvt{x}{\stval}{\tau}{\R} \in \p.\Mprom \setminus \underset{\tId}{\bigcup} \p.\TSfprom(\tId).\PromSet) \land \\
    \quad \quad (\cpath'' \ge \p.\TSfprom(\tId'').\cpath => \writeEvt{x}{\stval}{\tau}{\R} \in \p.\TSfprom(\tId'').\PromSet) \\
   \end{array}\]

   Fix $\R$. If $\tId'' = \tId$, then, as we know that $\cpath'' < \cpath$ by $\invAReadWriteOne(\aT)$ (\app{\ref{inv:invAReadWriteOne}}),
   the first conjunction holds.
   Suppose, $\tId'' \not = \tId$.
   If $\cpath'' < \p.\TSfprom(\tId'').\cpath$, then
      $\writeEvt{x}{\stval}{\tau}{\R} \in \p.\Mprom \setminus \p.\TSfprom(\tId).\PromSet$ holds.
   If $\cpath'' \ge \p.\TSfprom(\tId'').\cpath$, then
      as $\p.\TSfprom(\tId'').\PromSet$ and $\p.\TSfprom(\tId).\PromSet$ are disjoint, the statement holds.

  Let's prove that $\Rcur(x) \le \tau$:
  \[\begin{array}{l}
    \Rcur(x) \le (\bigsqcup \readsSatisfiedR(\lastLD(\tape, \cpath), \tape, \hmap)
                  \sqcup \bigsqcup \opstau(\tId, \cpath, \tape, \hmap))(x) \\
    \quad \quad \quad
        \le (\bigsqcup \readsSatisfiedR(\lastCF(\tape, \cpath), \tape, \hmap)
                  \sqcup \bigsqcup \opstau(\tId, \cpath, \tape, \hmap))(x) \\
    \quad \quad \quad
     \le \tau, \text{by } \invAview(\aT).
  \end{array}\]
  
  We know that $\Rcur' = \Rcur \sqcup [x@\tau]$, $\Racq' = \Racq \sqcup \R, \R \le \R'$,
  $\cpath' = \cpath ++ [\lastInstr{\cpath} + 1]$, $\PromState' = \PromState[\reg \mapsto \stval]$.

  $\invMemOne(\aT, \p')$ holds for the same reason as in $\tape(\cpath) = \Nop$.
  $\invState(\aT, \p')$ holds as
  $\PromState' = \PromState[\reg \mapsto \stval] =$
  $\regstcom(\Carm, \tape, \cpath)[\reg \mapsto \stval] = \regstcom(\Carm, \tape, \cpath')$.
  
  We check $\invView (\aT, \p')$.
  \[\begin{array}{l}
    \Racq' \le \bigsqcup \readsSatisfiedR(\cpath', \tape, \hmap) \sqcup \bigsqcup \opstau(\tId, \cpath', \tape, \hmap) \\
    \quad (\Racq' = \Racq \sqcup \R) \\
    \Racq \sqcup \R \le \bigsqcup \readsSatisfiedR(\cpath', \tape, \hmap) \sqcup \bigsqcup \opstau(\tId, \cpath', \tape, \hmap) \\
    \Racq \sqcup \R \le \R' \sqcup \bigsqcup \readsSatisfiedR(\cpath, \tape, \hmap) \sqcup
          [x@\tau] \sqcup \bigsqcup \opstau(\tId, \cpath, \tape, \hmap) \\
    \quad (\R' \sqcup [x@\tau] = \R') \\
    \Racq \sqcup \R \le \R' \sqcup \bigsqcup \readsSatisfiedR(\cpath, \tape, \hmap)
          \sqcup \bigsqcup \opstau(\tId, \cpath, \tape, \hmap) \\
  \end{array}\]
  We know that $\Racq \le \bigsqcup \readsSatisfiedR(\cpath, \tape, \hmap)$ $\sqcup \bigsqcup \opstau(\tId, \cpath, \tape, \hmap)$
  by $\invView(\aT, \p)$, and $\R \le \R'$. The statement holds.

  \[\begin{array}{l}
    \Rcur' \le \bigsqcup \readsSatisfiedR(\lastLD(\tape, \cpath'), \tape, \hmap)
                 \sqcup \bigsqcup \opstau(\tId, \cpath', \tape, \hmap) \\
    \quad (\Rcur' = \Rcur \sqcup [x@\tau]) \\
    \Rcur \sqcup [x @ \tau] \le \bigsqcup \readsSatisfiedR(\lastLD(\tape, \cpath'), \tape, \hmap)
                 \sqcup \bigsqcup \opstau(\tId, \cpath', \tape, \hmap) \\
    \quad (\tape(\cpath').type \not = Fence) \\
    \Rcur \sqcup [x @ \tau] \le \bigsqcup \readsSatisfiedR(\lastLD(\tape, \cpath), \tape, \hmap)
                 \sqcup [x @ \tau] \sqcup \bigsqcup \opstau(\tId, \cpath', \tape, \hmap) \\
    \Rcur \le \bigsqcup \readsSatisfiedR(\lastLD(\tape, \cpath), \tape, \hmap)
                 \sqcup \bigsqcup \opstau(\tId, \cpath', \tape, \hmap) \\
    \Rcur \le \bigsqcup \readsSatisfiedR(\aT, \tId, \lastLD(\tape, \cpath)) \sqcup \bigsqcup \opstau(\aT, \tId, \cpath) \\
  \end{array}\]
    Follows from $\invView(\aT, \p)$.
    
    $\Rrel' = \Rrel$, so the $\invView(\aT, \p')$ holds.

    $\invComWrite(\aT, \p')$: Obviously holds as $\tape(\cpath)$ is not a write or a fence and $\invComWrite(\aT, \p)$ holds.

  \item $\tape(\cpath) = \tapeWrite{(\tapeWriteCommitted{\InMemory}{x}{\stval})}$.
    We know $(\tau, \_, \R') \triangleq \aT.\tmap(\tId, \cpath) \not = \bot$, as the write is committed.
    $\exists \expr_0, \expr_1, \Carm[\lastInstr{\cpath}] = ``[\expr_0] := \expr_1''$.
    By $\invATapeCfState(\aT)$ (\app{\ref{inv:invATapeCfState}}) we know that $x = \semfcom{\expr_0}{}$ and
    $\stval = \semfcom{\expr_1}{}$. The Promise machine is able to write the same value to the same locaiton
    due to $\invState(\aT, \p)$.

    $\writeEvt{x}{\stval}{\tau}{\R} \in \p.\TSfprom(\tId).\PromSet$
    directly follows from $\invMemOne(\aT, \p)$.
    $\R = \Rrel \sqcup [x@\tau]$ as the Promise machine wasn't able to promise the write through a release fence---
    $\Rrel$ remained the same starting from the moment the message was promised.

    $\invState(\aT, \p')$ holds trivially (no changes in the variable states for both machines).

    We need to check $\Rcur(x) < \tau$.
    \[\begin{array}{l}
  \Rcur(x) \le (\bigsqcup \readsSatisfiedR(\lastLD(\tape, \cpath), \tape, \hmap) \sqcup \bigsqcup \opstau(\tId, \cpath, \tape, \hmap))(x) \\
  \quad \quad \quad \le (\bigsqcup \readsSatisfiedR(\lastCF(\tape, \cpath), \tape, \hmap) \sqcup \bigsqcup \opstau(\tId, \cpath, \tape, \hmap))(x) \\
  \quad \quad \quad < \tau\\
    \end{array}\]
  by $\invAview(\aT)$ (\app{\ref{thm:invAview}}).
  
    We need to check $\R \le R'$.
    \[\begin{array}{l}
  \Rrel \le \bigsqcup \readsSatisfiedR(\aT, \tId, \lastLDSY(\tape, \cpath)) \sqcup \bigsqcup \opstau(\aT, \tId, \lastSY(\tape, \cpath)) \\
  \quad (\R = \Rrel \sqcup [x@\tau]) \\
  \R \le [x@\tau] \sqcup \bigsqcup \readsSatisfiedR(\aT, \tId, \lastLDSY(\tape, \cpath)) \sqcup \bigsqcup \opstau(\aT, \tId, \lastSY(\tape, \cpath)) \\
  \quad (\lastLDSY(\tape, \cpath) \le \lastSY) \\
  \R \le [x@\tau] \sqcup \bigsqcup \readsSatisfiedR(\aT, \tId, \lastSY  (\tape, \cpath)) \sqcup \bigsqcup \opstau(\aT, \tId, \lastSY(\tape, \cpath)) \\
  \quad (\text{by definition of the $\rmap$ component}) \\
  \R \le \R'.
    \end{array}\]

We know that:
    \[\begin{array}{l}
\Rcur' = \Rcur \sqcup [x@\tau];\\
\Racq' = \Racq \sqcup [x@\tau];\\
\PromSet' = \PromSet \setminus \{ \writeEvt{x}{\stval}{\tau}{\R} \};\\
\cpath' = \cpath ++ [\lastInstr{\cpath} + 1].
    \end{array}\]

Let's check $\invMemOne(\aT, \p')$.
We need to show:
\[\begin{array}{l}
  \forall \tId', y, \stval', \tau', \R'', \cpath'', \\
  \tapeWrite{(\tapeWriteCommitted{\_}{y}{\stval'})} = \aT'.\tapef(\tId', \cpath''),
    (\tau', \_, \R'') = \aT'.\hmap(\tId', \cpath'') => \\
  \exists \R''' \le \R'': \\
  \quad (\cpath'' <   \p'.\TSfprom(\tId').\cpath => \writeEvt{y}{\stval'}{\tau'}{\R'''} \in
      \p'.\Mprom \setminus \underset{\tId}{\bigcup} \p'.\TSfprom(\tId).\PromSet) \land \\
  \quad (\cpath'' \ge \p'.\TSfprom(\tId').\cpath => \writeEvt{y}{\stval'}{\tau'}{\R'''} \in \p'.\TSfprom(\tId').\PromSet).\\
\end{array}\]
  Fix $\tId', y, \stval', \tau', \R'', \cpath''$.
  We know that
  $\p'.\TSfprom(\tId).\PromSet = \p.\TSfprom(\tId).\PromSet \setminus \{ \writeEvt{x}{\stval}{\tau}{\R} \}$.
  We also know that
  $\p'.\Mprom \setminus \underset{\tId}{\bigcup} \p'.\TSfprom(\tId).\PromSet = \{ \writeEvt{x}{\stval}{\tau}{\R} \} \cup \p.\Mprom \setminus \underset{\tId}{\bigcup} \p.\TSfprom(\tId).\PromSet$.
    
  If $\tId' \not = \tId$, then
    $\p'.\TSfprom(\tId').\cpath = \p.\TSfprom(\tId').\cpath$ and
    $\p'.\TSfprom(\tId').\PromSet = \p.\TSfprom(\tId').\PromSet$.
  The simplified statement follows from $\invMemOne(\aT, \p)$.
  
  Suppose, $\tId' = \tId$. If $\cpath'' \not = \cpath$, then $(y, \tau') \not = (x, \tau)$ by uniqueness of timestamps,
  so the simplified statement follows from $\invMemOne(\aT, \p)$.
  If $\cpath'' = \cpath$, then, as we know, $\cpath < \cpath'$, so exists $\R''' = \R$,
  $\writeEvt{y}{\stval'}{\tau'}{\R'''} = \writeEvt{x}{\stval}{\tau}{\R}$. The statement holds.
  

  Let's show $\invView (\aT, \p')$.
  \[\begin{array}{l}
    \Racq' \le \bigsqcup \readsSatisfiedR(\cpath', \tape, \hmap) \sqcup \bigsqcup \opstau(\tId, \cpath', \tape, \hmap) \\
    \quad (\Racq' = \Racq \sqcup [x@\tau]) \\
    \Racq \sqcup [x@\tau]
      \le \bigsqcup \readsSatisfiedR(\cpath', \tape, \hmap) \sqcup \bigsqcup \opstau(\tId, \cpath', \tape, \hmap) \\
    \quad (\tape(\cpath).type \not = Read) \\
    \Racq \sqcup [x@\tau]
      \le \bigsqcup \readsSatisfiedR(\cpath, \tape, \hmap) \sqcup \bigsqcup \opstau(\tId, \cpath', \tape, \hmap) \\
    \Racq \sqcup [x@\tau]
      \le \bigsqcup \readsSatisfiedR(\cpath, \tape, \hmap) \sqcup [x@\tau] \sqcup \bigsqcup \opstau(\tId, \cpath, \tape, \hmap) \\
  \end{array}\]
  Follows from $\invView(\aT, \p')$

  The same proof for $\Rcur'$. $\Rrel' = \Rrel$, the statement holds.

  $\invComWrite(\aT, \p')$: Obviously holds as there is no $\cpath$-previous instruction, which is not executed by the Promise thread.
  \end{itemize}

Now we show that the Promise machine can certify the step $\p \promStepgen{} \p'$,
\ie, show that the thread $\tId$ can make a finite number of steps
and fulfill all its promises at these steps.
As we know that $\simrelBase(\aT, \p')$, we apply \app{\ref{cert-thm}}.

%% \[
%% \begin{array}{l}
%% \textLet \TS' \triangleq \p'.\TSfprom(\tId) \textIn \\
%% \exists k, \{\ptid_i\}_{i \in [0..k]}, (\forall i \in [0..k), \ptid_i \promTStepgen{} \ptid_{i + 1}), \\
%% \quad \ptid_0 = \angled{\p'.\Mprom, \TS'.\cpath, \TS'.\PromState, \TS'.\V, \TS'.\PromSet, \p'.\Cfprom(\tId)},
%%       \ptid_k.\PromSet = \emptyset.
%% \end{array}
%% \]
%% Note that the promises of the thread $\tId$ at the state $\p'$ ($\p'.\TSfprom(\tId).\PromSet$)
%% correspond to writes committed by the same $\tId$ in the ARM+$\tau$ machine, each of which are indexed
%% by $\cpath^{*}$ greater than the Promise thread pointer ($\cpath' = \p'.\TSfprom(\tId).\cpath$).

%% We have to show that the Promise machine can make transitions corresponding to instruction instances (elements of
%% $\aT'.\tapef(\tId)$) in the $\cpath' .. \cpath^{end}$ range.
%% The transitions of the Promise machine are determined by a certificate constructed from the program
%% and $\aT'.\tapef(\tId)$.

%% The $\Certificate : \Path \rightharpoonup \CommandState$ is a partial function, where
%% \[
%% \CommandState ::= \angled{\loc@\tau} \mid \loc \mid \Any
%%   %% \Committed \mid \Taken \mid \Ignored \mid \tapeAssign \mid \tapeNop \mid \Any
%% \]
%% A $\Certificate$ is a guidence for a thread which steps should it take to
%% fulfill its promises. We construct a $\Certificate$ from the corresponding ARM thread's state
%% via the following function:
%% \[
%% \tapeToCertificate(\tape, \Carm, \tId, \hmap) \triangleq \\
%% \]
%% \[
%% \quad \lambda \cpath .
%% \begin{cases}
%% \angled{\loc@\tmap(\tId, \cpath)} &
%%                      \textIf \exists \loc, \tape(\cpath) = \tapeWrite{(\tapeWriteCommitted{\_}{\loc}{\_})}; \\
%% \angled{\loc@\tmap(\tId', \cpath')} &
%%                      \textIf \exists \loc, \tId', \cpath', \tape(\cpath) =
%%                        \tapeRead(\tapeSatisfied{\Committed}{\stRequestWrite{\tId'}{\cpath'}{\loc}{\_}}); \\
%% \loc & \textIf \exists \reg, \expr_0, \expr_1, \loc = |[\expr_0|]_{com}^{\cpath}, \\
%%      & \Carm[\lastInstr{\cpath}] \in \{``\reg := [\expr_0]", ``[\expr_0] := \expr_1"\}; \\
%% %% \Fstate  & \textIf \exists \Fstate, \tape(\cpath) = \tapeFence{\Fstate}{\_};\\
%% %% \IfState & \textIf \exists \IfState, \tape(\cpath) = \tapeIfGoto{\IfState}{\_};\\
%% %% \tape(\cpath) & \textIf \tape(\cpath) \in \{ \tapeAssign, \tapeNop \};\\
%% \Any    & \text{otherwise}.
%% \end{cases}
%% \]
%% %% and check if the Promise thread may certify its step with the constructed $\Certificate$.

%% A $\Certificate$ guides a Promise thread execution in the following way. If
%% the next instruction to be executed, $\Cprom[\lastInstr{\cpath}]$, is:
%% \begin{description}
%%   \item[a read,] then $\Certificate(\cpath)$ is
%%     either $\angled{x@\tau}$ for some $x$ and $\tau$---the thread
%%     has to read from a write to $x$ with a timestamp $\tau$;
%%     or $x$---the thread has to read from $x$ with a minimal
%%     timestamp possible ($\Rcur(x)$).
%%   \item[a write,] then $\Certificate(\cpath)$ is
%%     either $\angled{x@\tau}$ for some $x$ and $\tau$---the thread
%%     has to promise to $x$ with a timestamp $\tau$,
%%     and fulfill the promise immediately;
%%     or $x$---the thread has write to $x$ with a minimal
%%     timestamp possible ($\Rcur(x) + \epsilon$).
%% \end{description}
%% In other cases the thread just executes the next instruction without restrictions.
%% %% The certification continues to execute the thread until it fulfills all its promises.

%% For the certification execution we show the following invariant:
%% \[\begin{array}{l}
%% \forall i \in [0..k], \PromState = \ptid_i.\PromState, \regstcom = \regstcom(\ptid_i.\Cprom, \tape, \ptid_i.\cpath),\\
%% \quad \forall \reg, \regstcom(\reg) \in \{\PromState(\reg), \bot\}.
%% \end{array}\]
%% This invariant guarantees that the Promise machine is able to evaluate expressions, which correspond to
%% committed instruction instances, getting the same values as the ARM+$\tau$ machine did:
%% take the same branches at conditional instructions,
%% read from the same locations, \emph{etc}.
%% As far as the certification execution reads the same values (the same writes exist due to $\invMemOne(\aT', \p')$
%% and $\invMemTwo(\aT', \p')$) on committed read instructions as the ARM+$\tau$ machine did, the invariant is trivially
%% holds.

%% The main problem is to show that the Promise machine can read with the same timestamps, and write with the
%% same timestamps and lesser message views as the ARM+$\tau$ machine did.
%% It's easy to show for the message views. We know that when the thread $\tId$ promised some write
%% $\writeEvt{y}{\stval'}{\tau'}{\R'}$, it assigned $\R'$ to be equal to combination of its $\Rrel$
%% at the moment and $[y@\tau']$. The only instuction, which is able to modify $\Rrel$, is
%% a release fence---an \SY fence in ARM+$\tau$.
%% As far as $\invPromUptoARM(\aT, \p)$ holds and $\cpath' = \p'.\TSfprom(\tId).\cpath = \p.\TSfprom(\tId).\cpath$,
%% we know that there are no \SY instances in the $\cpath' .. \cpath^{end}$ range.
%% So, if the current view of the thread $\tId$ is less than a necessary timestamp at the moment it has to fulfill
%% a promise ($\ptid_i.\cpath$ points to a committed write), then the release view is correct and doesn't prevent
%% of making a transition.

%% We know that all instruction instances of the ARM execution in the $\cpath' .. \cpath^{end}$ range
%% have fully determined addresses, $\ie$, the constructed certificate determines target locations for all
%% write and read instances in the range.
%% By the introduced state invariant, the addresses are determined in the same
%% way for the Promise machine. We also know that if there are $\LD$ fences in the $\cpath' .. \cpath^{end}$ range,
%% then they are committed (by the precondition of \transenv{Write commit} transition), and every read before
%% the last $\LD$ fence is committed as well.
%% By $\invView(\aT', \p')$ we know that views of the Promise machine at $\ptid_0$ are less than the corresponding
%% views of the ARM machine.
%% Due to the way the certificate guides the execution,
%% we know that for every $i \in [0..k]$ and $\ptid_k.\cpath$ in the $\cpath' .. \cpath^{end}$ range,
%% \[\begin{array}{l}
%% \textLet viewf = \lambda \cpath. \Rarm(\aT', \tId, \cpath) \textIn \\
%% \ptid_k.\Rcur \le viewf(\ptid_k.\cpath) \sqcup\\
%% \quad \bigsqcup \{[y@viewf(\cpath') + \epsilon] \mid \cpath' < \ptid_k.\cpath, \Carm[\lastInstr{\cpath'}] = ``[\expr] := \_'',\\
%% \quad \quad  \quad \quad y \triangleq |[\expr|]^{\ptid_k.\PromState}, \tape(\cpath') \; \text{is not committed}\}
%% \end{array}\]
%% where the set of singleton views corresponds to writes, which are not committed by the ARM machine, but have to be committed
%% by the Promise machine. Due to \cref{thm:invAview} and as we may choose $\epsilon$ in such a way
%% that $[y@viewf(\cpath') + \epsilon]$ is smaller than a timestamp of any promised write to the location $y$,
%% the Promise machine is able to fulfill its promises during the certification.
\end{proof}

%% \lemmaFst*
\begin{proof}
  Let's consider the first clause.
  Fix $\aT, \p$, and $\aT'$ such that $\aT \armStepgen{\lnot \; \transenv{Write commit} \; \tId} \aT'$.
  Some notations:
\[
\begin{array}{l l l}
\multicolumn{3}{l}{\angled{\cpath', \PromState, \angled{\Rcur, \Racq, \Rrel}, \PromSet} \triangleq \p.\TSfprom(\tId);}\\
\tape     & \triangleq & \aT.\tapef(\tId); \\
%% \cpathSY & \triangleq & \lastSY(\tape, \cpath');\\
%% \cpathLDSY & \triangleq & \lastLDSY(\tape, \cpath').\\
\end{array}
\]
  We need to show that $(\simrel \cup \simrelPre)(\aT', \p)$ holds.
  $(\aT', \p) \in \invReach \cap \invMemThree$ obviously holds.
  $\invPrefix(\aT', \p), \invView(\aT', \p)$ hold because
  the rules of the ARM machine don't change the committed prefix of $\aT.\tapef(\tId)$.

  As $\simrel(\aT, \p)$ holds and every instruction before committed $\SY$ fence has to be committed according to
  {\sf \bf Fence commit} $\SY$ rule requirements, every $\SY$ fence committed by the ARM machine is executed by the Promise
  machine. If the step $\aT \armStepgen{} \aT'$ is {\sf \bf Fence commit} $\SY$, then there has to be committed writes,
  which are $\cpath$-after the fence (according to requirements of {\sf \bf Write commit} rule). Thus,
  $\invComWrite(\aT', \p)$ holds. If the step $\aT \armStepgen{} \aT'$ is not {\sf \bf Fence commit} $\SY$ and {\sf \bf Write commit},
  then $\invComWrite(\aT', \p)$ directly follows from $\invComWrite(\aT, \p)$.

We check that either $\invPromUptoARM(\aT', \p)$ or $\invPromUptoARMnot(\aT', \p)$ holds.
If the transition is \transenv{Propagate}, then $\invPromUptoARM(\aT', \p)$ holds, as the ARM machine
doesn't commit on the step. Otherwise, it operates on some instruction instance $(\tId, \cpath)$.

Consider, if $\cpath'$ is equal to $\cpath$ or not:
\begin{itemize}
  \item $\cpath' = \cpath$. 
    If $\aT'.\tapef(\tId, \cpath)$ is not committed, then,
    as we know that $\forall \tId' \not = \tId, \invPromUptoARMtId(\tId', \aT', \p)$,
    $\invPromUptoARM(\aT', \p)$ holds.

    Suppose, $\aT'.\tapef(\tId, \cpath)$ is committed.\\
    $\forall \tId' \not = \tId, \invPromUptoARMtId(\tId', \aT', \p')$: \\
      As $\invPromUptoARMtId(\tId', \aT, \p)$ holds,
      and $\forall \tId' \not = \tId$, $\aT'.\tapef(\tId') = \aT.\tapef(\tId') \land$
          $\p'.\TSfprom(\tId').\cpath = \p.\TSfprom(\tId').\cpath$.
    Thus, $\invPromUptoARMnot(\aT', \p')$ holds.
  \item $\cpath' \not = \cpath$.
    We know that $\aT'.\TSfprom(\tId).\tape(\cpath) = \aT.\TSfprom(\tId).\tape(\cpath)$ and it is not committed
    because $\invPromUptoARM(\aT, \p)$ holds.
    $\forall \tId' \not = \tId, \invPromUptoARMtId(\tId', \aT', \p')$.
    Thus, $\invPromUptoARM(\aT', \p')$ holds.
\end{itemize}

$(\aT', \p) \in \invMemOne \cap \invMemTwo$ holds because
  $\lnot$ \transenv{Write commit} transitions of the ARM machine don't increase a number of committed write instances and
  don't change committed cells of $\aT.\tapef(\tId)$.
  
The first clause is proved.

  Let's consider the second clause. \\
  Fix $\aT, \p$, and $\aT'$ such that $\aT \armStepWriteCommit \aT'$.
  Some notations:
\[
\begin{array}{l l l}
\multicolumn{3}{l}{\angled{\cpath', \PromState, \angled{\Rcur, \Racq, \Rrel}, \PromSet} \triangleq \p.\TSfprom(\tId);}\\
\R' & \triangleq & \Rrel \sqcup [x @ \tau]; \\
\PromSet' & \triangleq & \PromSet \cup \{\writeEvt{x}{\stval}{\tau}{\R'}\}; \\
\TSfprom' & \triangleq & \p.\TSfprom[\tId \mapsto \angled{\cpath', \PromState, \angled{\Rcur, \Racq, \Rrel}, \PromSet'}]; \\
\tape     & \triangleq & \aT.\tapef(\tId); \\
\hmap     & \triangleq & \aT.\hmap; \\
%% \cpathSY & \triangleq & \lastSY(\tape, \cpath');\\
%% \cpathLDSY & \triangleq & \lastLDSY(\tape, \cpath').\\
\end{array}
\]
We know that $\tau$ hasn't been used for messages to the location $x$, as
it hasn't been used for committed writes in $\aT$
(by ARM+$\tau$ \transenv{Write commit} preconditions)
and $\invMemOne(\aT, \p)$, $\invMemTwo(\aT, \p)$ hold.
The \transenv{Promise write} transition doesn't have any other preconditions
(except for certification, which holds as in the previous lemma), so
it can make a step $\p \promStepPromiseRPrime \p'$,
where $\p' = \angled{\p.\Mprom \cup \{\writeEvt{x}{\stval}{\tau}{\R'}\}, \TSfprom'}$.

We need to show that $(\simrelPre \cup \simrel)(\aT', \p')$ holds.
$(\aT', \p') \in \invReach \cap \invMemThree$ obviously holds. \\
$\invPrefix(\aT', \p'), \invView(\aT', \p'), \invState(\aT', \p')$ hold because
  the \transenv{Write Commit} transition doesn't change the committed prefix of $\aT.\tapef(\tId)$.

  As $\simrel(\aT, \p)$ holds and every instruction before committed $\SY$ fence has to be committed according to
  {\sf \bf Fence commit} $\SY$ rule requirements, every $\SY$ fence committed by the ARM machine is executed by the Promise
  machine. Thus, $\invComWrite(\aT', \p)$ obviously holds.

\noindent
Let's show that $\invMemOne(\aT', \p')$ holds.
  We need to show: \\
  \[\begin{array}{l}
  \forall \tId', y, \stval', \tau', \R'', \cpath'', \\
  \tapeWrite{(\tapeWriteCommitted{\_}{y}{\stval'})} = \aT'.\tapef(\tId', \cpath''),
    (\tau', \_, \R'') = \aT'.\hmap(\tId', \cpath'') => \\
  \exists \R''' \le \R'': \\
  \quad (\cpath'' <   \p'.\TSfprom(\tId').\cpath => \writeEvt{y}{\stval'}{\tau'}{\R'''} \in
      \p'.\Mprom \setminus \underset{\tId}{\bigcup} \p'.\TSfprom(\tId).\PromSet) \land \\
  \quad (\cpath'' \ge \p'.\TSfprom(\tId').\cpath => \writeEvt{y}{\stval'}{\tau'}{\R'''} \in \p'.\TSfprom(\tId').\PromSet).\\
  \end{array}\]
  Fix $\tId', y, \stval', \tau', \R'', \cpath''$.
  We know that the set of fulfilled messages are the same for $\p$ and $\p'$:
  \[\begin{array}{l}
  \p'.\Mprom \setminus \underset{\tId}{\bigcup} \p'.\TSfprom(\tId).\PromSet = \\
    (\p.\Mprom \cup \{\writeEvt{x}{\stval}{\tau}{\R'}\}) \setminus
       (\{\writeEvt{x}{\stval}{\tau}{\R'}\} \cup \underset{\tId}{\bigcup} \p.\TSfprom(\tId).\PromSet)\} = \\
    \p.\Mprom \setminus \underset{\tId}{\bigcup} \p.\TSfprom(\tId).\PromSet.
  \end{array}\]
  
  Let's check two cases: $\tId'$ is equal to $\tId$ or not.
  \begin{itemize}
    \item $\tId' = \tId$.
    We know $\p'.\TSfprom(\tId).\cpath = \p.\TSfprom(\tId).\cpath = \cpath'$.
    We want to show:
    \[\begin{array}{l}
    \tapeWrite{(\tapeWriteCommitted{\_}{y}{\stval'})} = \aT'.\tapef(\tId, \cpath''),
       (\tau', \_, \R'') = \aT'.\hmap(\tId, \cpath'') => \\
    \exists \R''' \le \R'': \\
    \quad (\cpath'' <   \cpath' => \writeEvt{y}{\stval'}{\tau'}{\R'''}
               \in \p.\Mprom \setminus \underset{\tId}{\bigcup} \p.\TSfprom(\tId).\PromSet) \land \\
    \quad (\cpath'' \ge \cpath' => \writeEvt{y}{\stval'}{\tau'}{\R'''}
               \in \p.\TSfprom(\tId).\PromSet \cup \{\writeEvt{x}{\stval}{\tau}{\R'}\}).\\
    \end{array}\]
    Let's check if $\cpath''$ is equal to $\cpath$ or not.
    \begin{itemize}
      \item $\cpath'' = \cpath$. We know that $(y, \stval', \tau') = (x, \stval, \tau)$ and $\cpath \ge \cpath'$.

        We want to show:
    \[\begin{array}{l}
    \tapeWrite{(\tapeWriteCommitted{\_}{x}{\stval})} = \aT'.\tapef(\tId, \cpath),
       (\tau, \_, \R'') = \aT'.\hmap(\tId, \cpath) => \\
    \exists \R''' \le \R'': \\
    \quad (\cpath <   \cpath' => \writeEvt{x}{\stval}{\tau}{\R'''}
               \in \p.\Mprom \setminus \underset{\tId}{\bigcup} \p.\TSfprom(\tId).\PromSet) \land \\
    \quad (\cpath \ge \cpath' => \writeEvt{x}{\stval}{\tau}{\R'''}
               \in \p.\TSfprom(\tId).\PromSet \cup \{\writeEvt{x}{\stval}{\tau}{\R'}\}).\\
    \end{array}\]

      Let's choose $\R'''$ to be equal to $\R'$. $\R' = \Rrel \sqcup [x @ \tau]$ We need to show that $\R' \le \R$.
We know that $\cpath \ge \cpath'$, as it follows from
$\invPrefix(\aT, \p)$ and $\tape(\cpath) = \tapeWrite{(\tapePending{x}{\stval})}$,
which is the precondition of the ARM+$\tau$ \transenv{Write commit} transition.
\[
\begin{array}{l}
\Rrel \sqcup [x @ \tau] \le\\
\quad [x @ \tau] \sqcup \bigsqcup \opstau(\tId, \lastSY(\tape, \cpath'), \tape, \hmap) \\
\quad \quad      \sqcup \bigsqcup \readsSatisfiedR(\lastLDSY(\tape, \cpath'), \tape, \hmap),
\text{by } \invView(\aT, \p); \\
\Rrel \sqcup [x @ \tau] \le\\
\quad [x @ \tau] \sqcup \bigsqcup \opstau(\tId, \lastSY(\tape, \cpath), \tape, \hmap) \\
\quad \quad      \sqcup \bigsqcup \readsSatisfiedR(\lastLDSY(\tape, \cpath), \tape, \hmap),

\text{because } \cpath \ge \cpath';\\
\R' \le \\
\quad [x @ \tau] \sqcup \bigsqcup \opstau(\tId, \lastSY(\tape, \cpath), \tape, \hmap) \\
\quad \quad      \sqcup \bigsqcup \readsSatisfiedR(\lastLDSY(\tape, \cpath), \tape, \hmap),
\text{by definition of } \Rrel';\\
=> \R' \le \aT'.\rmap(\tId', \cpath),
\text{by definition of } \aT'.\rmap(\tId', \cpath).\\
\end{array}
\]
      Thus, the simplified statement:
      $\writeEvt{x}{\stval}{\tau}{\R'} \in \p.\TSfprom(\tId).\PromSet \cup \{\writeEvt{x}{\stval}{\tau}{\R'}\}$. It obviously holds.
    \end{itemize}

      \item $\cpath'' \not = \cpath$.
        We know:
        \begin{itemize}
          \item $\forall \cpath^{*}, \aT.\tapef(\tId, \cpath^{*})$ is committed $=>$ 
                $\aT'.\tapef(\tId, \cpath^{*}) = \aT.\tapef(\tId, \cpath^{*})$;
          \item $\forall \cpath^{*} \not = \cpath, \aT'.\hmap(\tId, \cpath^{*}) = \aT.\hmap(\tId, \cpath^{*})$;
          \item $(y, \tau') \not = (x, \tau)$ (by $\invAuniqWrite(\aT')$) $=>$ \\
                $\forall \Mprom, \writeEvt{y}{\stval'}{\tau'}{\R'''} \in \Mprom \Leftrightarrow \writeEvt{y}{\stval'}{\tau'}{\R'''} \in \Mprom \cup \{\writeEvt{x}{\stval}{\tau}{\R'}\}$.
        \end{itemize}

        After simplifications:
    \[\begin{array}{l}
    \tapeWrite{(\tapeWriteCommitted{\_}{y}{\stval'})} = \aT'.\tapef(\tId, \cpath''),
       (\tau', \_, \R'') = \aT.\hmap(\tId, \cpath'') => \\
    \exists \R''' \le \R'': \\
    \quad (\cpath'' <   \cpath' => \writeEvt{y}{\stval'}{\tau'}{\R'''}
               \in \p.\Mprom \setminus \underset{\tId}{\bigcup} \p.\TSfprom(\tId).\PromSet) \land \\
    \quad (\cpath'' \ge \cpath' => \writeEvt{y}{\stval'}{\tau'}{\R'''}
               \in \p.\TSfprom(\tId).\PromSet \}).\\
    \end{array}\]
      It follows from $\invMemOne(\aT, \p)$.
    \item $\tId' \not = \tId$.
    We know:
    \[\begin{array}{l}
      \aT'.\tapef(\tId') = \aT.\tapef(\tId'); \\
      \forall \cpath, \aT'.\hmap(\tId', \cpath) = \aT.\hmap(\tId', \cpath); \\
      \p'.\TSfprom(\tId') = \p.\TSfprom(\tId');\\
      \p'.\TSfprom(\tId').\PromSet = \p.\TSfprom(\tId').\PromSet.
    \end{array}\]
        After simplifications:
    \[\begin{array}{l}
    \tapeWrite{(\tapeWriteCommitted{\_}{y}{\stval'})} = \aT'.\tapef(\tId, \cpath''),
       (\tau', \_, \R'') = \aT.\hmap(\tId', \cpath'') => \\
    \exists \R''' \le \R'': \\
    \quad (\cpath'' <   \p.\TSfprom(\tId').\cpath => \writeEvt{y}{\stval'}{\tau'}{\R'''}
               \in \p.\Mprom \setminus \underset{\tId}{\bigcup} \p.\TSfprom(\tId).\PromSet) \land \\
    \quad (\cpath'' \ge \p.\TSfprom(\tId').\cpath => \writeEvt{y}{\stval'}{\tau'}{\R'''}
               \in \p.\TSfprom(\tId').\PromSet \}).\\
    \end{array}\]
    It follows from $\invMemOne(\aT, \p)$.
  \end{itemize}

  \noindent
Let's show that $\invMemTwo(\aT', \p')$ holds.
  We need to show:
  \[\begin{array}{l}
  \forall \writeEvt{y}{\stval'}{\tau'}{\R''} \in \p'.\Mprom, \tau' \not = \tstamp{0} => \\
  \quad \exists \tId', \cpath'', \R''' \ge \R'', \\
  \quad \quad \tapeWrite{(\tapeWriteCommitted{\_}{y}{\stval'})} = \aT'.\tapef(\tId', \cpath''),
        \aT'.\hmap(\tId', \cpath'') = (\tau', \_, \R'''). \\
  \end{array}\]
Fix $\writeEvt{y}{\stval'}{\tau'}{\R''}, \tau' \not = \tstamp{0}$. \\
  \[\begin{array}{l}
  \writeEvt{y}{\stval'}{\tau'}{\R''} \in \p.\Mprom \cup \{\writeEvt{x}{\stval}{\tau}{\R'}\} => \\
  \quad \exists \tId', \cpath'', \R''' \ge \R'', \\
  \quad \quad \tapeWrite{(\tapeWriteCommitted{\_}{y}{\stval'})} = \aT'.\tapef(\tId', \cpath''),
        \aT'.\hmap(\tId', \cpath'') = (\tau', \_, \R'''). \\
  \end{array}\]
  
  Let's check two options: \\
  \begin{itemize}
    \item $\writeEvt{y}{\stval'}{\tau'}{\R''} \in \p.\Mprom$. We know:
    \[\begin{array}{l}
      \forall \tId', \cpath^{*}, \aT.\tapef(\tId', \cpath^{*}) \text{ is committed} =>
        \aT'.\tapef(\tId', \cpath^{*}) = \aT.\tapef(\tId', \cpath^{*}); \\
      \forall (\tId', \cpath^{*}) \not = (\tId, \cpath), \aT'.\hmap(\tId', \cpath^{*}) = \aT.\tmap(\tId', \cpath^{*}).
    \end{array}\]
    The simplified statement:
    \[
      \exists \tId', \cpath'', \R''' \ge \R'',
      \tapeWrite{(\tapeWriteCommitted{\_}{y}{\stval'})} = \aT.\tapef(\tId', \cpath''),
      \aT.\hmap(\tId', \cpath'') = (\tau', \_, \R''').
    \]
    It follows from $\invMemTwo(\aT, \p)$.

    \item $\writeEvt{y}{\stval'}{\tau'}{\R''} = \writeEvt{x}{\stval}{\tau}{\R'}$. The simplified statement:
      \[\begin{array}{l}
    \exists \tId', \cpath'', \R''' \ge \R',
      \tapeWrite{(\tapeWriteCommitted{\_}{x}{\stval})} = \aT'.\tapef(\tId, \cpath''),
      \aT'.\hmap(\tId', \cpath'') = (\tau, \_, \R'''). \\
      \end{array}\]

    Let's take $\tId' = \tId, \cpath'' = \cpath, \R''' = \R$.\\
    The simplified statement:
    $\tapeWrite{(\tapeWriteCommitted{\_}{x}{\stval})} = \aT'.\tapef(\tId, \cpath), \aT'.\hmap(\tId, \cpath) = (\tau, \_, \R)$. \\
    It holds.
  \end{itemize}

\noindent
$\invPromUptoARM(\aT', \p) \cup \invPromUptoARMnot(\aT', \p)$: The same proof as in the previous case.
\end{proof}


%% \begin{figure}[h]
%% \begin{minipage}{0.6\textwidth}
%% $\begin{array}{l @{~} r @{~} l}
%% \Carm    & : & \ListOf{\StmtARM} \\  
%% \StmtARM & ::= & \readInst{\reg}{\expr} \\
%%          & |   & \writeInst{\expr_0}{\expr_1} \\
%%          & |   & \fenceInst{\FtypeProm} \\
%%          & |   & \ifGotoInst{\expr}{\z} \\
%%          & |   & \assignInst{\reg}{\expr} \mid \nop \\
%% \FtypeProm & ::= & \rel \mid \acq\\
%% \expr    & ::= & \val \mid \reg \mid \loc \mid uop \; \expr \\
%%          & |   & bop \; \expr_0 \; \expr_1 \\
%%          %% & :   & \Expr \\
%% \reg : \Reg & - & a, b, c, ...  \quad \; \text{(локальные переменные)} \\
%% \loc : \Loc & - & x, y, z, ... \quad \text{(локации)} \\
%% uop, bop & - & \text{арифметические операции} \\
%% \val, \z       & \in & \mathbb{Z}
%% \end{array}$
%% \captionof{figure}{Синтаксис программ}
%% \label{fig:syn-prog}
%% \end{minipage}
%% %
%% \begin{minipage}{0.4\textwidth}
%% $\begin{array}{l @{~} r @{~} l}
%%     \Path  & ::= & \ListOf{Label} \\
%%     \Label & ::= & \rlab{}{\loc}{\val} \\
%%            & |   & \wlab{}{\loc}{\val} \\
%%            & |   & \flab{\FtypeProm} \\
%%            & |   & \epsilon \\
%% \end{array}$
%% \captionof{figure}{Метки переходов}
%% \label{fig:lts}
%% \end{minipage}
%% \end{figure}

%% В этом разделе мы (i) опишем синтаксис исходного языка, который используется в нашей работе,
%% приведем алгоритм построения по программе (ii) помеченной системы переходов для обещающей машины
%% и (iii) \ARM-согласованных исполнений, а также предъявим теорему, которая их свяжет.

%% %% \section{Исходный язык}
%% Синтаксис исходного языка представлен на рисунке \ref{fig:syn-prog}.
%% В нашем синтаксисе многопоточная программа является параллельной композицией потоков на верхнем уровне, где
%% код каждого потока есть список инструкций ($\ListOf{\StmtARM}$).
%% Инструкция может быть чтением ($\readInst{\reg}{\expr}$),
%% записью ($\writeInst{\expr_0}{\expr_1}$),
%% барьером памяти ($\fenceInst{\FtypeProm}$), 
%% условным переходом на $\z$ позиций в списке инструкций ($\ifGotoInst{\expr}{\z}$),
%% присваиванием в локальную переменную ($\assignInst{\reg}{\expr}$),
%% пустой операцией ($\nop$).

\chapter{Компиляция программ в помеченную систему переходов}
Первым шагом для построения помеченной системы переходов по программе является
генерация всех возможных \emph{путей} ($\Path$) исполнения -- списков меток (см. рис. \ref{fig:lts}).
Далее по множеству путей строится конечный автомат, принимающий упомянутые пути.

Функция построения путей $\cmdsToLbls$ использует вспомогательную функцию $\instToLbl$,
которая принимает список инструкций потока $ilist$, указатель на текущую инструкцию $pos$ и
состояние локальных переменных $st$.
\[
\inarr{
  \cmdsToLbls : \Carm \rightarrow \Pset(\Path) \\
  \cmdsToLbls \; ilist = \instToLbl \; ilist \; 0 \; (\lambda \reg. 0) \\
  \\
  \instToLbl : \Carm \rightarrow \mathbb{N} \rightarrow (\Reg \rightarrow \mathbb{N}) \rightarrow \Pset(\Path) \\
  \instToLbl \; ilist \; pos \; \PromState \defeq \\
  \quad \textIf   pos < 0 \; || \; pos > length(ilist) \; \textThen \{ [] \} \\
  \quad \textElse \\
  \qquad \kw{match} \; ilist[pos] \; \kw{with} \\
  \qquad
    \begin{array}{@{}l}
    | ``\nop" \rightarrow \instToLbl \; ilist \; (pos + 1) \; \PromState \\
    | ``\readInst{\reg}{\expr}" \rightarrow \\
      \quad \{ \rlab{}{\semState{\expr}{\PromState}}{\val} : l \mid
      \forall \val \in \Val, l \in \instToLbl \; ilist \; (pos + 1) \; \PromState[\reg \mapsto \val] \} \\
    | ``\assignInst{\reg}{\expr}" \rightarrow
                 \instToLbl \; (pos + 1) \; \PromState[\reg \mapsto \semState{\expr}{\PromState}] \\
    | ``\writeInst{\expr_0}{\expr_1}" \rightarrow \\
      \quad \{ \wlab{}{\semState{\expr_0}{\PromState}}{\semState{\expr_1}{\PromState}} : l \mid
      \forall l \in \instToLbl \; ilist \; (pos + 1) \; \PromState \} \\
    | ``\fenceInst{\FtypeProm}" \rightarrow
      \{ \flab{\FtypeProm} : l \mid \forall l \in \instToLbl \; ilist \; (pos + 1) \; \PromState \} \\
    | ``\ifGotoInst{\expr}{k}" \rightarrow
      \letdef{step}{\textIf \semState{\expr}{\PromState} \; \textThen k \; \textElse 1} \\
      \quad \instToLbl \; (pos + step) \; \PromState \\
  \end{array} \\
}
\]

\chapter{Компиляция программ в предзапуски}

Для построения \ARM-согласованных исполнений по программе стандартно
используется следующая схема \cite{Vafeiadis-Narayan:OOPSLA13}.
В начале по программе строятся \emph{предзапуски} --- графы исполнений, в которых определена
только часть нужных отношений между событиями, а именно отношения порядка $\lPO$ и зависимостей
по данным $\lDATA$, управлению $\lCTRL$ и адресу $\lADDR$.
\[
\inarr{
  \PreExecutions \defeq \Pset(\langle set : \Pset(E), \lab : set \rightharpoonup \Label, \\
   \quad \lPO : \Pset(set \times set), \lCTRL : \Pset(set \times set),
         \lADDR : \Pset(set \times set), \lDATA : \Pset(set \times set) \rangle) \\
     %% , \state : \Reg \rightarrow \Val, \dep : \Pset(\Reg \times E)}) \\
}\]
При этом для каждой инструкции чтения
генерируется столько вариантов соответствующего события, сколько существует возможных значений
соответствующего типа данных.
Далее для каждого предзапуска недетерменированно выбираются отношения согласованности $\lCO$ и чтения из $\lRF$
таким образом, чтобы получалось \ARM-согласованное исполнение. Важно отменить, что не для всех предзапусков
существуют подходящие $\lCO$ и $\lRF$.

\[
\inarr{
  \cmdsToVrtxs : \Carm \rightarrow \PreExecutions \\
  \cmdsToVrtxs \; ilist = \instToVrtx \; ilist \; 0 \; (\lambda \reg. 0) \; \emptyset \\
}\]
Функция построения предзапусков по программе $\cmdsToVrtxs$ использует вспомогательную функцию $\instToVrtx$,
которая принимает список инструкций потока $ilist$, указатель на текущую инструкцию $pos$,
состояние локальных переменных $st$ и отношение зависимости локальной переменной от сгенерированного события $\dep$.
Последний параметр нужен для отношений зависимости.

\[
\inarr{
  \regs : \Expr \rightarrow \Pset(\Reg) \\
  \regs \; e \defeq \\
  \quad \kw{match} \; e \; \kw{with} \\
  \quad | \; \loc \; | \; \val \rightarrow \emptyset \\
  \quad | \; \reg \rightarrow \{\reg\} \\
  \quad | \; uop \; \expr \rightarrow \regs \; \expr \\
  \quad | \; bop \; \expr_o \; \expr_1 \rightarrow (\regs \; \expr_0) \cup (\regs \; \expr_1) \\
}\]
%% \[\inarr{
%%   \sembr{-} : \Carm \rightarrow
%%     \{\tup{\emptyset, \bot, \emptyset, \emptyset, \emptyset, \emptyset, \emptyset, \lambda \reg. \; 0, \emptyset} \} \\
%%   \sembr{[]} = \{\tup{\emptyset, \bot, \emptyset, \emptyset, \emptyset, \emptyset, \emptyset, \lambda \reg. \; 0, \emptyset} \} \\
%% }\]
{\small
\[
\inarr{
  \instToVrtx : \Carm \rightarrow \mathbb{N} \rightarrow (\Reg \rightarrow \mathbb{N}) \rightarrow \Pset(\Reg \times E)
    \rightarrow \PreExecutions \\
    %% \ListOf{(\Label \times \mathbb{N} \times (\Reg \rightarrow \mathbb{N}))} \\
  \instToVrtx \; ilist \; pos \; \PromState \; \dep \defeq \\
  \quad \textIf   pos < 0 \; || \; pos > length(ilist) \; \textThen
    \{\tup{\emptyset, \bot, \emptyset, \emptyset, \emptyset, \emptyset}\} \\
  \quad \textElse \\
  \qquad
    \letdef{a}{{\rm fresh\text{-}vertex} } \\
  \qquad \kw{match} \; ilist[pos] \; \kw{with} \\
  \qquad
    \begin{array}{@{}l}
    | ``\nop" \rightarrow \instToVrtx \; ilist \; (pos + 1) \; \PromState \; \dep \\
    | ``\readInst{\reg}{\expr}" \rightarrow \\
      \quad \inarr{
        \letdef{\dep'}{(\dep \setminus \{\reg\} \times E) \cup \{\tup{\reg, a}\}} \\
        \letdef{\loc}{\semState{\expr}{\PromState}} \\
        \letdef{\lADDR'}{ \codom{[\regs(\expr)]; \dep} \times \{a\} } \\
        \{\tup{set \cup\{a\}, \lab[a \mapsto \rlab{}{\loc}{\val}], \lPO \cup \{a\} \times set, \lCTRL, \lADDR \cup \lADDR', \lDATA}
        \mid \forall \val \in \Val, \\
        \quad \tup{set, \lab, \lPO, \lCTRL, \lADDR, \lDATA} \in
          \instToVrtx \; ilist \; (pos + 1) \; \PromState[\reg \mapsto \val] \; \dep'\} \\
      } \\
    | ``\assignInst{\reg}{\expr}" \rightarrow \\
      \quad \inarr{
        \letdef{\dep'}{(\dep \setminus \{\reg\} \times E) \cup \{\reg\} \times \regs(\expr) \}} \\
        \instToVrtx \; ilist \; (pos + 1) \; \PromState[\reg \mapsto \semState{\expr}{\PromState}] \; \dep' \\
      } \\
    | ``\writeInst{\expr_0}{\expr_1}" \rightarrow \\
      \quad \inarr{
        \letdef{\loc}{\semState{\expr_0}{\PromState}} \\
        \letdef{\val}{\semState{\expr_1}{\PromState}} \\
        \letdef{\lADDR'}{ \codom{[\regs(\expr_0)]; \dep} \times \{a\} } \\
        \letdef{\lDATA'}{ \codom{[\regs(\expr_1)]; \dep} \times \{a\} } \\
        \{\tup{set \cup\{a\}, \lab[a \mapsto \wlab{}{\loc}{\val}], \lPO \cup \{a\} \times set,
          \lCTRL, \lADDR \cup \lADDR', \lDATA \cup \lDATA'}
        \mid  \\
        \quad \tup{set, \lab, \lPO, \lCTRL, \lADDR, \lDATA} \in
          \instToVrtx \; ilist \; (pos + 1) \; \PromState \; \dep\} \\
      } \\
    | ``\fenceInst{\FtypeProm}" \rightarrow \\
      \quad \inarr{
        \{\tup{set \cup\{a\}, \lab[a \mapsto \flab{\FtypeProm}], \lPO \cup \{a\} \times set,
          \lCTRL, \lADDR, \lDATA}
        \mid  \\
        \quad \tup{set, \lab, \lPO, \lCTRL, \lADDR, \lDATA} \in
          \instToVrtx \; ilist \; (pos + 1) \; \PromState \; \dep\} \\
      } \\
    | ``\ifGotoInst{\expr}{k}" \rightarrow \\
      \quad \inarr{
        \letdef{step}{\textIf \semState{\expr}{\PromState} \; \textThen k \; \textElse 1} \\
        \letdef{dver}{\codom{[\regs(\expr)]; \dep}} \\
        \{\tup{set, \lab, \lPO, \lCTRL \cup dver \times set, \lADDR, \lDATA}
        \mid  \\
        \quad \tup{set, \lab, \lPO, \lCTRL, \lADDR, \lDATA} \in
          \instToVrtx \; ilist \; (pos + step) \; \PromState \; \dep \} \\
      } \\
  \end{array} \\
  %% \\
  %% \cmdsToVrtxsAux : \Carm \rightarrow \mathbb{N} \times (\Reg \rightarrow \mathbb{N}) \rightarrow \ListOf{\Path}\\
  %% \cmdsToVrtxsAux \; ilist \; (pos, \; \PromState) = \letdef{l}{\instToLbl \; ilist \; (pos, \; \PromState)} \\
  %% \quad %\textup{\sf flatten} \;
  %%       [lbl':lbls \mid (lbl', pos', \PromState') \in l, lbls \in \cmdsToLblsAux \; ilist \; (pos', \; \PromState') ] \\
  %% \\
}
\]
}



\chapter{Доказательство леммы о шаге симуляции между обещающей и ARMv8.3 моделями}
\label{sec:sim-step-proof}

  {\bf Лемма \ref{lem:sim-step}.}
  Пусть для некоторых конфигураций обхода $\tup{C, \IssuedSet}$ и $\tup{C', \IssuedSet'}$ сценария $G$,
  а также некоторого состояния обещающей машины $\tup{\TSf, M}$ выполняется
  $G \vdash \tup{C, \IssuedSet} \travConfigStep \tup{C', \IssuedSet'}$ и
  $\simRel(C, \IssuedSet, \TSf, M)$.
  Тогда существуют такие $\TSf'$ и $M'$, что $\tup{\TSf, M} \stepp\!\!^{+} \tup{\TSf', M'}$ и
  $\simRel(C', \IssuedSet', \TSf', M')$.
\begin{proof}%[Доказательство леммы \ref{lem:sim-step}]
  Существует два варианта: $G \vdash C, \IssuedSet \travConfigStep \tup{C', \IssuedSet'}$ соответствует покрытию или
  выпуску некоторого события $e \in \lE$. Введем обозначения $tid \defeq \lTID(e)$ и
  $\tup{\pstate, \View, \PromSet} \defeq \TSf(tid)$.
  %% \[\begin{array}{l@{~}l}
  %%    tid & \defeq \lTID(e) \\
  %%    \tup{\pstate, \View, \PromSet} & \defeq \TSf(tid) \\
  %% \end{array}\]
  Начнем с рассмотрения варианта, когда $G \vdash C, \IssuedSet \travConfigStep \tup{C', \IssuedSet'}$ соответствует
  выпуску события записи $e$.

  {\bf Выпуск события $e$}. Из определения $\travConfigStep$ следует, что
  $C' = C$, $\IssuedSet' = \IssuedSet \cup \{e\}$ и $e \in \issuable(G, C, \IssuedSet) \setminus \IssuedSet$.
      Введем обозначения:
      \[\begin{array}{l l}
        \loc, \; \val, \; \tau & \defeq \lLOC(e), \; \lVALW(e), \; T(e) \\
        mview & \defeq [\loc @ \tau] \sqcup \View.\viewRel \\
        msg & \defeq \msg{\loc}{\val}{\tau}{mview} \\
      \end{array}\]
      Мы знаем, что в $M$ нету сообщения для локации $\loc$ с меткой времени $\tau$,
      т.к. функция $T$ выдает уникальные для одной локации метки времени для событий записи ($\correctTmap(G, T)$),
      и у каждого сообщения из $M$ существует соответствующее ему сообщение в $\IssuedSet$ ($\invMemTwo(C, \IssuedSet, M)$).
      \[\begin{array}{l l}
        M', \; \PromSet' & \defeq \addToMemory{M}{msg}, \; \addToMemory{\PromSet}{msg} \\
        \TSf' & \defeq \TSf[tid \mapsto \tup{\pstate, \View, \PromSet'}]
      \end{array}\]
      $mview \in M'$, т.к. $[\loc @ \tau] \in M'$ и $\View.\viewRel \in M$.
      Таким образом $\tup{\tup{\pstate, \View, \PromSet}, M} \stepptid \tup{\tup{\pstate, \View, \PromSet'}, M'}$ выполняется.
      Осталось проверить, что $\simRel(C, \IssuedSet', \TSf', M')$ выполняется.
      \begin{itemize}
        \item $\invMemOne(C, \IssuedSet', \TSf', M') \land \invMemTwo(C, \IssuedSet', M')$: \\
          Единственным нетривиальным утверждением, которое нужно проверить, является $mview \le \domView(\msgRel; [e])$.
          По определению, $mview = [\loc @ \tau] \sqcup \View.\viewRel$
          $[\loc @ \tau] = [\lLOC(e) @ T(e)] \le \domView(\msgRel; [e])$, т.к. $\tup{e, e} \in \msgRel$.
          Из $\invView(C, \TSf)$ следует $\View.\viewRel \le \domView(\relRel; [e'])$,
          где $e' \in \nextset(G, C)$ и $\lTID(e') = \lTID(e)$. Т.к. $e \nin C$ и по определению $\nextset$,
          $\tup{e', e} \in \lPO^{?}$. Из определения $\relRel$ и $\msgRel$ следует
          $\relRel; [e']; \lPO^{?}; [e] \suq \msgRel; [e]$.
          %% \[\dom{\relRel; [e']} = \dom{\relRel; [e']; \lPO^{?}; [e]} \suq \dom{\msgRel; [e]}\]
          Утверждение выполняется, т.к. $\View.\viewRel \le \domView(\relRel; \lPO; [e'])$.
        
        \item $\invViewRel(\TSf')$: выполняется по $\invViewRel(\TSf)$ и определениям $mview$ и $\TSf'$.

        \item $\invView(C, \TSf')$: т.к. $\forall tid. \; \TSf'(tid).\View = \TSf(tid).\View$ и $\invView(C, \TSf')$,
          инвариант выполняется.

        \item $\invState(C, \TSf')$:
          следует из того, что $\invState(C, \TSf)$ выполняется и, для любого $tid$, $\TSf(tid).\pstate = \TSf'(tid).\pstate$.
      \end{itemize}
      %% $\tup{\TSf', M'}$ is consistent by \cref{prop:sim-cert}.

  
  {\bf Покрытие события $e$.} В этом случае $C' = C \cup \{e\}, \IssuedSet' = \IssuedSet$.
  %% \begin{itemize}
    %% \item $C' = C \cup \{e\}, \IssuedSet' = \IssuedSet$, and $e \nin \dom{\lRMW}$: \\
      Т.к. $\invState(C, \TSf)$ выполняется, существуют такие $t$ и $\pstate'$, что $t \approx \lLAB(e)$.
      Рассмотрим варианты $e$.
      \begin{itemize}
        \item $e \in \lDMBLD$.
          В этом случае $\labelF(t) = \fenceLbl{\acq}$ по $\invState(C, \TSf)$.
          \[\begin{array}{l@{~}l}
            \View'    & \defeq \tup{\View.\viewAcq, \View.\viewAcq, \View.\viewRel} \\
            \TSf', M' & \defeq \TSf[tid \mapsto \tup{\pstate', \View', \PromSet}], M \\
          \end{array}\]
      Проверим, что $\simRel(C', \IssuedSet, \TSf', M)$ выполняется.
      \begin{itemize}
        \item $\invMemOne(C', \IssuedSet, \TSf', M) \land \invMemTwo(C', \IssuedSet, M)$:
          выполняется, т.к. $e \nin \lW$ и $\simRel(C, \IssuedSet, \TSf, M)$ выполняется.

        \item $\invViewRel(\TSf')$: выполняется, т.к. $\invViewRel(\TSf)$ выполняется и\\
          $\TSf'(tid).\{\View.\viewRel, \PromSet\} = \TSf(tid).\{\View.\viewRel, \PromSet\}$.

        \item $\invView(C', \TSf')$:
          \[\inarr{
  \forall e' \in \nextset(G, C'). \; \letdef{\tup{\viewCur, \viewAcq, \viewfRel}}{\TSf'(\lTID(e')).\View}\\
  \quad
  \begin{array}{@{}r@{~}l@{~}l}
    \viewCur  & \le \domView(\curRel; [e']) & \land {} \\
    \viewAcq  & \le \domView(\acqRel; [e']) & \land {} \\
    \viewfRel & \le \domView(\relRel; [e']). \\
  \end{array}
          }\]
          Зафиксируем $e'$. Если $\lTID(e') \neq tid$, то утверждение следует из $\invView(C, \TSf)$.
          Предположим, что $\lTID(e') = tid$. Тогда $\imm{\lPO}(e, e')$, т.к. $e \in \nextset(G, C)$.
          Мы знаем, что $\View.\viewAcq \le \domView(\acqRel; \lPO; [e])$, т.к. $\invView(C, \TSf)$ выполняется.
          Нам нужно показать, что
          \[\inarr{
  \begin{array}{@{}r@{~}l@{~}l}
    \View.\viewAcq  & \le \domView(\curRel; [e']) & \land {} \\
    \View.\viewAcq  & \le \domView(\acqRel; [e']) & \land {} \\
    \View.\viewfRel & \le \domView(\relRel; [e']). \\
  \end{array}
          }\]
          Т.к. $\dom{\acqRel; [e]} \suq \dom{\acqRel; [e']}$
          и $\acqRel;[e];\lPO;[e'] \suq \curRel; [e']$, утверждение выполняется.
        \item $\invState(C', \TSf')$:
          очевидно следует из $\invState(C, \TSf)$ и определений $C', \TSf'$.
      \end{itemize}

        \item $e \in \lDMBSY$. В этом случае $\labelF(t) = \fenceLbl{\rel}$ по $\invState(C, \TSf)$.
          \[\begin{array}{l@{~}l}
            \View'   & \defeq \tup{\viewCur, \viewAcq, \viewCur} \\
            \TSf'    & \defeq \TSf[tid \mapsto \tup{\pstate', \View', \PromSet}] \\
            M' & \defeq M \\
          \end{array}\]
          Не существует $w \in \IssuedSet$ такого, что $\lPO(e, w)$. Иначе не
          это противоречило бы $w \in \issuable(G, C, \IssuedSet)$ по лемме \ref{prop:trav-prop-preserve} 
          Из этого следует, что не существует $w \in \IssuedSet \setminus C$ такого, что $\lTID(w) = tid$, и
          $\PromSet = \emptyset$ по $\invMemOne(C, \IssuedSet, \TSf, M)$.
          
      Проверим $\simRel(C', \IssuedSet, \TSf', M)$.
      \begin{itemize}
        \item $\invMemOne(C', \IssuedSet, \TSf', M) \land \invMemTwo(C', \IssuedSet, M)$:
          выполняется, т.к. $e \nin \lW$ и $\simRel(C, \IssuedSet, \TSf, M)$ выполняется.
        \item $\invViewRel(\TSf')$: \\
          Зафиксируем поток $tid'$. Если $tid' \neq tid$, то утверждение выполняется по $\invViewRel(\TSf)$.
          Если $tid' = tid$, то утверждение выполняется, т.к. $\TSf'(tid).\PromSet = \TSf(tid).\PromSet = \emptyset$.
        \item $\invView(C', \TSf')$:
          \[\inarr{
  \forall e' \in \nextset(G, C'). \; \letdef{\tup{\viewCur, \viewAcq, \viewfRel}}{\TSf'(\lTID(e')).\View}\\
  \quad
  \begin{array}{@{}r@{~}l@{~}l}
    \viewCur  & \le \domView(\curRel; [e']) & \land {} \\
    \viewAcq  & \le \domView(\acqRel; [e']) & \land {} \\
    \viewfRel & \le \domView(\relRel; [e']). \\
  \end{array}
          }\]
          Зафиксируем $e'$. Если $\lTID(e') \neq tid$, то утверждение следует из $\invView(C, \TSf)$.
          Если $\lTID(e') = tid$, то $\imm{\lPO}(e, e')$, т.к. $e \in \nextset(G, C)$.
          Нам нужно показать, что
          \[\inarr{
  \begin{array}{@{}r@{~}l@{~}l}
    \View.\viewCur & \le \domView(\curRel; [e']) & \land {} \\
    \View.\viewAcq & \le \domView(\acqRel; [e']) & \land {} \\
    \View.\viewCur & \le \domView(\relRel; [e']). \\
  \end{array}
          }\]
          Т.к. $\dom{\curRel; [e]} \suq \dom{\curRel; [e']}$
          и $\relRel;[e];\lPO;[e'] \suq \curRel; [e']$, утверждение выполняется.
          
        \item $\invState(C', \TSf')$:
          очевидно следует из $\invState(C, \TSf)$ и определений $C', \TSf'$.
      \end{itemize}
        \item $e \in \lR$. 
          В этом случае $\labelF(t) = \readLbl{\loc}{\val}$ по $\invState(C, \TSf)$.
          \[\begin{array}{l@{~}l}
            \loc, \val       & \defeq \lLOC(e), \lVALR(e) \\
            {\pstate, \View, \PromSet} & \defeq \TSf(tid) \\
          \end{array}\]
          Т.к. $e \in \lR \cap \coverable(G, C, \IssuedSet)$ из определения $\travConfigStep$,
          существует событие записи $w \in \IssuedSet \cap \dom{\lRF; [e]}$. По $\invMemOne(C, \IssuedSet, \TSf, M)$
          существует фронт $view$ такой, что $msg \defeq \msg{\loc}{\val}{T(w)}{view} \in M$.
          $\View.\viewCur \le T(w)$ по лемме \ref{lem:curset-acyclic}.

          \[\begin{array}{l@{~}l}
            \View'   & \defeq \tup{\View.\viewCur \sqcup [\loc @ T(w)], \View.\viewAcq \sqcup view, \View.\viewRel} \\
            \TSf'    & \defeq \TSf[tid \mapsto \tup{\pstate', \View', \PromSet}] \\
          \end{array}\]
          Нужно проверить $\simRel(C', \IssuedSet, \TSf', M)$.
        \begin{itemize}
        \item $\invMemOne(C', \IssuedSet, \TSf', M) \land \invMemTwo(C', \IssuedSet, M)$:
          выполняется, т.к. $e \nin \lW$ и $\simRel(C, \IssuedSet, \TSf, M)$ выполняется.

        \item $\invViewRel(\TSf')$: выполняется, т.к. $\invViewRel(\TSf)$ выполняется и\\
          $\TSf'(tid).\{\View.\viewRel, \PromSet\} = \TSf(tid).\{\View.\viewRel, \PromSet\}$.

        \item $\invView(C', \TSf')$:
          \[\inarr{
  \forall e' \in \nextset(G, C'). \; \letdef{\tup{\viewCur, \viewAcq, \viewfRel}}{\TSf'(\lTID(e')).\View}\\
  \quad
  \begin{array}{@{}r@{~}l@{~}l}
    \viewCur  & \le \domView(\curRel; [e']) & \land {} \\
    \viewAcq  & \le \domView(\acqRel; [e']) & \land {} \\
    \viewfRel & \le \domView(\relRel; [e']). \\
  \end{array}
          }\]
          Зафиксируем $e'$. Если $\lTID(e') \neq tid$, то утверждение следует из $\invView(C, \TSf)$.
          Предположим, что $\lTID(e') = tid$. Тогда $\imm{\lPO}(e, e')$, т.к. $e \in \nextset(G, C)$.
          %% Мы знаем, что $\View.\viewAcq \le \domView(\acqRel; \lPO; [e])$, т.к. $\invView(C, \TSf)$ выполняется.
          Нам нужно показать, что
          \[\inarr{
  \begin{array}{@{}l@{~}l@{~}l}
    \View.\viewCur \sqcup [\loc @ T(w)] & \le \domView(\curRel; [e']) & \land {} \\
    \View.\viewAcq \sqcup view & \le \domView(\acqRel; [e']) & \land {} \\
    \View.\viewfRel & \le \domView(\relRel; [e']). \\
  \end{array}
          }\]
          %% $\View.\viewCur \le \domView(\curRel; [e]) \le \domView(\curRel; [e'])$ по $\invView(C, \TSf')$.
          Т.к. $\tup{w, e'} \in \curRel$, $[\loc @ T(w)] \le \domView(\curRel; [e'])$.
          Из $\invViewRel(\TSf)$ следует, что $view = [\loc @ T(w)] \sqcup \TSf(\lTID(w)).\View.\viewfRel$.
          Т.к. $\tup{w, e'} \in \acqRel$, $[\loc @ T(w)] \le \domView(\acqRel; [e'])$.
          Т.к. $\TSf(\lTID(w)).\View.\viewfRel \le \domView(\relRel; [w])$ и
          $\dom{\relRel; [w]} \suq \dom{\acqRel; [e']}$, утверждение выполняется.
        \item $\invState(C', \TSf')$:
          очевидно следует из $\invState(C, \TSf)$ и определений $C', \TSf'$.
      \end{itemize}

        \item $e \in \lW$.
          В этом случае $\labelF(t) = \writeLbl{\loc}{\val}$ по $\invState(C, \TSf)$.
          \[\begin{array}{l@{~}l}
            \loc, \val, \tau        & \defeq \lLOC(e), \lVALW(e), T(e) \\
            \tup{\pstate, \View, \PromSet} & \defeq \TSf(tid) \\
            \tup{\viewCur, \viewAcq, \viewRel} & \defeq \View. \\
          \end{array}\]
          $e \in \IssuedSet$, т.е. $e \in \lW \cap \coverable(G, C, \IssuedSet)$.
          Из $\invMemOne(C, \IssuedSet, \TSf, M)$ следует, что существует $view$ такое, что
          $msg \defeq \msg{\loc}{\val}{\tau}{view} \in M$.
          $\viewCur(\loc) < \tau$ следует по лемме \ref{lem:curset-acyclic}.
          \[\begin{array}{l@{~}l}
            \viewCur', \viewAcq' & \defeq \viewCur \sqcup [\loc @ \tau], \viewAcq \sqcup [\loc @ \tau] \\
            \View'    & \defeq \tup{\viewCur', \viewAcq', \viewRel} \\
            \TSf'     & \defeq \TSf[tid \mapsto \tup{\pstate', \View', \PromSet \setminus msg}] \\
          \end{array}\]
          $view = \viewRel$ по $\invViewRel(\TSf)$.

          Нужно проверить $\simRel(C', \IssuedSet, \TSf', M)$.
        \begin{itemize}
        \item $\invMemOne(C', \IssuedSet, \TSf', M) \land \invMemTwo(C', \IssuedSet, M)$:
          выполняется, т.к. мы добавили $e$ во множество покрытых событий и убрали $msg$ из множества
          обещанных сообщений потока $tid$.

        \item $\invViewRel(\TSf')$: выполняется, т.к. $\invViewRel(\TSf)$ выполняется, \\
          $\TSf'(tid).\View.\viewRel = \TSf(tid).\View.\viewRel$ и
          $\TSf'(tid).\PromSet \subset \TSf(tid).\PromSet$.

        \item $\invView(C', \TSf')$:
          \[\inarr{
  \forall e' \in \nextset(G, C'). \; \letdef{\tup{\viewCur, \viewAcq, \viewfRel}}{\TSf'(\lTID(e')).\View}\\
  \quad
  \begin{array}{@{}r@{~}l@{~}l}
    \viewCur  & \le \domView(\curRel; [e']) & \land {} \\
    \viewAcq  & \le \domView(\acqRel; [e']) & \land {} \\
    \viewfRel & \le \domView(\relRel; [e']). \\
  \end{array}
          }\]
          Зафиксируем $e'$. Если $\lTID(e') \neq tid$, то утверждение следует из $\invView(C, \TSf)$.
          Предположим, что $\lTID(e') = tid$. Тогда $\imm{\lPO}(e, e')$, т.к. $e \in \nextset(G, C)$.
          %% Мы знаем, что $\View.\viewAcq \le \domView(\acqRel; \lPO; [e])$, т.к. $\invView(C, \TSf)$ выполняется.
          Нам нужно показать, что
          \[\inarr{
  \begin{array}{@{}l@{~}l@{~}l}
    \View.\viewCur \sqcup [\loc @ \tau] & \le \domView(\curRel; [e']) & \land {} \\
    \View.\viewAcq \sqcup [\loc @ \tau] & \le \domView(\acqRel; [e']) & \land {} \\
    \View.\viewfRel & \le \domView(\relRel; [e']). \\
  \end{array}
          }\]
          Т.к. $\tup{w, e'} \in \curRel \suq \acqRel$,
          $[\loc @ \tau] \le \domView(\curRel; [e'])$
          ${} \le \domView(\acqRel; [e'])$.
        \item $\invState(C', \TSf')$:
          очевидно следует из $\invState(C, \TSf)$ и определений $C', \TSf'$.
      \end{itemize}

      \end{itemize}

      %% $\tup{\TSf', M'}$ is consistent by \cref{prop:sim-cert}.

  %% \end{itemize}
\end{proof}

\begin{lemma}
  \label{lem:curset-acyclic}
  Для любой локации $\loc$ выполняется $[\lW_{\loc}]; \curRel; [\lW_{\loc}] \suq \lCO$.
\end{lemma}
\begin{proof}
  Зафиксируем $\tup{w, w'} \in [\lW_{\loc}]; \curRel; [\lW_{\loc}]$. Тогда по определению $\lCO$,
  либо $\lCO(w, w')$, либо $\lCO(w', w)$. Если выполняется первое, то выполняется утверждение.
  Пусть выполняется второе.
  \[\inarr{
    {} [\lW_{\loc}]; \curRel; [\lW_{\loc}] = \\
    {} [\lW_{\loc}]; \lRF^{?}; (\lPO \cup \lSW)^{+}; [\lW_{\loc}] = \\
    \quad \text{(по транзитивности $\lPO$ и определению $\lSW$)} \\
    {} [\lW_{\loc}]; \lRF^{?}; \lPO; (\lSW; \lPO)^{*}; [\lW_{\loc}] = \\
    {} [\lW_{\loc}]; \lRF^{?}; \lPO; [\lW_{\loc}] \cup {} \\
    {} [\lW_{\loc}]; \lRFE^{?}; \lPO; (\lSW; \lPO)^{+}; [\lW_{\loc}] \cup {} \\
  }\]
  $\tup{w, w'} \nin [\lW_{\loc}]; \lRF^{?}; \lPO; [\lW_{\loc}]$, т.к. $\tup{w', w} \in \lCO$ и выполняется \ref{ax:internal}.
  
  Введем вспомогательное отношение $\lEORD \defeq (\lOBS \cup \lDOB \cup \lBOB)^{+}$,
  которое антирефлексивно по определению \ARM-согласованности (\ref{ax:external}).
  Из определений отношений следует, что $\lPO^{?}; \lSW; \lPO \suq \lBOB^{?}; \lBOB \cup \lBOB^{?}; \lBOB; \lRFE; \lBOB \suq \lEORD$.
  По транзитивности $\lEORD$, $\tup{w, w'} \in \lEORD$.
  
  Мы знаем, что $\tup{w', w} \in \lCO$. Есть два варианта: $\tup{w', w} \in \lCOE$ или $\tup{w', w} \in \lCOI$.
  Первый вариант противоречит ацикличности $\lEORD$, т.к. $\lCOE \suq \lOBS$. Опровергнем второй вариант, показав, что
  $(\lEORD \cup \lCOI)^{+} = (\lOBS \cup \lDOB \cup \lBOB \cup \lCOI)^{+}$ антирефлексивно.

  Предположим обратное и рассмотрим цикл минимальной длины, в котором точно есть $\lCOI$ по антирефлексивности $\lEORD$.
  Рассмотрим предыдущее ребро и покажем, что во всех случаях цикл можно укоротить.  
  \begin{itemize}
    \item $\lCOI; \lCOI \suq \lCOI$
    \item $\lOBS; \lCOI \suq \lOBS$: $\quad \lRFE; \lCOI = \emptyset \quad \lFRE; \lCOI \suq \lFRE \quad \lCOE; \lCOI \suq \lCOE$
    \item $\lDOB; \lCOI \suq \lDOB$:
      \begin{itemize}
        \item $\lADDR; \lPO^{?}; \lCOI \suq \lDOB \quad \lDATA; \lCOI \suq \lDOB$
        \item $(\lADDR \cup \lDATA);\lRFI; \lCOI = \emptyset \quad (\lCTRL \cup \lDATA); [\lW]; \lCOI^{?}; \lCOI \suq \lDOB$
      \end{itemize}
    \item $\lBOB; \lCOI \suq \lBOB$: $\lPO; \lCOI \suq \lPO$
    \qedhere
  \end{itemize}
\end{proof}

\chapter{Связь между системой переходов и вершинами в \ARM-исполнении}
\label{sec:lts-rel}

\[\inarr{
  \approx : \Label_{\Promise} \rightarrow \Label_{\ARM} \rightarrow {\rm Boolean} \\
  lbl_{\Promise} \approx lbl_{\ARM} \defeq \\
  \qquad \kw{match} \; lbl_{\Promise}, \; lbl_{\ARM} \; \kw{with} \\
  \qquad
    \begin{array}{@{}l}
      | \; \rlab{}{\loc}{\val}, \rlab{}{\loc}{\val} \;
      | \; \wlab{}{\loc}{\val}, \wlab{}{\loc}{\val} \\
      | \; \flab{\rel}, \flab{\SY} \;
      | \; \flab{\acq}, \flab{\LD} \rightarrow {\rm true} \\
      | \; \_, \_ \rightarrow {\rm false}
    \end{array} \\
  \\
  \nthf : \{A : Set\} \rightarrow \Pset(A \times A) \rightarrow \mathbb{N} \rightarrow \Pset(A) \\
  \nthf \; rel \; n \defeq \codom{rel^{n}} \setminus \codom{rel^{n + 1}}.
}\]

\begin{theorem}
\[\inarr{
\forall \Carm. \\
\quad (\forall \tup{set, lbl, \lPO, \_, \_, \_} \in \cmdsToVrtxs \; \Carm. \\
\qquad \exists path \in \cmdsToLbls \; \Carm. \; \forall n \in \mathbb{N}, a \in \nthf \; \lPO \; n. \\
\qquad \quad \exists k \in \mathbb{N}. \; path[n + k] \approx lbl \; a) \land {} \\
\quad (\forall path \in \cmdsToLbls \; \Carm. \\
\qquad \exists \tup{set, lbl, \lPO, \_, \_, \_} \in \cmdsToVrtxs \; \Carm. \;
  \forall n \in \mathbb{N}. \\
\qquad \quad path[n] \neq \epsilon \Rightarrow \exists k \in \mathbb{N}, a \in \nthf \; \lPO \; (n - k). \; path[n] \approx lbl \; a).
}\]
\end{theorem}
\begin{proof}
  Верно по определению функций $\instToLbl$ и $\instToVrtx$.
\end{proof}

%% \chapter{Примеры вставки листингов программного кода} \label{AppendixA}

%% Для крупных листингов есть два способа. Первый красивый, но в нём могут быть проблемы с поддержкой кириллицы (у вас может встречаться в комментариях и
%% печатаемых сообщениях), он представлен на листинге~\ref{list:hwbeauty}.
%% \begin{ListingEnv}[!h]% настройки floating аналогичны окружению figure
%% %    \captionsetup{format=tablenocaption}% должен стоять до самого caption
%%     \caption{Программа “Hello, world” на \protect\cpp}
%%     % далее метка для ссылки:
%%     \label{list:hwbeauty}
%%     % окружение учитывает пробелы и табуляции и применяет их в сответсвии с настройками
%%     \begin{lstlisting}[language={[ISO]C++}]
%% 	#include <iostream>
%% 	using namespace std;

%% 	int main() //кириллица в комментариях при xelatex и lualatex имеет проблемы с пробелами
%% 	{
%% 		cout << "Hello, world" << endl; //latin letters in commentaries
%% 		system("pause");
%% 		return 0;
%% 	}
%%     \end{lstlisting}
%% \end{ListingEnv}%
%% Второй не такой красивый, но без ограничений (см.~листинг~\ref{list:hwplain}).
%% \begin{ListingEnv}[!h]
%%     \caption{Программа “Hello, world” без подсветки}
%%     \label{list:hwplain}
%%     \begin{Verb}
        
%%         #include <iostream>
%%         using namespace std;
        
%%         int main() //кириллица в комментариях
%%         {
%%             cout << "Привет, мир" << endl;
%%         }
%%     \end{Verb}
%% \end{ListingEnv}

%% Можно использовать первый для вставки небольших фрагментов
%% внутри текста, а второй для вставки полного
%% кода в приложении, если таковое имеется.

%% Если нужно вставить совсем короткий пример кода (одна или две строки), то выделение  линейками и нумерация может смотреться чересчур громоздко. В таких случаях можно использовать окружения \texttt{lstlisting} или \texttt{Verb} без \texttt{ListingEnv}. Приведём такой пример с указанием языка программирования, отличного от заданного по умолчанию:
%% \begin{lstlisting}[language=Haskell]
%% fibs = 0 : 1 : zipWith (+) fibs (tail fibs)
%% \end{lstlisting}
%% Такое решение~--- со вставкой нумерованных листингов покрупнее
%% и вставок без выделения для маленьких фрагментов~--- выбрано,
%% например, в книге Эндрю Таненбаума и Тодда Остина по архитектуре
%% %компьютера~\autocite{TanAus2013} (см.~рис.~\ref{fig:tan-aus}).

%% Наконец, для оформления идентификаторов внутри строк
%% (функция \lstinline{main} и тому подобное) используется
%% \texttt{lstinline} или, самое простое, моноширинный текст
%% (\texttt{\textbackslash texttt}).


%% Пример~\ref{list:internal3}, иллюстрирующий подключение переопределённого языка. Может быть полезным, если подсветка кода работает криво. Без дополнительного окружения, с подписью и ссылкой, реализованной встроенным средством.
%% \begin{lstlisting}[language={Renhanced},caption={Пример листинга c подписью собственными средствами},label={list:internal3}]
%% ## Caching the Inverse of a Matrix

%% ## Matrix inversion is usually a costly computation and there may be some
%% ## benefit to caching the inverse of a matrix rather than compute it repeatedly
%% ## This is a pair of functions that cache the inverse of a matrix.

%% ## makeCacheMatrix creates a special "matrix" object that can cache its inverse

%% makeCacheMatrix <- function(x = matrix()) {#кириллица в комментариях при xelatex и lualatex имеет проблемы с пробелами
%%     i <- NULL
%%     set <- function(y) {
%%         x <<- y
%%         i <<- NULL
%%     }
%%     get <- function() x
%%     setSolved <- function(solve) i <<- solve
%%     getSolved <- function() i
%%     list(set = set, get = get,
%%     setSolved = setSolved,
%%     getSolved = getSolved)
    
%% }


%% ## cacheSolve computes the inverse of the special "matrix" returned by
%% ## makeCacheMatrix above. If the inverse has already been calculated (and the
%% ## matrix has not changed), then the cachesolve should retrieve the inverse from
%% ## the cache.

%% cacheSolve <- function(x, ...) {
%%     ## Return a matrix that is the inverse of 'x'
%%     i <- x$getSolved()
%%     if(!is.null(i)) {
%%         message("getting cached data")
%%         return(i)
%%     }
%%     data <- x$get()
%%     i <- solve(data, ...)
%%     x$setSolved(i)
%%     i  
%% }
%% \end{lstlisting} %$ %Комментарий для корректной подсветки синтаксиса
%%                  %вне листинга

%% Листинг~\ref{list:external1} подгружается из внешнего файла. Приходится загружать без окружения дополнительного. Иначе по страницам не переносится.
%%     \lstinputlisting[lastline=78,language={R},caption={Листинг из внешнего файла},label={list:external1}]{listings/run_analysis.R}






%% \chapter{Очень длинное название второго приложения, в котором продемонстрирована работа с длинными таблицами} \label{AppendixB}

%%  \section{Подраздел приложения}\label{AppendixB1}
%% Вот размещается длинная таблица:
%% \fontsize{10pt}{10pt}\selectfont
%% \begin{longtable*}[c]{|l|c|l|l|} %longtable* появляется из пакета caption и даёт ненумерованную таблицу
%% % \caption{Описание входных файлов модели}\label{Namelists} 
%% %\\ 
%%  \hline
%%  %\multicolumn{4}{|c|}{\textbf{Файл puma\_namelist}}        \\ \hline
%%  Параметр & Умолч. & Тип & Описание               \\ \hline
%%                                               \endfirsthead   \hline
%%  \multicolumn{4}{|c|}{\small\slshape (продолжение)}        \\ \hline
%%  Параметр & Умолч. & Тип & Описание               \\ \hline
%%                                               \endhead        \hline
%% % \multicolumn{4}{|c|}{\small\slshape (окончание)}        \\ \hline
%% % Параметр & Умолч. & Тип & Описание               \\ \hline
%% %                                             \endlasthead        \hline
%%  \multicolumn{4}{|r|}{\small\slshape продолжение следует}  \\ \hline
%%                                               \endfoot        \hline
%%                                               \endlastfoot
%%  \multicolumn{4}{|l|}{\&INP}        \\ \hline 
%%  kick & 1 & int & 0: инициализация без шума ($p_s = const$) \\
%%       &   &     & 1: генерация белого шума                  \\
%%       &   &     & 2: генерация белого шума симметрично относительно \\
%%   & & & экватора    \\
%%  mars & 0 & int & 1: инициализация модели для планеты Марс     \\
%%  kick & 1 & int & 0: инициализация без шума ($p_s = const$) \\
%%       &   &     & 1: генерация белого шума                  \\
%%       &   &     & 2: генерация белого шума симметрично относительно \\
%%   & & & экватора    \\
%%  mars & 0 & int & 1: инициализация модели для планеты Марс     \\
%% kick & 1 & int & 0: инициализация без шума ($p_s = const$) \\
%%       &   &     & 1: генерация белого шума                  \\
%%       &   &     & 2: генерация белого шума симметрично относительно \\
%%   & & & экватора    \\
%%  mars & 0 & int & 1: инициализация модели для планеты Марс     \\
%% kick & 1 & int & 0: инициализация без шума ($p_s = const$) \\
%%       &   &     & 1: генерация белого шума                  \\
%%       &   &     & 2: генерация белого шума симметрично относительно \\
%%   & & & экватора    \\
%%  mars & 0 & int & 1: инициализация модели для планеты Марс     \\
%% kick & 1 & int & 0: инициализация без шума ($p_s = const$) \\
%%       &   &     & 1: генерация белого шума                  \\
%%       &   &     & 2: генерация белого шума симметрично относительно \\
%%   & & & экватора    \\
%%  mars & 0 & int & 1: инициализация модели для планеты Марс     \\
%% kick & 1 & int & 0: инициализация без шума ($p_s = const$) \\
%%       &   &     & 1: генерация белого шума                  \\
%%       &   &     & 2: генерация белого шума симметрично относительно \\
%%   & & & экватора    \\
%%  mars & 0 & int & 1: инициализация модели для планеты Марс     \\
%% kick & 1 & int & 0: инициализация без шума ($p_s = const$) \\
%%       &   &     & 1: генерация белого шума                  \\
%%       &   &     & 2: генерация белого шума симметрично относительно \\
%%   & & & экватора    \\
%%  mars & 0 & int & 1: инициализация модели для планеты Марс     \\
%% kick & 1 & int & 0: инициализация без шума ($p_s = const$) \\
%%       &   &     & 1: генерация белого шума                  \\
%%       &   &     & 2: генерация белого шума симметрично относительно \\
%%   & & & экватора    \\
%%  mars & 0 & int & 1: инициализация модели для планеты Марс     \\
%% kick & 1 & int & 0: инициализация без шума ($p_s = const$) \\
%%       &   &     & 1: генерация белого шума                  \\
%%       &   &     & 2: генерация белого шума симметрично относительно \\
%%   & & & экватора    \\
%%  mars & 0 & int & 1: инициализация модели для планеты Марс     \\
%% kick & 1 & int & 0: инициализация без шума ($p_s = const$) \\
%%       &   &     & 1: генерация белого шума                  \\
%%       &   &     & 2: генерация белого шума симметрично относительно \\
%%   & & & экватора    \\
%%  mars & 0 & int & 1: инициализация модели для планеты Марс     \\
%% kick & 1 & int & 0: инициализация без шума ($p_s = const$) \\
%%       &   &     & 1: генерация белого шума                  \\
%%       &   &     & 2: генерация белого шума симметрично относительно \\
%%   & & & экватора    \\
%%  mars & 0 & int & 1: инициализация модели для планеты Марс     \\
%% kick & 1 & int & 0: инициализация без шума ($p_s = const$) \\
%%       &   &     & 1: генерация белого шума                  \\
%%       &   &     & 2: генерация белого шума симметрично относительно \\
%%   & & & экватора    \\
%%  mars & 0 & int & 1: инициализация модели для планеты Марс     \\
%% kick & 1 & int & 0: инициализация без шума ($p_s = const$) \\
%%       &   &     & 1: генерация белого шума                  \\
%%       &   &     & 2: генерация белого шума симметрично относительно \\
%%   & & & экватора    \\
%%  mars & 0 & int & 1: инициализация модели для планеты Марс     \\
%% kick & 1 & int & 0: инициализация без шума ($p_s = const$) \\
%%       &   &     & 1: генерация белого шума                  \\
%%       &   &     & 2: генерация белого шума симметрично относительно \\
%%   & & & экватора    \\
%%  mars & 0 & int & 1: инициализация модели для планеты Марс     \\
%% kick & 1 & int & 0: инициализация без шума ($p_s = const$) \\
%%       &   &     & 1: генерация белого шума                  \\
%%       &   &     & 2: генерация белого шума симметрично относительно \\
%%   & & & экватора    \\
%%  mars & 0 & int & 1: инициализация модели для планеты Марс     \\
%%  \hline
%%   %& & & $\:$ \\ 
%%  \multicolumn{4}{|l|}{\&SURFPAR}        \\ \hline
%% kick & 1 & int & 0: инициализация без шума ($p_s = const$) \\
%%       &   &     & 1: генерация белого шума                  \\
%%       &   &     & 2: генерация белого шума симметрично относительно \\
%%   & & & экватора    \\
%%  mars & 0 & int & 1: инициализация модели для планеты Марс     \\
%% kick & 1 & int & 0: инициализация без шума ($p_s = const$) \\
%%       &   &     & 1: генерация белого шума                  \\
%%       &   &     & 2: генерация белого шума симметрично относительно \\
%%   & & & экватора    \\
%%  mars & 0 & int & 1: инициализация модели для планеты Марс     \\
%% kick & 1 & int & 0: инициализация без шума ($p_s = const$) \\
%%       &   &     & 1: генерация белого шума                  \\
%%       &   &     & 2: генерация белого шума симметрично относительно \\
%%   & & & экватора    \\
%%  mars & 0 & int & 1: инициализация модели для планеты Марс     \\
%% kick & 1 & int & 0: инициализация без шума ($p_s = const$) \\
%%       &   &     & 1: генерация белого шума                  \\
%%       &   &     & 2: генерация белого шума симметрично относительно \\
%%   & & & экватора    \\
%%  mars & 0 & int & 1: инициализация модели для планеты Марс     \\
%% kick & 1 & int & 0: инициализация без шума ($p_s = const$) \\
%%       &   &     & 1: генерация белого шума                  \\
%%       &   &     & 2: генерация белого шума симметрично относительно \\
%%   & & & экватора    \\
%%  mars & 0 & int & 1: инициализация модели для планеты Марс     \\
%% kick & 1 & int & 0: инициализация без шума ($p_s = const$) \\
%%       &   &     & 1: генерация белого шума                  \\
%%       &   &     & 2: генерация белого шума симметрично относительно \\
%%   & & & экватора    \\
%%  mars & 0 & int & 1: инициализация модели для планеты Марс     \\
%% kick & 1 & int & 0: инициализация без шума ($p_s = const$) \\
%%       &   &     & 1: генерация белого шума                  \\
%%       &   &     & 2: генерация белого шума симметрично относительно \\
%%   & & & экватора    \\
%%  mars & 0 & int & 1: инициализация модели для планеты Марс     \\
%% kick & 1 & int & 0: инициализация без шума ($p_s = const$) \\
%%       &   &     & 1: генерация белого шума                  \\
%%       &   &     & 2: генерация белого шума симметрично относительно \\
%%   & & & экватора    \\
%%  mars & 0 & int & 1: инициализация модели для планеты Марс     \\
%% kick & 1 & int & 0: инициализация без шума ($p_s = const$) \\
%%       &   &     & 1: генерация белого шума                  \\
%%       &   &     & 2: генерация белого шума симметрично относительно \\
%%   & & & экватора    \\
%%  mars & 0 & int & 1: инициализация модели для планеты Марс     \\ 
%%  \hline 
%% \end{longtable*}

%% \normalsize% возвращаем шрифт к нормальному
%% \section{Ещё один подраздел приложения} \label{AppendixB2}

%% Нужно больше подразделов приложения!

%% Пример длинной таблицы с записью продолжения по ГОСТ 2.105
%% \begingroup
%%     \centering
%% 	\small
%%     \begin{longtable}[c]{|l|c|l|l|}
%% 	\caption{Наименование таблицы средней длины}%
%%     \label{tbl:test5}% label всегда желательно идти после caption
%%     \\
%%     \hline
%%      %\multicolumn{4}{|c|}{\textbf{Файл puma\_namelist}}        \\ \hline
%%      Параметр & Умолч. & Тип & Описание\\ \hline
%%      \endfirsthead%
%% %     \multicolumn{4}{|c|}{\small\slshape (продолжение)}        \\ \hline
%%  \captionsetup{format=tablenocaption,labelformat=continued}% должен стоять до самого caption
%%     \caption[]{}\\
%%     \hline
%%      Параметр & Умолч. & Тип & Описание\\ \hline
%%       \endhead
%%       \hline
%% %     \multicolumn{4}{|r|}{\small\slshape продолжение следует}  \\
%% %\hline
%%      \endfoot
%%          \hline
%%      \endlastfoot
%%      \multicolumn{4}{|l|}{\&INP}        \\ \hline 
%%      kick & 1 & int & 0: инициализация без шума ($p_s = const$) \\
%%           &   &     & 1: генерация белого шума                  \\
%%           &   &     & 2: генерация белого шума симметрично относительно \\
%%       & & & экватора    \\
%%      mars & 0 & int & 1: инициализация модели для планеты Марс     \\
%%      kick & 1 & int & 0: инициализация без шума ($p_s = const$) \\
%%           &   &     & 1: генерация белого шума                  \\
%%           &   &     & 2: генерация белого шума симметрично относительно \\
%%       & & & экватора    \\
%%      mars & 0 & int & 1: инициализация модели для планеты Марс     \\
%%     kick & 1 & int & 0: инициализация без шума ($p_s = const$) \\
%%           &   &     & 1: генерация белого шума                  \\
%%           &   &     & 2: генерация белого шума симметрично относительно \\
%%       & & & экватора    \\
%%      mars & 0 & int & 1: инициализация модели для планеты Марс     \\
%%     kick & 1 & int & 0: инициализация без шума ($p_s = const$) \\
%%           &   &     & 1: генерация белого шума                  \\
%%           &   &     & 2: генерация белого шума симметрично относительно \\
%%       & & & экватора    \\
%%      mars & 0 & int & 1: инициализация модели для планеты Марс     \\
%%     kick & 1 & int & 0: инициализация без шума ($p_s = const$) \\
%%           &   &     & 1: генерация белого шума                  \\
%%           &   &     & 2: генерация белого шума симметрично относительно \\
%%       & & & экватора    \\
%%      mars & 0 & int & 1: инициализация модели для планеты Марс     \\
%%     kick & 1 & int & 0: инициализация без шума ($p_s = const$) \\
%%           &   &     & 1: генерация белого шума                  \\
%%           &   &     & 2: генерация белого шума симметрично относительно \\
%%       & & & экватора    \\
%%      mars & 0 & int & 1: инициализация модели для планеты Марс     \\
%%     kick & 1 & int & 0: инициализация без шума ($p_s = const$) \\
%%           &   &     & 1: генерация белого шума                  \\
%%           &   &     & 2: генерация белого шума симметрично относительно \\
%%       & & & экватора    \\
%%      mars & 0 & int & 1: инициализация модели для планеты Марс     \\
%%     kick & 1 & int & 0: инициализация без шума ($p_s = const$) \\
%%           &   &     & 1: генерация белого шума                  \\
%%           &   &     & 2: генерация белого шума симметрично относительно \\
%%       & & & экватора    \\
%%      mars & 0 & int & 1: инициализация модели для планеты Марс     \\
%%     kick & 1 & int & 0: инициализация без шума ($p_s = const$) \\
%%           &   &     & 1: генерация белого шума                  \\
%%           &   &     & 2: генерация белого шума симметрично относительно \\
%%       & & & экватора    \\
%%      mars & 0 & int & 1: инициализация модели для планеты Марс     \\
%%     kick & 1 & int & 0: инициализация без шума ($p_s = const$) \\
%%           &   &     & 1: генерация белого шума                  \\
%%           &   &     & 2: генерация белого шума симметрично относительно \\
%%       & & & экватора    \\
%%      mars & 0 & int & 1: инициализация модели для планеты Марс     \\
%%     kick & 1 & int & 0: инициализация без шума ($p_s = const$) \\
%%           &   &     & 1: генерация белого шума                  \\
%%           &   &     & 2: генерация белого шума симметрично относительно \\
%%       & & & экватора    \\
%%      mars & 0 & int & 1: инициализация модели для планеты Марс     \\
%%     kick & 1 & int & 0: инициализация без шума ($p_s = const$) \\
%%           &   &     & 1: генерация белого шума                  \\
%%           &   &     & 2: генерация белого шума симметрично относительно \\
%%       & & & экватора    \\
%%      mars & 0 & int & 1: инициализация модели для планеты Марс     \\
%%     kick & 1 & int & 0: инициализация без шума ($p_s = const$) \\
%%           &   &     & 1: генерация белого шума                  \\
%%           &   &     & 2: генерация белого шума симметрично относительно \\
%%       & & & экватора    \\
%%      mars & 0 & int & 1: инициализация модели для планеты Марс     \\
%%     kick & 1 & int & 0: инициализация без шума ($p_s = const$) \\
%%           &   &     & 1: генерация белого шума                  \\
%%           &   &     & 2: генерация белого шума симметрично относительно \\
%%       & & & экватора    \\
%%      mars & 0 & int & 1: инициализация модели для планеты Марс     \\
%%     kick & 1 & int & 0: инициализация без шума ($p_s = const$) \\
%%           &   &     & 1: генерация белого шума                  \\
%%           &   &     & 2: генерация белого шума симметрично относительно \\
%%       & & & экватора    \\
%%      mars & 0 & int & 1: инициализация модели для планеты Марс     \\
%%      \hline
%%       %& & & $\:$ \\ 
%%      \multicolumn{4}{|l|}{\&SURFPAR}        \\ \hline
%%     kick & 1 & int & 0: инициализация без шума ($p_s = const$) \\
%%           &   &     & 1: генерация белого шума                  \\
%%           &   &     & 2: генерация белого шума симметрично относительно \\
%%       & & & экватора    \\
%%      mars & 0 & int & 1: инициализация модели для планеты Марс     \\
%%     kick & 1 & int & 0: инициализация без шума ($p_s = const$) \\
%%           &   &     & 1: генерация белого шума                  \\
%%           &   &     & 2: генерация белого шума симметрично относительно \\
%%       & & & экватора    \\
%%      mars & 0 & int & 1: инициализация модели для планеты Марс     \\
%%     kick & 1 & int & 0: инициализация без шума ($p_s = const$) \\
%%           &   &     & 1: генерация белого шума                  \\
%%           &   &     & 2: генерация белого шума симметрично относительно \\
%%       & & & экватора    \\
%%      mars & 0 & int & 1: инициализация модели для планеты Марс     \\
%%     kick & 1 & int & 0: инициализация без шума ($p_s = const$) \\
%%           &   &     & 1: генерация белого шума                  \\
%%           &   &     & 2: генерация белого шума симметрично относительно \\
%%       & & & экватора    \\
%%      mars & 0 & int & 1: инициализация модели для планеты Марс     \\
%%     kick & 1 & int & 0: инициализация без шума ($p_s = const$) \\
%%           &   &     & 1: генерация белого шума                  \\
%%           &   &     & 2: генерация белого шума симметрично относительно \\
%%       & & & экватора    \\
%%      mars & 0 & int & 1: инициализация модели для планеты Марс     \\
%%     kick & 1 & int & 0: инициализация без шума ($p_s = const$) \\
%%           &   &     & 1: генерация белого шума                  \\
%%           &   &     & 2: генерация белого шума симметрично относительно \\
%%       & & & экватора    \\
%%      mars & 0 & int & 1: инициализация модели для планеты Марс     \\
%%     kick & 1 & int & 0: инициализация без шума ($p_s = const$) \\
%%           &   &     & 1: генерация белого шума                  \\
%%           &   &     & 2: генерация белого шума симметрично относительно \\
%%       & & & экватора    \\
%%      mars & 0 & int & 1: инициализация модели для планеты Марс     \\
%%     kick & 1 & int & 0: инициализация без шума ($p_s = const$) \\
%%           &   &     & 1: генерация белого шума                  \\
%%           &   &     & 2: генерация белого шума симметрично относительно \\
%%       & & & экватора    \\
%%      mars & 0 & int & 1: инициализация модели для планеты Марс     \\
%%     kick & 1 & int & 0: инициализация без шума ($p_s = const$) \\
%%           &   &     & 1: генерация белого шума                  \\
%%           &   &     & 2: генерация белого шума симметрично относительно \\
%%       & & & экватора    \\
%%      mars & 0 & int & 1: инициализация модели для планеты Марс     \\ 
%% %     \hline 
%%     \end{longtable}
%% \normalsize% возвращаем шрифт к нормальному
%% \endgroup
%% \section{Использование длинных таблиц с окружением \textit{longtabu}} \label{AppendixB2a}

%% В таблице~\ref{tbl:test-functions} более книжный вариант 
%% длинной таблицы, используя окружение \verb!longtabu! и разнообразные
%% \verb!toprule! \verb!midrule! \verb!bottomrule! из пакета
%% \verb!booktabs!. Чтобы визуально таблица смотрелась лучше, можно
%% использовать следующие параметры: в самом начале задаётся расстояние
%% между строчками с~помощью \verb!arraystretch!. Таблица задаётся на
%% всю ширину, \verb!longtabu! позволяет делить ширину колонок
%% пропорционально "--- тут три колонки в пропорции 1.1:1:4 "--- для каждой
%% колонки первый параметр в описании \verb!X[]!. Кроме того, в~таблице
%% убраны отступы слева и справа с помощью \verb!@{}! в
%% преамбуле таблицы. К первому и второму столбцу применяется
%% модификатор 

%% \verb!>{\setlength{\baselineskip}{0.7\baselineskip}}!,

%% \noindent который уменьшает межстрочный интервал в для текста таблиц (иначе
%% заголовок второго столбца значительно шире, а двухстрочное имя
%% сливается с окружающими). Для первой и второй колонки текст в ячейках
%% выравниваются по~центру как по вертикали, так и по горизонтали -
%% задаётся буквами \verb!m! и \verb!c! в~описании столбца \verb!X[]!. 

%% Так как формулы большие "--- используется окружение \verb!alignedat!,
%% чтобы отступ был одинаковый у всех формул "--- он сделан для всех, хотя
%% для большей части можно было и не использовать.  Чтобы формулы
%% занимали поменьше места в~каждом столбце формулы (где надо)
%% используется \verb!\textstyle! "--- он делает дроби меньше, у знаков
%% суммы и произведения "--- индексы сбоку. Иногда формулы слишком большая,
%% сливается со следующей, поэтому после неё ставится небольшой
%% дополнительный отступ \verb!\vspace*{2ex}!  Для штрафных функций "---
%% размер фигурных скобок задан вручную \verb!\Big\{!, т.к. не умеет
%% \verb!alignedat! работать с~\verb!\left! и \verb!\right! через
%% несколько строк/колонок.


%% В примечании к таблице наоборот, окружение \verb!cases! даёт слишком
%% большие промежутки между вариантами, чтобы их уменьшить, в конце
%% каждой строчки окружения использовался отрицательный дополнительный
%% отступ \verb!\\[-0.5em]!.



%% \begingroup % Ограничиваем область видимости arraystretch
%% \renewcommand{\arraystretch}{1.6}%% Увеличение расстояния между рядами, для улучшения восприятия.
%% \begin{longtabu} to \textwidth
%% {%
%% @{}>{\setlength{\baselineskip}{0.7\baselineskip}}X[1.1mc]%
%% >{\setlength{\baselineskip}{0.7\baselineskip}}X[mc]%
%% X[4]@{}%
%% }
%%         \caption{Тестовые функции для оптимизации, $D$ "---
%%           размерность. Для всех функций значение в точке глобального
%%           минимума равно нулю.\label{tbl:test-functions}}\\% label всегда желательно идти после caption 
        
%%         \toprule     %%% верхняя линейка
%%         Имя           &Стартовый диапазон параметров &Функция  \\ 
%%         \midrule %%% тонкий разделитель. Отделяет названия столбцов. Обязателен по ГОСТ 2.105 пункт 4.4.5 
%%         \endfirsthead

%%         \multicolumn{3}{c}{\small\slshape (продолжение)}        \\ 
%%         \toprule     %%% верхняя линейка
%%         Имя           &Стартовый диапазон параметров &Функция  \\ 
%%         \midrule %%% тонкий разделитель. Отделяет названия столбцов. Обязателен по ГОСТ 2.105 пункт 4.4.5 
%%         \endhead
        
%%         \multicolumn{3}{c}{\small\slshape (окончание)}        \\ 
%%         \toprule     %%% верхняя линейка
%%         Имя           &Стартовый диапазон параметров &Функция  \\ 
%%         \midrule %%% тонкий разделитель. Отделяет названия столбцов. Обязателен по ГОСТ 2.105 пункт 4.4.5 
%%         \endlasthead

%%         \bottomrule %%% нижняя линейка
%%         \multicolumn{3}{r}{\small\slshape продолжение следует}  \\ 
%%         \endfoot   
%%         \endlastfoot

%%         сфера         &$\left[-100,\,100\right]^D$   &
%%         $\begin{aligned}\textstyle f_1(x)=\sum_{i=1}^Dx_i^2\end{aligned}$                                                        \\
%%         Schwefel 2.22 &$\left[-10,\,10\right]^D$     &
%%         $\begin{aligned}\textstyle f_2(x)=\sum_{i=1}^D|x_i|+\prod_{i=1}^D|x_i|\end{aligned}$                                     \\
%%         Schwefel 1.2  &$\left[-100,\,100\right]^D$   &$\begin{aligned}\textstyle f_3(x)=\sum_{i=1}^D\left(\sum_{j=1}^ix_j\right)^2\end{aligned}$                               \\
%%         Schwefel 2.21 &$\left[-100,\,100\right]^D$   &$\begin{aligned}\textstyle f_4(x)=\max_i\!\left\{\left|x_i\right|\right\}\end{aligned}$                             \\
%%         Rosenbrock    &$\left[-30,\,30\right]^D$     &$\begin{aligned}\textstyle f_5(x)=\sum_{i=1}^{D-1}\left[100\!\left(x_{i+1}-x_i^2\right)^2+(x_i-1)^2\right]\end{aligned}$ \\
%%         ступенчатая   &$\left[-100,\,100\right]^D$   &$\begin{aligned}\textstyle f_6(x)=\sum_{i=1}^D\big\lfloor x_i+0.5\big\rfloor^2\end{aligned}$                             \\ 
%% зашумлённая квартическая  &$\left[-1.28,\,1.28\right]^D$ &$\begin{aligned}\textstyle f_7(x)=\sum_{i=1}^Dix_i^4+rand[0,1)\end{aligned}$\vspace*{2ex}\\
%%         Schwefel 2.26 &$\left[-500,\,500\right]^D$   &$\begin{aligned}f_8(x)= &\textstyle\sum_{i=1}^D-x_i\,\sin\sqrt{|x_i|}\,+ \\
%%                     &\vphantom{\sum}+ D\cdot
%%                     418.98288727243369 \end{aligned}$\\
%%         Rastrigin     &$\left[-5.12,\,5.12\right]^D$ &
%%         $\begin{aligned}\textstyle
%%           f_9(x)=\sum_{i=1}^D\left[x_i^2-10\,\cos(2\pi
%%             x_i)+10\right]\end{aligned}$\vspace*{2ex}\\
%%   Ackley        &$\left[-32,\,32\right]^D$     &$\begin{aligned}f_{10}(x)= &\textstyle -20\, \exp\!\left(-0.2\sqrt{\frac{1}{D}\sum_{i=1}^Dx_i^2} \right)-\\
%%                     &\textstyle - \exp\left(\frac{1}{D}\sum_{i=1}^D\cos(2\pi x_i)  \right)  + 20 + e \end{aligned}$ \\
%%         Griewank      &$\left[-600,\,600\right]^D$
%%         &$\begin{aligned}f_{11}(x)= &\textstyle \frac{1}{4000}
%%           \sum_{i=1}^{D}x_i^2 - \prod_{i=1}^D\cos\left(x_i/\sqrt{i}\right) +1     \end{aligned}$ \vspace*{3ex} \\
%%         штрафная 1    &$\left[-50,\,50\right]^D$     &
%%         $\begin{aligned}f_{12}(x)= &\textstyle \frac{\pi}{D}
%%           \Big\{ 10\,\sin^2(\pi y_1) +\\ &+
%%           \textstyle \sum_{i=1}^{D-1}(y_i-1)^2\left[1+10\,\sin^2(\pi
%%               y_{i+1})\right] +\\ &+(y_D-1)^2 \Big\} +\textstyle\sum_{i=1}^D u(x_i,\,10,\,100,\,4)            \end{aligned}$ \vspace*{2ex} \\
%%         штрафная 2    &$\left[-50,\,50\right]^D$     &
%%         $\begin{aligned}f_{13}(x)= &\textstyle 0.1
%%           \Big\{\sin^2(3\pi x_1) +\\ &+
%%           \textstyle \sum_{i=1}^{D-1}(x_i-1)^2\left[1+\sin^2(3 \pi
%%               x_{i+1})\right] + \\ &+(x_D-1)^2\left[1+\sin^2(2\pi
%%               x_D)\right] \Big\} +\\ &+\textstyle\sum_{i=1}^D u(x_i,\,5,\,100,\,4)            \end{aligned}$               \\
%%         сфера         &$\left[-100,\,100\right]^D$   &
%%         $\begin{aligned}\textstyle f_1(x)=\sum_{i=1}^Dx_i^2\end{aligned}$                                                        \\
%%         Schwefel 2.22 &$\left[-10,\,10\right]^D$     &
%%         $\begin{aligned}\textstyle f_2(x)=\sum_{i=1}^D|x_i|+\prod_{i=1}^D|x_i|\end{aligned}$                                     \\
%%         Schwefel 1.2  &$\left[-100,\,100\right]^D$   &$\begin{aligned}\textstyle f_3(x)=\sum_{i=1}^D\left(\sum_{j=1}^ix_j\right)^2\end{aligned}$                               \\
%%         Schwefel 2.21 &$\left[-100,\,100\right]^D$   &$\begin{aligned}\textstyle f_4(x)=\max_i\!\left\{\left|x_i\right|\right\}\end{aligned}$                             \\
%%         Rosenbrock    &$\left[-30,\,30\right]^D$     &$\begin{aligned}\textstyle f_5(x)=\sum_{i=1}^{D-1}\left[100\!\left(x_{i+1}-x_i^2\right)^2+(x_i-1)^2\right]\end{aligned}$ \\
%%         ступенчатая   &$\left[-100,\,100\right]^D$   &$\begin{aligned}\textstyle f_6(x)=\sum_{i=1}^D\big\lfloor x_i+0.5\big\rfloor^2\end{aligned}$                             \\ 
%% зашумлённая квартическая  &$\left[-1.28,\,1.28\right]^D$ &$\begin{aligned}\textstyle f_7(x)=\sum_{i=1}^Dix_i^4+rand[0,1)\end{aligned}$\vspace*{2ex}\\
%%         Schwefel 2.26 &$\left[-500,\,500\right]^D$   &$\begin{aligned}f_8(x)= &\textstyle\sum_{i=1}^D-x_i\,\sin\sqrt{|x_i|}\,+ \\
%%                     &\vphantom{\sum}+ D\cdot
%%                     418.98288727243369 \end{aligned}$\\
%%         Rastrigin     &$\left[-5.12,\,5.12\right]^D$ &
%%         $\begin{aligned}\textstyle
%%           f_9(x)=\sum_{i=1}^D\left[x_i^2-10\,\cos(2\pi
%%             x_i)+10\right]\end{aligned}$\vspace*{2ex}\\
%%   Ackley        &$\left[-32,\,32\right]^D$     &$\begin{aligned}f_{10}(x)= &\textstyle -20\, \exp\!\left(-0.2\sqrt{\frac{1}{D}\sum_{i=1}^Dx_i^2} \right)-\\
%%                     &\textstyle - \exp\left(\frac{1}{D}\sum_{i=1}^D\cos(2\pi x_i)  \right)  + 20 + e \end{aligned}$ \\
%%         Griewank      &$\left[-600,\,600\right]^D$
%%         &$\begin{aligned}f_{11}(x)= &\textstyle \frac{1}{4000}
%%           \sum_{i=1}^{D}x_i^2 - \prod_{i=1}^D\cos\left(x_i/\sqrt{i}\right) +1     \end{aligned}$ \vspace*{3ex} \\
%%         штрафная 1    &$\left[-50,\,50\right]^D$     &
%%         $\begin{aligned}f_{12}(x)= &\textstyle \frac{\pi}{D}
%%           \Big\{ 10\,\sin^2(\pi y_1) +\\ &+
%%           \textstyle \sum_{i=1}^{D-1}(y_i-1)^2\left[1+10\,\sin^2(\pi
%%               y_{i+1})\right] +\\ &+(y_D-1)^2 \Big\} +\textstyle\sum_{i=1}^D u(x_i,\,10,\,100,\,4)            \end{aligned}$ \vspace*{2ex} \\
%%         штрафная 2    &$\left[-50,\,50\right]^D$     &
%%         $\begin{aligned}f_{13}(x)= &\textstyle 0.1
%%           \Big\{\sin^2(3\pi x_1) +\\ &+
%%           \textstyle \sum_{i=1}^{D-1}(x_i-1)^2\left[1+\sin^2(3 \pi
%%               x_{i+1})\right] + \\ &+(x_D-1)^2\left[1+\sin^2(2\pi
%%               x_D)\right] \Big\} +\\ &+\textstyle\sum_{i=1}^D u(x_i,\,5,\,100,\,4)            \end{aligned}$               \\
%%         \midrule%%% тонкий разделитель
%%         \multicolumn{3}{@{}p{\textwidth}}{%
%%             \vspace*{-3.5ex}% этим подтягиваем повыше
%%             \hspace*{2.5em}% абзацный отступ - требование ГОСТ 2.105
%%             Примечание "---  Для функций $f_{12}$ и $f_{13}$
%%             используется $y_i = 1 + \frac{1}{4}(x_i+1)$ и
%%             $u(x_i,\,a,\,k,\,m)=\begin{cases}
%% k(x_i-a)^m,\quad &x_i >a\\[-0.5em]
%% 0,\quad &-a\leq x_i \leq a\\[-0.5em]
%% k(-x_i-a)^m,\quad &x_i <-a
%% \end{cases}$  }   \\        \bottomrule %%% нижняя линейка 
%% \end{longtabu} 
%% \endgroup


%% \section{Форматирование внутри таблиц} \label{AppendixB3}

%% В таблице~\ref{tbl:other-row} пример с чересстрочным
%% форматированием. В \verb+userstyles.tex+  задаётся счётчик
%% \verb+\newcounter{rowcnt}+ который увеличивается на 1 после каждой
%% строчки (как указано в преамбуле таблицы). Кроме того, задаётся
%% условный макрос \verb+\altshape+ который выдаёт одно из
%% двух типов форматирования в зависимости от чётности счётчика. 

%% В таблице~\ref{tbl:other-row} каждая чётная строчка --- синяя,
%% нечётная --- с наклоном и слегка поднята вверх. Визуально это приводит
%% к тому, что среднее значение и среднеквадратичное изменение
%% группируются и хорошо выделяются взглядом в таблице. Сохраняется
%% возможность отдельные значения в таблице выделить цветом или
%% шрифтом. К первому и второму столбцу форматирование не применяется по
%% сути таблицы, к шестому общее форматирование не применяетсся для
%% наглядности.

%% Так как заголовок таблицы тоже считается за строчку, то перед ним (для
%% первого, промежуточного и финального варианта) счётчик обнуляется, а в
%% \verb+\altshape+ для нулевого значения счётчика форматирования не
%% применяется. 


%% \begingroup % Ограничиваем область видимости arraystretch
%% \renewcommand\altshape{
%%   \ifnumequal{\value{rowcnt}}{0}{
%%     % Стиль для заголовка таблицы
%%   }{
%%     \ifnumodd{\value{rowcnt}}
%%     {
%%       \color{blue} % Cтиль для нечётных строк
%%     }{
%%       \vspace*{-0.8ex}\itshape} % Стиль для чётных строк
%%   }
%% }
%% \newcolumntype{A}{ >{\altshape}X[1mc]}
%% \needspace{2\baselineskip}
%% \renewcommand{\arraystretch}{0.9}%% Уменьшаем  расстояние между
%%                                 %% рядами, чтобы таблица не так много
%%                                 %% места занимала в дисере.
%% \begin{longtabu} to \textwidth {@{}X[0.2ml]X[0.9mc]AAAX[0.99mc]>{\setlength{\baselineskip}{0.7\baselineskip}}AA<{\stepcounter{rowcnt}}@{}}
%% % \begin{longtabu} to \textwidth {@{}X[0.2ml]X[1mc]X[1mc]X[1mc]X[1mc]X[1mc]>{\setlength{\baselineskip}{0.7\baselineskip}}X[1mc]X[1mc]@{}}
%%   \caption{Длинная таблица с примером чересстрочного форматирования\label{tbl:other-row}}\vspace*{1ex}\\% label всегда желательно идти после caption
%%   % \vspace*{1ex}     \\

%%   \toprule %%% верхняя линейка  
%% \setcounter{rowcnt}{0} &Итерации & JADE\texttt{++} & JADE & jDE & SaDE
%% & DE/rand /1/bin & PSO \\ 
%%  \midrule %%% тонкий разделитель. Отделяет названия столбцов. Обязателен по ГОСТ 2.105 пункт 4.4.5 
%%  \endfirsthead

%%  \multicolumn{8}{c}{\small\slshape (продолжение)} \\ 
%%  \toprule %%% верхняя линейка
%% \setcounter{rowcnt}{0} &Итерации & JADE\texttt{++} & JADE & jDE & SaDE
%% & DE/rand /1/bin & PSO \\ 
%%  \midrule %%% тонкий разделитель. Отделяет названия столбцов. Обязателен по ГОСТ 2.105 пункт 4.4.5 
%%  \endhead
 
%%  \multicolumn{8}{c}{\small\slshape (окончание)} \\ 
%%  \toprule %%% верхняя линейка
%% \setcounter{rowcnt}{0} &Итерации & JADE\texttt{++} & JADE & jDE & SaDE
%% & DE/rand /1/bin & PSO \\ 
%%  \midrule %%% тонкий разделитель. Отделяет названия столбцов. Обязателен по ГОСТ 2.105 пункт 4.4.5 
%%  \endlasthead

%%  \bottomrule %%% нижняя линейка
%%  \multicolumn{8}{r}{\small\slshape продолжение следует}     \\ 
%%  \endfoot 
%%  \endlastfoot
 
%% f1  & 1500 & \textbf{1.8E-60}   & 1.3E-54   & 2.5E-28   & 4.5E-20   & 9.8E-14   & 9.6E-42   \\\nopagebreak
%%     &      & (8.4E-60) & (9.2E-54) & \color{red}(3.5E-28) & (6.9E-20) & (8.4E-14) & (2.7E-41) \\
%% f2  & 2000 & 1.8E-25   & 3.9E-22   & 1.5E-23   & 1.9E-14   & 1.6E-09   & 9.3E-21   \\\nopagebreak
%%     &      & (8.8E-25) & (2.7E-21) & (1.0E-23) & (1.1E-14) & (1.1E-09) & (6.3E-20) \\
%% f3  & 5000 & 5.7E-61   & 6.0E-87   & 5.2E-14   & \color{green}9.0E-37   & 6.6E-11   & 2.5E-19   \\\nopagebreak
%%     &      & (2.7E-60) & (1.9E-86) & (1.1E-13) & (5.4E-36) & (8.8E-11) & (3.9E-19) \\
%% f4  & 5000 & 8.2E-24   & 4.3E-66   & 1.4E-15   & 7.4E-11   & 4.2E-01   & 4.4E-14   \\\nopagebreak
%%     &      & (4.0E-23) & (1.2E-65) & (1.0E-15) & (1.8E-10) & (1.1E+00) & (9.3E-14) \\
%% f5  & 3000 & 8.0E-02   & 3.2E-01   & 1.3E+01   & 2.1E+01   & 2.1E+00   & 2.5E+01   \\\nopagebreak
%%     &      & (5.6E-01) & (1.1E+00) & (1.4E+01) & (7.8E+00) & (1.5E+00) & (3.2E+01) \\
%% f6  & 100  & 2.9E+00   & 5.6E+00   & 1.0E+03   & 9.3E+02   & 4.7E+03   & 4.5E+01   \\\nopagebreak
%%     &      & (1.2E+00) & (1.6E+00) & (2.2E+02) & (1.8E+02) & (1.1E+03) & (2.4E+01) \\
%% f7  & 3000 & 6.4E-04   & 6.8E-04   & 3.3E-03   & 4.8E-03   & 4.7E-03   & 2.5E-03   \\\nopagebreak
%%     &      & (2.5E-04) & (2.5E-04) & (8.5E-04) & (1.2E-03) & (1.2E-03) & (1.4E-03) \\
%% f8  & 1000 & 3.3E-05   & 7.1E+00   & 7.9E-11   & 4.7E+00   & 5.9E+03   & 2.4E+03   \\\nopagebreak
%%     &      & (2.3E-05) & (2.8E+01) & (1.3E-10) & (3.3E+01) & (1.1E+03) & (6.7E+02) \\
%% f9  & 1000 & 1.0E-04   & 1.4E-04   & 1.5E-04   & 1.2E-03   & 1.8E+02   & 5.2E+01   \\\nopagebreak
%%     &      & (6.0E-05) & (6.5E-05) & (2.0E-04) & (6.5E-04) & (1.3E+01) & (1.6E+01) \\
%% f10 & 500  & 8.2E-10   & 3.0E-09   & 3.5E-04   & 2.7E-03   & 1.1E-01   & 4.6E-01   \\\nopagebreak
%%     &      & (6.9E-10) & (2.2E-09) & (1.0E-04) & (5.1E-04) & (3.9E-02) & (6.6E-01) \\
%% f11 & 500  & 9.9E-08   & 2.0E-04   & 1.9E-05   & 7.8E-04)  & 2.0E-01   & 1.3E-02   \\\nopagebreak
%%     &      & (6.0E-07) & (1.4E-03) & (5.8E-05) & (1.2E-03  & (1.1E-01) & (1.7E-02) \\
%% f12 & 500  & 4.6E-17   & 3.8E-16   & 1.6E-07   & 1.9E-05   & 1.2E-02   & 1.9E-01   \\\nopagebreak
%%     &      & (1.9E-16) & (8.3E-16) & (1.5E-07) & (9.2E-06) & (1.0E-02) & (3.9E-01) \\
%% f13 & 500  & 2.0E-16   & 1.2E-15   & 1.5E-06   & 6.1E-05   & 7.5E-02   & 2.9E-03   \\\nopagebreak
%%     &      & (6.5E-16) & (2.8E-15) & (9.8E-07) & (2.0E-05) & (3.8E-02) & (4.8E-03) \\
%% f1  & 1500 & \textbf{1.8E-60}   & 1.3E-54   & 2.5E-28   & 4.5E-20   & 9.8E-14   & 9.6E-42   \\\nopagebreak
%%     &      & (8.4E-60) & (9.2E-54) & \color{red}(3.5E-28) & (6.9E-20) & (8.4E-14) & (2.7E-41) \\
%% f2  & 2000 & 1.8E-25   & 3.9E-22   & 1.5E-23   & 1.9E-14   & 1.6E-09   & 9.3E-21   \\\nopagebreak
%%     &      & (8.8E-25) & (2.7E-21) & (1.0E-23) & (1.1E-14) & (1.1E-09) & (6.3E-20) \\
%% f3  & 5000 & 5.7E-61   & 6.0E-87   & 5.2E-14   & 9.0E-37   & 6.6E-11   & 2.5E-19   \\\nopagebreak
%%     &      & (2.7E-60) & (1.9E-86) & (1.1E-13) & (5.4E-36) & (8.8E-11) & (3.9E-19) \\
%% f4  & 5000 & 8.2E-24   & 4.3E-66   & 1.4E-15   & 7.4E-11   & 4.2E-01   & 4.4E-14   \\\nopagebreak
%%     &      & (4.0E-23) & (1.2E-65) & (1.0E-15) & (1.8E-10) & (1.1E+00) & (9.3E-14) \\
%% f5  & 3000 & 8.0E-02   & 3.2E-01   & 1.3E+01   & 2.1E+01   & 2.1E+00   & 2.5E+01   \\\nopagebreak
%%     &      & (5.6E-01) & (1.1E+00) & (1.4E+01) & (7.8E+00) & (1.5E+00) & (3.2E+01) \\
%% f6  & 100  & 2.9E+00   & 5.6E+00   & 1.0E+03   & 9.3E+02   & 4.7E+03   & 4.5E+01   \\\nopagebreak
%%     &      & (1.2E+00) & (1.6E+00) & (2.2E+02) & (1.8E+02) & (1.1E+03) & (2.4E+01) \\
%% f7  & 3000 & 6.4E-04   & 6.8E-04   & 3.3E-03   & 4.8E-03   & 4.7E-03   & 2.5E-03   \\\nopagebreak
%%     &      & (2.5E-04) & (2.5E-04) & (8.5E-04) & (1.2E-03) & (1.2E-03) & (1.4E-03) \\
%% f8  & 1000 & 3.3E-05   & 7.1E+00   & 7.9E-11   & 4.7E+00   & 5.9E+03   & 2.4E+03   \\\nopagebreak
%%     &      & (2.3E-05) & (2.8E+01) & (1.3E-10) & (3.3E+01) & (1.1E+03) & (6.7E+02) \\
%% f9  & 1000 & 1.0E-04   & 1.4E-04   & 1.5E-04   & 1.2E-03   & 1.8E+02   & 5.2E+01   \\\nopagebreak
%%     &      & (6.0E-05) & (6.5E-05) & (2.0E-04) & (6.5E-04) & (1.3E+01) & (1.6E+01) \\
%% f10 & 500  & 8.2E-10   & 3.0E-09   & 3.5E-04   & 2.7E-03   & 1.1E-01   & 4.6E-01   \\\nopagebreak
%%     &      & (6.9E-10) & (2.2E-09) & (1.0E-04) & (5.1E-04) & (3.9E-02) & (6.6E-01) \\
%% f11 & 500  & 9.9E-08   & 2.0E-04   & 1.9E-05   & 7.8E-04)  & 2.0E-01   & 1.3E-02   \\\nopagebreak
%%     &      & (6.0E-07) & (1.4E-03) & (5.8E-05) & (1.2E-03  & (1.1E-01) & (1.7E-02) \\
%% f12 & 500  & 4.6E-17   & 3.8E-16   & 1.6E-07   & 1.9E-05   & 1.2E-02   & 1.9E-01   \\\nopagebreak
%%     &      & (1.9E-16) & (8.3E-16) & (1.5E-07) & (9.2E-06) & (1.0E-02) & (3.9E-01) \\
%% f13 & 500  & 2.0E-16   & 1.2E-15   & 1.5E-06   & 6.1E-05   & 7.5E-02   & 2.9E-03   \\\nopagebreak
%%     &      & (6.5E-16) & (2.8E-15) & (9.8E-07) & (2.0E-05) & (3.8E-02) & (4.8E-03) \\

%%     % \vspace*{1ex}     \\
%% %         \midrule%%% тонкий разделитель
%% %         \multicolumn{3}{@{}p{\textwidth}}{%
%% %             % \vspace*{-4ex}% этим подтягиваем повыше
%% %             % \hspace*{2.5em}% абзацный отступ - требование ГОСТ 2.105
%% %             Примечание "---  Для функций $f_{12}$ и $f_{13}$
%% %             используется $y_i = 1 + \frac{1}{4}(x_i+1)$ и
%% %             $u(x_i,\,a,\,k,\,m)=\begin{cases}
%% % k(x_i-a)^m,\quad  & x_i >a     \\[-0.5em]
%% % 0,\quad           & -a\leq x_i \leq a        \\[-0.5em]
%% % k(-x_i-a)^m,\quad & x_i <-a
%% % \end{cases}$  }     \\
%% \bottomrule %%% нижняя линейка 
%% \end{longtabu} \endgroup

%% \section{Очередной подраздел приложения} \label{AppendixB3}

%% Нужно больше подразделов приложения!

%% \section{И ещё один подраздел приложения} \label{AppendixB4}

%% Нужно больше подразделов приложения!



\end{document}

