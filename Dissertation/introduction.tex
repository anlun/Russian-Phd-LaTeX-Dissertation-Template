\chapter*{Введение}							% Заголовок
\addcontentsline{toc}{chapter}{Введение}	% Добавляем его в оглавление

\newcommand{\actuality}{{\textbf{\actualityTXT}}}
\newcommand{\progress}{\textbf{\progressTXT}}
\newcommand{\aim}{{\textbf\aimTXT}}
\newcommand{\tasks}{\textbf{\tasksTXT}}
\newcommand{\novelty}{\textbf{\noveltyTXT}}
\newcommand{\influence}{\textbf{\influenceTXT}}
\newcommand{\methods}{\textbf{\methodsTXT}}
\newcommand{\defpositions}{\textbf{\defpositionsTXT}}
\newcommand{\reliability}{\textbf{\reliabilityTXT}}
\newcommand{\probation}{\textbf{\probationTXT}}
\newcommand{\contribution}{\textbf{\contributionTXT}}
\newcommand{\publications}{\textbf{\publicationsTXT}}

\newcommand{\ccite}[1]{\cite{#1}}
\newcommand{\cscite}[1]{~\cite{#1}}
\input{common/characteristic} % Характеристика работы по структуре во введении и в автореферате не отличается (ГОСТ Р 7.0.11, пункты 5.3.1 и 9.2.1), потому её загружаем из одного и того же внешнего файла, предварительно задав форму выделения некоторым параметрам

%% \textbf{Объем и структура работы.} Диссертация состоит из~введения, пяти глав, заключения и~двух приложений.
%% %% на случай ошибок оставляю исходный кусок на месте, закомментированным
%% %Полный объём диссертации составляет  \ref*{TotPages}~страницу с~\totalfigures{}~рисунками и~\totaltables{}~таблицами. Список литературы содержит \total{citenum}~наименований.
%% %
%% Полный объём диссертации составляет
%% \formbytotal{TotPages}{страниц}{у}{ы}{}, включая
%% \formbytotal{totalcount@figure}{рисун}{ок}{ка}{ков} и
%% \formbytotal{totalcount@table}{таблиц}{у}{ы}{}.   Список литературы содержит  
%% \formbytotal{citenum}{наименован}{ие}{ия}{ий}.

\textbf{Благодарности.} \\
Я хочу выразить признательность своим коллегам, друзьям и семье, присутствие которых
в моей жизни сделало эту работу возможной.
Начать я хотел бы с выражения благодарности Д. Ю. Булычеву, А. В. Иванову и компании JetBrains,
которые предоставили мне уникальную возможность заниматься наукой как основной деятельностью.
Я признателен своему научному руководителю Д. В. Кознову, который помог мне пройти через тернистый путь
создания диссертации. 
Я благодарен И. Д. Сергею и А. Наневски (A. Nanevski) за бескорыстную поддержку и руководство
моей работой на первых этапах.
Я хотел бы выразить признательность старшим коллегам В. Вафеядису (V. Vafeiadis) и О. Лахаву (O. Lahav)
за помощь в исследованиях и сотрудничество.
Я очень благодарен коллективам MPI-SWS, IMDEA Software и кафедры системного программирования,
а также лично А. Н. Терехову за создание душевной и плодотворной атмосферы для работы.

Я признателен за удивительную отзывчивость и дружбу Д. А. Березуну,
за помощь в тяжелейших ситуациях и человеческую доброту А. М. Карачуну,
за немыслимые поддержку и понимание А. В., Т. П. и В. Е. Турецким и моей семье.
Я горячо признателен моей маме за её самоотверженное стремление помочь мне на каждом этапе моей жизни.
Я безмерно благодарен моему папе за всё.
