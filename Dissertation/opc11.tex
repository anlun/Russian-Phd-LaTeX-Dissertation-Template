\chapter{Операционная модель памяти C/C++11} \label{sec:opc11}
В главе описана операционная модель памяти C/C++11 \cite{Podkopaev-al:CoRR16}.


Модель представлена как семейство операционных семантик, которые описывают
различные аспекты модели памяти C/C++11 \cite{Batty-al:POPL11}.
Все семантики базируются на комбинации двух основных идей: 
\emph{фронтов} и \emph{операционных буферов}.
Фронты используются для представления осведомленности потоков о текущем состоянии общей памяти,
тогда как операционные буферы позволяют откладывать исполнение инструкций и
производить спекулятивные вычисления.
Модель представлена в виде абстрактной машины и покрывает 

\section{Модель на примерах}
В разделе мы рассмотрим базовые концепции модели на нескольких примерах.

\subsection{Синхронизация потоков. Фронты}
\subsection{Спекулятивное исполнение. Операционные буферы}

\section{Полная модель}
В разделе описываются более сложные аспекты модели памяти C/C++, такие как
$\sco$-обращения, неатомарные инструкции и гонки по данным,
$\con$-чтения, соединение потоков (thread's joining),
а также расслабленные обращения.

\section{Формальное определение модели}
В разделе приводится математическое определение операционной модели C/C++11.

\section{Интерпретация и тестирование модели}

\subsection{Тестирование алгоритма RCU}

\newcommand{\nullPtr}{\kw{null}}
\newcommand{\funcSt}[1]{{\rm #1}}

{\small{
\textbf{Программа:}
\vspace{-17pt}
\[
\writeInstParam{\na}{cw}{0};
\writeInstParam{\na}{cr1}{0};
\writeInstParam{\na}{cr2}{0};
\writeInstParam{\na}{lhead}{\nullPtr};
\]
%% \begin{lstlisting}
%%                         |$[$|cw|$]_{na}$| := 0; |$[$|cr1|$]_{na}$| := 0; |$[$|cr2|$]_{na}$| := 0; |$[$|lhead|$]_{na}$| := null;
%% \end{lstlisting}
\begin{tabular}{L || L || L}
  \begin{array}{l}
    \writeInstParam{\rlx}{a}{(1, \nullPtr)}; \\
    \writeInstParam{\na}{ltail}{a}; \\
    \writeInstParam{\rel}{lhead}{a}; \\
    \funcSt{append}(b, 10, ltail); \\
    \funcSt{append}(c, 100, ltail); \\
    \funcSt{updateSecondNode}(d, 1000) \\
  \end{array}
&
  \begin{array}{l}
    \writeInstParam{\na}{sum11}{0}; \\
    \funcSt{rcuOnline}(cw, cr1); \\
    \funcSt{traverse}(lhead, cur1, sum11); \\
    \funcSt{rcuOffline}(cw, cr1); \\
    \\
    \writeInstParam{\na}{sum12}{0}; \\
    \funcSt{rcuOnline}(cw, cr1); \\
    \funcSt{traverse}(lhead, cur1, sum12); \\
    \funcSt{rcuOffline}(cw, cr1); \\
    \\
    \readInstParam{na}{r11}{sum11}; \\
    \readInstParam{na}{r11}{sum12} \\
  \end{array}
&
  \begin{array}{l}
    \writeInstParam{\na}{sum21}{0}; \\
    \funcSt{rcuOnline}(cw, cr2); \\
    \funcSt{traverse}(lhead, cur2, sum21); \\
    \funcSt{rcuOffline}(cw, cr2); \\
    \\
    \writeInstParam{\na}{sum22}{0}; \\
    \funcSt{rcuOnline}(cw, cr2); \\
    \funcSt{traverse}(lhead, cur2, sum22); \\
    \funcSt{rcuOffline}(cw, cr2); \\
    \\
    \readInstParam{na}{r21}{sum21}; \\
    \readInstParam{na}{r21}{sum22} \\
  \end{array}
\end{tabular}

\vspace{5pt}
\hrule
\vspace{5pt}

\textbf{Функции:}

\begin{tabular}{@{}L L l}

\begin{array}{@{}l}
  \funcSt{append}(loc, value, ltail) \triangleq \\
  \quad \writeInstParam{\rlx}{loc}{(value, \nullPtr)}; \\
  \quad \readInstParam{\na}{rt}{ltail}; \\
  \quad \readInstParam{\rlx}{rtc}{rt}; \\
  \quad \writeInstParam{\rel}{rt}{(\kw{fst} \; rtc, loc)}; \\
  \quad \writeInstParam{\na}{ltail}{loc} \\
  \\
  \funcSt{updateSecondNode}(loc, value) \triangleq \\
  \quad \readInstParam{\rlx}{r1}{lhead}; \\
  \quad \readInstParam{\rlx}{r1c}{r1}; \\
  \quad \assignInst{r2}{\kw{snd} \; r1c}; \\
  \quad \readInstParam{\rlx}{r2c}{r2}; \\
  \quad \assignInst{r3}{\kw{snd} \; r2c}; \\
  \quad \writeInstParam{\rel}{loc}{(value, r3)}; \\
  \quad \writeInstParam{\rel}{r1}{(\kw{fst} \; r1c, loc)}; \\
  \quad \funcSt{sync}(cw, cr1, cr2); \\
  \quad \kw{delete} \; r2
\end{array}

&

\begin{array}{l}
  \funcSt{traverse}(lhead, curNodeLoc, resLoc) \triangleq \\
  \quad \readInstParam{\acq}{rh}{lhead}; \\
  \quad \writeInstParam{\na}{curNodeLoc}{rh}; \\
  \quad \kw{repeat} \\
  \qquad \readInstParam{\na}{rCurNode}{curNodeLoc}; \\
  \qquad \kw{if} \; rCurNode \neq \nullPtr \\
  \qquad \begin{array}[t]{@{}l l}
         \kw{then}& {\begin{array}[t]{@{}l}
                      \readInstParam{\acq}{rNode}{rCurNode}; \\
                      \readInstParam{\na}{rRes}{resLoc}; \\
                      \assignInst{rVal}{\kw{fst} \; rNode}; \\
                      \writeInstParam{\na}{resLoc}{rVal + rRes}; \\
                      \writeInstParam{\na}{curNodeLoc}{\kw{snd} \; rNode}; \\
                      0
                    \end{array}} \\
         \kw{else}& 1 \\
         \end{array}\\
  \qquad \kw{fi}
  \quad \kw{end} \\

\end{array}
&

\begin{tabular}{@{}l@{}}
\begin{lstlisting}
sync(cw, cr1, cr2) |$\triangleq$|
  rcw  = |$[$|cw|$]_{rlx}$|;
  rcwn = rcw + 2;
  |$[$|cw|$]_{rel}$| := rcwn;
  |\graybox{\texttt{syncWithReader(rcwn, cr1);}}|
  |\graybox{\texttt{syncWithReader(rcwn, cr2)}}|
\end{lstlisting}
\\
\\
\begin{lstlisting}
syncWithReader(rcwn, cr) |$\triangleq$| 
  repeat |$[$|cr|$]_{acq}$| >= rcwn end
\end{lstlisting}
\\
\\
\begin{lstlisting}
rcuOnline(cw, cr) |$\triangleq$| 
  |$[$|cr|$]_{rlx}$| := |$[$|cw|$]_{acq}$| + 1
\end{lstlisting}
\\
\\
\begin{lstlisting}
rcuOffline(cw, cr) |$\triangleq$| 
  |$[$|cr|$]_{rel}$| := |$[$|cw|$]_{rlx}$|
\end{lstlisting}
\end{tabular}


\end{tabular}
}}


\section{Свойства модели. Выводы}
