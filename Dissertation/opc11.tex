\chapter{Операционная модель памяти C/C++11} \label{sec:opc11}
В главе описана операционная модель памяти C/C++11 \cite{Podkopaev-al:CoRR16}, или \OpCpp,
и реализованный интерпретатор для неё \app{TODO: \url{}}.
Сама модель представлена как семейство аспектов, каждый из которых описывает некоторую особенность
изначальной модели C/C++11 \cite{Batty-al:POPL11}.
Такое представление позволяет упрощать модель для задач, в которых рассматривается только подмножество
языка модели C/C++11.
Также, поскольку некоторые особенности оригинальной модели, такие как поддержка секвенциализации (см. раздел \ref{sec:opc11:join}),
неоднозначны, аспектное представление позволяет настраивать интерпретатор под интересующий вариант семантики.

Ключевыми понятиями в модели являются \emph{фронты} и \emph{операционные буферы}.
Фронты используются для представления осведомленности потоков о текущем состоянии общей памяти,
тогда как операционные буферы позволяют откладывать исполнение инструкций и
производить спекулятивные вычисления.

%% Операционное представление различных аспектов оригинальной модели памяти C/C++11 \cite{Batty-al:POPL11}
%% базируется на комбинации двух основных идей: \emph{фронтов} и \emph{операционных буферов}.

Описание модели структурировано следующим образом.
В разделе \ref{sec:opc11:base} рассматриваются базовые концепции модели на нескольких примерах,
которые требуются для представления исполнения расслабленных инструкций, приобретающих чтений и
высвобождающих записей.
Далее в разделе \ref{sec:opc11:fullmodel} рассматриваются части модели, которые описывают
поведение $\sco$-инструкций, неатомарных обращений к памяти, потребляющих чтений, а также
высвобождающих цепочек (release sequences).
\app{TODO}

\section{Основные концепции модели}
\label{sec:opc11:base}
Модель памяти $\OpCpp$ задана операционным способом, а, значит, существует
некоторая абстрактная машина, связанная с моделью, которая исполняет программы
по шагам. Далее мы будем использовать термин \emph{машина} $\OpCpp$ для её обозначения.

\subsection{Память и базовый фронт}
Состояние машины $\OpCpp$ включается в себя множество сообщений, которое называется 
\emph{памятью}. Сообщение --- это тройка из локации, значения и \emph{метки времени} (timestamp).
Метки времени являются натуральными числами и используются для упорядочивания сообщений, которые связаны с одной и той же локацией.
Такой порядок аналогичен отношению $\lMO$ в аксиоматической модели C/C++11 \cite{Batty-al:POPL11}.

Рассмотрим машину $\OpCpp$ на примере исполнения следующей программы.
\begin{equation*}
\tag{MP-rlx-2}
\begin{tabular}{c}
  $\writeInstParam{\rlx}{x}{0}; \writeInstParam{\rlx}{y}{0};$ \\
\begin{tabular}{L || L}
  \writeInstParam{\rlx}{x}{1}; & \readInstParam{\rlx}{a}{y}; \\
  \writeInstParam{\rlx}{y}{1}  & \readInstParam{\rlx}{b}{x}; \\
                               & \readInstParam{\rlx}{c}{x}\\
\end{tabular}
\end{tabular}
\end{equation*}
Это очередная вариация программы ${\rm MP}$, в которой во второй поток было добавлено дополнительное
чтение из локации $x$.

После исполнения первых двух инструкций программы
($\writeInstParam{\rlx}{x}{0}; \writeInstParam{\rlx}{y}{0}$) память машины будет содержать два сообщения:
\[
M = \{\angled{x:0@\tstamp{0}}, \angled{y:0@\tstamp{0}}\},
\]
где $\tstamp{0}$ -- метка времени.
После того, как левый поток закончит своё исполнение, в памяти будет четыре сообщения:
\[
M = \{\angled{x:0@\tstamp{0}}, \angled{y:0@\tstamp{0}},
      \angled{y:1@\tstamp{1}},\angled{x:1@\tstamp{1}}\}.
\]

Заметим, что в оригинальной модели C/C++11 \cite{Batty-al:POPL11} у рассматриваемой программы есть сценарий поведения
с результатом $[a = 1, b = 0, c = 0]$.
Для того, чтобы разрешить результат $[a = 1, b = 0, c = 0]$, потоки машины $\OpCpp$ имеют право при чтении выбирать из памяти
сообщение не с самым большой меткой времени (как этого требовала бы абстрактная машина, представляющая модель SC), т.е.
не самую последнюю запись в локацию.
Так, после завершения исполнения левого потока, правый поток может сначала прочитать сообщение $\angled{y:1@\tstamp{1}}$,
присвоив в регистр $a$ значение $1$, а после --- два раза из старого сообщения $\angled{x:0@\tstamp{0}}$,
получив $b = c = 0$.

В то же время модель C/C++11 запрещает сценарий поведения $[a = 1, b = 1, c = 0]$, поскольку гарантирует, что если поток
``увидел'' более новую запись в локацию, в данном случае сообщение $\angled{x:1@\tstamp{1}}$,
то более он не может прочитать более старое, с точки зрения отношения $\lMO$, сообщение.
Для того, чтобы поддержать данное ограничение, в модели $\OpCpp$ у каждого потока есть т.н. \emph{базовый фронт}.
Базовый фронт --- это частичная функция, которая по локации возвращает максимальную метку времени связанного с локацией сообщения, о
котором ``осведомлен'' поток.
Рассмотрим действие базового фронта на примере сценария поведения программы ${\rm MP\text{-}rlx\text{-}2}$,
который приводит к результату $[a = 1, b = 1, c = \_]$.

Так, после того, как поток читает или записывает сообщение в локацию $\loc$ с меткой времени $\tau$, он обновляет
свой базовый фронт по этой локации до $\tau$ и более не может читать $\loc$-сообщения с меньшей меткой.

Изначально в системе один поток $T0$ с пустым базовым фронтом.
После исполнения первых двух инструкций программы
($\writeInstParam{\rlx}{x}{0}; \writeInstParam{\rlx}{y}{0}$) базовый фронт потока, $T0.\Rcur$, будет указывать на
соответствующие сообщения из памяти:
\[
T0.\Rcur = [x@\tstamp{0}, y@\tstamp{0}]
\]
Далее машина стартует два потока, $T1$ и $T2$, базовые фронты которых будут равняться $T0.\Rcur$.
\[
T1.\Rcur = [x@\tstamp{0}, y@\tstamp{0}] \quad T2.\Rcur = [x@\tstamp{0}, y@\tstamp{0}]
\]
Далее, исполняются инструкции (левого) потока $T1$ и память содержит два новых сообщения,
а базовый фронт потока $T1$ увеличивается до $\tstamp{1}$ по обоим локациям,
т.к. поток, сделавший запись, естественным образом осведомлен о ней.
\[
\begin{array}{l}
T1.\Rcur = [\graybox{x@\tstamp{1}}, \graybox{y@\tstamp{1}}] \quad T2.\Rcur = [x@\tstamp{0}, y@\tstamp{0}]
\end{array}
\]
После этого (правый) поток $T2$ может прочитать новое сообщение в локацию $y$, присвоив $1$ в регистр $a$
и обновив свой фронт по локации $y$ до $\tstamp{1}$.
\[
\begin{array}{l}
T1.\Rcur = [x@\tstamp{1}, y@\tstamp{1}] \quad T2.\Rcur = [x@\tstamp{0}, \graybox{y@\tstamp{1}}]
\end{array}
\]
Т.к. базовый фронт потока $T2$ по локации $x$ равен $\tstamp{0}$, т.е. поток ещё не осведомлён о новой записи в $x$,
поток может прочитать либо сообщение $\angled{x:0@\tstamp{0}}$, либо сообщение $\angled{x:1@\tstamp{1}}$.
Для того, чтобы $b$ равнялось $1$, поток $T2$ должен прочитать из более нового сообщения, что обновит его
базовый фронт по локации $x$.
\[
\begin{array}{l}
T1.\Rcur = [x@\tstamp{1}, y@\tstamp{1}] \quad T2.\Rcur = [\graybox{x@\tstamp{1}}, y@\tstamp{1}]
\end{array}
\]
После этого потоку $T2$ остаётся только выполнить последнее чтение ($\readInstParam{\rlx}{c}{x}$),
и поскольку его базовый фронт по локации $x$ равен $\tstamp{1}$, то он может прочитать только
из нового сообщения в локацию $x$ ($\angled{x:1@\tstamp{1}}$).
Таким образом модель $\OpCpp$ запрещает результат $[a = 1, b = 1, c = 0]$.

%% Для рассматриваемой программы модель $\OpCpp$ должна разрешать сценарий поведения с результатом $[a = 1, b = 0, c = 0]$,
%% т.к. такой результат разрешает оригинальная модель C/C++11 \cite{Batty-al:POPL11}.
%% Рассмотрим сценарий поведения машины $\OpCpp$, который приводит к нему.

\subsection{Синхронизация потоков}
Рассмотрим программу ${\rm MP\text{-}rel\text{-}acq}$, которую мы уже обсуждали в главе \ref{sec:overview}.
\begin{equation*}
  \tag{MP-rel-acq}
\begin{tabular}{c}
  $\writeInstParam{\rlx}{x}{0}; \writeInstParam{\rlx}{y}{0};$ \\
\begin{tabular}{L || L}
  \writeInstParam{\rlx}{x}{1}; & \readInstParam{\acq}{a}{y}; \\
  \writeInstParam{\rel}{y}{1} & \readInstParam{\rlx}{b}{x}\\
\end{tabular}
\end{tabular}
\end{equation*}
Результат $[a = 1, b = 0]$ запрещён в C/C++11 MM для этой программы, т.к. если $a = 1$, то между потоками произошла
синхронизация (в соответствующем графе есть ребро отношения $\lSW$),
и перед выполнением $\readInstParam{\rlx}{b}{x}$ правый поток должен быть осведомлен
о записи $\writeInstParam{\rlx}{x}{1}$.

Для того, чтобы представить такую синхронизацию, у каждого сообщения машины $\OpCpp$
есть четвертая дополнительная компонента --- \emph{фронт сообщения}.
Фронт сообщения $m$ хранит информацию о сообщениях, о которых станет осведомлен
поток, который выполнит приобретающее ($\acq$) чтение сообщения $m$.
Если поток $T$ выполняет высвобождающую ($\rel$) запись, то фронтом сообщения,
которое будет добавлено в память как результат исполнения записи, будет базовый
фронт потока $T$ на момент выполнения записи.
Расслабленные ($\rlx$) записи также помещают некоторый фронт в соответствующие сообщения,
но по более сложным правилам, которые будут описаны в разделе \ref{sec:opc11:fullmodel}.

Рассмотрим сценарий поведения ${\rm MP\text{-}rel\text{-}acq}$, в котором $a = 1$, т.е. происходит синхронизация.
После того, как выполнены две инициализирующие записи и запущены два потока, память и базовые фронты потоков
выглядят следующим образом:
\[
\begin{array}{l}
M = \{\angled{x:0@\tstamp{0}, [x@\tstamp{0}]}, \angled{y:0@\tstamp{0},[y@\tstamp{0}]} \} \\
T1.\Rcur = [x@\tstamp{0}, y@\tstamp{0}] \quad T2.\Rcur = [x@\tstamp{0}, y@\tstamp{0}]
\end{array}
\]
После исполнения двух записей (левым) потоком $T1$, в память попадает два новых сообщения, одно из которых
было сделано высвобождающей записью:
\[
\begin{array}{l}
M = \{
\angled{x:0@\tstamp{0}, [x@\tstamp{0}]}, \angled{y:0@\tstamp{0},[y@\tstamp{0}]}, \\
\qquad \angled{x:1@\tstamp{1}, [x@\tstamp{1}]}, \angled{y:1@\tstamp{1},\graybox{[x@\tstamp{1},y@\tstamp{1}]}}
 \} \\
T1.\Rcur = [x@\tstamp{1}, y@\tstamp{1}] \quad T2.\Rcur = [x@\tstamp{0}, y@\tstamp{0}]
\end{array}
\]
После того, как (правый) поток выполняет приобретающее чтение из сообщения 
$\angled{y:1@\tstamp{1},[x@\tstamp{1},y@\tstamp{1}]}$, его базовый фронт увеличивается по обоим компонентам.
\[
\begin{array}{l}
T2.\Rcur = [\graybox{x@\tstamp{1}}, y@\tstamp{1}]
\end{array}
\]
После этого (правый) поток $T2$ не может прочитать старое сообщение $\angled{x:0@\tstamp{0}, [x@\tstamp{0}]}$.
Таким образом модель $\OpCpp$ запрещает результат $[a = 1, b = 0]$ для программы ${\rm MP\text{-}rel\text{-}acq}$.

\subsection{Операционные буферы}
Тем не менее, не все слабые сценарии поведения программ, наблюдаемые в модели C/C++11, могут быть описаны
приведенными выше механизмами.
Одной из таких является программа ${\rm LB\text{-}rlx}$ (load buffering, буферизация записи):
\begin{equation*}
\tag{LB-rlx}\label{ex:LBrlx}
\begin{tabular}{c}
  $\writeInstParam{\rlx}{x}{0}; \writeInstParam{\rlx}{y}{0};$ \\
\begin{tabular}{L || L}
  \readInstParam{\rlx}{a}{x}; & \readInstParam{\rlx}{b}{y}; \\
  \writeInstParam{\rlx}{y}{1} & \writeInstParam{\rlx}{x}{1} \\
\end{tabular}
\end{tabular}
\end{equation*}
Модель памяти C/C++11 разрешает сценарий поведения этой программы с результатом $[a = 1, b = 1]$.
Такой результат требует, чтобы в момент исполнения инструкции $\readInstParam{\rlx}{a}{x}$
сообщени, соответствующая инструкции $\writeInstParam{\rlx}{x}{1}$, было уже в памяти,
и аналогично для пары инструкций
$\readInstParam{\rlx}{b}{y}$ и $\writeInstParam{\rlx}{y}{1}$.
Таким образом, хотя бы в одном из потоков инструкция записи должна быть исполнена раньше чтения.

Для решения этой проблемы модель $\OpCpp$ добавляет в состояние каждого потока по
\emph{операционному буферу}.
Операционный буфер --- это список записей об отложенных инструкциях, которые 
хранят всю необходимую информацию для дальнейшего их исполнения. 
Так, в частности, когда поток откладывает инструкцию чтения, он заменяет её в программе
на новое, уникальное символьное значение%
\footnote{Семантика $\OpCpp$ задана в стиле редукционных контекстов \cite{Felleisen-Hieb:TCS92,Felleisen-al:BOOK09},
т.е. программа в ней представляется как выражение, которое постепенно редуцируется.
Как следствие, инструкция чтения в этой семантике --- это некоторое подвыражение, которое, будучи вычисленным,
заменяется на прочитанное значение.
Подробное описание семантики в виде редукционных контекстов приведено в разделе \ref{sec:opc11:formal}.
}, а в буфер добавляет пару, состоящую из символьного значения
и целевой локации.
В то время как для отложенной инструкции записи в буфер сохраняется целевая локация и значение,
которое нужно записать.
Далее поток машины $\OpCpp$ может недетерминировано выбрать отложенную инструкцию из буфера и,
если в буфере перед выбранной инструкцией нет инструкции, которая может непосредственно повлиять на результат
выбранной, исполнить её.
Так, данный механизм позволяет в программе ${\rm LB\text{-}rlx}$ отложить исполнение инструкции чтения в левом потоке,
что даёт возможность получить результат $[a = 1, b = 1]$.

\subsection{Спекулятивное исполнение}
Оригинальная модель C/C++11 поддерживает оптимизацию, при которой инструкция, которая
встречается в обоих ветках условного оператора, может быть вынесена за пределы этого
оператора.
Так, в следующей программе инструкция $\writeInstParam{\rlx}{y}{1}$
семантически не зависит от условия $\kw{if}$-оператора, а значит может быть
выполнена перед инструкцией $\readInstParam{\rlx}{a}{x}$.
Это позволяет получить $c = 1$ после выполнения программы.
\begin{equation*}
\tag{SE-simple}
\begin{tabular}{c}
  $\writeInstParam{\rlx}{x}{0}; \writeInstParam{\rlx}{y}{0}; \writeInstParam{\rlx}{z}{0};$ \\
\begin{tabular}{L || L}
  \begin{array}{@{}l@{}}
    \readInstParam{\rlx}{a}{x}; \\
    \iteml{a}{
      \writeInstParam{\rlx}{z}{1}; \\
      \writeInstParam{\rlx}{y}{1}
    }
    {\writeInstParam{\rlx}{y}{1}}
  \end{array} &
  \begin{array}{@{}l@{}}
    \readInstParam{\rlx}{b}{y}; \\
    \iteml{b}{
      \writeInstParam{\rlx}{x}{1}
    }
    {\skipc}
  \end{array}
\end{tabular} \\
  $\readInstParam{\rlx}{c}{z}$
\end{tabular}
\end{equation*}

Для того, чтобы поддержать такие сценарии поведения в $\OpCpp$, операционные буферы могут быть \emph{вложенными}.
Когда исполнение потока подходит к условному оператору, условие которого зависит от ранее отложенной
операции, семантика добавляет в буфер кортеж, который содержит символическое представление условия
и два пустых подбуфера. Эти подбуферы в дальнейшем будут пополняться инструкциями $\kw{then}$ и $\kw{else}$
веток оператора.

Рассмотрим сценарий поведения программы ${\rm SE-simple}$, в котором $c = 1$.
После исполнения инициализирующих инструкций записи память машины $\OpCpp$ содержит три сообщения:
\[
M = \{\angled{x:0@\tstamp{0}, [x@\tstamp{0}]}, \angled{y:0@\tstamp{0},[y@\tstamp{0}]},
      \angled{z:0@\tstamp{0},[z@\tstamp{0}]}\}.
\]
Далее левый поток откладывает выполнение инструкции чтения и начинает спекулятивно исполнять условный оператор.
Так, в буфере левого потока оказывается две записи:
\[\angled{\readInstParam{\rlx}{a}{x}; \kw{if} \; a \; \angled{} \; \angled{}}.\]
Продолжая (спекулятивно) откладывать инструкции, левый поток помешает все инструкции из
условного оператора в соответствующие подбуферы:
\[\angled{\readInstParam{\rlx}{a}{x};
  \kw{if} \; a \; \angled{\writeInstParam{\rlx}{z}{1}; \writeInstParam{\rlx}{y}{1}} \;
  \angled{\writeInstParam{\rlx}{y}{1}}}.\]
После того, как в подбуферах оказываются одинаковые инструкции
(в данном случае, $\writeInstParam{\rlx}{y}{1}$), и перед ними в подбуферах нет
конфликтующих инструкций, модель $\OpCpp$ может вынести их на предыдущей уровень буфера:
\[\angled{\readInstParam{\rlx}{a}{x}; \writeInstParam{\rlx}{y}{1};
  \kw{if} \; a \; \angled{\writeInstParam{\rlx}{z}{1}} \;
  \angled{} }.\]
Далее, поскольку $\readInstParam{\rlx}{a}{x}$ и $\writeInstParam{\rlx}{y}{1}$
независимы, может быть исполнена запись в $y$, после чего в памяти появляется новое сообщение:
\[
\begin{array}{@{}l@{}}
M = \{\angled{x:0@\tstamp{0}, [x@\tstamp{0}]}, \angled{y:0@\tstamp{0},[y@\tstamp{0}]},
      \angled{z:0@\tstamp{0},[z@\tstamp{0}]}, \graybox{\angled{y:1@\tstamp{1},[y@\tstamp{1}]}}\}.
\end{array}
\]
После этого результат $c = 1$ получается следующим образом.
Правый поток читает из нового сообщения, присваивая $1$ в $b$.
Поскольку в правом потоке условие вычисляется к $1$, происходит исполнение записи в $x$:
\[
\begin{array}{@{}l@{}}
M = \{\angled{x:0@\tstamp{0}, [x@\tstamp{0}]}, \angled{y:0@\tstamp{0},[y@\tstamp{0}]}, \angled{z:0@\tstamp{0},[z@\tstamp{0}]}, \\
\qquad \angled{y:1@\tstamp{1},[y@\tstamp{1}]}, \graybox{\angled{x:1@\tstamp{1},[x@\tstamp{1}]}}\}.
\end{array}
\]
После этого, левый поток может выполнить отложенное чтение $\readInstParam{\rlx}{a}{x}$ из добавленного сообщения.
Так, в буфере левого потока остаётся только одна запись, соответствующая отложенной конструкции $\kw{if}$, где
условие было вычислено к $1$:
\[\angled{
  \kw{if} \; 1 \; \angled{\writeInstParam{\rlx}{z}{1}} \;
  \angled{}}.\]
После вычисления отложенной конструкции $\kw{if}$ и выполнения записи в $z$ в памяти находится шесть сообщений:
\[
\begin{array}{@{}l@{}}
M = \{\angled{x:0@\tstamp{0}, [x@\tstamp{0}]}, \angled{y:0@\tstamp{0},[y@\tstamp{0}]}, \angled{z:0@\tstamp{0},[z@\tstamp{0}]}, \\
\qquad \angled{y:1@\tstamp{1},[y@\tstamp{1}]}, \angled{x:1@\tstamp{1},[x@\tstamp{1}]}, \graybox{\angled{z:1@\tstamp{1},[z@\tstamp{1}]}}\}.
\end{array}
\]
Чтение из последней записи в локацию $z$ приводит к результату $c = 1$.

Идея перевода повторяющихся инструкций из вложенных буферов естественным образом обобщается на
случай вложенных условных операторов.

\section{Продвинутые детали модели}
\label{sec:opc11:fullmodel}
В это разделе описывается то, как с помощью фронтов представляется поведение
$\sco$- и $\con$-инструкции, и ищутся гонки по данным на неатомарных обращениях к памяти.
Кроме того, рассматривается проблема соединения потоков (thread's joining) в контексте
оптимизации по секвенциализации потоков (thread's sequentialization).

\subsection{$\sco$-инструкции}
Оригинальная модель C/C++11 гарантирует наличие такого тотального порядка, $\lSC$, на $\sco$-событиях,
который не противоречит программному порядку, $\lPO$, отношению ``синхронизируется с'', $\lSW$,
и порядка памяти, $\lMO$.
Помимо этого, $\sco$-чтения обладает теми же свойствами, как и приобретающие чтения,
а $\sco$-записи --- как высвобождающие записи.

Рассмотрим программу ${\rm SB-sc}$ (store buffering, буферизация записи).
\begin{equation*}
\tag{SB-sc}
\begin{tabular}{c}
  $\writeInstParam{\sco}{x}{0}; \writeInstParam{\sco}{y}{0};$ \\
\begin{tabular}{L || L}
  \writeInstParam{\sco}{x}{1}; & \writeInstParam{\sco}{y}{1} \\
 \readInstParam{\sco}{a}{y};   & \readInstParam{\sco}{b}{x}; \\
\end{tabular}
\end{tabular}
\end{equation*}
Модель C/C++11 запрещает результат $[a = 0, b = 0]$ для неё.
Это объясняется тем, что как минимум одно из чтений находится позже все остальных событий в отношении $\lSC$,
а это значит, что соответствующий поток в момент исполнения этого чтения осведомлён о всех ($\sco$-)записях в программе.

Для того, чтобы гарантировать аналогичное ограничение, в состоянии машины $\OpCpp$ добавляется глобальный
(общий для всех потоков) компонент --- \emph{$\sco$-фронт}, $\Rsc$.
Этот фронт используется следующим образом: при выполнении $\sco$-чтения из локации $\loc$
поток обновляет свой базовый фронт на $[\loc @ \Rsc(\loc)]$, а при выполнении $\sco$-записи
в локацию $\loc$ сообщения с меткой времени $\tau$ поток обновляет $\sco$-фронт на $[\loc@\tau]$.
Таким образом, $\sco$-чтение из локации $\loc$ не может прочитать сообщение в памяти, которое имеет
меньшую метку времени, чем метка сообщения в ту же локацию, которое было записано последней на тот момент $\sco$-записью.

\subsection{Неатомарные обращения}
Согласно модели C/C++11, программы с гонками по данным, в которых участвуют неатомарные обращения к памяти,
обладают неопределенном поведением. Так, рассмотрим следующую программу.
\begin{equation*}
\tag{DR-rlx-na}
\begin{tabular}{c}
  $\writeInstParam{\na}{d}{0};$ \\
\begin{tabular}{L || L}
 \writeInstParam{\rlx}{d}{1};   & \readInstParam{\na}{a}{d}; \\
\end{tabular}
\end{tabular}
\end{equation*}
В ней есть гонка по данным между инструкциями $\writeInstParam{\rlx}{d}{1}$ и $\readInstParam{\na}{a}{d}$.
Мы можем идентифицировать эту гонку с помощью базовых фронтов в случае сценария исполнения, в котором
сначала выполняется запись левым потоком, а потом --- чтение правым.
В такой ситуации правый поток будет выполнять неатомарную операцию над локацией $d$, при этом не
являясь осведомленным о последний записи в данную локацию.
Если модель $\OpCpp$ идентифицировала гонку по данным, в которую вовлечена неатомарная операция,
хотя бы в одном сценарии поведения программы, считается, что программа в целом обладает неопределенным
поведением.

Рассмотрим похожую программу, также с гонкой по данным.
\begin{equation*}
\tag{DR-na-rlx}
\begin{tabular}{c}
  $\writeInstParam{\na}{d}{0};$ \\
\begin{tabular}{L || L}
 \writeInstParam{\na}{d}{1};  & \readInstParam{\rlx}{a}{d}; \\
\end{tabular}
\end{tabular}
\end{equation*}
Идентификация гонки в данном случае не может быть проведена только с помощью базовых фронтов потоков.
Для решения проблемы мы вводим в состояние машины $\OpCpp$ ещё один глобальный фронт --- \emph{$\na$-фронт}, $\Rna$.
Подобно $\sco$-фронту, $\na$-фронт по локации возвращает метку времени последней $\na$-записи в неё.
Так, при выполнении любой операции над локацией $\loc$ поток проверяет, что он осведомлён о последней
$\na$-записи в $\loc$ (т.е. его базовый фронт по $\loc$ больше или равен $\Rna(\loc)$), и если нет, то
машина $\OpCpp$ сигнализирует о гонке.

С помощью такой техники гонка по данным в программе ${\rm DR\text{-}na\text{-}rlx}$ идентифицируется в сценарии
поведения, в котором $\na$-запись левого потока выполняется до $\rlx$-чтения правого потока.

\subsection{Потребляющие чтения}
Потребляющие (consume, $\con$) чтения являются более слабой версией приобретающих чтений.
Так, приобретающее чтение обновляет базовый фронт потока с помощью фронта прочитанного сообщения
и тем самым влияет на все последующие инструкции, тогда как потребляющее чтение действует только
на последующие чтения, которые находятся в зависимости по адресу%
\footnote{Инструкция чтения $i$ \emph{зависит по адресу} от инструкции чтения $j$, если
целевой адрес $i$ зависит от результата исполнения $j$.}
от него.

Рассмотрим программу, в которой используется потребляющее чтение.
\begin{equation*}
\tag{MP-con-na-2}
\begin{tabular}{c}
  $\writeInstParam{\na}{p}{\nullPtr}; \writeInstParam{\na}{d}{0}; \writeInstParam{\na}{x}{0};$ \\
\begin{tabular}{L || L}
  \begin{array}{@{}l@{}}
    \writeInstParam{\rlx}{x}{1}; \\
    \writeInstParam{\na}{d}{1}; \\
    \writeInstParam{\rel}{p}{d} \\
  \end{array}
  &
  \begin{array}{@{}l@{}}
    \readInstParam{\con}{a}{p}; \\
    \iteml{a \neq \nullPtr}
          {\readInstParam{\na}{b}{a}; \\
           \readInstParam{\rlx}{c}{x}}
          {\assignInst{b}{0}; \\ \assignInst{c}{0}}
  \end{array}
\end{tabular}
\end{tabular}
\end{equation*}
В этой программе левый поток передаёт информацию, записанную в локацию $d$, правому потоку через указатель $p$.
Правый поток с помощью $\con$-чтения получает содержимое указателя $p$ в переменную $a$.
Если переменная $a$ не $\nullPtr$, то поток далее читает из той локации, на которую указывает $a$ (в данном случае на локацию $d$),
а потом --- из локации $x$.
У этой программы существует три возможных результата исполнения в модели C/C++11:
$[a = \nullPtr, b = 0, c = 0]$,
$[a = d, b = 1, c = 1]$ и
$[a = d, b = 1, c = 0]$.
Если в программе $\con$-чтение заменить на $\acq$-чтение, то последний результат станет невозможным,
т.к. после выполнения $\readInstParam{\acq}{a}{p}$ с результатом $d$, базовый фронт правого потока будет
указывать на сообщения, которые были получены в результате исполнения $\writeInstParam{\rlx}{x}{1}$ и
$\writeInstParam{\na}{d}{1}$.
В то же время $\con$-чтение $\readInstParam{\con}{a}{p}$ ``синхронизирует'' только последующее чтение $\readInstParam{\na}{b}{a}$,
т.к. оно является разыменованием результата $\con$-чтения, и не влияет на независимое чтение $\readInstParam{\rlx}{c}{x}$.

Для поддержания подобного поведения модель $\OpCpp$ может \emph{аннотировать} с помощью фронтов инструкции чтения, которые зависят
от $\con$-чтений. Когда потребляющее чтение получает из памяти сообщение $\angled{\loc:\val@\tau,\R}$,
оно, вместо того, чтобы обновить базовый фронт потока с помощью $\R$, как это сделало бы приобретающее чтение,
помечает все зависимые от него инструкции чтения фронтом $\R$.
В дальнейшем, когда помеченная инструкция чтения будет исполняться, она скомбинирует базовый фронт потока на тот момент
с фронтом-пометкой для того, чтобы вычислить минимальную метку времени, сообщение с которой доступно для чтения.
Аналогичным образом помечаются и обрабатываются отложенные чтения в операционных буферах, которые ссылаются
на символьный результат $\con$-чтения.

\subsection{Соединение потоков}
\label{sec:opc11:join}
В момент соединения потоков, т.е. когда оба потока закончили своё исполнение,
естественно ожидать, что все отложенные операции были выполнены, и, соответственно,
операционные буферы пусты.
Это подтверждается тем, что в оригинальной модели C/C++11 \cite{Batty-al:POPL11}
используется отношение ``дополнительно синхронизируется с'' (additional-synchronizes-with, $\lASW$),
которое связывает последние событие потока с первой следующей за потоком инструкцией родительского потока.
Отношение $\lASW$ является частью отношения $\lHB$. Это означает, что родительский поток
становится осведомлен о всех тех сообщениях, которые были прочитаны дочерними потоками.
В модели $\OpCpp$ это выражается тем, что соединение потоков возможно, только если их операционные буферы
пусты, а после соединения потоков родительский поток получает базовый фронт,
равный комбинации базовых фронтов соединяемых потоков.

Тем не менее, это противоречит требованию, исходящему от стандартов C и C++11 --- 
оптимизация секвенциализации ($C_1\;||\;C_2 \optarrow C_1;\;C_2$) должна быть корректной.
Это делает предыдущие предположения некорректными.
Данная проблема может быть проиллюстрирована на следующей программе,
в которой параллельная композиция с пустым потоком может быть заменена на
не пустой поток.
\begin{equation*}
\tag{LB-rlx-join}
\begin{tabular}{c}
  $\writeInstParam{\rlx}{x}{0}; \writeInstParam{\rlx}{y}{0};$ \\
\begin{tabular}{L || L || L || L}
  \begin{array}{@{}l@{}}
    \readInstParam{\rlx}{a}{y} \\
  \end{array}
  &
\skipc
  &
  \begin{array}{@{}l@{}}
    \readInstParam{\rlx}{b}{x} \\
  \end{array}
  &
\skipc \\
\multicolumn{2}{c ||}{$\writeInstParam{\rlx}{x}{1}$} &
\multicolumn{2}{c}{$\writeInstParam{\rlx}{y}{1}$}
\end{tabular}
\end{tabular}
\end{equation*}
Если применить такую оптимизацию к одной из композиций потоков,
то результат $[a = 1, b = 1]$ станет возможным.
Для того, чтобы поддержать данный сценарий поведения,
мы допускаем альтернативное правило соединения потоков как отдельный аспект в интерпретаторе модели.

\subsection{Расслабленные обращения и синхронизация}
Между расслабленными операциями, высвобождающими записями и
приобретающими чтениями существует тонкое взаимодействие.
В этом подразделе мы опишем варианты этого взаимодействия, а также
принятые в связи с ним технические решения в устройстве модели $\OpCpp$.

\subsubsection{Высвобождающие цепочки}
В связи с тем, что расслабленное чтение не может участвовать в синхронизации,
его исполнение не обновляет базовый фронт потока фронтом прочитанного сообщения,
как это происходит в случае приобретающего чтения.
Тем не менее, если приобретающее чтение потока $T_2$ читает из сообщения, которое
было добавлено расслабленной записью $w_{\rlx}$ потока $T_1$, то оно должно синхронизироваться
с высвобождающей записью в ту же локацию $w_{\rel}$, предшествующей $w_{\rlx}$ в потоке $T_1$,
если такая запись существует.
Это свойство является аналогом высвобождающих цепочек (release sequences) из аксиоматической
модели C/C++11 \cite{Batty-al:POPL11}.

Рассмотрим вариант программы ${\rm MP}$, в сценариях поведения которого имеется описанная выше синхронизация.
\begin{equation*}
\tag{MP-rel-acq-na-rlx-2}
\begin{tabular}{c}
  $\writeInstParam{\na}{f}{0}; \writeInstParam{\na}{d}{0}; \writeInstParam{\na}{x}{0};$ \\
\begin{tabular}{L || L}
  \begin{array}{@{}l@{}}
    \writeInstParam{\na}{d}{5}; \\
    \writeInstParam{\rel}{f}{1}; \\
    \writeInstParam{\rel}{x}{1}; \\
    \writeInstParam{\rlx}{f}{2} \\
  \end{array}
  &
  \begin{array}{@{}l@{}}
    \repeatInst{\readInstParam{\acq}{c}{f}; c == 2}; \\
    \readInstParam{\na}{a}{d}; \\
    \readInstParam{\rlx}{b}{x} \\
  \end{array}
\end{tabular}
\end{tabular}
\end{equation*}
Единственным значением, которое может получить регистр $a$, является $5$.
Это объясняется тем, что 
последнее приобретающее чтение $\readInstParam{\acq}{c}{f}$ в цикле должно прочитать значение $2$,
a, значит, синхронизироваться с высвобождающей записью $\writeInstParam{\rel}{f}{1}$.
В то же время регистр $b$ может получить как значение $0$, так и $1$, поскольку
$\writeInstParam{\rel}{f}{1}$, с которым синхронизируется приобретающее чтение, предшествует
записи $\writeInstParam{\rel}{x}{1}$, и, как следствие, информация об этой записи $x$ не попадает
во фронт сообщения, записанного $\writeInstParam{\rel}{f}{1}$.

Для того, чтобы выразить такую синхронизацию, у каждого потока в машине $\OpCpp$ имеется
\emph{фронт записи} (write-front), $\Rwrite$.
Этот фронт для каждой локации возвращает метку времени последней высвобождающей записи,
сделанной потоком в эту локацию.
Когда поток выполняет расслабленную запись в локацию $\loc$ с меткой времени $\tau$,
в соответствующее сообщение записывается фронт, который является комбинацией $[\loc@\tau]$
и фронта сообщения в $\loc$ с меткой времени $\Rwrite(\tau)$, которое было записано
высвобождающей записью того же потока.

\subsubsection{Отложенные расслабленные обращения и синхронизация}
Рассмотрим вариант программы ${\rm LB}$, в которой чтения являются расслабленными, а записи --- высвобождающими.
\begin{equation*}
\tag{LB-rel-rlx}
\begin{tabular}{c}
  $\writeInstParam{\rlx}{x}{0}; \writeInstParam{\rlx}{y}{0};$ \\
\begin{tabular}{L || L}
  \readInstParam{\rlx}{a}{x}; & \readInstParam{\rlx}{b}{y}; \\
  \writeInstParam{\rel}{y}{1} & \writeInstParam{\rel}{x}{1} \\
\end{tabular}
\end{tabular}
\end{equation*}
Согласно оригинальной модели C/C++11 эта программа имеет сценарий поведения с результатом $[a = 1, b = 1]$,
поскольку высвобождающие записи не накладывают никаких дополнительных ограничений в отсутствии
приобретающих чтений.
Поэтому модель $\OpCpp$ разрешает исполнять высвобождающие записи даже в тот момент, когда
предшествующие отложенные чтения ещё не выполнены.

Тем не менее, требуются ввести некоторые ограничения на откладывание чтений за высвобождающие записи.
Рассмотрим ещё один вариант ${\rm LB}$.
\begin{equation*}
\tag{LB-rel-acq-rlx}
\begin{tabular}{c}
  $\writeInstParam{\rlx}{x}{0}; \writeInstParam{\rlx}{y}{0};$ \\
\begin{tabular}{L || L}
  \readInstParam{\acq}{a}{x}; & \readInstParam{\rlx}{b}{y}; \\
  \writeInstParam{\rlx}{y}{1} & \writeInstParam{\rel}{x}{1} \\
\end{tabular}
\end{tabular}
\end{equation*}
В этой программе в левом потоке есть приобретающее чтение, а в правом --- высвобождающая запись.
Если это чтение прочитает сообщение, сделанное инструкцией $\writeInstParam{\rel}{x}{1}$,
т.е. регистр $a$ получит значение $1$, то, в терминах оригинальной модели, между
$\readInstParam{\rlx}{b}{y}$ и $\writeInstParam{\rlx}{y}{1}$ возникнет ребро отношения ``предшествует'', $\lHB$,
Как следствие, регистр $b$ не может получить значение $1$, т.к. чтение не может прочитать то, что записано
последующей за ним записью.
Таким образом, модель $\OpCpp$ должна предотвращять ситуацию, когда чтение, которое было отложено через
высвобождающую запись $W$, выполняется после того, как другой поток прочитал сообщение $W$ с помощью приобретающего
чтения.

Для реализации данного ограничения мы добавляем глобальный компонент $\stGamma$ в состояние машины $\OpCpp$,
который является списком троек, которые состоят из локации, метки времени некоторого существующего сообщения
и символического значения, идентифицирующего отложенное чтение.
При выполнении высвобождающей записи $W$, которая добавляет сообщение локации $\loc$ с меткой времени $\tau$,
для каждого отложенного потоком чтения в список $\stGamma$ добавляется по тройке $\angled{\loc,\tau, \vName}$, где
$\vName$ --- символическим значение чтения.
При этом приобретающее чтение из сообщения локации $\loc$ с меткой времени $\tau$ возможно, только
если в $\stGamma$ не осталось записей вида $\angled{\loc, \tau, \_}$,
а выполнение отложенного чтения с символическим значением $\vName$ удаляет из $\stGamma$ все тройки
$\angled{\_,\_,\vName}$.

Следующая программа показывает ещё один тонкий момент, на этот раз связанный с откладыванием
расслабленных записей через высвобождающие.
\begin{equation*}
\tag{WR-rlx-rel}
\begin{tabular}{c}
  $\writeInstParam{\rlx}{x}{0}; \writeInstParam{\rlx}{y}{0};$ \\
\begin{tabular}{L || L}
  \writeInstParam{\rlx}{x}{1}; & \writeInstParam{\rlx}{y}{1}; \\
  \writeInstParam{\rel}{y}{1} & \writeInstParam{\rel}{x}{1} \\
\end{tabular}\\
  $\readInstParam{\rlx}{a}{x}; \readInstParam{\rlx}{b}{y}$ \\
\end{tabular}
\end{equation*}
Согласно оригинальной модели C/C++11 результат $[a = 1, b = 1]$ возможен для этой
программы, что требует, чтобы сообщения расслабленных записей попали в память
после сообщений высвобождающих.
При этом, если другой (третий) поток выполнит приобретающее чтение из сообщения,
сделанного одной из высвобождающих записей, то он должен стать осведомлённым о
предшествующей расслабленной записи, а для этого соответствующее сообщение должно
находиться в памяти. 

Аналогично предыдущему пункту, мы решаем данную проблему с помощью списка $\stGamma$.
При исполнении высвобождающей записи также, как и для отложенных чтений, в списке $\stGamma$
появляются тройки, связанные с отложенными записями.
Единственным отличием от обработки отложенных чтений является то, что при выполнении отложенной
записи $W$ мы не только удаляем все тройки $\angled{\loc, \tau, \vName}$, где $\vName$ --- символьный
идентификатор записи, но и увеличиваем фронт сообщения локации $\loc$ с меткой времени $\tau$ на
$[\loc'@\tau']$, где $\loc'$ и $\tau'$ связаны с сообщением $W$.

\section{Формальное определение модели}
\label{sec:opc11:formal}
В разделе приводится математическое определение операционной модели C/C++11,
начиная с определения синтаксиса языка, для которого задана модель,
представления памяти и фронтов.

\subsection{Синтаксис языка модели и базовые правила редукции}
Синтаксис языка модели представлен на рис. \ref{fig:opc11:syntax}.
Мета-переменная $\Expr$ представляет выражения, которые могут быть
целыми числами $z$, локациями $\loc$, неизменяемыми локальными
переменными $\vName$, парами, проекциями пар или бинарными выражениями.
Конструкция $\kw{choice}$ является недетерминированным выбором между двумя
выражениями.

\begin{figure}%[t]\small
\[\begin{array}{rcl}
   %% \multicolumn{3}{c}{\text{\textbf{Statements and Expressions}}} 
   %% \\[3pt] 
\Expr   & ::= & \vName \mid z~(\in \Number) \mid \Expr_{1}~\op~\Expr_{2}
                \mid \Choice~\Expr_{1}~\Expr_{2} \\
        &     & \First{\Expr} \mid \Second{\Expr} \mid \Pair{\Expr_{1}}{\Expr_{2}} \mid \locVar\\
\op     & ::= & + \mid - \mid * \mid / \mid \% \mid \texttt{==} \mid \neq \\
\locVar & ::= & \loc \mid \vName \\
\loc    & \text{---} & \text{идентификатор локации} \\
\vName  & \text{---} & \text{локальная переменная} \\ 
\mvalSubst & ::= & \loc \mid z \mid \Pair{\mvalSubst_{1}}{\mvalSubst_{2}} \\  
\\
\AST & ::=  & \Ret{\Expr} \mid\Bind{\vName}{\AST_{1}}{\AST_{2}} \mid
              \Spw{\AST_{1}}{\AST_{2}}  \mid\\
     &      & \itesl{\Expr}{\AST_{1}}{\AST_{2}} \mid
              \repeatInst{\AST} \mid \\
     &      & \readExpr{\RM}{\locVar} \mid
              \writeInstParam{\WM}{\locVar}{\Expr} 
              \mid \Cas{SM}{FM}{\locVar}{\Expr_{1}}{\Expr_{2}} \\
\RT & ::=  & \Stuck \mid \Par{\AST_{1}}{\AST_{2}}\\
% \\
% \auxX   & \text{---} & \text{a generic state (to be defined)}\\ 
              \\
\RM   & ::= & \sco \mid \acq \mid \con \mid \rlx \mid \na \\
\WM   & ::= & \sco \mid \rel \mid \rlx \mid \na \\
\SM   & ::= & \sco \mid \relAcq \mid \rel \mid \acq \mid \con \mid \rlx \\
\FM   & ::= & \sco \mid \acq \mid \con \mid \rlx
\end{array}\]
\caption{Синтаксис операций и выражений языка модели $\OpCpp$}
\label{fig:opc11:syntax}
\end{figure}

Программы представляются как операторы $\AST$, где $\Bind{\vName}{\AST_{1}}{\AST_{2}}$ --- это let-выражение,
$\Spw{\AST_{1}}{\AST_{2}}$ --- параллельная композиция потоков,
$\readExpr{\RM}{\locVar}$ --- выражение, читающее из памяти.
В наших примерах мы также используем $\skipc$ как альтернативное представление константного выражения $0$;
$\readInstParam{\RM}{\vName}{\locVar}$ и $\assignInst{\vName}{\Expr}$ для обозначения
$\Bind{\vName}{\readExpr{\RM}{\locVar}}{\Ret{\vName}}$ и $\Bind{\vName}{\Expr}{\Ret{\vName}}$;
$\AST_{1}; \AST_{2}$ вместо $\Bind{\vName'}{\AST_{1}}{\AST_{2}}$, где $\vName'$ не встречается в $\AST_{2}$;
а также $\AST_{1} {\sf ||} \AST_{2}$ для обозначения $\Spw{\AST_{1}}{\AST_{2}}$.

Результатом полностью вычисленной программы является либо значение $\mvalSubst$,
либо значение времени исполнения (run-time value) $\Stuck$, которое символизирует, что-то пошло не так ---
в исполнении найдена гонка по данным, или была попытка чтения из неинициализированной переменной.
Другой конструкцией времени исполнения является $\Par{\AST_{1}}{\AST_{2}}$, которая получается
в результате редукции $\Spw{\AST_{1}}{\AST_{2}}$ и представляет находящиеся в исполнении потоки.
На шаге операционной семантики, который редуцирует $\Spw{\AST_{1}}{\AST_{2}}$ в $\Par{\AST_{1}}{\AST_{2}}$,
происходит необходимая инициализация компонент машины $\OpCpp$.

Мета-переменная $\auxX$ представляет динамическое состояние машины с точностью до программы; оно описано ниже.
Вычисление программы $\AST$ в машине $\OpCpp$ начинается со стартового состояния $\angled{\AST, \auxX_{init}}$,
где $\auxX_{init}$ содержит пустую память и пустой базовый фронт для единственного стартового потока.
Переходы машины заданы в редукционном стиле \cite{Felleisen-Hieb:TCS92}, и большинство из них имеют
следующую форму:
\begin{mathpar}
\inferrule{
  \dots
}
{{\tup{\EvalContext[\AST], \auxX}}  \astep{}
 {\tup{\EvalContext[\AST'], \auxX'}}}
\end{mathpar}
где $\EvalContext$ --- это \emph{редукционный контекст}, заданный так: 
%
{
\[\begin{array}{rcl}
   %% \multicolumn{3}{c}{\text{\textbf{Evaluation contexts}}} 
   %% \\[3pt]
\EvalContext   & ::= & \hole
                         \mid \Bind{\vName}{\EvalContext}{\AST}  \mid \Par{\EvalContext}{\AST}
                       \mid \Par{\AST}{\EvalContext}. \\  
%% \EvalEUContext & ::= & \hole \mid (\EvalEUContext) \mid
%%                          \EvalEUContext~\op~\Expr \mid
%%                          \Expr~\op~\EvalEUContext \\
%%                &     &   \mid \Pair{\mval}{\EvalEUContext}
%%                          \mid \Pair{\EvalEUContext}{\mval} \\ 
%%                &     &   \mid \First{\EvalEUContext}
%%                          \mid \Second{\EvalEUContext} \\ 
%%                &     &   \mid \Choice{~\EvalEUContext}{\Expr} \mid
%%                          \Choice{~\Expr}{\EvalEUContext} \\
%%                &     &   \mid \Bind{\vName}{\EvalEUContext}{\AST} \\
%%                &     &   \mid \IfThenElse{\EvalEUContext}{\AST_{1}}{\AST_{2}} \\
%%                &     &   \mid \Write{\WM}{\locVar}{\EvalEUContext} \\
%%                &     &   \mid \Cas{SM}{FM}{\locVar}{\EvalEUContext}{\mval}
%%                          \mid \Cas{SM}{FM}{\locVar}{\mval}{\EvalEUContext} \\
\end{array}\]
}
Если в момент исполнения программы запущено более одного потока, т.е. в программном выражении
присутствует узел $\kw{par}$, то программное выражение может быть разбито на пару контекст и подпрограмма
недетерминированно несколькими способами.

\begin{figure*}
\begin{mathpar}
\inferrule[Subst]{\quad
}
{
\angled{\EvalContext[\Bind{\vName}{\Ret{\mvalSubst}}{\AST}], \auxX} 
 \astep{}
\angled{\EvalContext[\AST\subst{\vName}{\mvalSubst}], \auxX}
} \\

\inferrule[If-False]{\quad}
{
 \angled{\EvalContext[\itesl{0}{\AST_{1}}{\AST_{2}}], \auxX}
 \astep{}
 \angled{\EvalContext[\AST_{2}]], \auxX}
} \\
\inferrule[If-True]{n \neq 0}
{
 \angled{\EvalContext[\itesl{n}{\AST_{1}}{\AST_{2}}], \auxX}
 \astep{}
 \angled{\EvalContext[\AST_{1}]], \auxX}
} \\

\inferrule[Repeat-Unroll]{
  \vName \text{ -- новая переменная}
}
{
 \angled{\EvalContext[\Repeat{\AST}], \auxX}
 \astep{} \\
 \quad \angled{\EvalContext[\Bind{\vName}{\AST}
                 {\IfThenElse{\vName}{\Ret{\vName}}
                              {\Repeat{\AST}}}],
            \auxX}
} \\

\inferrule[Spawn]{
\auxX' = \spawn{\EvalContext}{\auxX}}
{
\angled{\EvalContext[\Spw{\AST_{1}}{\AST_{2}}], \auxX}
 \astep{}
\angled{\EvalContext[\Par{\AST_{1}}{\AST_{2}}], \auxX'}
} \and

\inferrule[Join]{
\auxX' = \joinP{\EvalContext}{\auxX}
}{
\angled{\EvalContext[\Par{\Ret{\mvalSubst_{1}}}{\Ret{\mvalSubst_{2}}}], \auxX}
 \astep{}
\angled{\EvalContext[\Ret{\Pair{\mvalSubst_{1}}{\mvalSubst_{2}}}], \auxX'}
} \\

\inferrule[Choice-Fst]{\quad
}{
  \angled{\EvalContext[\EvalEUContext[\Choice{~\Expr_{1}}{\Expr_{2}}]], \auxX} \astep{} %\\
  \angled{\EvalContext[\EvalEUContext[\Expr_{1}]], \auxX}
} \\

\inferrule[Choice-Snd]{\quad
}{
  \angled{\EvalContext[\EvalEUContext[\Choice{~\Expr_{1}}{\Expr_{2}}]], \auxX} \astep{} %\\
  \angled{\EvalContext[\EvalEUContext[\Expr_{2}]], \auxX}
} \\
\end{mathpar}
\caption{Базовые правила модели $\OpCpp$}
\label{fig:opc11:baseSem}
\end{figure*}
Базовые правила семантики, которые не затрагивают операций над памятью, приведены на рисунке \ref{fig:opc11:baseSem}.
Из них интерес представляют правила ${\rm Spawn}$ и ${\rm Join}$, которые описывают старт и соединение дочерних
потоков соответственно. Эти правила модифицируют компоненты состояния машины $\OpCpp$, которые являются локальными для
потоков, например, базовый фронт и операционный буфер.
Конкретные представления правил ${\rm Spawn}$ и ${\rm Join}$ зависят от многопоточных аспектов модели,
которые определяют мета-функции ${\sf spawn}$ и ${\sf join}$.

\subsection{Представление памяти и фронтов}
В самом базовом представлении состояние $\auxX$ машины $\OpCpp$ включает в себя память $M$ и функцию
$\stPsiRead$, которая по идентификатору потока $\stpath$ возвращает его базовый фронт
(см. рис. \ref{fig:auxXrelAcq}).

\begin{figure}
\[\begin{array}{l r c l}
\text{Состояние} & \auxX      & ::= & \angled{\stEta, \stPsiRead} \\
\text{Память} &\stEta     & ::= & (\loc, \stTau) \prarrow \angled{\mvalSubst, \stSigma}\\ 
%% \stEtaLoc  & ::= & \stTau \prarrow \angled{\mvalSubst, \stSigma}\\
\text{Функция базового фронта} & \stPsiRead & ::= & \stpath \prarrow \stSigma\\
\text{Фронт} & \stSigma   & ::= & \loc \prarrow \stTau\\
\text{Идентификатор потока} & \stpath    & \text{---} & (l|r)^{*}\\
\text{Метка времени} & \stTau \in \mathbb{N}   & \text{---} & \text{timestamp}
\end{array}\]
\caption{Базовое состояние машины $\OpCpp$}
\label{fig:auxXrelAcq}
\end{figure}

Память $\stEta$ это частичная функция, которая по локации и метке времени возвращает
пару сохраненное значение и синхронизационный фронт.
Разные аспекты модели используют фронты по-разному, но каждый фронт является частичной
функцией, которая по локации возвращает некоторую метку времени.
Функция базового фронта возвращает базовый фронт потока по его идентификатору $\stpath$,
где идентификатор потока --- это список направлений лево (l) / право (r), который показывает,
как найти подвыражение потока во всем выражении программы, проходя по вершинам $\kw{par}$.
Это путь уникально идентифицирует поток.
Для вычисления пути по редукционному контексту мы используем вспомогательную функцию $\textsf{path}$.

Стартующие дочерние потоки наследуют базовый фронт родительского потока, что
в терминах самой простой версии функции ${\sf spawn}$ определяется следующим образом:
%
\[
\spawn{\EvalContext}{\angled{\stEta,\stPsiRead}} \triangleq
\angled{\stEta, \stPsiRead
[\stpath\:l \mapsto \stSigmaRead,~\stpath\:r \mapsto \stSigmaRead]},
\]
где $\stpath = {\sf path}(\EvalContext)$ и $\stSigmaRead = \stPsiRead(\stpath)$.
Когда потоки соединяются, базовый фронт их родительского потока становится покомпонентным
максимумом базовых фронтов дочерних потоков:
\[
\joinP{\EvalContext}{\angled{\AST,\stPsiRead}} =
\angled{\AST, \stPsiRead[\stpath \mapsto \stSigmaRead^{l} \sqcup \stSigmaRead^{r}]},
\]
%
где $\stpath = \Path(\EvalContext)$,
$\stSigmaRead^{l} = \stPsiRead(\stpath\:l)$ и
$\stSigmaRead^{r} = \stPsiRead(\stpath\:r)$.

\begin{figure}
\begin{mathpar}
  \inferrule[Read-Uninit]{
  \auxX = \angled{\stEta, \stPsiRead} \quad \stpath = \Path(\EvalContext) \quad
  \stSigmaRead = \stPsiRead(\stpath) \quad \stSigmaRead(\loc) = \bot
  }{
\angled{\EvalContext[\readExpr{\RM}{\loc}], \auxX} \astep{} 
\angled{\Stuck, \auxX_{init}}
  }\\

  \inferrule[CAS-Uninit]{
  \auxX = \angled{\stEta, \stPsiRead} \quad \stpath = \Path(\EvalContext) 
  \quad
  \stSigmaRead = \stPsiRead(\stpath) \quad \stSigmaRead(\loc) = \bot
  }{
  \angled{\EvalContext[\Cas{\SM}{\FM}{\loc}{\Expr_{1}}{\Expr_{2}}], \auxX} \astep{} 
  \angled{\Stuck, \auxX_{init}}
  }
\end{mathpar}
\caption{Правила чтения из неинициализированной локации}
\label{fig:uninit-stuckRules}
\end{figure}

Далее мы определяем первый набор правил о ``плохих'' сценариях поведения ---
чтение из неинициализированной локации (см. рис. \ref{fig:uninit-stuckRules}).
Эти правила срабатывают в том случае, если поток, пытающийся прочитать значение из локации $\loc$,
не осведомлен ни об одной записи в неё, т.е. его базовый фронт не определён для $\loc$.
Как частный случай, правила применимы, если в локацию не было сделано ни одной записи в принципе.

\subsection{Высвобождающие и приобретающие операции}

\begin{figure}
\begin{mathpar}
  \inferrule[WriteRel]{
  \auxX = \angled{\stEta, \stPsiRead} \quad \stpath = \Path(\EvalContext) \quad
  \stTau = \NextTau{\stEta}{\loc} \\
  \stSigmaRead = \stPsiRead(\stpath) \quad
  \stSigma = \stSigmaRead[\loc \mapsto \stTau] \\
  \auxX' = \angled{\stEta[(\loc, \stTau) \mapsto (\mvalSubst, \stSigma)],
                   \stPsiRead[\stpath \mapsto \stSigma]}
  }{
\angled{\EvalContext[\Write{\rel}{\loc}{\mvalSubst}], \auxX} \astep{} 
\angled{\EvalContext[\Ret{\mvalSubst}], \auxX'}
  } \\

  \inferrule[ReadAcq]{
  \auxX = \angled{\stEta, \stPsiRead} \quad \stpath = \Path(\EvalContext) \quad 
  \stEta(\loc, \stTau) = (\mvalSubst, \stSigma) \\
  \stSigmaRead = \stPsiRead(\stpath) \quad
  \stSigmaRead(\loc) \leq \stTau \\
  \auxX' = \angled{\stEta, \stPsiRead[\stpath \mapsto \stSigmaRead \sqcup \stSigma]}
  }{
\angled{\EvalContext[\readExpr{\acq}{\loc}], \auxX} \astep{} 
\angled{\EvalContext[\Ret{\mvalSubst}], \auxX'}
  }
\end{mathpar}
\caption{Правила высвобождающей записи и приобретающего чтения}
\label{fig:rel/acq-sem}
\end{figure}


\section{Интерпретация и тестирование модели}

\subsection{Тестирование алгоритма RCU}

%% \begin{tabular}{| l ||@{~}c@{~}|@{~}c@{~}|@{~}c@{~}|@{~}c@{~}|@{~}c@{~}|@{~}c@{~}|@{~}c@{~}|@{~}c@{~}|@{~}c@{~}||@{~}c@{~}|}
\begin{tabular}{| l ||@{~}c@{~}|@{~}c@{~}|@{~}c@{~}|@{~}c@{~}|@{~}c@{~}|@{~}c@{~}|@{~}c@{~}|@{~}c@{~}||@{~}c@{~}|}
  \hline
  \textbf{Название теста} & \textsf{VF} & \textsf{WF} & \textsf{SCF}
  & \textsf{NAF} & \textsf{PO} & \textsf{ARR} % & \textsf{VS}
  & \textsf{CR} & \textsf{JN} & \textbf{C11} \\
%
% \textbf{Test name} & History & Viewfronts & Write-fronts & SC-front
%           & NA-front & Postponed operations & Acquire Read Restrictions (& Value Stealing) & Consume & Joining threads w/ non-empty operation buffers & \textbf{Full Support} \\

\hline\hline
\multicolumn{10}{|c|}{Буферизация записи (\textsf{SB})\ifext{, \S\ref{app:sb}}{}} \\
\hline
\textsf{rel-acq}   & \tick & &       & & & & & & \tick\\ 
\textsf{sc}        & \tick & & \tick & & & & & & \tick\\ 
\textsf{sc-rel}    & \tick & & \tick & & & & & & \tick\\ 
\textsf{sc-acq}    & \tick & & \tick & & & & & & \tick\\ 

\hline
\multicolumn{10}{|c|}{Буферизация чтения (\textsf{LB})\ifext{, \S\ref{app:lb}}{}} \\
\hline
\textsf{rlx}         & \tick & & & & \tick & & & & \tick\\ 
\textsf{rel-rlx}     & \tick & & & & \tick & & & & \tick\\ 
\textsf{acq-rlx}     & \tick & & & & \tick & & & & \fail\\ 
\textsf{rel-acq-rlx} & \tick & & & & \tick & \tick & & & \tick\\ 
\textsf{rlx-use}     & \tick & & & & \tick & & & & \tick\\ 
\textsf{rlx-let}     & \tick & & & & \tick & & & & \tick\\ 
\textsf{rlx-join}    & \tick & & & & \tick & & & \tick & \tickP\\ 
\textsf{rel-rlx-join} & \tick & & & & \tick & & & \tick & \tickP\\ 
\textsf{acq-rlx-join} & \tick & & & & \tick & & & \tick & \fail\\ 

\hline
\multicolumn{10}{|c|}{Передача сообщения (\textsf{MP})\ifext{, \S\ref{app:mp}}{}} \\
\hline
\textsf{rlx-na}            & \tick &       & & \tick & & &       & & \tick\\ 
\textsf{rel-rlx-na}        & \tick &       & & \tick & & &       & & \tick\\ 
\textsf{rlx-acq-na}        & \tick &       & & \tick & & &       & & \tick\\ 
\textsf{rel-acq-na}        & \tick &       & & \tick & & \tick & & & \tick\\ 
\textsf{rel-acq-na-rlx(\_2)} & \tick & \tick & & \tick & & \tick & & & \tick\\ 
\textsf{con-na(\_2)}       & \tick &       & & \tick & & & \tick & & \tick\\ 
\textsf{cas-rel-acq-na}    & \tick &       & & \tick & & \tick & & & \tick\\ 
\textsf{cas-rel-rlx-na}    & \tick &       & & \tick & & &     & & \tick\\ 

\hline
\multicolumn{10}{|c|}{Корректность повторного чтения (\textsf{CoRR})\ifext{, \S\ref{app:corr}}{}} \\
\hline
\textsf{rlx}      & \tick & &       & & &  & & & \tick\\ 
\textsf{rel-acq}  & \tick & &       & & &  & & & \tick\\ 

\hline
\multicolumn{10}{|c|}{Независимые чтения независимых записей (\textsf{IRIW})\ifext{, \S\ref{app:iriw}}{}} \\
\hline
\textsf{rlx}      & \tick & &       & & &  & & & \tick\\ 
\textsf{rel-acq}  & \tick & &       & & &  & & & \tick\\ 
\textsf{sc}       & \tick & & \tick & & &  & & & \tick\\ 

\hline
\multicolumn{10}{|c|}{Зависимость запись-чтение (\textsf{WRC})\ifext{, \S\ref{app:wrc}}{}} \\
\hline
\textsf{rlx}      & \tick & &       & & & &  & & \tick\\ 
\textsf{rel-acq}  & \tick & &       & & & &  & & \tick\\ 
\textsf{cas-rel}  & \tick & &       & & & \tick & & & \tick\\ 
\textsf{cas-rlx}  & \tick & &       & & & &  & & \tick\\ 

\hline
\multicolumn{10}{|c|}{``Значения из воздуха'' (\textsf{OOTA})\ifext{, \S\ref{app:ota}}{}} \\
\hline
\textsf{lb}       & \tick & &       & & \tick & & & & \fail\\ 
\textsf{if}       & \tick & &       & & \tick & & & & \fail\\ 

\hline
\multicolumn{10}{|c|}{Независимые записи (\textsf{WR})\ifext{, \S\ref{app:wr}}{}} \\
\hline
\textsf{rlx}      & \tick & &       & & \tick & & & & \tick\\ 
\textsf{rlx-rel}  & \tick & &       & & \tick & \tick & & & \tick\\ 
\textsf{rel}      & \tick & &       & & \tick & \tick & & & \tick\\ 

%% \hline
%% \multicolumn{11}{|c|}{Value Stealing (\textsf{VS})\ifext{, \S\ref{app:ss}}{}} \\
%% \hline
%% \textsf{rlx}      & \tick & &       & & \tick & \tick & \tick & & & \tick\\ 

\hline
\multicolumn{10}{|c|}{Спекулятивное исполнение (\textsf{SE})\ifext{, \S\ref{app:se}}{}} \\
\hline
\textsf{simple}      & \tick & &       & & \tick & & & & \tick\\ 
\textsf{prop}        & \tick & &       & & \tick & & & & \tick\\ 
\textsf{nested}      & \tick & &       & & \tick & & & & \tick\\ 

  \hline
  \multicolumn{10}{|c|}{Блокировки \ifext{, \S\ref{app:locks}}{}} \\
  \hline
  Деккера & \tick & & & \tick & & & & & \tick\\ 
  Коэна~\cite{Turon-al:OOPSLA14}  & \tick & & & \tick & & & & & \tick\\ 

%% LB & $\checkmark$ & & \\
%% \multicolumn{4}{l}{ A snippet code here
%% }\\
%% SB & $\checkmark$ & & \\

\hline

\end{tabular}


\begin{figure*}[t]
\newcommand{\nullPtr}{\kw{null}}
\newcommand{\funcSt}[1]{{\rm #1}}

{\small{
\textbf{Программа:}
\vspace{-17pt}
\[
\writeInstParam{\na}{cw}{0};
\writeInstParam{\na}{cr1}{0};
\writeInstParam{\na}{cr2}{0};
\writeInstParam{\na}{lhead}{\nullPtr};
\]
%% \begin{lstlisting}
%%                         |$[$|cw|$]_{na}$| := 0; |$[$|cr1|$]_{na}$| := 0; |$[$|cr2|$]_{na}$| := 0; |$[$|lhead|$]_{na}$| := null;
%% \end{lstlisting}
\begin{tabular}{L || L || L}
  \begin{array}{l}
    \writeInstParam{\rlx}{a}{(1, \nullPtr)}; \\
    \writeInstParam{\na}{ltail}{a}; \\
    \writeInstParam{\rel}{lhead}{a}; \\
    \funcSt{append}(b, 10, ltail); \\
    \funcSt{append}(c, 100, ltail); \\
    \funcSt{updateSecondNode}(d, 1000) \\
  \end{array}
&
  \begin{array}{l}
    \writeInstParam{\na}{sum11}{0}; \\
    \funcSt{rcuOnline}(cw, cr1); \\
    \funcSt{traverse}(lhead, cur1, sum11); \\
    \funcSt{rcuOffline}(cw, cr1); \\
    \\
    \writeInstParam{\na}{sum12}{0}; \\
    \funcSt{rcuOnline}(cw, cr1); \\
    \funcSt{traverse}(lhead, cur1, sum12); \\
    \funcSt{rcuOffline}(cw, cr1); \\
    \\
    \readInstParam{na}{r11}{sum11}; \\
    \readInstParam{na}{r11}{sum12} \\
  \end{array}
&
  \begin{array}{l}
    \writeInstParam{\na}{sum21}{0}; \\
    \funcSt{rcuOnline}(cw, cr2); \\
    \funcSt{traverse}(lhead, cur2, sum21); \\
    \funcSt{rcuOffline}(cw, cr2); \\
    \\
    \writeInstParam{\na}{sum22}{0}; \\
    \funcSt{rcuOnline}(cw, cr2); \\
    \funcSt{traverse}(lhead, cur2, sum22); \\
    \funcSt{rcuOffline}(cw, cr2); \\
    \\
    \readInstParam{na}{r21}{sum21}; \\
    \readInstParam{na}{r21}{sum22} \\
  \end{array}
\end{tabular}

\vspace{5pt}
\hrule
\vspace{5pt}

\textbf{Функции:}

\begin{tabular}{@{}L L l}

\begin{array}{@{}l}
  \funcSt{append}(loc, value, ltail) \triangleq \\
  \quad \writeInstParam{\rlx}{loc}{(value, \nullPtr)}; \\
  \quad \readInstParam{\na}{rt}{ltail}; \\
  \quad \readInstParam{\rlx}{rtc}{rt}; \\
  \quad \writeInstParam{\rel}{rt}{(\kw{fst} \; rtc, loc)}; \\
  \quad \writeInstParam{\na}{ltail}{loc} \\
  \\
  \funcSt{updateSecondNode}(loc, value) \triangleq \\
  \quad \readInstParam{\rlx}{r1}{lhead}; \\
  \quad \readInstParam{\rlx}{r1c}{r1}; \\
  \quad \assignInst{r2}{\kw{snd} \; r1c}; \\
  \quad \readInstParam{\rlx}{r2c}{r2}; \\
  \quad \assignInst{r3}{\kw{snd} \; r2c}; \\
  \quad \writeInstParam{\rel}{loc}{(value, r3)}; \\
  \quad \writeInstParam{\rel}{r1}{(\kw{fst} \; r1c, loc)}; \\
  \quad \funcSt{sync}(cw, cr1, cr2); \\
  \quad \kw{delete} \; r2
\end{array}

&

\begin{array}{l}
  \funcSt{traverse}(lhead, curNodeLoc, resLoc) \triangleq \\
  \quad \readInstParam{\acq}{rh}{lhead}; \\
  \quad \writeInstParam{\na}{curNodeLoc}{rh}; \\
  \quad \kw{repeat} \\
  \qquad \readInstParam{\na}{rCurNode}{curNodeLoc}; \\
  \qquad \kw{if} \; rCurNode \neq \nullPtr \\
  \qquad \begin{array}[t]{@{}l l}
         \kw{then}& {\begin{array}[t]{@{}l}
                      \readInstParam{\acq}{rNode}{rCurNode}; \\
                      \readInstParam{\na}{rRes}{resLoc}; \\
                      \assignInst{rVal}{\kw{fst} \; rNode}; \\
                      \writeInstParam{\na}{resLoc}{rVal + rRes}; \\
                      \writeInstParam{\na}{curNodeLoc}{\kw{snd} \; rNode}; \\
                      0
                    \end{array}} \\
         \kw{else}& 1 \\
         \end{array}\\
  \qquad \kw{fi}
  \quad \kw{end} \\

\end{array}
&

\begin{tabular}{@{}l@{}}
\begin{lstlisting}
sync(cw, cr1, cr2) |$\triangleq$|
  rcw  = |$[$|cw|$]_{rlx}$|;
  rcwn = rcw + 2;
  |$[$|cw|$]_{rel}$| := rcwn;
  |\graybox{\texttt{syncWithReader(rcwn, cr1);}}|
  |\graybox{\texttt{syncWithReader(rcwn, cr2)}}|
\end{lstlisting}
\\
\\
\begin{lstlisting}
syncWithReader(rcwn, cr) |$\triangleq$| 
  repeat |$[$|cr|$]_{acq}$| >= rcwn end
\end{lstlisting}
\\
\\
\begin{lstlisting}
rcuOnline(cw, cr) |$\triangleq$| 
  |$[$|cr|$]_{rlx}$| := |$[$|cw|$]_{acq}$| + 1
\end{lstlisting}
\\
\\
\begin{lstlisting}
rcuOffline(cw, cr) |$\triangleq$| 
  |$[$|cr|$]_{rel}$| := |$[$|cw|$]_{rlx}$|
\end{lstlisting}
\end{tabular}


\end{tabular}
}}

\caption[Реализация алгоритма QSBR RCU]
{Реализация алгоритма QSBR RCU.
 При тестировании была рассмотрена также версия без фрагментов, выделенных серым фоном
 (Раздел~\ref{sec:testing}).}
\label{fig:rcuProg}
\end{figure*}

\section{Свойства модели. Выводы}
