\chapter{Операционная модель памяти C/C++11} \label{sec:opc11}
В главе описана операционная модель памяти C/C++11 \cite{Podkopaev-al:CoRR16}.

Модель представлена как семейство операционных семантик, которые описывают
различные аспекты модели памяти C/C++11 \cite{Batty-al:POPL11}.
Все семантики базируются на комбинации двух основных идей: 
\emph{фронтов} и \emph{операционных буферов}.
Фронты используются для представления осведомленности потоков о текущем состоянии общей памяти,
тогда как операционные буферы позволяют откладывать исполнение инструкций и
производить спекулятивные вычисления.
Модель представлена в виде абстрактной машины и покрывает 

\section{Основные концепции модели}
В разделе мы рассмотрим базовые концепции модели на нескольких примерах.
Начнём мы с программы ${\rm MP\text{-}rel\text{-}acq}$ (см. рис. \ref{fig:MPrelAcq}),
которую мы уже обсуждали в контексте модели C/C++11.
В аксиоматической C/C++11 MM программа имеет три сценария поведения (см. рис. \ref{fig:MPrelAcqSem}).
Рассмотрим один из них:
\[
\begin{tikzpicture}[yscale=1,xscale=1.8]
  \node (01)  at (-2,3) {$p: \wlab{\na}{x}{0}$ };
  \node (02)  at (0,3) {$q: \wlab{\na}{y}{0}$ };
  \node (1)  at (-2,1.5) {$r: \wlab{\rlx}{x}{1}$ };
  \node (2)  at (-2,0) {$s: \wlab{\rel}{y}{1}$ };
  \node (11) at (0,1.5)  {$t: \rlab{\acq}{y}{1}$ };
  \node (12) at (0,0)  {$u: \rlab{\rlx}{x}{1}$ };

  \draw[po] (1)  edge  (2);
  \draw[po] (11) edge (12);
  \draw[po] (01) edge (11);
  \draw[po] (02) edge (11);
  %% \draw[hb] (01) edge node[right] {\small $\lHB$} (1);
  \draw[po] (01) edge (1);
  \draw[po] (02) edge  (1);
  %% \draw[deps,bend left=20] (11)  edge node[right] {\small $\lDEPS$} (12);
  \draw[sw,bend right=20] (2) edge node[below] {\small $\lSW$} (11);
  \draw[rf] (2) edge node[above] {\small $\lRF$} (11);
  \draw[rf] (1) edge node[below] {} (12);
  \draw[mo,bend right=20] (01)  edge node[left] {\small $\lMO$} (1);
  \draw[mo,bend right=90] (02)  edge node[above] {} (2); %{\small $\lCO$} (2);
\end{tikzpicture}
\]

%% На рис. \ref{fig:MPrelAcq} представлена сама программа и одни из сценариев поведения
%% этой программы в рамках C/C++11 MM.
%% \begin{figure}
%%   \begin{minipage}{.4\textwidth}
%% \begin{equation*}
%% %% \tag{MP-rel-acq}
%% \begin{tabular}{c}
%%   $\writeInstParam{\na}{x}{0}; \writeInstParam{\na}{y}{0};$ \\
%% \begin{tabular}{L || L}
%%   \writeInstParam{\rlx}{x}{1}; & \readInstParam{\acq}{a}{y}; \\
%%   \writeInstParam{\rel}{y}{1} & \readInstParam{\rlx}{b}{x} \\
%% \end{tabular}
%% \end{tabular}
%% \end{equation*}
%%   \end{minipage}
%%   \begin{minipage}{.4\textwidth}
%% \[
%% \begin{tikzpicture}[yscale=1,xscale=1.8]
%%   \node (01)  at (-2,3) {$p: \wlab{\na}{x}{0}$ };
%%   \node (02)  at (0,3) {$q: \wlab{\na}{y}{0}$ };
%%   \node (1)  at (-2,1.5) {$r: \wlab{\rlx}{x}{1}$ };
%%   \node (2)  at (-2,0) {$s: \wlab{\rel}{y}{1}$ };
%%   \node (11) at (0,1.5)  {$t: \rlab{\acq}{y}{1}$ };
%%   \node (12) at (0,0)  {$u: \rlab{\rlx}{x}{1}$ };

%%   \draw[po] (1)  edge  (2);
%%   \draw[po] (11) edge (12);
%%   \draw[po] (01) edge (11);
%%   \draw[po] (02) edge (11);
%%   %% \draw[hb] (01) edge node[right] {\small $\lHB$} (1);
%%   \draw[po] (01) edge (1);
%%   \draw[po] (02) edge  (1);
%%   %% \draw[deps,bend left=20] (11)  edge node[right] {\small $\lDEPS$} (12);
%%   \draw[sw,bend right=20] (2) edge node[below] {\small $\lSW$} (11);
%%   \draw[rf] (2) edge node[above] {\small $\lRF$} (11);
%%   \draw[rf] (1) edge node[below] {} (12);
%%   \draw[mo,bend right=20] (01)  edge node[left] {\small $\lMO$} (1);
%%   \draw[mo,bend right=90] (02)  edge node[above] {} (2); %{\small $\lCO$} (2);
%% \end{tikzpicture}
%% \]
%%   \end{minipage}
%% \caption{Программа ${\rm MP\text{-}rel\text{-}acq}$ и её сценарий исполнения в модели C/C++11}
%% \label{fig:MPrelAcq}
%% \end{figure}


\subsection{Синхронизация потоков. Фронты}
\subsection{Спекулятивное исполнение. Операционные буферы}

\section{Полная модель}
В разделе описываются более сложные аспекты модели памяти C/C++, такие как
$\sco$-обращения, неатомарные инструкции и гонки по данным,
$\con$-чтения, соединение потоков (thread's joining),
а также расслабленные обращения.

\section{Формальное определение модели}
В разделе приводится математическое определение операционной модели C/C++11.

\section{Интерпретация и тестирование модели}

\subsection{Тестирование алгоритма RCU}

%% \begin{tabular}{| l ||@{~}c@{~}|@{~}c@{~}|@{~}c@{~}|@{~}c@{~}|@{~}c@{~}|@{~}c@{~}|@{~}c@{~}|@{~}c@{~}|@{~}c@{~}||@{~}c@{~}|}
\begin{tabular}{| l ||@{~}c@{~}|@{~}c@{~}|@{~}c@{~}|@{~}c@{~}|@{~}c@{~}|@{~}c@{~}|@{~}c@{~}|@{~}c@{~}||@{~}c@{~}|}
  \hline
  \textbf{Название теста} & \textsf{VF} & \textsf{WF} & \textsf{SCF}
  & \textsf{NAF} & \textsf{PO} & \textsf{ARR} % & \textsf{VS}
  & \textsf{CR} & \textsf{JN} & \textbf{C11} \\
%
% \textbf{Test name} & History & Viewfronts & Write-fronts & SC-front
%           & NA-front & Postponed operations & Acquire Read Restrictions (& Value Stealing) & Consume & Joining threads w/ non-empty operation buffers & \textbf{Full Support} \\

\hline\hline
\multicolumn{10}{|c|}{Буферизация записи (\textsf{SB})\ifext{, \S\ref{app:sb}}{}} \\
\hline
\textsf{rel-acq}   & \tick & &       & & & & & & \tick\\ 
\textsf{sc}        & \tick & & \tick & & & & & & \tick\\ 
\textsf{sc-rel}    & \tick & & \tick & & & & & & \tick\\ 
\textsf{sc-acq}    & \tick & & \tick & & & & & & \tick\\ 

\hline
\multicolumn{10}{|c|}{Буферизация чтения (\textsf{LB})\ifext{, \S\ref{app:lb}}{}} \\
\hline
\textsf{rlx}         & \tick & & & & \tick & & & & \tick\\ 
\textsf{rel-rlx}     & \tick & & & & \tick & & & & \tick\\ 
\textsf{acq-rlx}     & \tick & & & & \tick & & & & \fail\\ 
\textsf{rel-acq-rlx} & \tick & & & & \tick & \tick & & & \tick\\ 
\textsf{rlx-use}     & \tick & & & & \tick & & & & \tick\\ 
\textsf{rlx-let}     & \tick & & & & \tick & & & & \tick\\ 
\textsf{rlx-join}    & \tick & & & & \tick & & & \tick & \tickP\\ 
\textsf{rel-rlx-join} & \tick & & & & \tick & & & \tick & \tickP\\ 
\textsf{acq-rlx-join} & \tick & & & & \tick & & & \tick & \fail\\ 

\hline
\multicolumn{10}{|c|}{Передача сообщения (\textsf{MP})\ifext{, \S\ref{app:mp}}{}} \\
\hline
\textsf{rlx-na}            & \tick &       & & \tick & & &       & & \tick\\ 
\textsf{rel-rlx-na}        & \tick &       & & \tick & & &       & & \tick\\ 
\textsf{rlx-acq-na}        & \tick &       & & \tick & & &       & & \tick\\ 
\textsf{rel-acq-na}        & \tick &       & & \tick & & \tick & & & \tick\\ 
\textsf{rel-acq-na-rlx(\_2)} & \tick & \tick & & \tick & & \tick & & & \tick\\ 
\textsf{con-na(\_2)}       & \tick &       & & \tick & & & \tick & & \tick\\ 
\textsf{cas-rel-acq-na}    & \tick &       & & \tick & & \tick & & & \tick\\ 
\textsf{cas-rel-rlx-na}    & \tick &       & & \tick & & &     & & \tick\\ 

\hline
\multicolumn{10}{|c|}{Корректность повторного чтения (\textsf{CoRR})\ifext{, \S\ref{app:corr}}{}} \\
\hline
\textsf{rlx}      & \tick & &       & & &  & & & \tick\\ 
\textsf{rel-acq}  & \tick & &       & & &  & & & \tick\\ 

\hline
\multicolumn{10}{|c|}{Независимые чтения независимых записей (\textsf{IRIW})\ifext{, \S\ref{app:iriw}}{}} \\
\hline
\textsf{rlx}      & \tick & &       & & &  & & & \tick\\ 
\textsf{rel-acq}  & \tick & &       & & &  & & & \tick\\ 
\textsf{sc}       & \tick & & \tick & & &  & & & \tick\\ 

\hline
\multicolumn{10}{|c|}{Зависимость запись-чтение (\textsf{WRC})\ifext{, \S\ref{app:wrc}}{}} \\
\hline
\textsf{rlx}      & \tick & &       & & & &  & & \tick\\ 
\textsf{rel-acq}  & \tick & &       & & & &  & & \tick\\ 
\textsf{cas-rel}  & \tick & &       & & & \tick & & & \tick\\ 
\textsf{cas-rlx}  & \tick & &       & & & &  & & \tick\\ 

\hline
\multicolumn{10}{|c|}{``Значения из воздуха'' (\textsf{OOTA})\ifext{, \S\ref{app:ota}}{}} \\
\hline
\textsf{lb}       & \tick & &       & & \tick & & & & \fail\\ 
\textsf{if}       & \tick & &       & & \tick & & & & \fail\\ 

\hline
\multicolumn{10}{|c|}{Независимые записи (\textsf{WR})\ifext{, \S\ref{app:wr}}{}} \\
\hline
\textsf{rlx}      & \tick & &       & & \tick & & & & \tick\\ 
\textsf{rlx-rel}  & \tick & &       & & \tick & \tick & & & \tick\\ 
\textsf{rel}      & \tick & &       & & \tick & \tick & & & \tick\\ 

%% \hline
%% \multicolumn{11}{|c|}{Value Stealing (\textsf{VS})\ifext{, \S\ref{app:ss}}{}} \\
%% \hline
%% \textsf{rlx}      & \tick & &       & & \tick & \tick & \tick & & & \tick\\ 

\hline
\multicolumn{10}{|c|}{Спекулятивное исполнение (\textsf{SE})\ifext{, \S\ref{app:se}}{}} \\
\hline
\textsf{simple}      & \tick & &       & & \tick & & & & \tick\\ 
\textsf{prop}        & \tick & &       & & \tick & & & & \tick\\ 
\textsf{nested}      & \tick & &       & & \tick & & & & \tick\\ 

  \hline
  \multicolumn{10}{|c|}{Блокировки \ifext{, \S\ref{app:locks}}{}} \\
  \hline
  Деккера & \tick & & & \tick & & & & & \tick\\ 
  Коэна~\cite{Turon-al:OOPSLA14}  & \tick & & & \tick & & & & & \tick\\ 

%% LB & $\checkmark$ & & \\
%% \multicolumn{4}{l}{ A snippet code here
%% }\\
%% SB & $\checkmark$ & & \\

\hline

\end{tabular}


\begin{figure*}[t]
\newcommand{\nullPtr}{\kw{null}}
\newcommand{\funcSt}[1]{{\rm #1}}

{\small{
\textbf{Программа:}
\vspace{-17pt}
\[
\writeInstParam{\na}{cw}{0};
\writeInstParam{\na}{cr1}{0};
\writeInstParam{\na}{cr2}{0};
\writeInstParam{\na}{lhead}{\nullPtr};
\]
%% \begin{lstlisting}
%%                         |$[$|cw|$]_{na}$| := 0; |$[$|cr1|$]_{na}$| := 0; |$[$|cr2|$]_{na}$| := 0; |$[$|lhead|$]_{na}$| := null;
%% \end{lstlisting}
\begin{tabular}{L || L || L}
  \begin{array}{l}
    \writeInstParam{\rlx}{a}{(1, \nullPtr)}; \\
    \writeInstParam{\na}{ltail}{a}; \\
    \writeInstParam{\rel}{lhead}{a}; \\
    \funcSt{append}(b, 10, ltail); \\
    \funcSt{append}(c, 100, ltail); \\
    \funcSt{updateSecondNode}(d, 1000) \\
  \end{array}
&
  \begin{array}{l}
    \writeInstParam{\na}{sum11}{0}; \\
    \funcSt{rcuOnline}(cw, cr1); \\
    \funcSt{traverse}(lhead, cur1, sum11); \\
    \funcSt{rcuOffline}(cw, cr1); \\
    \\
    \writeInstParam{\na}{sum12}{0}; \\
    \funcSt{rcuOnline}(cw, cr1); \\
    \funcSt{traverse}(lhead, cur1, sum12); \\
    \funcSt{rcuOffline}(cw, cr1); \\
    \\
    \readInstParam{na}{r11}{sum11}; \\
    \readInstParam{na}{r11}{sum12} \\
  \end{array}
&
  \begin{array}{l}
    \writeInstParam{\na}{sum21}{0}; \\
    \funcSt{rcuOnline}(cw, cr2); \\
    \funcSt{traverse}(lhead, cur2, sum21); \\
    \funcSt{rcuOffline}(cw, cr2); \\
    \\
    \writeInstParam{\na}{sum22}{0}; \\
    \funcSt{rcuOnline}(cw, cr2); \\
    \funcSt{traverse}(lhead, cur2, sum22); \\
    \funcSt{rcuOffline}(cw, cr2); \\
    \\
    \readInstParam{na}{r21}{sum21}; \\
    \readInstParam{na}{r21}{sum22} \\
  \end{array}
\end{tabular}

\vspace{5pt}
\hrule
\vspace{5pt}

\textbf{Функции:}

\begin{tabular}{@{}L L l}

\begin{array}{@{}l}
  \funcSt{append}(loc, value, ltail) \triangleq \\
  \quad \writeInstParam{\rlx}{loc}{(value, \nullPtr)}; \\
  \quad \readInstParam{\na}{rt}{ltail}; \\
  \quad \readInstParam{\rlx}{rtc}{rt}; \\
  \quad \writeInstParam{\rel}{rt}{(\kw{fst} \; rtc, loc)}; \\
  \quad \writeInstParam{\na}{ltail}{loc} \\
  \\
  \funcSt{updateSecondNode}(loc, value) \triangleq \\
  \quad \readInstParam{\rlx}{r1}{lhead}; \\
  \quad \readInstParam{\rlx}{r1c}{r1}; \\
  \quad \assignInst{r2}{\kw{snd} \; r1c}; \\
  \quad \readInstParam{\rlx}{r2c}{r2}; \\
  \quad \assignInst{r3}{\kw{snd} \; r2c}; \\
  \quad \writeInstParam{\rel}{loc}{(value, r3)}; \\
  \quad \writeInstParam{\rel}{r1}{(\kw{fst} \; r1c, loc)}; \\
  \quad \funcSt{sync}(cw, cr1, cr2); \\
  \quad \kw{delete} \; r2
\end{array}

&

\begin{array}{l}
  \funcSt{traverse}(lhead, curNodeLoc, resLoc) \triangleq \\
  \quad \readInstParam{\acq}{rh}{lhead}; \\
  \quad \writeInstParam{\na}{curNodeLoc}{rh}; \\
  \quad \kw{repeat} \\
  \qquad \readInstParam{\na}{rCurNode}{curNodeLoc}; \\
  \qquad \kw{if} \; rCurNode \neq \nullPtr \\
  \qquad \begin{array}[t]{@{}l l}
         \kw{then}& {\begin{array}[t]{@{}l}
                      \readInstParam{\acq}{rNode}{rCurNode}; \\
                      \readInstParam{\na}{rRes}{resLoc}; \\
                      \assignInst{rVal}{\kw{fst} \; rNode}; \\
                      \writeInstParam{\na}{resLoc}{rVal + rRes}; \\
                      \writeInstParam{\na}{curNodeLoc}{\kw{snd} \; rNode}; \\
                      0
                    \end{array}} \\
         \kw{else}& 1 \\
         \end{array}\\
  \qquad \kw{fi}
  \quad \kw{end} \\

\end{array}
&

\begin{tabular}{@{}l@{}}
\begin{lstlisting}
sync(cw, cr1, cr2) |$\triangleq$|
  rcw  = |$[$|cw|$]_{rlx}$|;
  rcwn = rcw + 2;
  |$[$|cw|$]_{rel}$| := rcwn;
  |\graybox{\texttt{syncWithReader(rcwn, cr1);}}|
  |\graybox{\texttt{syncWithReader(rcwn, cr2)}}|
\end{lstlisting}
\\
\\
\begin{lstlisting}
syncWithReader(rcwn, cr) |$\triangleq$| 
  repeat |$[$|cr|$]_{acq}$| >= rcwn end
\end{lstlisting}
\\
\\
\begin{lstlisting}
rcuOnline(cw, cr) |$\triangleq$| 
  |$[$|cr|$]_{rlx}$| := |$[$|cw|$]_{acq}$| + 1
\end{lstlisting}
\\
\\
\begin{lstlisting}
rcuOffline(cw, cr) |$\triangleq$| 
  |$[$|cr|$]_{rel}$| := |$[$|cw|$]_{rlx}$|
\end{lstlisting}
\end{tabular}


\end{tabular}
}}

\caption[Реализация алгоритма QSBR RCU]
{Реализация алгоритма QSBR RCU.
 При тестировании была рассмотрена также версия без фрагментов, выделенных серым фоном
 (Раздел~\ref{sec:testing}).}
\label{fig:rcuProg}
\end{figure*}

\section{Свойства модели. Выводы}
