\chapter{Обзор предметной области} \label{sec:overview}

В данной главе вводится понятие модели памяти, слабой модели памяти,
приводится примеры слабых сценариев поведения программ. Описываются
классы моделей памяти: операционные, аксиоматические (декларативные),
денотационные.
Рассматриваются существующие модели памяти языков программирования и
процессорных архитектур, а также требования, предъявляемые к ним.
Приводится подробное описание модели памяти C/C++11 \cite{Batty-al:POPL11}.
Описывается проблема ``значений из воздуха'' (OOTA, out-of-thin-air values).
В конце главы приведены выводы о состоянии предметной области и о существующих
направлениях исследования в ней.

\section{Модели памяти}
Под \emph{моделью памяти} мы будем понимать семантику системы с многопоточностью.
В рамках диссертационного исследования рассматриваются два типа таких систем:
языки программирования и процессорные архитектуры.

Модели памяти разделяются по принципу того, какие ограничения на сценарии поведения
программ, запущенных в системе опеределения модели, они предоставляют \cite{Kshemkalyani-Singhal:2011}.
Так, \emph{строгая консистентность} гарантирует, что любая запись в память становится мгновенно видна
всем потокам в системе. Эта модель требует наличия некоторого абсолютного времени во всех потоках системы,
что зачастую нереализуемо. Менее строгая модель \emph{последовательной консистентности}




%% План
%% \begin{itemize} 
%%   \item Определение. Модель памяти
%%   \item Определение. Модель последовательной консистентности
%%   \item Мотивировка. Пример слабого исполнения
%%   \item Определение. Слабый сценарий исполнения
%%   \item Определение. Слабая модель памяти
%%   \item Существующие модели
%%   \begin{itemize}
%%     \item Языки
%%       \begin{itemize}
%%         \item C/C++
%%         \item Java
%%         \item ?? LLVM ??
%%         \item ?? .Net ??
%%       \end{itemize}
%%     \item Процессоры
%%       \begin{itemize}
%%         \item x86
%%         \item Power
%%         \item ARM
%%         \item DEC Alpha (видимая спекуляция потоков)
%%       \end{itemize}
%%   \end{itemize}
%%   \begin{itemize}
%%     \item Модель памяти C/C++11. Недостатки
%%       \begin{itemize}
%%         \item OOTA (out-of-thin-air)
%%         \item Нет операционной интуиции
%%       \end{itemize}
%%     \item Модель памяти Java. Недостатки
%%       \begin{itemize}
%%         \item Оптимизации некорректны
%%         \item Нет операционной интуиции
%%       \end{itemize}
%%   \end{itemize}
%%   \item Определение. Операционная модель памяти
%%   \item Определение. Декларативная (аксиоматическая) модель памяти
%% \end{itemize}


%% \section{Модели памяти}
%% \subsection{Виды моделей памяти}
