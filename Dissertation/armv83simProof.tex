\chapter{Доказательство леммы о шаге симуляции между обещающей и ARMv8.3 моделями}
\label{sec:sim-step-proof}

  {\bf Лемма \ref{lem:sim-step}.}
  Пусть для некоторых конфигураций обхода $\tup{C, \IssuedSet}$ и $\tup{C', \IssuedSet'}$ сценария $G$,
  а также некоторого состояния обещающей машины $\tup{\TSf, M}$ выполняется
  $G \vdash \tup{C, \IssuedSet} \travConfigStep \tup{C', \IssuedSet'}$ и
  $\simRel(C, \IssuedSet, \TSf, M)$.
  Тогда существуют такие $\TSf'$ и $M'$, что $\tup{\TSf, M} \stepp\!\!^{+} \tup{\TSf', M'}$ и
  $\simRel(C', \IssuedSet', \TSf', M')$.
\begin{proof}%[Доказательство леммы \ref{lem:sim-step}]
  Существует два варианта: $G \vdash C, \IssuedSet \travConfigStep \tup{C', \IssuedSet'}$ соответствует покрытию или
  выпуску некоторого события $e \in \lE$. Введем обозначения $tid \defeq \lTID(e)$ и
  $\tup{\pstate, \View, \PromSet} \defeq \TSf(tid)$.
  %% \[\begin{array}{l@{~}l}
  %%    tid & \defeq \lTID(e) \\
  %%    \tup{\pstate, \View, \PromSet} & \defeq \TSf(tid) \\
  %% \end{array}\]
  Начнем с рассмотрения варианта, когда $G \vdash C, \IssuedSet \travConfigStep \tup{C', \IssuedSet'}$ соответствует
  выпуску события записи $e$.

  {\bf Выпуск события $e$}. Из определения $\travConfigStep$ следует, что
  $C' = C$, $\IssuedSet' = \IssuedSet \cup \{e\}$ и $e \in \issuable(G, C, \IssuedSet) \setminus \IssuedSet$.
      Введем обозначения:
      \[\begin{array}{l l}
        \loc, \; \val, \; \tau & \defeq \lLOC(e), \; \lVALW(e), \; T(e) \\
        mview & \defeq [\loc @ \tau] \sqcup \View.\viewRel \\
        msg & \defeq \msg{\loc}{\val}{\tau}{mview} \\
      \end{array}\]
      Мы знаем, что в $M$ нету сообщения для локации $\loc$ с меткой времени $\tau$,
      т.к. функция $T$ выдает уникальные для одной локации метки времени для событий записи ($\correctTmap(G, T)$),
      и у каждого сообщения из $M$ существует соответствующее ему сообщение в $\IssuedSet$ ($\invMemTwo(C, \IssuedSet, M)$).
      \[\begin{array}{l l}
        M', \; \PromSet' & \defeq \addToMemory{M}{msg}, \; \addToMemory{\PromSet}{msg} \\
        \TSf' & \defeq \TSf[tid \mapsto \tup{\pstate, \View, \PromSet'}]
      \end{array}\]
      $mview \in M'$, т.к. $[\loc @ \tau] \in M'$ и $\View.\viewRel \in M$.
      Таким образом $\tup{\tup{\pstate, \View, \PromSet}, M} \stepptid \tup{\tup{\pstate, \View, \PromSet'}, M'}$ выполняется.
      Осталось проверить, что $\simRel(C, \IssuedSet', \TSf', M')$ выполняется.
      \begin{itemize}
        \item $\invMemOne(C, \IssuedSet', \TSf', M') \land \invMemTwo(C, \IssuedSet', M')$: \\
          Единственным нетривиальным утверждением, которое нужно проверить, является $mview \le \domView(\msgRel; [e])$.
          По определению, $mview = [\loc @ \tau] \sqcup \View.\viewRel$
          $[\loc @ \tau] = [\lLOC(e) @ T(e)] \le \domView(\msgRel; [e])$, т.к. $\tup{e, e} \in \msgRel$.
          Из $\invView(C, \TSf)$ следует $\View.\viewRel \le \domView(\relRel; [e'])$,
          где $e' \in \nextset(G, C)$ и $\lTID(e') = \lTID(e)$. Т.к. $e \nin C$ и по определению $\nextset$,
          $\tup{e', e} \in \lPO^{?}$. Из определения $\relRel$ и $\msgRel$ следует
          $\relRel; [e']; \lPO^{?}; [e] \suq \msgRel; [e]$.
          %% \[\dom{\relRel; [e']} = \dom{\relRel; [e']; \lPO^{?}; [e]} \suq \dom{\msgRel; [e]}\]
          Утверждение выполняется, т.к. $\View.\viewRel \le \domView(\relRel; \lPO; [e'])$.
        
        \item $\invViewRel(\TSf')$: выполняется по $\invViewRel(\TSf)$ и определениям $mview$ и $\TSf'$.

        \item $\invView(C, \TSf')$: т.к. $\forall tid. \; \TSf'(tid).\View = \TSf(tid).\View$ и $\invView(C, \TSf')$,
          инвариант выполняется.

        \item $\invState(C, \TSf')$:
          следует из того, что $\invState(C, \TSf)$ выполняется и, для любого $tid$, $\TSf(tid).\pstate = \TSf'(tid).\pstate$.
      \end{itemize}
      %% $\tup{\TSf', M'}$ is consistent by \cref{prop:sim-cert}.

  
  {\bf Покрытие события $e$.} В этом случае $C' = C \cup \{e\}, \IssuedSet' = \IssuedSet$.
  %% \begin{itemize}
    %% \item $C' = C \cup \{e\}, \IssuedSet' = \IssuedSet$, and $e \nin \dom{\lRMW}$: \\
      Т.к. $\invState(C, \TSf)$ выполняется, существуют такие $t$ и $\pstate'$, что $t \approx \lLAB(e)$.
      Рассмотрим варианты $e$.
      \begin{itemize}
        \item $e \in \lDMBLD$.
          В этом случае $\labelF(t) = \fenceLbl{\acq}$ по $\invState(C, \TSf)$.
          \[\begin{array}{l@{~}l}
            \View'    & \defeq \tup{\View.\viewAcq, \View.\viewAcq, \View.\viewRel} \\
            \TSf', M' & \defeq \TSf[tid \mapsto \tup{\pstate', \View', \PromSet}], M \\
          \end{array}\]
      Проверим, что $\simRel(C', \IssuedSet, \TSf', M)$ выполняется.
      \begin{itemize}
        \item $\invMemOne(C', \IssuedSet, \TSf', M) \land \invMemTwo(C', \IssuedSet, M)$:
          выполняется, т.к. $e \nin \lW$ и $\simRel(C, \IssuedSet, \TSf, M)$ выполняется.

        \item $\invViewRel(\TSf')$: выполняется, т.к. $\invViewRel(\TSf)$ выполняется и\\
          $\TSf'(tid).\{\View.\viewRel, \PromSet\} = \TSf(tid).\{\View.\viewRel, \PromSet\}$.

        \item $\invView(C', \TSf')$:
          \[\inarr{
  \forall e' \in \nextset(G, C'). \; \letdef{\tup{\viewCur, \viewAcq, \viewfRel}}{\TSf'(\lTID(e')).\View}\\
  \quad
  \begin{array}{@{}r@{~}l@{~}l}
    \viewCur  & \le \domView(\curRel; [e']) & \land {} \\
    \viewAcq  & \le \domView(\acqRel; [e']) & \land {} \\
    \viewfRel & \le \domView(\relRel; [e']). \\
  \end{array}
          }\]
          Зафиксируем $e'$. Если $\lTID(e') \neq tid$, то утверждение следует из $\invView(C, \TSf)$.
          Предположим, что $\lTID(e') = tid$. Тогда $\imm{\lPO}(e, e')$, т.к. $e \in \nextset(G, C)$.
          Мы знаем, что $\View.\viewAcq \le \domView(\acqRel; \lPO; [e])$, т.к. $\invView(C, \TSf)$ выполняется.
          Нам нужно показать, что
          \[\inarr{
  \begin{array}{@{}r@{~}l@{~}l}
    \View.\viewAcq  & \le \domView(\curRel; [e']) & \land {} \\
    \View.\viewAcq  & \le \domView(\acqRel; [e']) & \land {} \\
    \View.\viewfRel & \le \domView(\relRel; [e']). \\
  \end{array}
          }\]
          Т.к. $\dom{\acqRel; [e]} \suq \dom{\acqRel; [e']}$
          и $\acqRel;[e];\lPO;[e'] \suq \curRel; [e']$, утверждение выполняется.
        \item $\invState(C', \TSf')$:
          очевидно следует из $\invState(C, \TSf)$ и определений $C', \TSf'$.
      \end{itemize}

        \item $e \in \lDMBSY$. В этом случае $\labelF(t) = \fenceLbl{\rel}$ по $\invState(C, \TSf)$.
          \[\begin{array}{l@{~}l}
            \View'   & \defeq \tup{\viewCur, \viewAcq, \viewCur} \\
            \TSf'    & \defeq \TSf[tid \mapsto \tup{\pstate', \View', \PromSet}] \\
            M' & \defeq M \\
          \end{array}\]
          Не существует $w \in \IssuedSet$ такого, что $\lPO(e, w)$. Иначе не
          это противоречило бы $w \in \issuable(G, C, \IssuedSet)$ по лемме \ref{prop:trav-prop-preserve} 
          Из этого следует, что не существует $w \in \IssuedSet \setminus C$ такого, что $\lTID(w) = tid$, и
          $\PromSet = \emptyset$ по $\invMemOne(C, \IssuedSet, \TSf, M)$.
          
      Проверим $\simRel(C', \IssuedSet, \TSf', M)$.
      \begin{itemize}
        \item $\invMemOne(C', \IssuedSet, \TSf', M) \land \invMemTwo(C', \IssuedSet, M)$:
          выполняется, т.к. $e \nin \lW$ и $\simRel(C, \IssuedSet, \TSf, M)$ выполняется.
        \item $\invViewRel(\TSf')$: \\
          Зафиксируем поток $tid'$. Если $tid' \neq tid$, то утверждение выполняется по $\invViewRel(\TSf)$.
          Если $tid' = tid$, то утверждение выполняется, т.к. $\TSf'(tid).\PromSet = \TSf(tid).\PromSet = \emptyset$.
        \item $\invView(C', \TSf')$:
          \[\inarr{
  \forall e' \in \nextset(G, C'). \; \letdef{\tup{\viewCur, \viewAcq, \viewfRel}}{\TSf'(\lTID(e')).\View}\\
  \quad
  \begin{array}{@{}r@{~}l@{~}l}
    \viewCur  & \le \domView(\curRel; [e']) & \land {} \\
    \viewAcq  & \le \domView(\acqRel; [e']) & \land {} \\
    \viewfRel & \le \domView(\relRel; [e']). \\
  \end{array}
          }\]
          Зафиксируем $e'$. Если $\lTID(e') \neq tid$, то утверждение следует из $\invView(C, \TSf)$.
          Если $\lTID(e') = tid$, то $\imm{\lPO}(e, e')$, т.к. $e \in \nextset(G, C)$.
          Нам нужно показать, что
          \[\inarr{
  \begin{array}{@{}r@{~}l@{~}l}
    \View.\viewCur & \le \domView(\curRel; [e']) & \land {} \\
    \View.\viewAcq & \le \domView(\acqRel; [e']) & \land {} \\
    \View.\viewCur & \le \domView(\relRel; [e']). \\
  \end{array}
          }\]
          Т.к. $\dom{\curRel; [e]} \suq \dom{\curRel; [e']}$
          и $\relRel;[e];\lPO;[e'] \suq \curRel; [e']$, утверждение выполняется.
          
        \item $\invState(C', \TSf')$:
          очевидно следует из $\invState(C, \TSf)$ и определений $C', \TSf'$.
      \end{itemize}
        \item $e \in \lR$. 
          В этом случае $\labelF(t) = \readLbl{\loc}{\val}$ по $\invState(C, \TSf)$.
          \[\begin{array}{l@{~}l}
            \loc, \val       & \defeq \lLOC(e), \lVALR(e) \\
            {\pstate, \View, \PromSet} & \defeq \TSf(tid) \\
          \end{array}\]
          Т.к. $e \in \lR \cap \coverable(G, C, \IssuedSet)$ из определения $\travConfigStep$,
          существует событие записи $w \in \IssuedSet \cap \dom{\lRF; [e]}$. По $\invMemOne(C, \IssuedSet, \TSf, M)$
          существует фронт $view$ такой, что $msg \defeq \msg{\loc}{\val}{T(w)}{view} \in M$.
          $\View.\viewCur \le T(w)$ по лемме \ref{lem:curset-acyclic}.

          \[\begin{array}{l@{~}l}
            \View'   & \defeq \tup{\View.\viewCur \sqcup [\loc @ T(w)], \View.\viewAcq \sqcup view, \View.\viewRel} \\
            \TSf'    & \defeq \TSf[tid \mapsto \tup{\pstate', \View', \PromSet}] \\
          \end{array}\]
          Нужно проверить $\simRel(C', \IssuedSet, \TSf', M)$.
        \begin{itemize}
        \item $\invMemOne(C', \IssuedSet, \TSf', M) \land \invMemTwo(C', \IssuedSet, M)$:
          выполняется, т.к. $e \nin \lW$ и $\simRel(C, \IssuedSet, \TSf, M)$ выполняется.

        \item $\invViewRel(\TSf')$: выполняется, т.к. $\invViewRel(\TSf)$ выполняется и\\
          $\TSf'(tid).\{\View.\viewRel, \PromSet\} = \TSf(tid).\{\View.\viewRel, \PromSet\}$.

        \item $\invView(C', \TSf')$:
          \[\inarr{
  \forall e' \in \nextset(G, C'). \; \letdef{\tup{\viewCur, \viewAcq, \viewfRel}}{\TSf'(\lTID(e')).\View}\\
  \quad
  \begin{array}{@{}r@{~}l@{~}l}
    \viewCur  & \le \domView(\curRel; [e']) & \land {} \\
    \viewAcq  & \le \domView(\acqRel; [e']) & \land {} \\
    \viewfRel & \le \domView(\relRel; [e']). \\
  \end{array}
          }\]
          Зафиксируем $e'$. Если $\lTID(e') \neq tid$, то утверждение следует из $\invView(C, \TSf)$.
          Предположим, что $\lTID(e') = tid$. Тогда $\imm{\lPO}(e, e')$, т.к. $e \in \nextset(G, C)$.
          %% Мы знаем, что $\View.\viewAcq \le \domView(\acqRel; \lPO; [e])$, т.к. $\invView(C, \TSf)$ выполняется.
          Нам нужно показать, что
          \[\inarr{
  \begin{array}{@{}l@{~}l@{~}l}
    \View.\viewCur \sqcup [\loc @ T(w)] & \le \domView(\curRel; [e']) & \land {} \\
    \View.\viewAcq \sqcup view & \le \domView(\acqRel; [e']) & \land {} \\
    \View.\viewfRel & \le \domView(\relRel; [e']). \\
  \end{array}
          }\]
          %% $\View.\viewCur \le \domView(\curRel; [e]) \le \domView(\curRel; [e'])$ по $\invView(C, \TSf')$.
          Т.к. $\tup{w, e'} \in \curRel$, $[\loc @ T(w)] \le \domView(\curRel; [e'])$.
          Из $\invViewRel(\TSf)$ следует, что $view = [\loc @ T(w)] \sqcup \TSf(\lTID(w)).\View.\viewfRel$.
          Т.к. $\tup{w, e'} \in \acqRel$, $[\loc @ T(w)] \le \domView(\acqRel; [e'])$.
          Т.к. $\TSf(\lTID(w)).\View.\viewfRel \le \domView(\relRel; [w])$ и
          $\dom{\relRel; [w]} \suq \dom{\acqRel; [e']}$, утверждение выполняется.
        \item $\invState(C', \TSf')$:
          очевидно следует из $\invState(C, \TSf)$ и определений $C', \TSf'$.
      \end{itemize}

        \item $e \in \lW$.
          В этом случае $\labelF(t) = \writeLbl{\loc}{\val}$ по $\invState(C, \TSf)$.
          \[\begin{array}{l@{~}l}
            \loc, \val, \tau        & \defeq \lLOC(e), \lVALW(e), T(e) \\
            \tup{\pstate, \View, \PromSet} & \defeq \TSf(tid) \\
            \tup{\viewCur, \viewAcq, \viewRel} & \defeq \View. \\
          \end{array}\]
          $e \in \IssuedSet$, т.е. $e \in \lW \cap \coverable(G, C, \IssuedSet)$.
          Из $\invMemOne(C, \IssuedSet, \TSf, M)$ следует, что существует $view$ такое, что
          $msg \defeq \msg{\loc}{\val}{\tau}{view} \in M$.
          $\viewCur(\loc) < \tau$ следует по лемме \ref{lem:curset-acyclic}.
          \[\begin{array}{l@{~}l}
            \viewCur', \viewAcq' & \defeq \viewCur \sqcup [\loc @ \tau], \viewAcq \sqcup [\loc @ \tau] \\
            \View'    & \defeq \tup{\viewCur', \viewAcq', \viewRel} \\
            \TSf'     & \defeq \TSf[tid \mapsto \tup{\pstate', \View', \PromSet \setminus msg}] \\
          \end{array}\]
          $view = \viewRel$ по $\invViewRel(\TSf)$.

          Нужно проверить $\simRel(C', \IssuedSet, \TSf', M)$.
        \begin{itemize}
        \item $\invMemOne(C', \IssuedSet, \TSf', M) \land \invMemTwo(C', \IssuedSet, M)$:
          выполняется, т.к. мы добавили $e$ во множество покрытых событий и убрали $msg$ из множества
          обещанных сообщений потока $tid$.

        \item $\invViewRel(\TSf')$: выполняется, т.к. $\invViewRel(\TSf)$ выполняется, \\
          $\TSf'(tid).\View.\viewRel = \TSf(tid).\View.\viewRel$ и
          $\TSf'(tid).\PromSet \subset \TSf(tid).\PromSet$.

        \item $\invView(C', \TSf')$:
          \[\inarr{
  \forall e' \in \nextset(G, C'). \; \letdef{\tup{\viewCur, \viewAcq, \viewfRel}}{\TSf'(\lTID(e')).\View}\\
  \quad
  \begin{array}{@{}r@{~}l@{~}l}
    \viewCur  & \le \domView(\curRel; [e']) & \land {} \\
    \viewAcq  & \le \domView(\acqRel; [e']) & \land {} \\
    \viewfRel & \le \domView(\relRel; [e']). \\
  \end{array}
          }\]
          Зафиксируем $e'$. Если $\lTID(e') \neq tid$, то утверждение следует из $\invView(C, \TSf)$.
          Предположим, что $\lTID(e') = tid$. Тогда $\imm{\lPO}(e, e')$, т.к. $e \in \nextset(G, C)$.
          %% Мы знаем, что $\View.\viewAcq \le \domView(\acqRel; \lPO; [e])$, т.к. $\invView(C, \TSf)$ выполняется.
          Нам нужно показать, что
          \[\inarr{
  \begin{array}{@{}l@{~}l@{~}l}
    \View.\viewCur \sqcup [\loc @ \tau] & \le \domView(\curRel; [e']) & \land {} \\
    \View.\viewAcq \sqcup [\loc @ \tau] & \le \domView(\acqRel; [e']) & \land {} \\
    \View.\viewfRel & \le \domView(\relRel; [e']). \\
  \end{array}
          }\]
          Т.к. $\tup{w, e'} \in \curRel \suq \acqRel$,
          $[\loc @ \tau] \le \domView(\curRel; [e'])$
          ${} \le \domView(\acqRel; [e'])$.
        \item $\invState(C', \TSf')$:
          очевидно следует из $\invState(C, \TSf)$ и определений $C', \TSf'$.
      \end{itemize}

      \end{itemize}

      %% $\tup{\TSf', M'}$ is consistent by \cref{prop:sim-cert}.

  %% \end{itemize}
\end{proof}

\begin{lemma}
  \label{lem:curset-acyclic}
  Для любой локации $\loc$ выполняется $[\lW_{\loc}]; \curRel; [\lW_{\loc}] \suq \lCO$.
\end{lemma}
\begin{proof}
  Зафиксируем $\tup{w, w'} \in [\lW_{\loc}]; \curRel; [\lW_{\loc}]$. Тогда по определению $\lCO$,
  либо $\lCO(w, w')$, либо $\lCO(w', w)$. Если выполняется первое, то выполняется утверждение.
  Пусть выполняется второе.
  \[\inarr{
    {} [\lW_{\loc}]; \curRel; [\lW_{\loc}] = \\
    {} [\lW_{\loc}]; \lRF^{?}; (\lPO \cup \lSW)^{+}; [\lW_{\loc}] = \\
    \quad \text{(по транзитивности $\lPO$ и определению $\lSW$)} \\
    {} [\lW_{\loc}]; \lRF^{?}; \lPO; (\lSW; \lPO)^{*}; [\lW_{\loc}] = \\
    {} [\lW_{\loc}]; \lRF^{?}; \lPO; [\lW_{\loc}] \cup {} \\
    {} [\lW_{\loc}]; \lRFE^{?}; \lPO; (\lSW; \lPO)^{+}; [\lW_{\loc}] \cup {} \\
  }\]
  $\tup{w, w'} \nin [\lW_{\loc}]; \lRF^{?}; \lPO; [\lW_{\loc}]$, т.к. $\tup{w', w} \in \lCO$ и выполняется \ref{ax:internal}.
  
  Введем вспомогательное отношение $\lEORD \defeq (\lOBS \cup \lDOB \cup \lBOB)^{+}$,
  которое антирефлексивно по определению \ARM-согласованности (\ref{ax:external}).
  Из определений отношений следует, что $\lPO^{?}; \lSW; \lPO \suq \lBOB^{?}; \lBOB \cup \lBOB^{?}; \lBOB; \lRFE; \lBOB \suq \lEORD$.
  По транзитивности $\lEORD$, $\tup{w, w'} \in \lEORD$.
  
  Мы знаем, что $\tup{w', w} \in \lCO$. Есть два варианта: $\tup{w', w} \in \lCOE$ или $\tup{w', w} \in \lCOI$.
  Первый вариант противоречит ацикличности $\lEORD$, т.к. $\lCOE \suq \lOBS$. Опровергнем второй вариант, показав, что
  $(\lEORD \cup \lCOI)^{+} = (\lOBS \cup \lDOB \cup \lBOB \cup \lCOI)^{+}$ антирефлексивно.

  Предположим обратное и рассмотрим цикл минимальной длины, в котором точно есть $\lCOI$ по антирефлексивности $\lEORD$.
  Рассмотрим предыдущее ребро и покажем, что во всех случаях цикл можно укоротить.  
  \begin{itemize}
    \item $\lCOI; \lCOI \suq \lCOI$
    \item $\lOBS; \lCOI \suq \lOBS$: $\quad \lRFE; \lCOI = \emptyset \quad \lFRE; \lCOI \suq \lFRE \quad \lCOE; \lCOI \suq \lCOE$
    \item $\lDOB; \lCOI \suq \lDOB$:
      \begin{itemize}
        \item $\lADDR; \lPO^{?}; \lCOI \suq \lDOB \quad \lDATA; \lCOI \suq \lDOB$
        \item $(\lADDR \cup \lDATA);\lRFI; \lCOI = \emptyset \quad (\lCTRL \cup \lDATA); [\lW]; \lCOI^{?}; \lCOI \suq \lDOB$
      \end{itemize}
    \item $\lBOB; \lCOI \suq \lBOB$: $\lPO; \lCOI \suq \lPO$
    \qedhere
  \end{itemize}
\end{proof}

\chapter{Связь между системой переходов и вершинами в \ARM-исполнении}
\label{sec:lts-rel}

\[\inarr{
  \approx : \Label_{\Promise} \rightarrow \Label_{\ARM} \rightarrow {\rm Boolean} \\
  lbl_{\Promise} \approx lbl_{\ARM} \defeq \\
  \qquad \kw{match} \; lbl_{\Promise}, \; lbl_{\ARM} \; \kw{with} \\
  \qquad
    \begin{array}{@{}l}
      | \; \rlab{}{\loc}{\val}, \rlab{}{\loc}{\val} \;
      | \; \wlab{}{\loc}{\val}, \wlab{}{\loc}{\val} \\
      | \; \flab{\rel}, \flab{\SY} \;
      | \; \flab{\acq}, \flab{\LD} \rightarrow {\rm true} \\
      | \; \_, \_ \rightarrow {\rm false}
    \end{array} \\
  \\
  \nthf : \{A : Set\} \rightarrow \Pset(A \times A) \rightarrow \mathbb{N} \rightarrow \Pset(A) \\
  \nthf \; rel \; n \defeq \codom{rel^{n}} \setminus \codom{rel^{n + 1}}.
}\]

\begin{theorem}
\[\inarr{
\forall \Carm. \\
\quad (\forall \tup{set, lbl, \lPO, \_, \_, \_} \in \cmdsToVrtxs \; \Carm. \\
\qquad \exists path \in \cmdsToLbls \; \Carm. \; \forall n \in \mathbb{N}, a \in \nthf \; \lPO \; n. \\
\qquad \quad \exists k \in \mathbb{N}. \; path[n + k] \approx lbl \; a) \land {} \\
\quad (\forall path \in \cmdsToLbls \; \Carm. \\
\qquad \exists \tup{set, lbl, \lPO, \_, \_, \_} \in \cmdsToVrtxs \; \Carm. \;
  \forall n \in \mathbb{N}. \\
\qquad \quad path[n] \neq \epsilon \Rightarrow \exists k \in \mathbb{N}, a \in \nthf \; \lPO \; (n - k). \; path[n] \approx lbl \; a).
}\]
\end{theorem}
\begin{proof}
  Верно по определению функций $\instToLbl$ и $\instToVrtx$.
\end{proof}
