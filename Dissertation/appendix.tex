\appendix
%%% Оформление заголовков приложений ближе к ГОСТ:
\setlength{\midchapskip}{20pt}
%% \renewcommand*{\afterchapternum}{\par\nobreak\vskip \midchapskip}
\renewcommand\thechapter{\Asbuk{chapter}} % Чтобы приложения русскими буквами нумеровались
   % Предварительные настройки для правильного подключения Приложений
%% \begin{figure*}[t]

\newcommand{\litmusTestStart}[3]{
\begin{minipage}[t]{0.2\linewidth}
\textbf{#1} \\
Fully Supported: #2 \\
Requires: #3\\
\end{minipage}
}
\newcommand{\litmusTestEnd}{
\vspace{.2cm}
\hrule
\vspace{.2cm}
}

\newcommand{\tickP}{\checkmark}%
\newcommand{\tickPP}{\checkmark}%

\newcommand{\sbTemplate}[6]{
\begin{minipage}[t]{0.3\linewidth}
\vspace{-.2cm}
\begin{equation*}
\begin{tabular}{c}
  $\writeInstParam{#1}{x}{0}; \writeInstParam{#2}{y}{0};$ \\
\begin{tabular}{L || L}
  \writeInstParam{#3}{x}{1}; & \writeInstParam{#5}{y}{1}; \\
  \readInstParam{#4}{a}{y} & \readInstParam{#6}{b}{x} \\
\end{tabular}
\end{tabular}
\end{equation*}
\end{minipage}
}

\newcommand{\lbTemplate}[6]{
\begin{minipage}[t]{0.3\linewidth}
\vspace{-.2cm}
\begin{equation*}
\begin{tabular}{c}
  $\writeInstParam{#1}{x}{0}; \writeInstParam{#2}{y}{0};$ \\
\begin{tabular}{L || L}
  \readInstParam{#3}{a}{y} & \readInstParam{#5}{b}{x} \\
  \writeInstParam{#4}{x}{1}; & \writeInstParam{#6}{y}{1}; \\
\end{tabular}
\end{tabular}
\end{equation*}
\end{minipage}
}

\chapter{Каталог тестов для модели C/C++11}
\label{sec:litmusTests}

\section{Store Buffering (SB)}
\label{app:sb}

\litmusTestStart{SB\_rel+acq}{\tick}{History + Viewfronts}
%
\begin{minipage}[t]{0.3\linewidth}
Possible outcomes:\\
\[\begin{array}{l}
a = 0 \land b = 0\\
a = 0 \land b = 1\\
a = 1 \land b = 0\\
a = 1 \land b = 1\\
\end{array}\]
\end{minipage}
%
\sbTemplate{\rel}{\rel}{\rel}{\acq}{\rel}{\acq}
\litmusTestEnd


\litmusTestStart{SB\_sc}{\tick}{SC + History + Viewfronts}
\begin{minipage}[t]{0.3\linewidth}
Forbidden outcomes:\\
\[\begin{array}{l}
a = 0 \land b = 0\\
\end{array}\]
\end{minipage}
%
\sbTemplate{\sco}{\sco}{\sco}{\sco}{\sco}{\sco}
\litmusTestEnd

\litmusTestStart{SB\_sc+rel}{\tick}{SC + History + Viewfronts}
\begin{minipage}[t]{0.3\linewidth}
Possible outcomes:\\
\[\begin{array}{l}
a = 0 \land b = 0\\
a = 0 \land b = 1\\
a = 1 \land b = 0\\
a = 1 \land b = 1\\
\end{array}\]
\end{minipage}
%
\sbTemplate{\sco}{\sco}{\rel}{\sco}{\sco}{\sco}
\litmusTestEnd

\litmusTestStart{SB\_sc+acq}{\tick}{SC + History + Viewfronts}
\begin{minipage}[t]{0.3\linewidth}
Possible outcomes:\\
\[\begin{array}{l}
a = 0 \land b = 0\\
a = 0 \land b = 1\\
a = 1 \land b = 0\\
a = 1 \land b = 1\\
\end{array}\]
\end{minipage}
%
\sbTemplate{\sco}{\sco}{\sco}{\acq}{\sco}{\sco}
\litmusTestEnd

%\newpage

\section{Load Buffering (LB)}
\label{app:lb}

\litmusTestStart{LB\_rlx}{\tick}{Postponed Reads + History + Viewfronts}
\begin{minipage}[t]{0.3\linewidth}
Possible outcomes:\\
\[\begin{array}{l}
a = 0 \land b = 0\\
a = 0 \land b = 1\\
a = 1 \land b = 0\\
a = 1 \land b = 1\\
\end{array}\]
\end{minipage}
%
\lbTemplate{\rlx}{\rlx}{\rlx}{\rlx}{\rlx}{\rlx}
\litmusTestEnd

\litmusTestStart{LB\_rel+rlx}{\tick}{Postponed Reads + History + Viewfronts}
\begin{minipage}[t]{0.3\linewidth}
Possible outcomes:\\
\[\begin{array}{l}
a = 0 \land b = 0\\
a = 0 \land b = 1\\
a = 1 \land b = 0\\
a = 1 \land b = 1\\
\end{array}\]
\end{minipage}
%
\lbTemplate{\rlx}{\rlx}{\rlx}{\rel}{\rlx}{\rel}
\litmusTestEnd

\litmusTestStart{LB\_acq+rlx}{\fail}{Postponed Reads + History + Viewfronts}
\begin{minipage}[t]{0.3\linewidth}
Possible outcomes:\\
\[\begin{array}{l}
a = 0 \land b = 0\\
a = 0 \land b = 1\\
a = 1 \land b = 0\\
a = 1 \land b = 1\\
\end{array}\]
\end{minipage}
%
\lbTemplate{\rlx}{\rlx}{\acq}{\rlx}{\acq}{\rlx}

Our semantics doesn't allow the $a = 1 \land b = 1$ outcome for the program.
It doesn't allow reordering of an acquire read with a subsequent write.
The known sound compilation schemes of acquire read to major platforms (x86, ARM, Power) don't
allow the behavior either. 

\litmusTestEnd

\litmusTestStart{LB\_rel+acq+rlx}{\tick}{Postponed Reads + History + Viewfronts}
\begin{minipage}[t]{0.3\linewidth}
Forbidden outcomes:\\
\[\begin{array}{l}
a = 1 \land b = 1\\
\end{array}\]
\end{minipage}
%
\lbTemplate{\rlx}{\rlx}{\acq}{\rlx}{\rlx}{\rel}
\litmusTestEnd

\litmusTestStart{LB\_rlx+use}{\tick}{Postponed Reads + History + Viewfronts}
\begin{minipage}[t]{0.3\linewidth}
Allowed outcome:\\
\lstinline{  r1 = 1 |$\land$| r2 = 1}\\
\end{minipage}
%
\begin{minipage}[t]{0.3\linewidth}
\vspace{-.2cm}
  \begin{tabular}{l@{\ \ \ }l}
    \begin{minipage}[l]{4.3cm} \small
\begin{lstlisting}
  |$[$|x|$]_{rlx}$| := 0; |$[$|y|$]_{rlx}$| := 0;
\end{lstlisting}
\vspace{-.2cm}
\begin{tabular}{l||l}
\begin{lstlisting}
r1 = |$[$|y|$]_{rlx}$|;
|$[$|z1|$]_{rlx}$| := r1;
|$[$|x|$]_{rlx}$| := 1
\end{lstlisting}
\hspace{.6cm}
&
\begin{lstlisting}
r2 = |$[$|x|$]_{rlx}$|;
|$[$|z2|$]_{rlx}$| := r2;
|$[$|y|$]_{rlx}$| := 1
\end{lstlisting}
\end{tabular}
    \end{minipage}
&
  \end{tabular}
\end{minipage}

%% Comment:
%% Unfortunately, our semantics isn't able to reorder writes,
%% thus, we can't store 1s to z1 and z2.
\litmusTestEnd

\litmusTestStart{LB\_rlx+let}{\tick}{Postponed Reads + History + Viewfronts}
\begin{minipage}[t]{0.3\linewidth}
Allowed outcome:\\
\lstinline{  r1 = 1 |$\land$| r'1 = 2 |$\land$| r2 = 1 |$\land$| r'2 = 2}\\
\end{minipage}
%
\begin{minipage}[t]{0.3\linewidth}
\vspace{-.2cm}
  \begin{tabular}{l@{\ \ \ }l}
    \begin{minipage}[l]{4.3cm} \small
\begin{lstlisting}
  |$[$|x|$]_{rlx}$| := 0; |$[$|y|$]_{rlx}$| := 0;
\end{lstlisting}
\vspace{-.2cm}
\begin{tabular}{l||l}
\begin{lstlisting}
r1 = |$[$|y|$]_{rlx}$|;
r'1 = r1 + 1;
|$[$|x|$]_{rlx}$| := 1
\end{lstlisting}
\hspace{.6cm}
&
\begin{lstlisting}
r2 = |$[$|x|$]_{rlx}$|;
r'2 = r2 + 1;
|$[$|y|$]_{rlx}$| := 1
\end{lstlisting}
\end{tabular}
    \end{minipage}
&
  \end{tabular}
\end{minipage}
\litmusTestEnd

\litmusTestStart{LB\_rlx+join}{\tickPP}{Postponed Reads + History + Viewfronts + JN}
\begin{minipage}[t]{0.2\linewidth}
Allowed outcomes:\\
\lstinline{  r1 = 1 |$\land$| r2 = 1}\\
\end{minipage}
%
\begin{minipage}[t]{0.4\linewidth}
\vspace{-.2cm}
  \begin{tabular}{l@{\ \ \ }l}
    \begin{minipage}[l]{4.3cm} \small
\begin{lstlisting}
            |$[$|x|$]_{rlx}$| := 0; |$[$|y|$]_{rlx}$| := 0;
\end{lstlisting}
\vspace{-.2cm}
\begin{tabular}{l||l||l||l}
\begin{lstlisting}
r1 = |$[$|y|$]_{rlx}$|;
|$[$|z1|$]_{rlx}$| := r1
\end{lstlisting}
\hspace{.6cm}
&
\begin{lstlisting}
  0
\end{lstlisting}
\hspace{.6cm}
&
\begin{lstlisting}
r2 = |$[$|x|$]_{rlx}$|;
|$[$|z2|$]_{rlx}$| := r2
\end{lstlisting}
\hspace{.6cm}
&
\begin{lstlisting}
  0
\end{lstlisting}
\end{tabular}

\vspace{-1pt}
\begin{tabular}{l||l}
  \begin{lstlisting}
            |$[$|x|$]_{rlx}$| := 1
  \end{lstlisting}
%% \hspace{.708cm}
\hspace{2.52em}
&
  \begin{lstlisting}
      |$[$|y|$]_{rlx}$| := 1
  \end{lstlisting}
\end{tabular}
    \end{minipage}
&
  \end{tabular}
\end{minipage}

%% Unfortunately, our semantics isn't complete for this snippet,
%% because it can't postpone write, or anything across a join point.
\litmusTestEnd

\litmusTestStart{LB\_rel+rlx+join}{\tickPP}{Postponed Reads + History + Viewfronts + JN}
\begin{minipage}[t]{0.2\linewidth}
Allowed outcomes:\\
\lstinline{  r1 = 1 |$\land$| r2 = 1}\\
\end{minipage}
%
\begin{minipage}[t]{0.4\linewidth}
\vspace{-.2cm}
  \begin{tabular}{l@{\ \ \ }l}
    \begin{minipage}[l]{4.3cm} \small
\begin{lstlisting}
            |$[$|x|$]_{rlx}$| := 0; |$[$|y|$]_{rlx}$| := 0;
\end{lstlisting}
\vspace{-.2cm}
\begin{tabular}{l||l||l||l}
\begin{lstlisting}
r1 = |$[$|y|$]_{rlx}$|;
|$[$|z1|$]_{rlx}$| := r1
\end{lstlisting}
\hspace{.6cm}
&
\begin{lstlisting}
  0
\end{lstlisting}
\hspace{.6cm}
&
\begin{lstlisting}
r2 = |$[$|x|$]_{rlx}$|;
|$[$|z2|$]_{rlx}$| := r2
\end{lstlisting}
\hspace{.6cm}
&
\begin{lstlisting}
  0
\end{lstlisting}
\end{tabular}

\vspace{-1pt}
\begin{tabular}{l||l}
  \begin{lstlisting}
            |$[$|x|$]_{rel}$| := 1
  \end{lstlisting}
%% \hspace{.708cm}
\hspace{2.52em}
&
  \begin{lstlisting}
      |$[$|y|$]_{rel}$| := 1
  \end{lstlisting}
\end{tabular}
    \end{minipage}
&
  \end{tabular}
\end{minipage}
\litmusTestEnd

\litmusTestStart{LB\_acq+rlx+join}{\fail}{Postponed Reads + History + Viewfronts + JN}
\begin{minipage}[t]{0.2\linewidth}
Allowed outcomes:\\
\lstinline{  r1 = 1 |$\land$| r2 = 1}\\
\end{minipage}
%
\begin{minipage}[t]{0.4\linewidth}
\vspace{-.2cm}
  \begin{tabular}{l@{\ \ \ }l}
    \begin{minipage}[l]{4.3cm} \small
\begin{lstlisting}
            |$[$|x|$]_{rlx}$| := 0; |$[$|y|$]_{rlx}$| := 0;
\end{lstlisting}
\vspace{-.2cm}
\begin{tabular}{l||l||l||l}
\begin{lstlisting}
r1 = |$[$|y|$]_{acq}$|;
|$[$|z1|$]_{rlx}$| := r1
\end{lstlisting}
\hspace{.6cm}
&
\begin{lstlisting}
  0
\end{lstlisting}
\hspace{.6cm}
&
\begin{lstlisting}
r2 = |$[$|x|$]_{acq}$|;
|$[$|z2|$]_{rlx}$| := r2
\end{lstlisting}
\hspace{.6cm}
&
\begin{lstlisting}
  0
\end{lstlisting}
\end{tabular}

\vspace{-1pt}
\begin{tabular}{l||l}
  \begin{lstlisting}
            |$[$|x|$]_{rlx}$| := 1
  \end{lstlisting}
%% \hspace{.708cm}
\hspace{2.52em}
&
  \begin{lstlisting}
      |$[$|y|$]_{rlx}$| := 1
  \end{lstlisting}
\end{tabular}
    \end{minipage}
&
  \end{tabular}
\end{minipage}
\litmusTestEnd

\section{Message Passing (MP)}
\label{app:mp}

\litmusTestStart{MP\_rlx+na}{\tick}{NA + History + Viewfronts}
\begin{minipage}[t]{0.3\linewidth}
Possible outcomes:\\
\lstinline{  r1 = 0}\\
\lstinline{  r1 = 5}\\
\lstinline{  stuck}\\
\end{minipage}
%
\begin{minipage}[t]{0.3\linewidth}
\vspace{-.2cm}
  \begin{tabular}{l@{\ \ \ }l}
    \begin{minipage}[l]{4.3cm} \small
\begin{lstlisting}
        |$[$|f|$]_{rlx}$| := 0; |$[$|d|$]_{na}$| := 0;
\end{lstlisting}
\vspace{-.2cm}
\begin{tabular}{l||l}
\begin{lstlisting}
|$[$|d|$]_{na}$| := 5;
|$[$|f|$]_{rlx}$| := 1
\end{lstlisting}
\hspace{.6cm}
&
\begin{lstlisting}
repeat |$[$|f|$]_{rlx}$| end;
r1 = |$[$|d|$]_{na}$|
\end{lstlisting}
\end{tabular}
    \end{minipage}
&
  \end{tabular}
\end{minipage}
\litmusTestEnd

\litmusTestStart{MP\_rel+rlx+na}{\tick}{NA + History + Viewfronts}
\begin{minipage}[t]{0.3\linewidth}
Possible outcomes:\\
\lstinline{  r1 = 0}\\
\lstinline{  r1 = 5}\\
\lstinline{  stuck}\\
\end{minipage}
%
\begin{minipage}[t]{0.3\linewidth}
\vspace{-.2cm}
  \begin{tabular}{l@{\ \ \ }l}
    \begin{minipage}[l]{4.3cm} \small
\begin{lstlisting}
        |$[$|f|$]_{rlx}$| := 0; |$[$|d|$]_{na}$| := 0;
\end{lstlisting}
\vspace{-.2cm}
\begin{tabular}{l||l}
\begin{lstlisting}
|$[$|d|$]_{na}$| := 5;
|$[$|f|$]_{rel}$| := 1
\end{lstlisting}
\hspace{.6cm}
&
\begin{lstlisting}
repeat |$[$|f|$]_{rlx}$| end;
r1 = |$[$|d|$]_{na}$|
\end{lstlisting}
\end{tabular}
    \end{minipage}
&
  \end{tabular}
\end{minipage}
\litmusTestEnd

\litmusTestStart{MP\_rlx+acq+na}{\tick}{NA + History + Viewfronts}
\begin{minipage}[t]{0.3\linewidth}
Possible outcomes:\\
\lstinline{  r1 = 0}\\
\lstinline{  r1 = 5}\\
\lstinline{  stuck}\\
\end{minipage}
%
\begin{minipage}[t]{0.4\linewidth}
\vspace{-.2cm}
  \begin{tabular}{l@{\ \ \ }l}
    \begin{minipage}[l]{4.3cm} \small
\begin{lstlisting}
        |$[$|f|$]_{rlx}$| := 0; |$[$|d|$]_{na}$| := 0;
\end{lstlisting}
\vspace{-.2cm}
\begin{tabular}{l||l}
\begin{lstlisting}
|$[$|d|$]_{na}$| := 5;
|$[$|f|$]_{rlx}$| := 1
\end{lstlisting}
\hspace{.6cm}
&
\begin{lstlisting}
repeat |$[$|f|$]_{acq}$| end;
r1 = |$[$|d|$]_{na}$|
\end{lstlisting}
\end{tabular}
    \end{minipage}
&
  \end{tabular}
\end{minipage}
\litmusTestEnd

\litmusTestStart{MP\_rel+acq+na}{\tick}{NA + History + Viewfronts}
\begin{minipage}[t]{0.3\linewidth}
Possible outcomes:\\
\lstinline{  r1 = 5}\\
\end{minipage}
%
\begin{minipage}[t]{0.3\linewidth}
\vspace{-.2cm}
  \begin{tabular}{l@{\ \ \ }l}
    \begin{minipage}[l]{4.3cm} \small
\begin{lstlisting}
        |$[$|f|$]_{rel}$| := 0; |$[$|d|$]_{na}$| := 0;
\end{lstlisting}
\vspace{-.2cm}
\begin{tabular}{l||l}
\begin{lstlisting}
|$[$|d|$]_{na}$| := 5;
|$[$|f|$]_{rel}$| := 1
\end{lstlisting}
\hspace{.6cm}
&
\begin{lstlisting}
repeat |$[$|f|$]_{acq}$| end;
r1 = |$[$|d|$]_{na}$|
\end{lstlisting}
\end{tabular}
    \end{minipage}
&
  \end{tabular}
\end{minipage}
\litmusTestEnd

\litmusTestStart{MP\_rel+acq+na+rlx}{\tick}{Write-fronts + NA + History + Viewfronts}
\begin{minipage}[t]{0.3\linewidth}
Possible outcomes:\\
\lstinline{  r1 = 5}\\
\end{minipage}
%
\begin{minipage}[t]{0.3\linewidth}
\vspace{-.2cm}
  \begin{tabular}{l@{\ \ \ }l}
    \begin{minipage}[l]{4.3cm} \small
\begin{lstlisting}
        |$[$|f|$]_{rel}$| := 0; |$[$|d|$]_{na}$| := 0;
\end{lstlisting}
\vspace{-.2cm}
\begin{tabular}{l||l}
\begin{lstlisting}
|$[$|d|$]_{na}$| := 5;
|$[$|f|$]_{rel}$| := 1;
|$[$|f|$]_{rlx}$| := 2
\end{lstlisting}
\hspace{.6cm}
&
\begin{lstlisting}
repeat |$[$|f|$]_{acq}$| == 2 end;
r1 = |$[$|d|$]_{na}$|
\end{lstlisting}
\end{tabular}
    \end{minipage}
&
  \end{tabular}
\end{minipage}
\litmusTestEnd

\litmusTestStart{MP\_rel+acq+na+rlx\_2}{\tick}{Write-fronts + NA + History + Viewfronts}
\begin{minipage}[t]{0.3\linewidth}
Possible outcomes:\\
\lstinline{  r1 = 5 /\ r2 = <0, 1>}\\
\end{minipage}
%
\begin{minipage}[t]{0.3\linewidth}
\vspace{-.2cm}
  \begin{tabular}{l@{\ \ \ }l}
    \begin{minipage}[l]{4.3cm} \small
\begin{lstlisting}
|$[$|f|$]_{na}$| := 0; |$[$|d|$]_{na}$| := 0; |$[$|x|$]_{na}$| := 0;
\end{lstlisting}
\vspace{-.2cm}
\begin{tabular}{l||l}
\begin{lstlisting}
|$[$|d|$]_{na}$| := 5;
|$[$|f|$]_{rel}$| := 1;
|$[$|x|$]_{rel}$| := 1;
|$[$|f|$]_{rlx}$| := 2
\end{lstlisting}
\hspace{.6cm}
&
\begin{lstlisting}
repeat |$[$|f|$]_{acq}$| == 2 end;
r1 := |$[$|d|$]_{na}$|;
r2 := |$[$|x|$]_{rlx}$|
\end{lstlisting}
\end{tabular}
    \end{minipage}
&
  \end{tabular}
\end{minipage}
\litmusTestEnd

\litmusTestStart{MP\_con+na}{\tick}{Consume + NA + History + Viewfronts}
\begin{minipage}[t]{0.3\linewidth}
Possible outcomes:\\
\lstinline{  r1 = 0}\\
\lstinline{  r1 = 5}\\
\end{minipage}
%
\begin{minipage}[t]{0.3\linewidth}
\vspace{-.2cm}
  \begin{tabular}{l@{\ \ \ }l}
    \begin{minipage}[l]{4.3cm} \small
\begin{lstlisting}
        |$[$|f|$]_{con}$| := null; |$[$|d|$]_{na}$| := 0;
\end{lstlisting}
\vspace{-.2cm}
\begin{tabular}{l||l}
\begin{lstlisting}
|$[$|d|$]_{na}$| := 5;
|$[$|f|$]_{rel}$| := d
\end{lstlisting}
\hspace{.6cm}
&
\begin{lstlisting}
r0 := |$[$|f|$]_{con}$|;
if r0 != null
then r1 = |$[$|r0|$]_{na}$|
else r1 = 0
fi
\end{lstlisting}
\end{tabular}
    \end{minipage}
&
  \end{tabular}
\end{minipage}
\litmusTestEnd

\litmusTestStart{MP\_con+na\_2}{\tick}{Consume + NA + History + Viewfronts}
\begin{minipage}[t]{0.3\linewidth}
Possible outcomes:\\
\lstinline{  r2 = 0 /\ r3 = <0, 1>}\\
\lstinline{  r2 = 5 /\ r3 = <0, 1>}\\
\end{minipage}
%
\begin{minipage}[t]{0.3\linewidth}
\vspace{-.2cm}
  \begin{tabular}{l@{\ \ \ }l}
    \begin{minipage}[l]{4.3cm} \small
\begin{lstlisting}
|$[$|p|$]_{na}$| := null; |$[$|d|$]_{na}$| := 0; |$[$|x|$]_{na}$| := 0;
\end{lstlisting}
\vspace{-.2cm}
\begin{tabular}{l||l}
\begin{lstlisting}
|$[$|x|$]_{rlx}$| := 1;
|$[$|d|$]_{na}$| := 1;
|$[$|p|$]_{rel}$| := d
\end{lstlisting}
\hspace{.6cm}
&
\begin{lstlisting}
r1 = |$[$|p|$]_{con}$|;
if   r1 != null
then r2 = |$[$|r1|$]_{na}$|;
     r3 = |$[$|x|$]_{rlx}$|
else r2 = 0; r3 = 0
fi
\end{lstlisting}
\end{tabular}
    \end{minipage}
&
  \end{tabular}
\end{minipage}
\litmusTestEnd

\litmusTestStart{MP\_cas+rel+acq+na from \cite{Vafeiadis-Narayan:OOPSLA13}}{\tick}{NA + History + Viewfronts}
\begin{minipage}[t]{0.2\linewidth}
Impossible outcomes:\\
\lstinline{  stuck}\\
\end{minipage}
%
\begin{minipage}[t]{0.4\linewidth}
\vspace{-.2cm}
  \begin{tabular}{l@{\ \ \ }l}
    \begin{minipage}[l]{4.3cm} \small
\begin{lstlisting}
        |$[$|f|$]_{rlx}$| := 1; |$[$|d|$]_{na}$| := 0;
\end{lstlisting}
\vspace{-.2cm}
\begin{tabular}{l||l@{\ \ \ \ }||l}
\begin{lstlisting}
|$[$|d|$]_{na}$| := 5;
|$[$|f|$]_{rel}$| := 0
\end{lstlisting}
\hspace{.6cm}
&
\begin{lstlisting}
r1 = cas|$_{acq,rlx}$|(f, 0, 1);
if r1 == 0
then |$[$|d|$]_{rlx}$| := 6
else 0
fi
\end{lstlisting}
\hspace{.6cm}
&
\begin{lstlisting}
r2 = cas|$_{acq,rlx}$|(f, 0, 1);
if r2 == 0
then |$[$|d|$]_{rlx}$| := 7
else 0
fi
\end{lstlisting}
\end{tabular}
    \end{minipage}
&
  \end{tabular}
\end{minipage}
\litmusTestEnd

\litmusTestStart{MP\_cas+rel+rlx+na}{\tick}{NA + History + Viewfronts}
\begin{minipage}[t]{0.2\linewidth}
Possible outcomes:\\
\lstinline{  stuck}\\
\end{minipage}
%
\begin{minipage}[t]{0.4\linewidth}
\vspace{-.2cm}
  \begin{tabular}{l@{\ \ \ }l}
    \begin{minipage}[l]{4.3cm} \small
\begin{lstlisting}
        |$[$|f|$]_{rlx}$| := 1; |$[$|d|$]_{na}$| := 0;
\end{lstlisting}
\vspace{-.2cm}
\begin{tabular}{l||l||l}
\begin{lstlisting}
|$[$|d|$]_{na}$| := 5
|$[$|f|$]_{rel}$| := 0;
\end{lstlisting}
\hspace{.6cm}
&
\begin{lstlisting}
r1 = cas|$_{rlx,rlx}$|(f, 0, 1);
if r1 == 0
then |$[$|d|$]_{rlx}$| := 6
else 0
fi
\end{lstlisting}
\hspace{.6cm}
&
\begin{lstlisting}
r2 = cas|$_{rlx,rlx}$|(f, 0, 1);
if r2 == 0
then |$[$|d|$]_{rlx}$| := 7
else 0
fi
\end{lstlisting}
\end{tabular}
    \end{minipage}
&
  \end{tabular}
\end{minipage}
\litmusTestEnd

\section{Coherence of Read-Read (CoRR)}
\label{app:corr}

\litmusTestStart{CoRR\_rlx}{\tick}{History + Viewfronts}
\begin{minipage}[t]{0.3\linewidth}
Impossible outcomes:\\
\lstinline{  r1 = 1 |$\land$| r2 = 2 |$\land$| r3 = 2 |$\land$| r4 = 1}\\
\lstinline{  r1 = 2 |$\land$| r2 = 1 |$\land$| r3 = 1 |$\land$| r4 = 2}\\
\end{minipage}
%
\begin{minipage}[t]{0.3\linewidth}
\vspace{-.2cm}
  \begin{tabular}{l@{\ \ \ }l}
    \begin{minipage}[l]{4.3cm} \small
\begin{lstlisting}
                    |$[$|x|$]_{rlx}$| := 0;
\end{lstlisting}
\vspace{-.2cm}
\begin{tabular}{l||l||l||l}
\begin{lstlisting}
|$[$|x|$]_{rlx}$| := 1
\end{lstlisting}
\hspace{.6cm}
&
\begin{lstlisting}
|$[$|x|$]_{rlx}$| := 2
\end{lstlisting}
\hspace{.6cm}
&
\begin{lstlisting}
r1 = |$[$|x|$]_{rlx}$|;
r2 = |$[$|x|$]_{rlx}$|
\end{lstlisting}
\hspace{.6cm}
&
\begin{lstlisting}
r3 = |$[$|x|$]_{rlx}$|;
r4 = |$[$|x|$]_{rlx}$|
\end{lstlisting}
\end{tabular}
    \end{minipage}
&
  \end{tabular}
\end{minipage}
\litmusTestEnd

\litmusTestStart{CoRR\_rel+acq}{\tick}{History + Viewfronts}
\begin{minipage}[t]{0.3\linewidth}
Impossible outcomes:\\
\lstinline{  r1 = 1 |$\land$| r2 = 2 |$\land$| r3 = 2 |$\land$| r4 = 1}\\
\lstinline{  r1 = 2 |$\land$| r2 = 1 |$\land$| r3 = 1 |$\land$| r4 = 2}\\
\end{minipage}
%
\begin{minipage}[t]{0.3\linewidth}
\vspace{-.2cm}
  \begin{tabular}{l@{\ \ \ }l}
    \begin{minipage}[l]{4.3cm} \small
\begin{lstlisting}
                   |$[$|x|$]_{rel}$| := 0;
\end{lstlisting}
\vspace{-.2cm}
\begin{tabular}{l||l||l||l}
\begin{lstlisting}
|$[$|x|$]_{rel}$| := 1
\end{lstlisting}
\hspace{.6cm}
&
\begin{lstlisting}
|$[$|x|$]_{rel}$| := 2
\end{lstlisting}
\hspace{.6cm}
&
\begin{lstlisting}
r1 = |$[$|x|$]_{acq}$|;
r2 = |$[$|x|$]_{acq}$|
\end{lstlisting}
\hspace{.6cm}
&
\begin{lstlisting}
r3 = |$[$|x|$]_{acq}$|;
r4 = |$[$|x|$]_{acq}$|
\end{lstlisting}
\end{tabular}
    \end{minipage}
&
  \end{tabular}
\end{minipage}
\litmusTestEnd

\section{Independent Reads of Independent Writes (IRIW)}
\label{app:iriw}

\litmusTestStart{IRIW\_rlx}{\tick}{History + Viewfronts}
\begin{minipage}[t]{0.3\linewidth}
Possible outcomes:\\
\lstinline{  r1 = <0, 1>; r2 = <0, 1>;}\\
\lstinline{  r3 = <0, 1>; r4 = <0, 1>}\\
\end{minipage}
%
\begin{minipage}[t]{0.5\linewidth}
\vspace{-.2cm}
  \begin{tabular}{l@{\ \ \ }l}
    \begin{minipage}[l]{4.3cm} \small
\begin{lstlisting}
               |$[$|x|$]_{rlx}$| := 0; |$[$|y|$]_{rlx}$| := 0;
\end{lstlisting}
\vspace{-.2cm}
\begin{tabular}{l||l||l||l}
\begin{lstlisting}
|$[$|x|$]_{rlx}$| := 1
\end{lstlisting}
\hspace{.6cm}
&
\begin{lstlisting}
|$[$|y|$]_{rlx}$| := 1
\end{lstlisting}
\hspace{.6cm}
&
\begin{lstlisting}
r1 = |$[$|x|$]_{rlx}$|;
r2 = |$[$|y|$]_{rlx}$|
\end{lstlisting}
\hspace{.6cm}
&
\begin{lstlisting}
r3 = |$[$|y|$]_{rlx}$|;
r4 = |$[$|x|$]_{rlx}$|
\end{lstlisting}
\end{tabular}
    \end{minipage}
&
  \end{tabular}
\end{minipage}

Comment:
It is possible to get
\lstinline{r1 = 1; r2 = 0; r3 = 1; r4 = 0}
\litmusTestEnd

\litmusTestStart{IRIW\_rel+acq}{\tick}{History + Viewfronts}
\begin{minipage}[t]{0.3\linewidth}
Possible outcomes:\\
\lstinline{  r1 = <0, 1>; r2 = <0, 1>;}\\
\lstinline{  r3 = <0, 1>; r4 = <0, 1>}\\
\end{minipage}
%
\begin{minipage}[t]{0.5\linewidth}
\vspace{-.2cm}
  \begin{tabular}{l@{\ \ \ }l}
    \begin{minipage}[l]{4.3cm} \small
\begin{lstlisting}
               |$[$|x|$]_{rel}$| := 0; |$[$|y|$]_{rel}$| := 0;
\end{lstlisting}
\vspace{-.2cm}
\begin{tabular}{l||l||l||l}
\begin{lstlisting}
|$[$|x|$]_{rel}$| := 1
\end{lstlisting}
\hspace{.6cm}
&
\begin{lstlisting}
|$[$|y|$]_{rel}$| := 1
\end{lstlisting}
\hspace{.6cm}
&
\begin{lstlisting}
r1 = |$[$|x|$]_{acq}$|;
r2 = |$[$|y|$]_{acq}$|
\end{lstlisting}
\hspace{.6cm}
&
\begin{lstlisting}
r3 = |$[$|y|$]_{acq}$|;
r4 = |$[$|x|$]_{acq}$|
\end{lstlisting}
\end{tabular}
    \end{minipage}
&
  \end{tabular}
\end{minipage}

Comment:
It is possible to get
\lstinline{r1 = 1; r2 = 0; r3 = 1; r4 = 0}
\litmusTestEnd

\litmusTestStart{IRIW\_sc}{\tick}{SC + History + Viewfronts}
\begin{minipage}[t]{0.3\linewidth}
Forbidden outcomes:\\
\lstinline{  r1 = 1 |$\land$| r2 = 0 |$\land$| r3 = 1 |$\land$| r4 = 0}\\
\end{minipage}
%
\begin{minipage}[t]{0.5\linewidth}
\vspace{-.2cm}
  \begin{tabular}{l@{\ \ \ }l}
    \begin{minipage}[l]{4.3cm} \small
\begin{lstlisting}
               |$[$|x|$]_{sc}$| := 0; |$[$|y|$]_{sc}$| := 0;
\end{lstlisting}
\vspace{-.2cm}
\begin{tabular}{l||l||l||l}
\begin{lstlisting}
|$[$|x|$]_{sc}$| := 1
\end{lstlisting}
\hspace{.6cm}
&
\begin{lstlisting}
|$[$|y|$]_{sc}$| := 1
\end{lstlisting}
\hspace{.6cm}
&
\begin{lstlisting}
r1 = |$[$|x|$]_{sc}$|;
r2 = |$[$|y|$]_{sc}$|
\end{lstlisting}
\hspace{.6cm}
&
\begin{lstlisting}
r3 = |$[$|y|$]_{sc}$|;
r4 = |$[$|x|$]_{sc}$|
\end{lstlisting}
\end{tabular}
    \end{minipage}
&
  \end{tabular}
\end{minipage}
\litmusTestEnd

\section{Write-to-Read Causality (WRC)}
\label{app:wrc}

\litmusTestStart{WRC\_rel+acq}{\tick}{History + Viewfronts}
\begin{minipage}[t]{0.3\linewidth}
Forbidden outcomes:\\
\lstinline{  r2 = 1 |$\land$| r3 = 0}\\
\end{minipage}
%
\begin{minipage}[t]{0.3\linewidth}
\vspace{-.2cm}
  \begin{tabular}{l@{\ \ \ }l}
    \begin{minipage}[l]{4.3cm} \small
\begin{lstlisting}
        |$[$|x|$]_{rel}$| := 0; |$[$|y|$]_{rel}$| := 0;
\end{lstlisting}
\vspace{-.2cm}
\begin{tabular}{l||l||l}
\begin{lstlisting}
|$[$|x|$]_{rel}$| := 1
\end{lstlisting}
\hspace{.6cm}
&
\begin{lstlisting}
r1 = |$[$|x|$]_{acq}$|;
|$[$|y|$]_{rel}$| := r1
\end{lstlisting}
\hspace{.6cm}
&
\begin{lstlisting}
r2 = |$[$|y|$]_{acq}$|;
r3 = |$[$|x|$]_{acq}$|
\end{lstlisting}
\end{tabular}
    \end{minipage}
&
  \end{tabular}
\end{minipage}
\litmusTestEnd

\litmusTestStart{WRC\_rlx}{\tick}{History + Viewfronts}
\begin{minipage}[t]{0.3\linewidth}
Possible outcomes:\\
\lstinline{  r2 = 0 |$\land$| r3 = 0}\\
\lstinline{  r2 = 0 |$\land$| r3 = 1}\\
\lstinline{  r2 = 1 |$\land$| r3 = 0}\\
\lstinline{  r2 = 1 |$\land$| r3 = 1}\\
\end{minipage}
%
\begin{minipage}[t]{0.3\linewidth}
\vspace{-.2cm}
  \begin{tabular}{l@{\ \ \ }l}
    \begin{minipage}[l]{4.3cm} \small
\begin{lstlisting}
        |$[$|x|$]_{rlx}$| := 0; |$[$|y|$]_{rlx}$| := 0;
\end{lstlisting}
\vspace{-.2cm}
\begin{tabular}{l||l||l}
\begin{lstlisting}
|$[$|x|$]_{rlx}$| := 1
\end{lstlisting}
\hspace{.6cm}
&
\begin{lstlisting}
r1 = |$[$|x|$]_{rlx}$|;
|$[$|y|$]_{rlx}$| := r1
\end{lstlisting}
\hspace{.6cm}
&
\begin{lstlisting}
r2 = |$[$|y|$]_{rlx}$|;
r3 = |$[$|x|$]_{rlx}$|
\end{lstlisting}
\end{tabular}
    \end{minipage}
&
  \end{tabular}
\end{minipage}

%% Comment:
%% It is possible to get
\lstinline{r2 = 1; r3 = 0}
\litmusTestEnd

\litmusTestStart{WRC\_cas+rel}{\tick}{History + Viewfronts}
\begin{minipage}[t]{0.3\linewidth}
Impossible outcomes:\\
\lstinline{  r2 = 2 |$\land$| r3 = 0}\\
\end{minipage}
%
\begin{minipage}[t]{0.3\linewidth}
\vspace{-.2cm}
  \begin{tabular}{l@{\ \ \ }l}
    \begin{minipage}[l]{4.3cm} \small
\begin{lstlisting}
        |$[$|x|$]_{rel}$| := 0; |$[$|y|$]_{rel}$| := 0;
\end{lstlisting}
\vspace{-.2cm}
\begin{tabular}{l||l||l}
\begin{lstlisting}
|$[$|x|$]_{rel}$| := 1;
|$[$|y|$]_{rel}$| := 1
\end{lstlisting}
\hspace{.6cm}
&
\begin{lstlisting}
cas|$_{rel,acq}$|(y, 1, 2)
\end{lstlisting}
\hspace{.6cm}
&
\begin{lstlisting}
r1 = |$[$|y|$]_{rel}$|;
r2 = |$[$|x|$]_{rel}$|
\end{lstlisting}
\end{tabular}
    \end{minipage}
&
  \end{tabular}
\end{minipage}
\vspace{.2cm}
\hrule
\vspace{.2cm}

\litmusTestStart{WRC\_cas+rlx}{\tick}{History + Viewfronts}
\begin{minipage}[t]{0.3\linewidth}
Impossible outcomes:\\
\lstinline{  r2 = 2 |$\land$| r3 = 0}\\
\end{minipage}
%
\begin{minipage}[t]{0.3\linewidth}
\vspace{-.2cm}
  \begin{tabular}{l@{\ \ \ }l}
    \begin{minipage}[l]{4.3cm} \small
\begin{lstlisting}
        |$[$|x|$]_{rlx}$| := 0; |$[$|y|$]_{rlx}$| := 0;
\end{lstlisting}
\vspace{-.2cm}
\begin{tabular}{l||l||l}
\begin{lstlisting}
|$[$|x|$]_{rlx}$| := 1;
|$[$|y|$]_{rel}$| := 1
\end{lstlisting}
\hspace{.6cm}
&
\begin{lstlisting}
cas|$_{rlx,rlx}$|(y, 1, 2)
\end{lstlisting}
\hspace{.6cm}
&
\begin{lstlisting}
r1 = |$[$|y|$]_{rlx}$|;
r2 = |$[$|x|$]_{rlx}$|
\end{lstlisting}
\end{tabular}
    \end{minipage}
&
  \end{tabular}
\end{minipage}
\litmusTestEnd


%\newpage

\section{Out-of-Thin-Air reads}
\label{app:ota}

In our semantics it is not possible to get out-of-thin-air results,
unlike the C11 standard. But such reads are considered to be an
undesirable behavior by most of the standard's
clients~\cite{Batty-al:ESOP15}.

\litmusTestStart{OTA\_lb}{\fail}{Postponed reads + History + Viewfronts}
\begin{minipage}[t]{0.3\linewidth}
Possible outcomes:\\
\lstinline{  r1 = 0 |$\land$| r2 = 0}\\
\end{minipage}
%
\begin{minipage}[t]{0.3\linewidth}
\vspace{-.2cm}
  \begin{tabular}{l@{\ \ \ }l}
    \begin{minipage}[l]{4.3cm} \small
\begin{lstlisting}
  |$[$|x|$]_{rlx}$| := 0; |$[$|y|$]_{rlx}$| := 0;
\end{lstlisting}
\vspace{-.2cm}
\begin{tabular}{l||l}
\begin{lstlisting}
r1 = |$[$|y|$]_{rlx}$|;
|$[$|x|$]_{rlx}$| := r1
\end{lstlisting}
\hspace{.6cm}
&
\begin{lstlisting}
r2 = |$[$|x|$]_{rlx}$|;
|$[$|y|$]_{rlx}$| := r2
\end{lstlisting}
\end{tabular}
    \end{minipage}
&
  \end{tabular}
\end{minipage}

Comment: According to the C11 standard \cite{C:11,CPP:11},
\lstinline{r1} and \lstinline{r2} can get arbitrary values.
\litmusTestEnd

\litmusTestStart{OTA\_if}{\fail}{Postponed reads + History + Viewfronts}
\begin{minipage}[t]{0.4\linewidth}
Possible outcomes:\\
\lstinline{  r1 = 0 |$\land$| r2 = 0}\\
\end{minipage}
%
\begin{minipage}[t]{0.4\linewidth}
\vspace{-.2cm}
  \begin{tabular}{l@{\ \ \ }l}
    \begin{minipage}[l]{4.3cm} \small
\begin{lstlisting}
  |$[$|x|$]_{rlx}$| := 0; |$[$|y|$]_{rlx}$| := 0;
\end{lstlisting}
\vspace{-.2cm}
\begin{tabular}{l||l}
\begin{lstlisting}
r1 = |$[$|y|$]_{rlx}$|;
if r1
then |$[$|x|$]_{rlx}$| := 1
else r1 = 0 
fi
\end{lstlisting}
\hspace{.6cm}
&
\begin{lstlisting}
r2 = |$[$|x|$]_{rlx}$|;
if r2
then |$[$|y|$]_{rlx}$| := 1
else r2 = 0
fi
\end{lstlisting}
\end{tabular}
    \end{minipage}
&
  \end{tabular}
\end{minipage}

Comment: According to the C11 standard \cite{C:11,CPP:11},
\lstinline{r1} and \lstinline{r2} can be 1s at the end of execution.
\litmusTestEnd

\section{Write Reorder (WR), or 2+2W from \cite{Lahav-al:POPL16}}
\label{app:wr}

\litmusTestStart{WR\_rlx}{\tick}{History + Viewfronts + Operational Buffers}
\begin{minipage}[t]{0.3\linewidth}
Possible outcomes:\\
\lstinline{  r1 = 1 |$\land$| r2 = 2}\\
\lstinline{  r1 = 2 |$\land$| r2 = 1}\\
\lstinline{  r1 = 2 |$\land$| r2 = 2}\\
\end{minipage}
%
\begin{minipage}[t]{0.3\linewidth}
\vspace{-.2cm}
  \begin{tabular}{l@{\ \ \ }l}
    \begin{minipage}[l]{4.3cm} \small
\begin{lstlisting}
  |$[$|x|$]_{rlx}$| := 0; |$[$|y|$]_{rlx}$| := 0;
\end{lstlisting}
\vspace{-.2cm}
\begin{tabular}{l||l}
\begin{lstlisting}
|$[$|x|$]_{rlx}$| := 1;
|$[$|y|$]_{rlx}$| := 2
\end{lstlisting}
\hspace{.6cm}
&
\begin{lstlisting}
|$[$|y|$]_{rlx}$| := 1;
|$[$|x|$]_{rlx}$| := 2
\end{lstlisting}
\end{tabular}
\begin{lstlisting}
  r1 = |$[$|x|$]_{rlx}$|; r2 = |$[$|y|$]_{rlx}$|
\end{lstlisting}
    \end{minipage}
&
  \end{tabular}
\end{minipage}
\litmusTestEnd

\litmusTestStart{WR\_rlx+rel}{\tick}{History + Viewfronts + Operational Buffers}
\begin{minipage}[t]{0.3\linewidth}
Possible outcomes:\\
\lstinline{  r1 = 1 |$\land$| r2 = 2}\\
\lstinline{  r1 = 2 |$\land$| r2 = 1}\\
\lstinline{  r1 = 2 |$\land$| r2 = 2}\\
\end{minipage}
%
\begin{minipage}[t]{0.3\linewidth}
\vspace{-.2cm}
  \begin{tabular}{l@{\ \ \ }l}
    \begin{minipage}[l]{4.3cm} \small
\begin{lstlisting}
  |$[$|x|$]_{rlx}$| := 0; |$[$|y|$]_{rlx}$| := 0;
\end{lstlisting}
\vspace{-.2cm}
\begin{tabular}{l||l}
\begin{lstlisting}
|$[$|x|$]_{rlx}$| := 1;
|$[$|y|$]_{rel}$| := 2
\end{lstlisting}
\hspace{.6cm}
&
\begin{lstlisting}
|$[$|y|$]_{rlx}$| := 1;
|$[$|x|$]_{rel}$| := 2
\end{lstlisting}
\end{tabular}
\begin{lstlisting}
  r1 = |$[$|x|$]_{rlx}$|; r2 = |$[$|y|$]_{rlx}$|
\end{lstlisting}
    \end{minipage}
&
  \end{tabular}
\end{minipage}
\litmusTestEnd

\litmusTestStart{WR\_rel}{\tick}{History + Viewfronts + Operational Buffers}
\begin{minipage}[t]{0.3\linewidth}
Possible outcomes:\\
\lstinline{  r1 = 1 |$\land$| r2 = 2}\\
\lstinline{  r1 = 2 |$\land$| r2 = 1}\\
\lstinline{  r1 = 2 |$\land$| r2 = 2}\\
\end{minipage}
%
\begin{minipage}[t]{0.3\linewidth}
\vspace{-.2cm}
  \begin{tabular}{l@{\ \ \ }l}
    \begin{minipage}[l]{4.3cm} \small
\begin{lstlisting}
  |$[$|x|$]_{rel}$| := 0; |$[$|y|$]_{rel}$| := 0;
\end{lstlisting}
\vspace{-.2cm}
\begin{tabular}{l||l}
\begin{lstlisting}
|$[$|x|$]_{rel}$| := 1;
|$[$|y|$]_{rel}$| := 2
\end{lstlisting}
\hspace{.6cm}
&
\begin{lstlisting}
|$[$|y|$]_{rel}$| := 1;
|$[$|x|$]_{rel}$| := 2
\end{lstlisting}
\end{tabular}
\begin{lstlisting}
  r1 = |$[$|x|$]_{acq}$|; r2 = |$[$|y|$]_{acq}$|
\end{lstlisting}
    \end{minipage}
&
  \end{tabular}
\end{minipage}

\vspace{10pt}

%% This is an example of the program from \cite{Lahav-al:POPL16}.
%% The weak behavior (\lstinline{r1 = r2 = 1}) is allowed in C11,
%% but never observable under any known sound compilation scheme
%% of the C11 release writes.
%% Our semantics doesn't allow such behaviour as well as SRA
%% \cite{Lahav-al:POPL16}. 

\litmusTestEnd

%% \section{Value Stealing}
%% \label{app:ss}

%% \litmusTestStart{VS\_rlx}{\tick}{}
%% \begin{minipage}[t]{0.3\linewidth}
%% Possible outcomes:\\
%% \lstinline{  r1 = 0}\\
%% \lstinline{  r1 = 1}\\
%% \end{minipage}
%% %
%% \begin{minipage}[t]{0.3\linewidth}
%% \vspace{-.2cm}
%%   \begin{tabular}{l@{\ \ \ }l}
%%     \begin{minipage}[l]{4.3cm} \small
%% \begin{lstlisting}
%%         |$[$|x|$]_{rlx}$| := 0; |$[$|y|$]_{rlx}$| := 0;
%% \end{lstlisting}
%% \vspace{-.2cm}
%% \begin{tabular}{l||l||l}
%% \begin{lstlisting}
%% r1 = |$[$|x|$]_{rlx}$|;
%% |$[$|x|$]_{rlx}$| := 1
%% \end{lstlisting}
%% \hspace{.6cm}
%% &
%% \begin{lstlisting}
%% r2 = |$[$|x|$]_{rlx}$|;
%% |$[$|y|$]_{rlx}$| := r2
%% \end{lstlisting}
%% \hspace{.6cm}
%% &
%% \begin{lstlisting}
%% r3 = |$[$|y|$]_{rlx}$|;
%% |$[$|x|$]_{rlx}$| := r3
%% \end{lstlisting}
%% \end{tabular}
%%     \end{minipage}
%% &
%%   \end{tabular}
%% \end{minipage}

%% \litmusTestEnd

\section{Speculative Execution}
\label{app:se}

\litmusTestStart{SE\_simple}{\tick}{}
\begin{minipage}[t]{0.3\linewidth}
Possible outcomes:\\
\lstinline{  r0 = 0}\\
\lstinline{  r0 = 1}\\
\end{minipage}
%
\begin{minipage}[t]{0.4\linewidth}
\vspace{-.2cm}
  \begin{tabular}{l@{\ \ \ }l}
    \begin{minipage}[l]{4.3cm} \small
%
\begin{lstlisting}
|$[$|x|$]_{rlx}$| := 0; |$[$|y|$]_{rlx}$| := 0; |$[$|z|$]_{rlx}$| := 0;
\end{lstlisting}
\vspace{-.2cm}
\begin{tabular}{l||l}
\begin{lstlisting}
r1 = |$[$|x|$]_{rlx}$|;
if r1
then |$[$|z|$]_{rlx}$| := 1;
     |$[$|y|$]_{rlx}$| := 1
else |$[$|y|$]_{rlx}$| := 1
fi
\end{lstlisting}
\hspace{.6cm}
&
\begin{lstlisting}
r2 = |$[$|y|$]_{rlx}$|;
if r2
then |$[$|x|$]_{rlx}$| := 1
else 0 
fi
\end{lstlisting}
\end{tabular}
    \end{minipage}
&
  \end{tabular}

\begin{lstlisting}
                  r0 = |$[$|z|$]_{rlx}$|
\end{lstlisting}
\end{minipage}

\litmusTestEnd

\litmusTestStart{SE\_prop}{\tick}{}
\begin{minipage}[t]{0.3\linewidth}
Possible outcomes:\\
\lstinline{  r0 = 0}\\
\lstinline{  r0 = 1}\\
\end{minipage}
%
\begin{minipage}[t]{0.4\linewidth}
\vspace{-.2cm}
  \begin{tabular}{l@{\ \ \ }l}
    \begin{minipage}[l]{4.3cm} \small
%
\begin{lstlisting}
  |$[$|x|$]_{rlx}$| := 0; |$[$|y|$]_{rlx}$| := 0; |$[$|z|$]_{rlx}$| := 0;
\end{lstlisting}
\vspace{-.2cm}
\begin{tabular}{l||l}
\begin{lstlisting}
r1 = |$[$|x|$]_{rlx}$|;
if r1
then |$[$|z|$]_{rlx}$| := 1;
     r1 = |$[$|z|$]_{rlx}$|;
     |$[$|y|$]_{rlx}$| := r1
else |$[$|y|$]_{rlx}$| := 1
fi
\end{lstlisting}
\hspace{.6cm}
&
\begin{lstlisting}
r2 = |$[$|y|$]_{rlx}$|;
if r2
then |$[$|x|$]_{rlx}$| := 1
else 0 
fi
\end{lstlisting}
\end{tabular}
    \end{minipage}
&
  \end{tabular}

\begin{lstlisting}
                  r0 = |$[$|z|$]_{rlx}$|
\end{lstlisting}
\end{minipage}

\litmusTestEnd

\litmusTestStart{SE\_nested}{\tick}{}
\begin{minipage}[t]{0.3\linewidth}
Possible outcomes:\\
\lstinline{  r0 = 0}\\
\lstinline{  r0 = 1}\\
\end{minipage}
%
\begin{minipage}[t]{0.4\linewidth}
\vspace{-.2cm}
  \begin{tabular}{l@{\ \ \ }l}
    \begin{minipage}[l]{4.3cm} \small
%
\begin{lstlisting}
|$[$|x|$]_{rlx}$| := 0; |$[$|y|$]_{rlx}$| := 0; |$[$|z|$]_{rlx}$| := 0; |$[$|f|$]_{rlx}$| := 0;
\end{lstlisting}
\vspace{-.2cm}
\begin{tabular}{l||l}
\begin{lstlisting}
r1 = |$[$|x|$]_{rlx}$|;
if r1
then r2 = |$[$|f|$]_{rlx}$|;
     if r2
     then |$[$|z|$]_{rlx}$| := 1;
          |$[$|y|$]_{rlx}$| := 1
     else |$[$|y|$]_{rlx}$| := 1
     fi
else |$[$|y|$]_{rlx}$| := 1
fi
\end{lstlisting}
\hspace{.6cm}
&
\begin{lstlisting}
r3 = |$[$|y|$]_{rlx}$|;
if r3
then |$[$|f|$]_{rlx}$| := 1;
     |$[$|x|$]_{rlx}$| := 1
else 0 
fi
\end{lstlisting}
\end{tabular}
    \end{minipage}
&
  \end{tabular}

\begin{lstlisting}
                  r0 = |$[$|z|$]_{rlx}$|
\end{lstlisting}
\end{minipage}

\litmusTestEnd


\section{Locks}
\label{app:locks}

\begin{minipage}[t]{0.4\linewidth}
\textbf{Dekker's lock}\\\\
Possible outcomes:\\
\lstinline{  stuck}\\
Requires: RA + na\\
Fully Supported: $\tick$\\
\end{minipage}
%
\begin{minipage}[t]{0.4\linewidth}
\vspace{-.2cm}
  \begin{tabular}{l@{\ \ \ }l}
    \begin{minipage}[l]{4.3cm} \small
\begin{lstlisting}
  |$[$|x|$]_{rel}$| := 0; |$[$|y|$]_{rel}$| := 0; |$[$|d|$]_{na}$| := 0;
\end{lstlisting}
\vspace{-.2cm}
\begin{tabular}{l||l}
\begin{lstlisting}
|$[$|x|$]_{rel}$| := 1;
r1 = |$[$|y|$]_{acq}$|
if r1 == 0
then |$[$|d|$]_{na}$| := 5
else 0 
fi
\end{lstlisting}
\hspace{.6cm}
&
\begin{lstlisting}
|$[$|y|$]_{rel}$| := 1;
r2 = |$[$|x|$]_{acq}$|
if r2 == 0
then |$[$|d|$]_{na}$| := 6
else 0 
fi
\end{lstlisting}
\end{tabular}
    \end{minipage}
&
  \end{tabular}
\end{minipage}
\litmusTestEnd

\begin{minipage}[t]{0.4\linewidth}
\textbf{Cohen's lock}\\\\
Impossible outcomes (according to \cite{Turon-al:OOPSLA14}):\\
\lstinline{  stuck}\\
Requires: RA + na\\
Fully Supported: $\tick$\\
\end{minipage}
%
\begin{minipage}[t]{0.4\linewidth}
\vspace{-.2cm}
  \begin{tabular}{l@{\ \ \ }l}
    \begin{minipage}[l]{4.3cm} \small
\begin{lstlisting}
  |$[$|x|$]_{rel}$| := 0; |$[$|y|$]_{rel}$| := 0; |$[$|d|$]_{na}$| := 0;
\end{lstlisting}
\vspace{-.2cm}
\begin{tabular}{l||l}
\begin{lstlisting}
|$[$|x|$]_{rel}$| := choice 1 2;
repeat |$[$|y|$]_{acq}$| end;
r1 = |$[$|x|$]_{acq}$|
r2 = |$[$|y|$]_{acq}$|
if r1 == r2
then |$[$|d|$]_{na}$| := 5
else 0 
fi
\end{lstlisting}
\hspace{.6cm}
&
\begin{lstlisting}
|$[$|y|$]_{rel}$| := choice 1 2;
repeat |$[$|x|$]_{acq}$| end;
r3 = |$[$|x|$]_{acq}$|
r4 = |$[$|y|$]_{acq}$|
if r3 != r4
then |$[$|d|$]_{na}$| := 6
else 0 
fi
\end{lstlisting}
\end{tabular}
    \end{minipage}
&
  \end{tabular}
\end{minipage}
\litmusTestEnd
%% \end{figure*}

\chapter{Правила переходов и вспомогательные функции машины ARMv8 POP}
\label{sec:armpopTrans}

{\small

\inference{
  \tape \triangleq \tapef(\tId) \quad \tape(\cpath) = \bot \quad
  \lastInstr{\cpath} < size(\Carm) \\
  \exists \cpath'. (\tape(\cpath') \not = \bot \lor \cpath' = []) \land \cpath \in \nextPathCom{\cpath'}{\Carm}{\tape} \\
  \tape' \triangleq \tape[\cpath \mapsto \getNewTapeCell(\Carm[\lastInstr{\cpath}])]
}{
  \Cfarm[\tId \mapsto \Carm] \vdash
    \angled{\Mpop, \IssuingOrderf, \tapef}
    \armStepPgen{\transenv{Fetch instruction} \; \tId \; \cpath}
    \angled{\Mpop, \IssuingOrderf, \tapef[\tId \mapsto \tape']}
}

\vspace{.5cm}

\inference{
  \e \in \Evt \quad \lnot\Prop(\tId,\e) \quad \forall \e' <_{\Ord} \e. \; \Prop(\tId,\e') \quad
  \Prop' \triangleq \Prop \cup \{ (\tId, \e) \} \\
  \Ord' \triangleq (\Ord \cup \{ (\e, \e') \mid \Prop(\tId,\e') \land \lnot \Prop(\e.\tId,\e'),
  \notReorderableRel{\e}{\e'}, \lnot (\e' <_{\Ord} \e) \})^{+}
}{
  \Cfarm \vdash
    \angled{\Mcomp{\Evt}{\Ord}{\Prop}, \IssuingOrderf, \tapef}
    \armStepPgen{\transenv{Propagate} \; \e \; \tId}
    \angled{\Mcomp{\Evt}{\Ord'}{\Prop'}, \IssuingOrderf, \tapef}
}

\vspace{.5cm}

\inference{
  \tape \triangleq \tapef(\tId) \quad \tape(\cpath) = \tapeIfGoto{\None}{\z} \quad
  \stval \triangleq |[\expr|] _{com} \in \mathbb{Z} \\
  \prevBrCommitted(\cpath, \tape) \\
  (\IfState, \tape') \triangleq \tapeUpdIf(\stval, \z, \cpath, \tape) \\
  \Mpop' \triangleq \deleteUpdReads(\tId, \tape', \Mpop) \\
}{
  \Cfarm \vdash
    \angled{\Mpop, \IssuingOrderf, \tapef}
    \armStepPgen{\transenv{Branch commit} \; \tId \; \cpath}
    \angled{\Mpop', \IssuingOrderf, \tapef'}
}

\vspace{.5cm}

\inference{
  \tapef(\tId, \cpath) = \tapeFence{\None}{\LD} \quad
  \tapef' \triangleq \tapef[(\tId, \cpath) \mapsto \tapeFence{\Committed}{\LD}] \\
  \prevReadsCommitted(\cpath, \tapef(\tId)) \quad
  \prevFencesCommitted(\cpath, \tape(\tId)) \\
  \prevBrCommitted(\cpath, \tapef(\tId)) \\
}{
  \Cfarm \vdash
    \angled{\Mpop, \IssuingOrderf, \tapef}
    \armStepPgen{\transenv{Fence commit} \; \LD \; \tId \; \cpath}
    \angled{\Mpop, \IssuingOrderf, \tapef'}
}

\vspace{.5cm}

\inference{
  \tapef(\tId, \cpath) = \tapeFence{\None}{\SY} \quad
  \tapef' \triangleq \tapef[(\tId, \cpath) \mapsto \tapeFence{\Committed}{\SY}] \\
  \prevInstrCommitted(\cpath, \tapef(\tId)) \\
  %% \tape' \triangleq \tape[\cpath \mapsto \tapeFence{\Committed}{\Ftype}] \quad
  \Mpop' \triangleq \acceptRequest(\stRequestFence{\tId}{ \cpath}, \Mpop)
}{
  \Cfarm \vdash
    \angled{\Mpop, \IssuingOrderf, \tapef}
    \armStepPgen{\transenv{Fence commit} \; \SY \; \tId \; \cpath}
    \angled{\Mpop', \IssuingOrderf, \tapef'}
}

\vspace{.5cm}

\inference{
  \tapef(\tId, \cpath) = \tapeWrite{\None} \quad
  \Cfarm(\tId)[\lastInstr{\cpath}] = ``[\expr_0] := \expr_1" \\
  |[\expr_0|] = \loc \quad |[\expr_1|] = \stval \quad
  \tapef' = \tapef[(\tId, \cpath) \mapsto \tapeWrite{(\tapePending{\loc}{\stval})}]
}{
  \Cfarm \vdash
    \angled{\Mpop, \IssuingOrderf, \tapef}
    \armStepPgen{\transenv{Write pending} \; \tId \; \cpath \; \loc \; \stval}
    \angled{\Mpop, \IssuingOrderf, \tapef'}
}

\vspace{.5cm}

\inference{
  \tape \triangleq \tapef(\tId) \quad \tape(\cpath) = \tapeWrite{(\tapePending{\loc}{\stval})} \quad \Carm \triangleq \Cfarm(\tId) \\
  \Carm[\lastInstr{\cpath}] = ``[\expr_0] := \expr_1" \quad |[\expr_0|] _{com} = \loc \quad |[\expr_1|] _{com} = \stval \\
  \prevFencesCommitted(\cpath, \tape) \quad
  \prevBrCommitted(\cpath, \tape) \\
  \prevCmdDetermined(\cpath, \tape) \quad
  \prevNoRestart(\loc, \cpath, \tape)\\
  im \triangleq \noFollowingWcom(\loc, \cpath, \tape) \\
  \tape' \triangleq \tapeUpdWcom(im, \loc, \stval, \Carm, \tId, \cpath, \tape) \\
  \tapef' = \tapef[\tId \mapsto \tape'] \quad
  \Mpop'' \triangleq \deleteUpdReads(\tId, \tape', \Mpop) \\
  \Mpop' \triangleq \textIf im \; \textThen \acceptRequest(\stRequestWrite{\tId}{ \cpath}{\loc}{\stval}, \Mpop'') \; \textElse \Mpop'' 
}{
  \angled{\Mpop, \IssuingOrderf, \tapef, \Cfarm}
  \armStepWriteCommitPLoc
  \angled{\Mpop', \IssuingOrderf, \tapef', \Cfarm}
}

%% \inference{
%%   \tape \triangleq \tapef(\tId) \quad \tape(\cpath) = \tapeWrite{(\tapePending{x}{\stval})} \quad \Carm \triangleq \Cfarm(\tId) \\
%%   \Carm[\lastInstr{\cpath}] = ``[\expr_0] := \expr_1" \quad |[\expr_0|] _{com} = x \quad |[\expr_1|] _{com} = \stval \\
%%   \prevFencesCommitted(\cpath, \tape) \quad
%%   \prevBrCommitted(\cpath, \tape) \\
%%   \prevCmdDetermined(\cpath, \tape) \quad
%%   \prevNoRestart(x, \cpath, \tape)\\
%%   im \triangleq \noFollowingWcom(x, \cpath, \tape) \\
%%   \tape' \triangleq \tapeUpdWcom(im, x, \stval, \Carm, \tId, \cpath, \tape) \\
%%   \tapef' = \tapef[\tId \mapsto \tape'] \quad
%%   \Mpop'' \triangleq \deleteUpdReads(\tId, \tape', \Mpop) \\
%%   \Mpop' \triangleq \textIf im \; \textThen \acceptRequest(\stRequestWrite{\tId}{ \cpath}{ x}{\stval}, \Mpop'') \; \textElse \Mpop'' 
%% }{
%%   \angled{\Mpop, \IssuingOrderf, \tapef, \Cfarm}
%%   \armStepWriteCommitP
%%   \angled{\Mpop', \IssuingOrderf, \tapef', \Cfarm}
%% }

\vspace{.5cm}

\inference{
  \tape \triangleq \tapef(\tId) \quad \tape(\cpath) = \tapeRead{\None} \\
   \Cfarm(\tId)[\lastInstr{\cpath}] = ``\reg = [\expr]" \quad |[\expr|] = \loc \quad
  \e \triangleq \stRequestRead{\tId}{\cpath}{\loc} \\
  \tapef' = \tapef[(\tId, \cpath) \mapsto \tapeRead{(\Issued{\loc})}] \quad \prevFencesCommitted(\cpath, \tape) \\
  \IssuingOrder' \triangleq append(\e, \IssuingOrderf(\tId)) \quad
  \IssuingOrderf' = \IssuingOrderf[\tId \mapsto \IssuingOrder'] \quad
  \Mpop' \triangleq \acceptRequest(\e, \Mpop)
}{
  \Cfarm \vdash
    \angled{\Mpop, \IssuingOrderf, \tapef}
    \armStepPgen{\transenv{Read issue} \; \tId \; \cpath \; \loc}
    \angled{\Mpop', \IssuingOrderf', \tapef'}
}

\vspace{.5cm}

\inference{
  \tape \triangleq \tapef(\tId) \quad
  \tape(\cpath) = \tapeRead{(\Issued{\loc})} \quad
  \e \triangleq \stRequestRead{\tId}{ \cpath}{\loc} \quad \e' \triangleq \stRequestWrite{\tId'}{ \cpath'}{\loc}{\stval} \\
     \angled{\Evt, \Ord, \Prop} \triangleq \Mpop \quad
  \{\e, \e'\} \subseteq \Evt \quad \e' <_{\Ord} \e \quad
  \samePropagated(\e, \e', \Prop) \\
  \forall \e^{*}, \e' <_{\Ord} \e^{*} <_{\Ord} \e, get\loc(\e^{*}) \not = \loc \land \fullyPropagated(\e^{*}, \Prop) \\
  \lnot \prevReadFromOther(\loc, \e', \tId, \cpath, \IssuingOrderf(\tId)) \\
  \tape' \triangleq \tapeUpdRsat(\Plain, \loc, \stval, \Carm, \tId, \cpath, \tId', \cpath', \tape) \\
  \tapef' = \tapef[\tId \mapsto \tape'] \quad \Mpop' \triangleq \deleteUpdReads(\tId, \tape', \Mpop) \\
}{
  \Cfarm \vdash
    \angled{\Mpop, \IssuingOrderf, \tapef}
    \armStepPrSatLoc
    \angled{\Mpop', \IssuingOrderf, \tapef'}
}

\vspace{.5cm}

\inference{
  \tape \triangleq \tapef(\tId) \quad
  \tape(\cpath) = \tapeRead{(\Issued{\loc})} \quad
  \e \triangleq \stRequestRead{\tId}{ \cpath}{\loc} \quad \e' \triangleq \stRequestWrite{\tId'}{ \cpath'}{\loc}{\stval} \\
     \angled{\Evt, \Ord, \Prop} \triangleq \Mpop \quad
  \{\e, \e'\} \subseteq \Evt \quad \e' <_{\Ord} \e \quad
  \samePropagated(\e, \e', \Prop) \\
  \forall \e^{*}, \e' <_{\Ord} \e^{*} <_{\Ord} \e, get\loc(\e^{*}) \not = \loc \land \fullyPropagated(\e^{*}, \Prop) \\
  \prevReadFromOther(\loc, \e', \tId, \cpath, \IssuingOrderf(\tId)) \\
  \tape' \triangleq \tape[\cpath \mapsto \tapeRead{\None}] \quad \tapef' = \tapef[\tId \mapsto \tape'] \\
  \Mpop' \triangleq \deleteUpdReads(\tId, \tape', \Mpop) \\
}{
  \Cfarm \vdash
    \angled{\Mpop, \IssuingOrderf, \tapef}
    \armStepPrSatFailLoc
    \angled{\Mpop', \IssuingOrderf, \tapef'}
}

\vspace{.5cm}

\inference{
  \tape \triangleq \tapef(\tId) \quad
  \tape(\cpath) = \tapeRead{\None} \quad \tape(\cpath') = \tapeWrite{(\tapePending{\loc}{\stval})} \\
  \cpath' < \cpath \quad \Carm \triangleq \Cfarm(\tId) \quad
    \Carm[\lastInstr{\cpath}] = ``\reg = [\expr]" \quad |[\expr|]^{\cpath} = \loc \\
  \noWritesBetween(\loc, \Carm, \cpath', \cpath, \tape) \\
  \noDiffReadsBetween(\loc, \tId, \cpath', \cpath, \tape) \\
  \tape' \triangleq \tapeUpdRsat(\SatisfiedInFlight, \loc, \stval, \Carm, \tId, \cpath, \tId, \cpath', \tape) \\
  \tapef' = \tapef[\tId \mapsto \tape'] \quad \Mpop \triangleq \deleteUpdReads(\tId, \tape', \Mpop) \\
}{
  \Cfarm \vdash
    \angled{\Mpop, \IssuingOrderf, \tapef}
    \armStepPrInFlightSatLoc
    \angled{\Mpop', \IssuingOrderf, \tapef'}
}

\vspace{.5cm}

\inference{
  \tape \triangleq \tapef(\tId) \quad
  \tape(\cpath) = \tapeRead{(\tapeSatisfied{\satisfiedState}{\stRequestWrite{\tId'}{ \cpath'}{\loc}{\stval}})} \\
  \satisfiedState \not = \Committed \quad \Carm \triangleq \Cfarm(\tId) \quad
  \prevCmdDetermined(\cpath, \tape) \\
  \prevFencesCommitted(\cpath, \tape) \quad
  \prevBrCommitted(\cpath, \tape) \\
  \cpath'' = \mathsf{max}\{ \cpath^{*} < \cpath | \Carm[\lastInstr{\cpath^{*}}] = ``[\expr_0] := \expr_1", |[\expr_0|] ^{\cpath^{*}} _{com} = \loc \} \\
  \textIf (\tId', \cpath') = (\tId, \cpath'') \; \textThen \tape(\cpath') \; \text{is fully determined}
                                                   \; \textElse \tape(\cpath') \; \text{is committed} \\
  \readsBetweenCommitted(\cpath'', \cpath, \tape, \Carm) \\
  \tape' \triangleq \tape[\cpath \mapsto \tapeRead{(\tapeSatisfied{\Committed}{\stRequestWrite{\tId'}{ \cpath'}{\loc}{\stval}})}] \quad
  \tapef' = \tapef[\tId \mapsto \tape'] \\
}{
  \Cfarm \vdash
    \angled{\Mpop, \IssuingOrderf, \tapef}
    \armStepPgen{\transenv{Read commit} \; \tId \; \cpath}
    \angled{\Mpop, \IssuingOrderf, \tapef'}
}

\[
\begin{array}{l}
\prevInstrCommitted(\cpath, \tape) \triangleq \forall \cpath' < \cpath, \tape(\cpath') \; \text{is committed}. \\
\prevReadsCommitted(\cpath, \tape) \triangleq
  \forall \cpath' < \cpath, \tape(\cpath') = \tapeRead{\_} => \tape(\cpath') \; \text{is committed}. \\
\prevFencesCommitted(\cpath, \tape) \triangleq
  \forall \cpath' < \cpath, \tape(\cpath') = \tapeFence{\Fstate}{\_} => \Fstate = \Committed. \\
\prevBrCommitted(\cpath, \tape) \triangleq
  \forall \cpath' < \cpath, \tape(\cpath') = \tapeIfGoto{\IfState}{\_} => \IfState \not = \None. \\
\prevCmdDetermined(\cpath, \tape) \triangleq
  \forall \cpath' < \cpath, \tape(\cpath') \; \text{has a fully determined address}. \\
\prevNoRestart(\loc, \cpath, \tape) \triangleq \\
\quad \forall \cpath' < \cpath, \reg, \expr, \tape(\cpath') = \tapeRead{\Rstate},
      \Carm[\lastInstr{\cpath'}] = ``\reg = [\expr]", |[\expr|] = \loc => \\
\quad \quad \Rstate = \tapeSatisfied{\_}{\_} \; \land \tape(\cpath') \; \text{can't be restarted}. \\
\\
\noFollowingWcom(\loc, \cpath, \tape) \triangleq
  \nexists \cpath' > \cpath, \; \tape(\cpath') = \tapeWrite{(\tapeWriteCommitted{\_}{\loc}{\_})}. \\
%% \\
%% \checkReorderings{S} \triangleq S \setminus \{ (\e_0, \e_1) \mid \reorderableRel{\e_0}{\e_1} \}. \\
\end{array}
\]

\[
\begin{array}{l}
\tapeUpdIf(\stval, \z, \cpath, \tape) \triangleq \\
\quad \text{let} \; \IfState \triangleq
      \textIf \stval \not = 0 \; \textThen \Taken \; \textElse \Ignored \; \text{in} \\
\quad \text{let} \; \cpath_{drop} \triangleq
      \textIf \stval \not = 0 \;
      \textThen \mathsf{append}(\lastInstr{\cpath} + 1, \cpath) \;
      \textElse \mathsf{append}(\lastInstr{\cpath} + k, \cpath) \; \text{in} \\
\quad (\IfState, \lambda \; \cpath' -> \\
\quad \quad \textIf \mathsf{prefix}(\cpath_{drop}, \cpath') \; \textThen \bot \\
\quad \quad \textElif \cpath' = \cpath \; \textThen \tapeIfGoto{\IfState}{\z} \\
\quad \quad \textElse \tape(\cpath')). \\
\end{array}
\]

\[
\begin{array}{l}
\readsBetweenCommitted(\cpath'', \cpath, \tape, \Carm) \triangleq \\
\quad \forall \cpath^{*} > \cpath'', \cpath^{*} < \cpath, \\
\quad \quad \Carm[\lastInstr{\cpath^{*}}] = ``\reg = [\expr]", |[\expr|] ^{\cpath^{*}} _{com} = x =>
            \tape(\cpath^{*}) \; \text{is committed}. \\
\\
\noWritesBetween(\loc, \Carm, \cpath', \cpath, \tape) \triangleq \\
\quad \not \exists \cpath^{*} > \cpath', \cpath^{*} < \cpath, \Carm[\lastInstr{\cpath^{*}}] = ``[\expr_0] := \expr_1",
      |[\expr_0|]^{\cpath^{*}} = \loc.\\
\\
\noDiffReadsBetween(\loc, \tId', \cpath', \cpath, \tape) \triangleq \\
\quad \not \exists \tId'' \not = \tId', \cpath'' \not = \cpath', \cpath^{*} > \cpath', \cpath^{*} < \cpath,
      \tape(\cpath^{*}) = \tapeRead{(\tapeSatisfied{\_}{\stRequestWrite{\tId''}{ \cpath''}{\loc}{\_}})}. \\
\end{array}
\]

\[
\begin{array}{l}
\deleteUpdReads(\tId, \tape, \angled{\Evt, \Ord, \Prop}) \triangleq \\
\quad \textLet \textit{to-delete} \triangleq
\{ \e \in \Evt \mid \e.\tId = \tId, \tape'(\e.\cpath) \not = \tapeRead{(\Issued{\_})} \} \; \textIn \\
\quad \textLet \Evt' \triangleq \Evt \setminus \textit{to-delete} \; \textIn \\
\quad \textLet \Prop' = \Prop \setminus (\mathbb{N} \times \textit{to-delete}) \; \textIn \\
\quad \textLet \Ord'  = (\checkReorderings{\Ord \setminus (\Evt \times \textit{to-delete} \cup \textit{to-delete} \times \Evt)})^{+}
 \; \textIn \\
\quad \angled{\Evt', \Ord', \Prop'}. \\
\\
\acceptRequest(\e, \angled{\Evt, \Ord, \Prop}) \triangleq \\
\quad \textLet \Evt' = \Evt \cup \{ \e \} \; \textIn \\
\quad \textLet \Prop' = \Prop[\tId \mapsto \Prop(\tId) \cup \{ \e \}]  \; \textIn \\
\quad \textLet \Ord'  = (\Ord \cup \{(\e', \e) | \e' \in \Prop(\tId), \notReorderableRel{\e'}{\e} \})^{+} \; \textIn \\
\quad \angled{\Evt', \Ord', \Prop'}.
\end{array}
\]

\[
\begin{array}{l}
\samePropagated(\e, \e', \Prop) \triangleq \;
\{ \tId \mid \Prop(\tId, e) \} = \{ \tId \mid \Prop(\tId, e') \} \\
\fullyPropagated(\e, \Prop) \triangleq \;
\forall \tId, \Prop(\tId, e) \lor \nexists e'. \Prop(\tId, e'). \\
get\loc(\e) \triangleq
\; \text{match} \; \e \; \text{with}
\;   \stRequestRead{\_}{ \_}{\loc}
\mid \stRequestWrite{\_}{ \_}{\loc}{\_} -> \loc
\mid \_ -> \bot
\; \text{end}. \\
\end{array}
\]

\[
\begin{array}{l}
\prevReadFromOther(\loc, \e', \tId, \cpath, \IssuingOrder) \triangleq \\
\quad \exists \cpath^{*} < \cpath, \\
\quad \quad last\_index(\stRequestRead{\tId}{ \cpath^{*}}{\loc}, \IssuingOrder) > last\_index(\stRequestRead{\tId}{ \cpath}{\loc}, \IssuingOrder), \\
\quad \quad \tape(\cpath^{*}) = \tapeRead{(\tapeSatisfied{\_}{\Request})}, \Request \not = \e'). \\
\\
\getNewTapeCell(\StmtARM) \triangleq \\
\quad \text{match} \; \StmtARM \; \text{with} \\
\quad | ``\readInst{\reg}{\expr}" -> \tapeRead{\None} \\
\quad | ``\writeInst{\expr_0}{\expr_1}" -> \tapeWrite{\None} \\
\quad | ``\fenceInst{\Ftype}" -> \tapeFence{\None}{\Ftype} \\
\quad | ``\ifGotoInst{\expr}{\z}" -> \tapeIfGoto{\None}{\z} \\
\quad | ``\reg = \expr" -> \tapeAssign \\
\quad | ``\nop" -> \tapeNop \\
\quad \text{end}. \\
\\
\e <_{\Ord} \e' \triangleq (\e, \e') \in \Ord. \\
\reorderableRel{\stRequest{\tId}{ \cpath}{ \RequestInfo}}{\stRequest{\tId'}{ \cpath'}{ \RequestInfo'}} \triangleq \\
\quad \textIf \SY \in \{\RequestInfo,\RequestInfo'\} \; \textThen false \\
\quad \textElif \RequestInfo.\loc = \RequestInfo'.\loc \not = \bot \; \textThen false \\
\quad \textElse true. \\
%% \quad \textIf \{\RequestInfo,\RequestInfo'\} \subseteq \{\LD, \SY\} \; \textThen false \\
%% \quad \textElif \SY \in \{\RequestInfo,\RequestInfo'\} \; \textThen false \\
%% \quad \textElif \exists \loc, \stval. \{\RequestInfo,\RequestInfo'\} \subseteq \{\loc, \loc:\stval\} \; \textThen false \\
%% \quad \textElif \RequestInfo' = \LD \land \tId = \tId' \land \exists \loc, \RequestInfo = \loc \; \textThen false \\
%% \quad \textElif \RequestInfo  = \LD \land \tId = \tId' \; \text{then}  \; false \\
%% \quad \textElse true. \\
\end{array}
\]

\[
\begin{array}{l}
\nextPathCom{\cpath}{\Carm}{\tape} \triangleq filter(\lambda \cpath' \rightarrow \lastInstr{\cpath'} < size(\Carm),  \\
\quad \textIf \cpath = [] \; \textThen \{ [0] \} \\
\quad \textElif \tape(\cpath) = \bot \; \textThen \emptyset \\
\quad \textElif \exists k, \tape(\cpath) = \tapeIfGoto{\None}{k} \; \textThen
      \{ snoc(\cpath, \lastInstr{\cpath} + 1), snoc(\cpath, \lastInstr{\cpath} + k) \} \\
\quad \textElif \exists k, \tape(\cpath) = \tapeIfGoto{\Taken}{k} \; \textThen
      \{ snoc(\cpath, \lastInstr{\cpath} + k) \} \\
\quad \textElse \{ snoc(\cpath, \lastInstr{\cpath} + 1) \}). \\
\end{array}
\]

\[
\begin{array}{l}
\tapeUpdRestart(\Carm, \tId, \tape) \triangleq \\
\quad fixpoint(\lambda \tape' -> \\
\qquad \lambda \cpath -> \\
\qquad \quad \textIf \Carm[\lastInstr{\cpath}] = ``\reg = [\expr]" \land \semf{\expr}{\cpath} = \bot \\
\qquad \qquad \textThen \tapeRead{\None} \\
\qquad \quad \textElif \exists \cpath'',
                  \tape'(\cpath) = \tapeRead{(\tapeSatisfied{\SatisfiedInFlight}{\stRequestWrite{\tId}{ \cpath''}{ \_}{\_}})} \land
                  \tape'(\cpath'') = \tapeWrite{\None} \\
\qquad \qquad \textThen \tapeRead{\None} \\
\qquad \quad \textElif \Carm[\lastInstr{\cpath}] = ``[\expr_0] = \expr_1" \land
                  (\semf{\expr_0}{\cpath} = \bot \lor \semf{\expr_1}{\cpath} = \bot) \\
\qquad \qquad \textThen \tapeWrite{\None} \\
\qquad \quad \textElse \tape'(\cpath))(\tape). \\
\end{array}
\]

\[
\begin{array}{l}
\tapeUpdWcom(im, \loc, \stval, \Carm, \tId, \cpath, \tape) \triangleq \\
\quad \tapeUpdRestart(\Carm, \tId, \\
\qquad \lambda \cpath' -> \\
\qquad \quad \textIf   \cpath' = \cpath \; \textThen \tapeWrite{(\tapeWriteCommitted{im}{\loc}{\stval})} \\
\qquad \quad \textElif \cpath' < \cpath \; \textThen \tape(\cpath') \\
\qquad \quad \textElif \tape(\cpath') = \tapeRead{(\tapeSatisfied{\SatisfiedInFlight}{\stRequestWrite{\tId}{ \cpath}{\loc}{\stval}})}
            \\
\qquad \qquad \textThen \tape(\cpath') = \tapeRead{(\tapeSatisfied{\Plain}{\stRequestWrite{\tId}{ \cpath}{\loc}{\stval}})} \\
\qquad \quad \textElif \tape(\cpath') = \tapeRead{(\Issued{\loc})} \; \textThen \tapeRead{\None} \\
\qquad \quad \textElif \exists \cpath'' < \cpath,
                      \tape(\cpath') =\tapeRead{(\tapeSatisfied{\SatisfiedInFlight}{\stRequestWrite{\tId}{ \cpath''}{\loc}{\_}})}
                      \; \textThen \tapeRead{\None} \\
\qquad \quad \textElif \exists \tId'', \cpath'', \lnot (\tId'' = \tId \land \cpath'' \ge \cpath) \land
                      \tape(\cpath') =\tapeRead{(\tapeSatisfied{\Plain}{\stRequestWrite{\tId''}{ \cpath''}{\loc}{\_}})}
                      \\
\qquad \qquad \textThen \tapeRead{\None} \\
\qquad \quad \textElse \tape(\cpath')). \\
\\
\tapeUpdRsat(\satisfiedState, \loc, \stval, \Carm, \tId, \cpath, \tId', \cpath', \tape) \triangleq \\
\quad \tapeUpdRestart(\Carm, \tId, \\
\qquad \lambda \cpath'' -> \\
\qquad \quad \textIf   \cpath'' = \cpath \;
                  \textThen \tapeRead{(\tapeSatisfied{\satisfiedState}{\stRequestWrite{\tId'}{ \cpath'}{\loc}{\stval}})} \\
\qquad \quad \textElif \cpath'' < \cpath \; \textThen \tape(\cpath'') \\
\qquad \quad \textElif \exists \cpath^{*} < \cpath,
                  \tape(\cpath'') = \tapeRead{(\tapeSatisfied{\SatisfiedInFlight}{\stRequestWrite{\tId}{ \cpath^{*}}{\loc}{\_}})}
                  \; \textThen \tapeRead{\None} \\
\qquad \quad \textElif \exists \tId^{*}, \cpath^{*},
                  \lnot (\tId^{*} = \tId  \land \cpath^{*} > \cpath ) \land
                  \lnot (\tId^{*} = \tId' \land \cpath^{*} = \cpath') \land \\
\qquad \qquad \tape(\cpath'') = \tapeRead{(\tapeSatisfied{\Plain}{\stRequestWrite{\tId^{*}}{ \cpath^{*}}{\loc}{\_}})}
                  \; \textThen \tapeRead{\None} \\
\qquad \quad \textElse \tape(\cpath'')).
\end{array}
\]

}



%% \begin{figure}[h]
%% \begin{minipage}{0.6\textwidth}
%% $\begin{array}{l @{~} r @{~} l}
%% \Carm    & : & \ListOf{\StmtARM} \\  
%% \StmtARM & ::= & \readInst{\reg}{\expr} \\
%%          & |   & \writeInst{\expr_0}{\expr_1} \\
%%          & |   & \fenceInst{\FtypeProm} \\
%%          & |   & \ifGotoInst{\expr}{\z} \\
%%          & |   & \assignInst{\reg}{\expr} \mid \nop \\
%% \FtypeProm & ::= & \rel \mid \acq\\
%% \expr    & ::= & \val \mid \reg \mid \loc \mid uop \; \expr \\
%%          & |   & bop \; \expr_0 \; \expr_1 \\
%%          %% & :   & \Expr \\
%% \reg : \Reg & - & a, b, c, ...  \quad \; \text{(локальные переменные)} \\
%% \loc : \Loc & - & x, y, z, ... \quad \text{(локации)} \\
%% uop, bop & - & \text{арифметические операции} \\
%% \val, \z       & \in & \mathbb{Z}
%% \end{array}$
%% \captionof{figure}{Синтаксис программ}
%% \label{fig:syn-prog}
%% \end{minipage}
%% %
%% \begin{minipage}{0.4\textwidth}
%% $\begin{array}{l @{~} r @{~} l}
%%     \Path  & ::= & \ListOf{Label} \\
%%     \Label & ::= & \rlab{}{\loc}{\val} \\
%%            & |   & \wlab{}{\loc}{\val} \\
%%            & |   & \flab{\FtypeProm} \\
%%            & |   & \epsilon \\
%% \end{array}$
%% \captionof{figure}{Метки переходов}
%% \label{fig:lts}
%% \end{minipage}
%% \end{figure}

%% В этом разделе мы (i) опишем синтаксис исходного языка, который используется в нашей работе,
%% приведем алгоритм построения по программе (ii) помеченной системы переходов для обещающей машины
%% и (iii) \ARM-согласованных исполнений, а также предъявим теорему, которая их свяжет.

%% %% \section{Исходный язык}
%% Синтаксис исходного языка представлен на рисунке \ref{fig:syn-prog}.
%% В нашем синтаксисе многопоточная программа является параллельной композицией потоков на верхнем уровне, где
%% код каждого потока есть список инструкций ($\ListOf{\StmtARM}$).
%% Инструкция может быть чтением ($\readInst{\reg}{\expr}$),
%% записью ($\writeInst{\expr_0}{\expr_1}$),
%% барьером памяти ($\fenceInst{\FtypeProm}$), 
%% условным переходом на $\z$ позиций в списке инструкций ($\ifGotoInst{\expr}{\z}$),
%% присваиванием в локальную переменную ($\assignInst{\reg}{\expr}$),
%% пустой операцией ($\nop$).

\chapter{Компиляция программ в помеченную систему переходов}
Первым шагом для построения помеченной системы переходов по программе является
генерация всех возможных \emph{путей} ($\Path$) исполнения -- списков меток (см. рис. \ref{fig:lts}).
Далее по множеству путей строится конечный автомат, принимающий упомянутые пути.

Функция построения путей $\cmdsToLbls$ использует вспомогательную функцию $\instToLbl$,
которая принимает список инструкций потока $ilist$, указатель на текущую инструкцию $pos$ и
состояние локальных переменных $st$.
\[
\inarr{
  \cmdsToLbls : \Carm \rightarrow \Pset(\Path) \\
  \cmdsToLbls \; ilist = \instToLbl \; ilist \; 0 \; (\lambda \reg. 0) \\
  \\
  \instToLbl : \Carm \rightarrow \mathbb{N} \rightarrow (\Reg \rightarrow \mathbb{N}) \rightarrow \Pset(\Path) \\
  \instToLbl \; ilist \; pos \; \PromState \defeq \\
  \quad \textIf   pos < 0 \; || \; pos > length(ilist) \; \textThen \{ [] \} \\
  \quad \textElse \\
  \qquad \kw{match} \; ilist[pos] \; \kw{with} \\
  \qquad
    \begin{array}{@{}l}
    | ``\nop" \rightarrow \instToLbl \; ilist \; (pos + 1) \; \PromState \\
    | ``\readInst{\reg}{\expr}" \rightarrow \\
      \quad \{ \rlab{}{\semState{\expr}{\PromState}}{\val} : l \mid
      \forall \val \in \Val, l \in \instToLbl \; ilist \; (pos + 1) \; \PromState[\reg \mapsto \val] \} \\
    | ``\assignInst{\reg}{\expr}" \rightarrow
                 \instToLbl \; (pos + 1) \; \PromState[\reg \mapsto \semState{\expr}{\PromState}] \\
    | ``\writeInst{\expr_0}{\expr_1}" \rightarrow \\
      \quad \{ \wlab{}{\semState{\expr_0}{\PromState}}{\semState{\expr_1}{\PromState}} : l \mid
      \forall l \in \instToLbl \; ilist \; (pos + 1) \; \PromState \} \\
    | ``\fenceInst{\FtypeProm}" \rightarrow
      \{ \flab{\FtypeProm} : l \mid \forall l \in \instToLbl \; ilist \; (pos + 1) \; \PromState \} \\
    | ``\ifGotoInst{\expr}{k}" \rightarrow
      \letdef{step}{\textIf \semState{\expr}{\PromState} \; \textThen k \; \textElse 1} \\
      \quad \instToLbl \; (pos + step) \; \PromState \\
  \end{array} \\
}
\]

\chapter{Компиляция программ в предзапуски}

Для построения \ARM-согласованных исполнений по программе стандартно
используется следующая схема \cite{Vafeiadis-Narayan:OOPSLA13}.
В начале по программе строятся \emph{предзапуски} --- графы исполнений, в которых определена
только часть нужных отношений между событиями, а именно отношения порядка $\lPO$ и зависимостей
по данным $\lDATA$, управлению $\lCTRL$ и адресу $\lADDR$.
\[
\inarr{
  \PreExecutions \defeq \Pset(\langle set : \Pset(E), \lab : set \rightharpoonup \Label, \\
   \quad \lPO : \Pset(set \times set), \lCTRL : \Pset(set \times set),
         \lADDR : \Pset(set \times set), \lDATA : \Pset(set \times set) \rangle) \\
     %% , \state : \Reg \rightarrow \Val, \dep : \Pset(\Reg \times E)}) \\
}\]
При этом для каждой инструкции чтения
генерируется столько вариантов соответствующего события, сколько существует возможных значений
соответствующего типа данных.
Далее для каждого предзапуска недетерменированно выбираются отношения согласованности $\lCO$ и чтения из $\lRF$
таким образом, чтобы получалось \ARM-согласованное исполнение. Важно отменить, что не для всех предзапусков
существуют подходящие $\lCO$ и $\lRF$.

\[
\inarr{
  \cmdsToVrtxs : \Carm \rightarrow \PreExecutions \\
  \cmdsToVrtxs \; ilist = \instToVrtx \; ilist \; 0 \; (\lambda \reg. 0) \; \emptyset \\
}\]
Функция построения предзапусков по программе $\cmdsToVrtxs$ использует вспомогательную функцию $\instToVrtx$,
которая принимает список инструкций потока $ilist$, указатель на текущую инструкцию $pos$,
состояние локальных переменных $st$ и отношение зависимости локальной переменной от сгенерированного события $\dep$.
Последний параметр нужен для отношений зависимости.

\[
\inarr{
  \regs : \Expr \rightarrow \Pset(\Reg) \\
  \regs \; e \defeq \\
  \quad \kw{match} \; e \; \kw{with} \\
  \quad | \; \loc \; | \; \val \rightarrow \emptyset \\
  \quad | \; \reg \rightarrow \{\reg\} \\
  \quad | \; uop \; \expr \rightarrow \regs \; \expr \\
  \quad | \; bop \; \expr_o \; \expr_1 \rightarrow (\regs \; \expr_0) \cup (\regs \; \expr_1) \\
}\]
%% \[\inarr{
%%   \sembr{-} : \Carm \rightarrow
%%     \{\tup{\emptyset, \bot, \emptyset, \emptyset, \emptyset, \emptyset, \emptyset, \lambda \reg. \; 0, \emptyset} \} \\
%%   \sembr{[]} = \{\tup{\emptyset, \bot, \emptyset, \emptyset, \emptyset, \emptyset, \emptyset, \lambda \reg. \; 0, \emptyset} \} \\
%% }\]
{\small
\[
\inarr{
  \instToVrtx : \Carm \rightarrow \mathbb{N} \rightarrow (\Reg \rightarrow \mathbb{N}) \rightarrow \Pset(\Reg \times E)
    \rightarrow \PreExecutions \\
    %% \ListOf{(\Label \times \mathbb{N} \times (\Reg \rightarrow \mathbb{N}))} \\
  \instToVrtx \; ilist \; pos \; \PromState \; \dep \defeq \\
  \quad \textIf   pos < 0 \; || \; pos > length(ilist) \; \textThen
    \{\tup{\emptyset, \bot, \emptyset, \emptyset, \emptyset, \emptyset}\} \\
  \quad \textElse \\
  \qquad
    \letdef{a}{{\rm fresh\text{-}vertex} } \\
  \qquad \kw{match} \; ilist[pos] \; \kw{with} \\
  \qquad
    \begin{array}{@{}l}
    | ``\nop" \rightarrow \instToVrtx \; ilist \; (pos + 1) \; \PromState \; \dep \\
    | ``\readInst{\reg}{\expr}" \rightarrow \\
      \quad \inarr{
        \letdef{\dep'}{(\dep \setminus \{\reg\} \times E) \cup \{\tup{\reg, a}\}} \\
        \letdef{\loc}{\semState{\expr}{\PromState}} \\
        \letdef{\lADDR'}{ \codom{[\regs(\expr)]; \dep} \times \{a\} } \\
        \{\tup{set \cup\{a\}, \lab[a \mapsto \rlab{}{\loc}{\val}], \lPO \cup \{a\} \times set, \lCTRL, \lADDR \cup \lADDR', \lDATA}
        \mid \forall \val \in \Val, \\
        \quad \tup{set, \lab, \lPO, \lCTRL, \lADDR, \lDATA} \in
          \instToVrtx \; ilist \; (pos + 1) \; \PromState[\reg \mapsto \val] \; \dep'\} \\
      } \\
    | ``\assignInst{\reg}{\expr}" \rightarrow \\
      \quad \inarr{
        \letdef{\dep'}{(\dep \setminus \{\reg\} \times E) \cup \{\reg\} \times \regs(\expr) \}} \\
        \instToVrtx \; ilist \; (pos + 1) \; \PromState[\reg \mapsto \semState{\expr}{\PromState}] \; \dep' \\
      } \\
    | ``\writeInst{\expr_0}{\expr_1}" \rightarrow \\
      \quad \inarr{
        \letdef{\loc}{\semState{\expr_0}{\PromState}} \\
        \letdef{\val}{\semState{\expr_1}{\PromState}} \\
        \letdef{\lADDR'}{ \codom{[\regs(\expr_0)]; \dep} \times \{a\} } \\
        \letdef{\lDATA'}{ \codom{[\regs(\expr_1)]; \dep} \times \{a\} } \\
        \{\tup{set \cup\{a\}, \lab[a \mapsto \wlab{}{\loc}{\val}], \lPO \cup \{a\} \times set,
          \lCTRL, \lADDR \cup \lADDR', \lDATA \cup \lDATA'}
        \mid  \\
        \quad \tup{set, \lab, \lPO, \lCTRL, \lADDR, \lDATA} \in
          \instToVrtx \; ilist \; (pos + 1) \; \PromState \; \dep\} \\
      } \\
    | ``\fenceInst{\FtypeProm}" \rightarrow \\
      \quad \inarr{
        \{\tup{set \cup\{a\}, \lab[a \mapsto \flab{\FtypeProm}], \lPO \cup \{a\} \times set,
          \lCTRL, \lADDR, \lDATA}
        \mid  \\
        \quad \tup{set, \lab, \lPO, \lCTRL, \lADDR, \lDATA} \in
          \instToVrtx \; ilist \; (pos + 1) \; \PromState \; \dep\} \\
      } \\
    | ``\ifGotoInst{\expr}{k}" \rightarrow \\
      \quad \inarr{
        \letdef{step}{\textIf \semState{\expr}{\PromState} \; \textThen k \; \textElse 1} \\
        \letdef{dver}{\codom{[\regs(\expr)]; \dep}} \\
        \{\tup{set, \lab, \lPO, \lCTRL \cup dver \times set, \lADDR, \lDATA}
        \mid  \\
        \quad \tup{set, \lab, \lPO, \lCTRL, \lADDR, \lDATA} \in
          \instToVrtx \; ilist \; (pos + step) \; \PromState \; \dep \} \\
      } \\
  \end{array} \\
  %% \\
  %% \cmdsToVrtxsAux : \Carm \rightarrow \mathbb{N} \times (\Reg \rightarrow \mathbb{N}) \rightarrow \ListOf{\Path}\\
  %% \cmdsToVrtxsAux \; ilist \; (pos, \; \PromState) = \letdef{l}{\instToLbl \; ilist \; (pos, \; \PromState)} \\
  %% \quad %\textup{\sf flatten} \;
  %%       [lbl':lbls \mid (lbl', pos', \PromState') \in l, lbls \in \cmdsToLblsAux \; ilist \; (pos', \; \PromState') ] \\
  %% \\
}
\]
}



\chapter{Доказательство леммы о шаге симуляции между обещающей и ARMv8.3 моделями}
\label{sec:sim-step-proof}

  {\bf Лемма \ref{lem:sim-step}.}
  Пусть для некоторых конфигураций обхода $\tup{C, \IssuedSet}$ и $\tup{C', \IssuedSet'}$ сценария $G$,
  а также некоторого состояния обещающей машины $\tup{\TSf, M}$ выполняется
  $G \vdash \tup{C, \IssuedSet} \travConfigStep \tup{C', \IssuedSet'}$ и
  $\simRel(C, \IssuedSet, \TSf, M)$.
  Тогда существуют такие $\TSf'$ и $M'$, что $\tup{\TSf, M} \stepp\!\!^{+} \tup{\TSf', M'}$ и
  $\simRel(C', \IssuedSet', \TSf', M')$.
\begin{proof}%[Доказательство леммы \ref{lem:sim-step}]
  Существует два варианта: $G \vdash C, \IssuedSet \travConfigStep \tup{C', \IssuedSet'}$ соответствует покрытию или
  выпуску некоторого события $e \in \lE$. Введем обозначения $tid \defeq \lTID(e)$ и
  $\tup{\pstate, \View, \PromSet} \defeq \TSf(tid)$.
  %% \[\begin{array}{l@{~}l}
  %%    tid & \defeq \lTID(e) \\
  %%    \tup{\pstate, \View, \PromSet} & \defeq \TSf(tid) \\
  %% \end{array}\]
  Начнем с рассмотрения варианта, когда $G \vdash C, \IssuedSet \travConfigStep \tup{C', \IssuedSet'}$ соответствует
  выпуску события записи $e$.

  {\bf Выпуск события $e$}. Из определения $\travConfigStep$ следует, что
  $C' = C$, $\IssuedSet' = \IssuedSet \cup \{e\}$ и $e \in \issuable(G, C, \IssuedSet) \setminus \IssuedSet$.
      Введем обозначения:
      \[\begin{array}{l l}
        \loc, \; \val, \; \tau & \defeq \lLOC(e), \; \lVALW(e), \; T(e) \\
        mview & \defeq [\loc @ \tau] \sqcup \View.\viewRel \\
        msg & \defeq \msg{\loc}{\val}{\tau}{mview} \\
      \end{array}\]
      Мы знаем, что в $M$ нету сообщения для локации $\loc$ с меткой времени $\tau$,
      т.к. функция $T$ выдает уникальные для одной локации метки времени для событий записи ($\correctTmap(G, T)$),
      и у каждого сообщения из $M$ существует соответствующее ему сообщение в $\IssuedSet$ ($\invMemTwo(C, \IssuedSet, M)$).
      \[\begin{array}{l l}
        M', \; \PromSet' & \defeq \addToMemory{M}{msg}, \; \addToMemory{\PromSet}{msg} \\
        \TSf' & \defeq \TSf[tid \mapsto \tup{\pstate, \View, \PromSet'}]
      \end{array}\]
      $mview \in M'$, т.к. $[\loc @ \tau] \in M'$ и $\View.\viewRel \in M$.
      Таким образом $\tup{\tup{\pstate, \View, \PromSet}, M} \stepptid \tup{\tup{\pstate, \View, \PromSet'}, M'}$ выполняется.
      Осталось проверить, что $\simRel(C, \IssuedSet', \TSf', M')$ выполняется.
      \begin{itemize}
        \item $\invMemOne(C, \IssuedSet', \TSf', M') \land \invMemTwo(C, \IssuedSet', M')$: \\
          Единственным нетривиальным утверждением, которое нужно проверить, является $mview \le \domView(\msgRel; [e])$.
          По определению, $mview = [\loc @ \tau] \sqcup \View.\viewRel$
          $[\loc @ \tau] = [\lLOC(e) @ T(e)] \le \domView(\msgRel; [e])$, т.к. $\tup{e, e} \in \msgRel$.
          Из $\invView(C, \TSf)$ следует $\View.\viewRel \le \domView(\relRel; [e'])$,
          где $e' \in \nextset(G, C)$ и $\lTID(e') = \lTID(e)$. Т.к. $e \nin C$ и по определению $\nextset$,
          $\tup{e', e} \in \lPO^{?}$. Из определения $\relRel$ и $\msgRel$ следует
          $\relRel; [e']; \lPO^{?}; [e] \suq \msgRel; [e]$.
          %% \[\dom{\relRel; [e']} = \dom{\relRel; [e']; \lPO^{?}; [e]} \suq \dom{\msgRel; [e]}\]
          Утверждение выполняется, т.к. $\View.\viewRel \le \domView(\relRel; \lPO; [e'])$.
        
        \item $\invViewRel(\TSf')$: выполняется по $\invViewRel(\TSf)$ и определениям $mview$ и $\TSf'$.

        \item $\invView(C, \TSf')$: т.к. $\forall tid. \; \TSf'(tid).\View = \TSf(tid).\View$ и $\invView(C, \TSf')$,
          инвариант выполняется.

        \item $\invState(C, \TSf')$:
          следует из того, что $\invState(C, \TSf)$ выполняется и, для любого $tid$, $\TSf(tid).\pstate = \TSf'(tid).\pstate$.
      \end{itemize}
      %% $\tup{\TSf', M'}$ is consistent by \cref{prop:sim-cert}.

  
  {\bf Покрытие события $e$.} В этом случае $C' = C \cup \{e\}, \IssuedSet' = \IssuedSet$.
  %% \begin{itemize}
    %% \item $C' = C \cup \{e\}, \IssuedSet' = \IssuedSet$, and $e \nin \dom{\lRMW}$: \\
      Т.к. $\invState(C, \TSf)$ выполняется, существуют такие $t$ и $\pstate'$, что $t \approx \lLAB(e)$.
      Рассмотрим варианты $e$.
      \begin{itemize}
        \item $e \in \lDMBLD$.
          В этом случае $\labelF(t) = \fenceLbl{\acq}$ по $\invState(C, \TSf)$.
          \[\begin{array}{l@{~}l}
            \View'    & \defeq \tup{\View.\viewAcq, \View.\viewAcq, \View.\viewRel} \\
            \TSf', M' & \defeq \TSf[tid \mapsto \tup{\pstate', \View', \PromSet}], M \\
          \end{array}\]
      Проверим, что $\simRel(C', \IssuedSet, \TSf', M)$ выполняется.
      \begin{itemize}
        \item $\invMemOne(C', \IssuedSet, \TSf', M) \land \invMemTwo(C', \IssuedSet, M)$:
          выполняется, т.к. $e \nin \lW$ и $\simRel(C, \IssuedSet, \TSf, M)$ выполняется.

        \item $\invViewRel(\TSf')$: выполняется, т.к. $\invViewRel(\TSf)$ выполняется и\\
          $\TSf'(tid).\{\View.\viewRel, \PromSet\} = \TSf(tid).\{\View.\viewRel, \PromSet\}$.

        \item $\invView(C', \TSf')$:
          \[\inarr{
  \forall e' \in \nextset(G, C'). \; \letdef{\tup{\viewCur, \viewAcq, \viewfRel}}{\TSf'(\lTID(e')).\View}\\
  \quad
  \begin{array}{@{}r@{~}l@{~}l}
    \viewCur  & \le \domView(\curRel; [e']) & \land {} \\
    \viewAcq  & \le \domView(\acqRel; [e']) & \land {} \\
    \viewfRel & \le \domView(\relRel; [e']). \\
  \end{array}
          }\]
          Зафиксируем $e'$. Если $\lTID(e') \neq tid$, то утверждение следует из $\invView(C, \TSf)$.
          Предположим, что $\lTID(e') = tid$. Тогда $\imm{\lPO}(e, e')$, т.к. $e \in \nextset(G, C)$.
          Мы знаем, что $\View.\viewAcq \le \domView(\acqRel; \lPO; [e])$, т.к. $\invView(C, \TSf)$ выполняется.
          Нам нужно показать, что
          \[\inarr{
  \begin{array}{@{}r@{~}l@{~}l}
    \View.\viewAcq  & \le \domView(\curRel; [e']) & \land {} \\
    \View.\viewAcq  & \le \domView(\acqRel; [e']) & \land {} \\
    \View.\viewfRel & \le \domView(\relRel; [e']). \\
  \end{array}
          }\]
          Т.к. $\dom{\acqRel; [e]} \suq \dom{\acqRel; [e']}$
          и $\acqRel;[e];\lPO;[e'] \suq \curRel; [e']$, утверждение выполняется.
        \item $\invState(C', \TSf')$:
          очевидно следует из $\invState(C, \TSf)$ и определений $C', \TSf'$.
      \end{itemize}

        \item $e \in \lDMBSY$. В этом случае $\labelF(t) = \fenceLbl{\rel}$ по $\invState(C, \TSf)$.
          \[\begin{array}{l@{~}l}
            \View'   & \defeq \tup{\viewCur, \viewAcq, \viewCur} \\
            \TSf'    & \defeq \TSf[tid \mapsto \tup{\pstate', \View', \PromSet}] \\
            M' & \defeq M \\
          \end{array}\]
          Не существует $w \in \IssuedSet$ такого, что $\lPO(e, w)$. Иначе не
          это противоречило бы $w \in \issuable(G, C, \IssuedSet)$ по лемме \ref{prop:trav-prop-preserve} 
          Из этого следует, что не существует $w \in \IssuedSet \setminus C$ такого, что $\lTID(w) = tid$, и
          $\PromSet = \emptyset$ по $\invMemOne(C, \IssuedSet, \TSf, M)$.
          
      Проверим $\simRel(C', \IssuedSet, \TSf', M)$.
      \begin{itemize}
        \item $\invMemOne(C', \IssuedSet, \TSf', M) \land \invMemTwo(C', \IssuedSet, M)$:
          выполняется, т.к. $e \nin \lW$ и $\simRel(C, \IssuedSet, \TSf, M)$ выполняется.
        \item $\invViewRel(\TSf')$: \\
          Зафиксируем поток $tid'$. Если $tid' \neq tid$, то утверждение выполняется по $\invViewRel(\TSf)$.
          Если $tid' = tid$, то утверждение выполняется, т.к. $\TSf'(tid).\PromSet = \TSf(tid).\PromSet = \emptyset$.
        \item $\invView(C', \TSf')$:
          \[\inarr{
  \forall e' \in \nextset(G, C'). \; \letdef{\tup{\viewCur, \viewAcq, \viewfRel}}{\TSf'(\lTID(e')).\View}\\
  \quad
  \begin{array}{@{}r@{~}l@{~}l}
    \viewCur  & \le \domView(\curRel; [e']) & \land {} \\
    \viewAcq  & \le \domView(\acqRel; [e']) & \land {} \\
    \viewfRel & \le \domView(\relRel; [e']). \\
  \end{array}
          }\]
          Зафиксируем $e'$. Если $\lTID(e') \neq tid$, то утверждение следует из $\invView(C, \TSf)$.
          Если $\lTID(e') = tid$, то $\imm{\lPO}(e, e')$, т.к. $e \in \nextset(G, C)$.
          Нам нужно показать, что
          \[\inarr{
  \begin{array}{@{}r@{~}l@{~}l}
    \View.\viewCur & \le \domView(\curRel; [e']) & \land {} \\
    \View.\viewAcq & \le \domView(\acqRel; [e']) & \land {} \\
    \View.\viewCur & \le \domView(\relRel; [e']). \\
  \end{array}
          }\]
          Т.к. $\dom{\curRel; [e]} \suq \dom{\curRel; [e']}$
          и $\relRel;[e];\lPO;[e'] \suq \curRel; [e']$, утверждение выполняется.
          
        \item $\invState(C', \TSf')$:
          очевидно следует из $\invState(C, \TSf)$ и определений $C', \TSf'$.
      \end{itemize}
        \item $e \in \lR$. 
          В этом случае $\labelF(t) = \readLbl{\loc}{\val}$ по $\invState(C, \TSf)$.
          \[\begin{array}{l@{~}l}
            \loc, \val       & \defeq \lLOC(e), \lVALR(e) \\
            {\pstate, \View, \PromSet} & \defeq \TSf(tid) \\
          \end{array}\]
          Т.к. $e \in \lR \cap \coverable(G, C, \IssuedSet)$ из определения $\travConfigStep$,
          существует событие записи $w \in \IssuedSet \cap \dom{\lRF; [e]}$. По $\invMemOne(C, \IssuedSet, \TSf, M)$
          существует фронт $view$ такой, что $msg \defeq \msg{\loc}{\val}{T(w)}{view} \in M$.
          $\View.\viewCur \le T(w)$ по лемме \ref{lem:curset-acyclic}.

          \[\begin{array}{l@{~}l}
            \View'   & \defeq \tup{\View.\viewCur \sqcup [\loc @ T(w)], \View.\viewAcq \sqcup view, \View.\viewRel} \\
            \TSf'    & \defeq \TSf[tid \mapsto \tup{\pstate', \View', \PromSet}] \\
          \end{array}\]
          Нужно проверить $\simRel(C', \IssuedSet, \TSf', M)$.
        \begin{itemize}
        \item $\invMemOne(C', \IssuedSet, \TSf', M) \land \invMemTwo(C', \IssuedSet, M)$:
          выполняется, т.к. $e \nin \lW$ и $\simRel(C, \IssuedSet, \TSf, M)$ выполняется.

        \item $\invViewRel(\TSf')$: выполняется, т.к. $\invViewRel(\TSf)$ выполняется и\\
          $\TSf'(tid).\{\View.\viewRel, \PromSet\} = \TSf(tid).\{\View.\viewRel, \PromSet\}$.

        \item $\invView(C', \TSf')$:
          \[\inarr{
  \forall e' \in \nextset(G, C'). \; \letdef{\tup{\viewCur, \viewAcq, \viewfRel}}{\TSf'(\lTID(e')).\View}\\
  \quad
  \begin{array}{@{}r@{~}l@{~}l}
    \viewCur  & \le \domView(\curRel; [e']) & \land {} \\
    \viewAcq  & \le \domView(\acqRel; [e']) & \land {} \\
    \viewfRel & \le \domView(\relRel; [e']). \\
  \end{array}
          }\]
          Зафиксируем $e'$. Если $\lTID(e') \neq tid$, то утверждение следует из $\invView(C, \TSf)$.
          Предположим, что $\lTID(e') = tid$. Тогда $\imm{\lPO}(e, e')$, т.к. $e \in \nextset(G, C)$.
          %% Мы знаем, что $\View.\viewAcq \le \domView(\acqRel; \lPO; [e])$, т.к. $\invView(C, \TSf)$ выполняется.
          Нам нужно показать, что
          \[\inarr{
  \begin{array}{@{}l@{~}l@{~}l}
    \View.\viewCur \sqcup [\loc @ T(w)] & \le \domView(\curRel; [e']) & \land {} \\
    \View.\viewAcq \sqcup view & \le \domView(\acqRel; [e']) & \land {} \\
    \View.\viewfRel & \le \domView(\relRel; [e']). \\
  \end{array}
          }\]
          %% $\View.\viewCur \le \domView(\curRel; [e]) \le \domView(\curRel; [e'])$ по $\invView(C, \TSf')$.
          Т.к. $\tup{w, e'} \in \curRel$, $[\loc @ T(w)] \le \domView(\curRel; [e'])$.
          Из $\invViewRel(\TSf)$ следует, что $view = [\loc @ T(w)] \sqcup \TSf(\lTID(w)).\View.\viewfRel$.
          Т.к. $\tup{w, e'} \in \acqRel$, $[\loc @ T(w)] \le \domView(\acqRel; [e'])$.
          Т.к. $\TSf(\lTID(w)).\View.\viewfRel \le \domView(\relRel; [w])$ и
          $\dom{\relRel; [w]} \suq \dom{\acqRel; [e']}$, утверждение выполняется.
        \item $\invState(C', \TSf')$:
          очевидно следует из $\invState(C, \TSf)$ и определений $C', \TSf'$.
      \end{itemize}

        \item $e \in \lW$.
          В этом случае $\labelF(t) = \writeLbl{\loc}{\val}$ по $\invState(C, \TSf)$.
          \[\begin{array}{l@{~}l}
            \loc, \val, \tau        & \defeq \lLOC(e), \lVALW(e), T(e) \\
            \tup{\pstate, \View, \PromSet} & \defeq \TSf(tid) \\
            \tup{\viewCur, \viewAcq, \viewRel} & \defeq \View. \\
          \end{array}\]
          $e \in \IssuedSet$, т.е. $e \in \lW \cap \coverable(G, C, \IssuedSet)$.
          Из $\invMemOne(C, \IssuedSet, \TSf, M)$ следует, что существует $view$ такое, что
          $msg \defeq \msg{\loc}{\val}{\tau}{view} \in M$.
          $\viewCur(\loc) < \tau$ следует по лемме \ref{lem:curset-acyclic}.
          \[\begin{array}{l@{~}l}
            \viewCur', \viewAcq' & \defeq \viewCur \sqcup [\loc @ \tau], \viewAcq \sqcup [\loc @ \tau] \\
            \View'    & \defeq \tup{\viewCur', \viewAcq', \viewRel} \\
            \TSf'     & \defeq \TSf[tid \mapsto \tup{\pstate', \View', \PromSet \setminus msg}] \\
          \end{array}\]
          $view = \viewRel$ по $\invViewRel(\TSf)$.

          Нужно проверить $\simRel(C', \IssuedSet, \TSf', M)$.
        \begin{itemize}
        \item $\invMemOne(C', \IssuedSet, \TSf', M) \land \invMemTwo(C', \IssuedSet, M)$:
          выполняется, т.к. мы добавили $e$ во множество покрытых событий и убрали $msg$ из множества
          обещанных сообщений потока $tid$.

        \item $\invViewRel(\TSf')$: выполняется, т.к. $\invViewRel(\TSf)$ выполняется, \\
          $\TSf'(tid).\View.\viewRel = \TSf(tid).\View.\viewRel$ и
          $\TSf'(tid).\PromSet \subset \TSf(tid).\PromSet$.

        \item $\invView(C', \TSf')$:
          \[\inarr{
  \forall e' \in \nextset(G, C'). \; \letdef{\tup{\viewCur, \viewAcq, \viewfRel}}{\TSf'(\lTID(e')).\View}\\
  \quad
  \begin{array}{@{}r@{~}l@{~}l}
    \viewCur  & \le \domView(\curRel; [e']) & \land {} \\
    \viewAcq  & \le \domView(\acqRel; [e']) & \land {} \\
    \viewfRel & \le \domView(\relRel; [e']). \\
  \end{array}
          }\]
          Зафиксируем $e'$. Если $\lTID(e') \neq tid$, то утверждение следует из $\invView(C, \TSf)$.
          Предположим, что $\lTID(e') = tid$. Тогда $\imm{\lPO}(e, e')$, т.к. $e \in \nextset(G, C)$.
          %% Мы знаем, что $\View.\viewAcq \le \domView(\acqRel; \lPO; [e])$, т.к. $\invView(C, \TSf)$ выполняется.
          Нам нужно показать, что
          \[\inarr{
  \begin{array}{@{}l@{~}l@{~}l}
    \View.\viewCur \sqcup [\loc @ \tau] & \le \domView(\curRel; [e']) & \land {} \\
    \View.\viewAcq \sqcup [\loc @ \tau] & \le \domView(\acqRel; [e']) & \land {} \\
    \View.\viewfRel & \le \domView(\relRel; [e']). \\
  \end{array}
          }\]
          Т.к. $\tup{w, e'} \in \curRel \suq \acqRel$,
          $[\loc @ \tau] \le \domView(\curRel; [e'])$
          ${} \le \domView(\acqRel; [e'])$.
        \item $\invState(C', \TSf')$:
          очевидно следует из $\invState(C, \TSf)$ и определений $C', \TSf'$.
      \end{itemize}

      \end{itemize}

      %% $\tup{\TSf', M'}$ is consistent by \cref{prop:sim-cert}.

  %% \end{itemize}
\end{proof}

\begin{lemma}
  \label{lem:curset-acyclic}
  Для любой локации $\loc$ выполняется $[\lW_{\loc}]; \curRel; [\lW_{\loc}] \suq \lCO$.
\end{lemma}
\begin{proof}
  Зафиксируем $\tup{w, w'} \in [\lW_{\loc}]; \curRel; [\lW_{\loc}]$. Тогда по определению $\lCO$,
  либо $\lCO(w, w')$, либо $\lCO(w', w)$. Если выполняется первое, то выполняется утверждение.
  Пусть выполняется второе.
  \[\inarr{
    {} [\lW_{\loc}]; \curRel; [\lW_{\loc}] = \\
    {} [\lW_{\loc}]; \lRF^{?}; (\lPO \cup \lSW)^{+}; [\lW_{\loc}] = \\
    \quad \text{(по транзитивности $\lPO$ и определению $\lSW$)} \\
    {} [\lW_{\loc}]; \lRF^{?}; \lPO; (\lSW; \lPO)^{*}; [\lW_{\loc}] = \\
    {} [\lW_{\loc}]; \lRF^{?}; \lPO; [\lW_{\loc}] \cup {} \\
    {} [\lW_{\loc}]; \lRFE^{?}; \lPO; (\lSW; \lPO)^{+}; [\lW_{\loc}] \cup {} \\
  }\]
  $\tup{w, w'} \nin [\lW_{\loc}]; \lRF^{?}; \lPO; [\lW_{\loc}]$, т.к. $\tup{w', w} \in \lCO$ и выполняется \ref{ax:internal}.
  
  Введем вспомогательное отношение $\lEORD \defeq (\lOBS \cup \lDOB \cup \lBOB)^{+}$,
  которое антирефлексивно по определению \ARM-согласованности (\ref{ax:external}).
  Из определений отношений следует, что $\lPO^{?}; \lSW; \lPO \suq \lBOB^{?}; \lBOB \cup \lBOB^{?}; \lBOB; \lRFE; \lBOB \suq \lEORD$.
  По транзитивности $\lEORD$, $\tup{w, w'} \in \lEORD$.
  
  Мы знаем, что $\tup{w', w} \in \lCO$. Есть два варианта: $\tup{w', w} \in \lCOE$ или $\tup{w', w} \in \lCOI$.
  Первый вариант противоречит ацикличности $\lEORD$, т.к. $\lCOE \suq \lOBS$. Опровергнем второй вариант, показав, что
  $(\lEORD \cup \lCOI)^{+} = (\lOBS \cup \lDOB \cup \lBOB \cup \lCOI)^{+}$ антирефлексивно.

  Предположим обратное и рассмотрим цикл минимальной длины, в котором точно есть $\lCOI$ по антирефлексивности $\lEORD$.
  Рассмотрим предыдущее ребро и покажем, что во всех случаях цикл можно укоротить.  
  \begin{itemize}
    \item $\lCOI; \lCOI \suq \lCOI$
    \item $\lOBS; \lCOI \suq \lOBS$: $\quad \lRFE; \lCOI = \emptyset \quad \lFRE; \lCOI \suq \lFRE \quad \lCOE; \lCOI \suq \lCOE$
    \item $\lDOB; \lCOI \suq \lDOB$:
      \begin{itemize}
        \item $\lADDR; \lPO^{?}; \lCOI \suq \lDOB \quad \lDATA; \lCOI \suq \lDOB$
        \item $(\lADDR \cup \lDATA);\lRFI; \lCOI = \emptyset \quad (\lCTRL \cup \lDATA); [\lW]; \lCOI^{?}; \lCOI \suq \lDOB$
      \end{itemize}
    \item $\lBOB; \lCOI \suq \lBOB$: $\lPO; \lCOI \suq \lPO$
    \qedhere
  \end{itemize}
\end{proof}

\chapter{Связь между системой переходов и вершинами в \ARM-исполнении}
\label{sec:lts-rel}

\[\inarr{
  \approx : \Label_{\Promise} \rightarrow \Label_{\ARM} \rightarrow {\rm Boolean} \\
  lbl_{\Promise} \approx lbl_{\ARM} \defeq \\
  \qquad \kw{match} \; lbl_{\Promise}, \; lbl_{\ARM} \; \kw{with} \\
  \qquad
    \begin{array}{@{}l}
      | \; \rlab{}{\loc}{\val}, \rlab{}{\loc}{\val} \;
      | \; \wlab{}{\loc}{\val}, \wlab{}{\loc}{\val} \\
      | \; \flab{\rel}, \flab{\SY} \;
      | \; \flab{\acq}, \flab{\LD} \rightarrow {\rm true} \\
      | \; \_, \_ \rightarrow {\rm false}
    \end{array} \\
  \\
  \nthf : \{A : Set\} \rightarrow \Pset(A \times A) \rightarrow \mathbb{N} \rightarrow \Pset(A) \\
  \nthf \; rel \; n \defeq \codom{rel^{n}} \setminus \codom{rel^{n + 1}}.
}\]

\begin{theorem}
\[\inarr{
\forall \Carm. \\
\quad (\forall \tup{set, lbl, \lPO, \_, \_, \_} \in \cmdsToVrtxs \; \Carm. \\
\qquad \exists path \in \cmdsToLbls \; \Carm. \; \forall n \in \mathbb{N}, a \in \nthf \; \lPO \; n. \\
\qquad \quad \exists k \in \mathbb{N}. \; path[n + k] \approx lbl \; a) \land {} \\
\quad (\forall path \in \cmdsToLbls \; \Carm. \\
\qquad \exists \tup{set, lbl, \lPO, \_, \_, \_} \in \cmdsToVrtxs \; \Carm. \;
  \forall n \in \mathbb{N}. \\
\qquad \quad path[n] \neq \epsilon \Rightarrow \exists k \in \mathbb{N}, a \in \nthf \; \lPO \; (n - k). \; path[n] \approx lbl \; a).
}\]
\end{theorem}
\begin{proof}
  Верно по определению функций $\instToLbl$ и $\instToVrtx$.
\end{proof}

%% \chapter{Примеры вставки листингов программного кода} \label{AppendixA}

%% Для крупных листингов есть два способа. Первый красивый, но в нём могут быть проблемы с поддержкой кириллицы (у вас может встречаться в комментариях и
%% печатаемых сообщениях), он представлен на листинге~\ref{list:hwbeauty}.
%% \begin{ListingEnv}[!h]% настройки floating аналогичны окружению figure
%% %    \captionsetup{format=tablenocaption}% должен стоять до самого caption
%%     \caption{Программа “Hello, world” на \protect\cpp}
%%     % далее метка для ссылки:
%%     \label{list:hwbeauty}
%%     % окружение учитывает пробелы и табуляции и применяет их в сответсвии с настройками
%%     \begin{lstlisting}[language={[ISO]C++}]
%% 	#include <iostream>
%% 	using namespace std;

%% 	int main() //кириллица в комментариях при xelatex и lualatex имеет проблемы с пробелами
%% 	{
%% 		cout << "Hello, world" << endl; //latin letters in commentaries
%% 		system("pause");
%% 		return 0;
%% 	}
%%     \end{lstlisting}
%% \end{ListingEnv}%
%% Второй не такой красивый, но без ограничений (см.~листинг~\ref{list:hwplain}).
%% \begin{ListingEnv}[!h]
%%     \caption{Программа “Hello, world” без подсветки}
%%     \label{list:hwplain}
%%     \begin{Verb}
        
%%         #include <iostream>
%%         using namespace std;
        
%%         int main() //кириллица в комментариях
%%         {
%%             cout << "Привет, мир" << endl;
%%         }
%%     \end{Verb}
%% \end{ListingEnv}

%% Можно использовать первый для вставки небольших фрагментов
%% внутри текста, а второй для вставки полного
%% кода в приложении, если таковое имеется.

%% Если нужно вставить совсем короткий пример кода (одна или две строки), то выделение  линейками и нумерация может смотреться чересчур громоздко. В таких случаях можно использовать окружения \texttt{lstlisting} или \texttt{Verb} без \texttt{ListingEnv}. Приведём такой пример с указанием языка программирования, отличного от заданного по умолчанию:
%% \begin{lstlisting}[language=Haskell]
%% fibs = 0 : 1 : zipWith (+) fibs (tail fibs)
%% \end{lstlisting}
%% Такое решение~--- со вставкой нумерованных листингов покрупнее
%% и вставок без выделения для маленьких фрагментов~--- выбрано,
%% например, в книге Эндрю Таненбаума и Тодда Остина по архитектуре
%% %компьютера~\autocite{TanAus2013} (см.~рис.~\ref{fig:tan-aus}).

%% Наконец, для оформления идентификаторов внутри строк
%% (функция \lstinline{main} и тому подобное) используется
%% \texttt{lstinline} или, самое простое, моноширинный текст
%% (\texttt{\textbackslash texttt}).


%% Пример~\ref{list:internal3}, иллюстрирующий подключение переопределённого языка. Может быть полезным, если подсветка кода работает криво. Без дополнительного окружения, с подписью и ссылкой, реализованной встроенным средством.
%% \begin{lstlisting}[language={Renhanced},caption={Пример листинга c подписью собственными средствами},label={list:internal3}]
%% ## Caching the Inverse of a Matrix

%% ## Matrix inversion is usually a costly computation and there may be some
%% ## benefit to caching the inverse of a matrix rather than compute it repeatedly
%% ## This is a pair of functions that cache the inverse of a matrix.

%% ## makeCacheMatrix creates a special "matrix" object that can cache its inverse

%% makeCacheMatrix <- function(x = matrix()) {#кириллица в комментариях при xelatex и lualatex имеет проблемы с пробелами
%%     i <- NULL
%%     set <- function(y) {
%%         x <<- y
%%         i <<- NULL
%%     }
%%     get <- function() x
%%     setSolved <- function(solve) i <<- solve
%%     getSolved <- function() i
%%     list(set = set, get = get,
%%     setSolved = setSolved,
%%     getSolved = getSolved)
    
%% }


%% ## cacheSolve computes the inverse of the special "matrix" returned by
%% ## makeCacheMatrix above. If the inverse has already been calculated (and the
%% ## matrix has not changed), then the cachesolve should retrieve the inverse from
%% ## the cache.

%% cacheSolve <- function(x, ...) {
%%     ## Return a matrix that is the inverse of 'x'
%%     i <- x$getSolved()
%%     if(!is.null(i)) {
%%         message("getting cached data")
%%         return(i)
%%     }
%%     data <- x$get()
%%     i <- solve(data, ...)
%%     x$setSolved(i)
%%     i  
%% }
%% \end{lstlisting} %$ %Комментарий для корректной подсветки синтаксиса
%%                  %вне листинга

%% Листинг~\ref{list:external1} подгружается из внешнего файла. Приходится загружать без окружения дополнительного. Иначе по страницам не переносится.
%%     \lstinputlisting[lastline=78,language={R},caption={Листинг из внешнего файла},label={list:external1}]{listings/run_analysis.R}






%% \chapter{Очень длинное название второго приложения, в котором продемонстрирована работа с длинными таблицами} \label{AppendixB}

%%  \section{Подраздел приложения}\label{AppendixB1}
%% Вот размещается длинная таблица:
%% \fontsize{10pt}{10pt}\selectfont
%% \begin{longtable*}[c]{|l|c|l|l|} %longtable* появляется из пакета caption и даёт ненумерованную таблицу
%% % \caption{Описание входных файлов модели}\label{Namelists} 
%% %\\ 
%%  \hline
%%  %\multicolumn{4}{|c|}{\textbf{Файл puma\_namelist}}        \\ \hline
%%  Параметр & Умолч. & Тип & Описание               \\ \hline
%%                                               \endfirsthead   \hline
%%  \multicolumn{4}{|c|}{\small\slshape (продолжение)}        \\ \hline
%%  Параметр & Умолч. & Тип & Описание               \\ \hline
%%                                               \endhead        \hline
%% % \multicolumn{4}{|c|}{\small\slshape (окончание)}        \\ \hline
%% % Параметр & Умолч. & Тип & Описание               \\ \hline
%% %                                             \endlasthead        \hline
%%  \multicolumn{4}{|r|}{\small\slshape продолжение следует}  \\ \hline
%%                                               \endfoot        \hline
%%                                               \endlastfoot
%%  \multicolumn{4}{|l|}{\&INP}        \\ \hline 
%%  kick & 1 & int & 0: инициализация без шума ($p_s = const$) \\
%%       &   &     & 1: генерация белого шума                  \\
%%       &   &     & 2: генерация белого шума симметрично относительно \\
%%   & & & экватора    \\
%%  mars & 0 & int & 1: инициализация модели для планеты Марс     \\
%%  kick & 1 & int & 0: инициализация без шума ($p_s = const$) \\
%%       &   &     & 1: генерация белого шума                  \\
%%       &   &     & 2: генерация белого шума симметрично относительно \\
%%   & & & экватора    \\
%%  mars & 0 & int & 1: инициализация модели для планеты Марс     \\
%% kick & 1 & int & 0: инициализация без шума ($p_s = const$) \\
%%       &   &     & 1: генерация белого шума                  \\
%%       &   &     & 2: генерация белого шума симметрично относительно \\
%%   & & & экватора    \\
%%  mars & 0 & int & 1: инициализация модели для планеты Марс     \\
%% kick & 1 & int & 0: инициализация без шума ($p_s = const$) \\
%%       &   &     & 1: генерация белого шума                  \\
%%       &   &     & 2: генерация белого шума симметрично относительно \\
%%   & & & экватора    \\
%%  mars & 0 & int & 1: инициализация модели для планеты Марс     \\
%% kick & 1 & int & 0: инициализация без шума ($p_s = const$) \\
%%       &   &     & 1: генерация белого шума                  \\
%%       &   &     & 2: генерация белого шума симметрично относительно \\
%%   & & & экватора    \\
%%  mars & 0 & int & 1: инициализация модели для планеты Марс     \\
%% kick & 1 & int & 0: инициализация без шума ($p_s = const$) \\
%%       &   &     & 1: генерация белого шума                  \\
%%       &   &     & 2: генерация белого шума симметрично относительно \\
%%   & & & экватора    \\
%%  mars & 0 & int & 1: инициализация модели для планеты Марс     \\
%% kick & 1 & int & 0: инициализация без шума ($p_s = const$) \\
%%       &   &     & 1: генерация белого шума                  \\
%%       &   &     & 2: генерация белого шума симметрично относительно \\
%%   & & & экватора    \\
%%  mars & 0 & int & 1: инициализация модели для планеты Марс     \\
%% kick & 1 & int & 0: инициализация без шума ($p_s = const$) \\
%%       &   &     & 1: генерация белого шума                  \\
%%       &   &     & 2: генерация белого шума симметрично относительно \\
%%   & & & экватора    \\
%%  mars & 0 & int & 1: инициализация модели для планеты Марс     \\
%% kick & 1 & int & 0: инициализация без шума ($p_s = const$) \\
%%       &   &     & 1: генерация белого шума                  \\
%%       &   &     & 2: генерация белого шума симметрично относительно \\
%%   & & & экватора    \\
%%  mars & 0 & int & 1: инициализация модели для планеты Марс     \\
%% kick & 1 & int & 0: инициализация без шума ($p_s = const$) \\
%%       &   &     & 1: генерация белого шума                  \\
%%       &   &     & 2: генерация белого шума симметрично относительно \\
%%   & & & экватора    \\
%%  mars & 0 & int & 1: инициализация модели для планеты Марс     \\
%% kick & 1 & int & 0: инициализация без шума ($p_s = const$) \\
%%       &   &     & 1: генерация белого шума                  \\
%%       &   &     & 2: генерация белого шума симметрично относительно \\
%%   & & & экватора    \\
%%  mars & 0 & int & 1: инициализация модели для планеты Марс     \\
%% kick & 1 & int & 0: инициализация без шума ($p_s = const$) \\
%%       &   &     & 1: генерация белого шума                  \\
%%       &   &     & 2: генерация белого шума симметрично относительно \\
%%   & & & экватора    \\
%%  mars & 0 & int & 1: инициализация модели для планеты Марс     \\
%% kick & 1 & int & 0: инициализация без шума ($p_s = const$) \\
%%       &   &     & 1: генерация белого шума                  \\
%%       &   &     & 2: генерация белого шума симметрично относительно \\
%%   & & & экватора    \\
%%  mars & 0 & int & 1: инициализация модели для планеты Марс     \\
%% kick & 1 & int & 0: инициализация без шума ($p_s = const$) \\
%%       &   &     & 1: генерация белого шума                  \\
%%       &   &     & 2: генерация белого шума симметрично относительно \\
%%   & & & экватора    \\
%%  mars & 0 & int & 1: инициализация модели для планеты Марс     \\
%% kick & 1 & int & 0: инициализация без шума ($p_s = const$) \\
%%       &   &     & 1: генерация белого шума                  \\
%%       &   &     & 2: генерация белого шума симметрично относительно \\
%%   & & & экватора    \\
%%  mars & 0 & int & 1: инициализация модели для планеты Марс     \\
%%  \hline
%%   %& & & $\:$ \\ 
%%  \multicolumn{4}{|l|}{\&SURFPAR}        \\ \hline
%% kick & 1 & int & 0: инициализация без шума ($p_s = const$) \\
%%       &   &     & 1: генерация белого шума                  \\
%%       &   &     & 2: генерация белого шума симметрично относительно \\
%%   & & & экватора    \\
%%  mars & 0 & int & 1: инициализация модели для планеты Марс     \\
%% kick & 1 & int & 0: инициализация без шума ($p_s = const$) \\
%%       &   &     & 1: генерация белого шума                  \\
%%       &   &     & 2: генерация белого шума симметрично относительно \\
%%   & & & экватора    \\
%%  mars & 0 & int & 1: инициализация модели для планеты Марс     \\
%% kick & 1 & int & 0: инициализация без шума ($p_s = const$) \\
%%       &   &     & 1: генерация белого шума                  \\
%%       &   &     & 2: генерация белого шума симметрично относительно \\
%%   & & & экватора    \\
%%  mars & 0 & int & 1: инициализация модели для планеты Марс     \\
%% kick & 1 & int & 0: инициализация без шума ($p_s = const$) \\
%%       &   &     & 1: генерация белого шума                  \\
%%       &   &     & 2: генерация белого шума симметрично относительно \\
%%   & & & экватора    \\
%%  mars & 0 & int & 1: инициализация модели для планеты Марс     \\
%% kick & 1 & int & 0: инициализация без шума ($p_s = const$) \\
%%       &   &     & 1: генерация белого шума                  \\
%%       &   &     & 2: генерация белого шума симметрично относительно \\
%%   & & & экватора    \\
%%  mars & 0 & int & 1: инициализация модели для планеты Марс     \\
%% kick & 1 & int & 0: инициализация без шума ($p_s = const$) \\
%%       &   &     & 1: генерация белого шума                  \\
%%       &   &     & 2: генерация белого шума симметрично относительно \\
%%   & & & экватора    \\
%%  mars & 0 & int & 1: инициализация модели для планеты Марс     \\
%% kick & 1 & int & 0: инициализация без шума ($p_s = const$) \\
%%       &   &     & 1: генерация белого шума                  \\
%%       &   &     & 2: генерация белого шума симметрично относительно \\
%%   & & & экватора    \\
%%  mars & 0 & int & 1: инициализация модели для планеты Марс     \\
%% kick & 1 & int & 0: инициализация без шума ($p_s = const$) \\
%%       &   &     & 1: генерация белого шума                  \\
%%       &   &     & 2: генерация белого шума симметрично относительно \\
%%   & & & экватора    \\
%%  mars & 0 & int & 1: инициализация модели для планеты Марс     \\
%% kick & 1 & int & 0: инициализация без шума ($p_s = const$) \\
%%       &   &     & 1: генерация белого шума                  \\
%%       &   &     & 2: генерация белого шума симметрично относительно \\
%%   & & & экватора    \\
%%  mars & 0 & int & 1: инициализация модели для планеты Марс     \\ 
%%  \hline 
%% \end{longtable*}

%% \normalsize% возвращаем шрифт к нормальному
%% \section{Ещё один подраздел приложения} \label{AppendixB2}

%% Нужно больше подразделов приложения!

%% Пример длинной таблицы с записью продолжения по ГОСТ 2.105
%% \begingroup
%%     \centering
%% 	\small
%%     \begin{longtable}[c]{|l|c|l|l|}
%% 	\caption{Наименование таблицы средней длины}%
%%     \label{tbl:test5}% label всегда желательно идти после caption
%%     \\
%%     \hline
%%      %\multicolumn{4}{|c|}{\textbf{Файл puma\_namelist}}        \\ \hline
%%      Параметр & Умолч. & Тип & Описание\\ \hline
%%      \endfirsthead%
%% %     \multicolumn{4}{|c|}{\small\slshape (продолжение)}        \\ \hline
%%  \captionsetup{format=tablenocaption,labelformat=continued}% должен стоять до самого caption
%%     \caption[]{}\\
%%     \hline
%%      Параметр & Умолч. & Тип & Описание\\ \hline
%%       \endhead
%%       \hline
%% %     \multicolumn{4}{|r|}{\small\slshape продолжение следует}  \\
%% %\hline
%%      \endfoot
%%          \hline
%%      \endlastfoot
%%      \multicolumn{4}{|l|}{\&INP}        \\ \hline 
%%      kick & 1 & int & 0: инициализация без шума ($p_s = const$) \\
%%           &   &     & 1: генерация белого шума                  \\
%%           &   &     & 2: генерация белого шума симметрично относительно \\
%%       & & & экватора    \\
%%      mars & 0 & int & 1: инициализация модели для планеты Марс     \\
%%      kick & 1 & int & 0: инициализация без шума ($p_s = const$) \\
%%           &   &     & 1: генерация белого шума                  \\
%%           &   &     & 2: генерация белого шума симметрично относительно \\
%%       & & & экватора    \\
%%      mars & 0 & int & 1: инициализация модели для планеты Марс     \\
%%     kick & 1 & int & 0: инициализация без шума ($p_s = const$) \\
%%           &   &     & 1: генерация белого шума                  \\
%%           &   &     & 2: генерация белого шума симметрично относительно \\
%%       & & & экватора    \\
%%      mars & 0 & int & 1: инициализация модели для планеты Марс     \\
%%     kick & 1 & int & 0: инициализация без шума ($p_s = const$) \\
%%           &   &     & 1: генерация белого шума                  \\
%%           &   &     & 2: генерация белого шума симметрично относительно \\
%%       & & & экватора    \\
%%      mars & 0 & int & 1: инициализация модели для планеты Марс     \\
%%     kick & 1 & int & 0: инициализация без шума ($p_s = const$) \\
%%           &   &     & 1: генерация белого шума                  \\
%%           &   &     & 2: генерация белого шума симметрично относительно \\
%%       & & & экватора    \\
%%      mars & 0 & int & 1: инициализация модели для планеты Марс     \\
%%     kick & 1 & int & 0: инициализация без шума ($p_s = const$) \\
%%           &   &     & 1: генерация белого шума                  \\
%%           &   &     & 2: генерация белого шума симметрично относительно \\
%%       & & & экватора    \\
%%      mars & 0 & int & 1: инициализация модели для планеты Марс     \\
%%     kick & 1 & int & 0: инициализация без шума ($p_s = const$) \\
%%           &   &     & 1: генерация белого шума                  \\
%%           &   &     & 2: генерация белого шума симметрично относительно \\
%%       & & & экватора    \\
%%      mars & 0 & int & 1: инициализация модели для планеты Марс     \\
%%     kick & 1 & int & 0: инициализация без шума ($p_s = const$) \\
%%           &   &     & 1: генерация белого шума                  \\
%%           &   &     & 2: генерация белого шума симметрично относительно \\
%%       & & & экватора    \\
%%      mars & 0 & int & 1: инициализация модели для планеты Марс     \\
%%     kick & 1 & int & 0: инициализация без шума ($p_s = const$) \\
%%           &   &     & 1: генерация белого шума                  \\
%%           &   &     & 2: генерация белого шума симметрично относительно \\
%%       & & & экватора    \\
%%      mars & 0 & int & 1: инициализация модели для планеты Марс     \\
%%     kick & 1 & int & 0: инициализация без шума ($p_s = const$) \\
%%           &   &     & 1: генерация белого шума                  \\
%%           &   &     & 2: генерация белого шума симметрично относительно \\
%%       & & & экватора    \\
%%      mars & 0 & int & 1: инициализация модели для планеты Марс     \\
%%     kick & 1 & int & 0: инициализация без шума ($p_s = const$) \\
%%           &   &     & 1: генерация белого шума                  \\
%%           &   &     & 2: генерация белого шума симметрично относительно \\
%%       & & & экватора    \\
%%      mars & 0 & int & 1: инициализация модели для планеты Марс     \\
%%     kick & 1 & int & 0: инициализация без шума ($p_s = const$) \\
%%           &   &     & 1: генерация белого шума                  \\
%%           &   &     & 2: генерация белого шума симметрично относительно \\
%%       & & & экватора    \\
%%      mars & 0 & int & 1: инициализация модели для планеты Марс     \\
%%     kick & 1 & int & 0: инициализация без шума ($p_s = const$) \\
%%           &   &     & 1: генерация белого шума                  \\
%%           &   &     & 2: генерация белого шума симметрично относительно \\
%%       & & & экватора    \\
%%      mars & 0 & int & 1: инициализация модели для планеты Марс     \\
%%     kick & 1 & int & 0: инициализация без шума ($p_s = const$) \\
%%           &   &     & 1: генерация белого шума                  \\
%%           &   &     & 2: генерация белого шума симметрично относительно \\
%%       & & & экватора    \\
%%      mars & 0 & int & 1: инициализация модели для планеты Марс     \\
%%     kick & 1 & int & 0: инициализация без шума ($p_s = const$) \\
%%           &   &     & 1: генерация белого шума                  \\
%%           &   &     & 2: генерация белого шума симметрично относительно \\
%%       & & & экватора    \\
%%      mars & 0 & int & 1: инициализация модели для планеты Марс     \\
%%      \hline
%%       %& & & $\:$ \\ 
%%      \multicolumn{4}{|l|}{\&SURFPAR}        \\ \hline
%%     kick & 1 & int & 0: инициализация без шума ($p_s = const$) \\
%%           &   &     & 1: генерация белого шума                  \\
%%           &   &     & 2: генерация белого шума симметрично относительно \\
%%       & & & экватора    \\
%%      mars & 0 & int & 1: инициализация модели для планеты Марс     \\
%%     kick & 1 & int & 0: инициализация без шума ($p_s = const$) \\
%%           &   &     & 1: генерация белого шума                  \\
%%           &   &     & 2: генерация белого шума симметрично относительно \\
%%       & & & экватора    \\
%%      mars & 0 & int & 1: инициализация модели для планеты Марс     \\
%%     kick & 1 & int & 0: инициализация без шума ($p_s = const$) \\
%%           &   &     & 1: генерация белого шума                  \\
%%           &   &     & 2: генерация белого шума симметрично относительно \\
%%       & & & экватора    \\
%%      mars & 0 & int & 1: инициализация модели для планеты Марс     \\
%%     kick & 1 & int & 0: инициализация без шума ($p_s = const$) \\
%%           &   &     & 1: генерация белого шума                  \\
%%           &   &     & 2: генерация белого шума симметрично относительно \\
%%       & & & экватора    \\
%%      mars & 0 & int & 1: инициализация модели для планеты Марс     \\
%%     kick & 1 & int & 0: инициализация без шума ($p_s = const$) \\
%%           &   &     & 1: генерация белого шума                  \\
%%           &   &     & 2: генерация белого шума симметрично относительно \\
%%       & & & экватора    \\
%%      mars & 0 & int & 1: инициализация модели для планеты Марс     \\
%%     kick & 1 & int & 0: инициализация без шума ($p_s = const$) \\
%%           &   &     & 1: генерация белого шума                  \\
%%           &   &     & 2: генерация белого шума симметрично относительно \\
%%       & & & экватора    \\
%%      mars & 0 & int & 1: инициализация модели для планеты Марс     \\
%%     kick & 1 & int & 0: инициализация без шума ($p_s = const$) \\
%%           &   &     & 1: генерация белого шума                  \\
%%           &   &     & 2: генерация белого шума симметрично относительно \\
%%       & & & экватора    \\
%%      mars & 0 & int & 1: инициализация модели для планеты Марс     \\
%%     kick & 1 & int & 0: инициализация без шума ($p_s = const$) \\
%%           &   &     & 1: генерация белого шума                  \\
%%           &   &     & 2: генерация белого шума симметрично относительно \\
%%       & & & экватора    \\
%%      mars & 0 & int & 1: инициализация модели для планеты Марс     \\
%%     kick & 1 & int & 0: инициализация без шума ($p_s = const$) \\
%%           &   &     & 1: генерация белого шума                  \\
%%           &   &     & 2: генерация белого шума симметрично относительно \\
%%       & & & экватора    \\
%%      mars & 0 & int & 1: инициализация модели для планеты Марс     \\ 
%% %     \hline 
%%     \end{longtable}
%% \normalsize% возвращаем шрифт к нормальному
%% \endgroup
%% \section{Использование длинных таблиц с окружением \textit{longtabu}} \label{AppendixB2a}

%% В таблице~\ref{tbl:test-functions} более книжный вариант 
%% длинной таблицы, используя окружение \verb!longtabu! и разнообразные
%% \verb!toprule! \verb!midrule! \verb!bottomrule! из пакета
%% \verb!booktabs!. Чтобы визуально таблица смотрелась лучше, можно
%% использовать следующие параметры: в самом начале задаётся расстояние
%% между строчками с~помощью \verb!arraystretch!. Таблица задаётся на
%% всю ширину, \verb!longtabu! позволяет делить ширину колонок
%% пропорционально "--- тут три колонки в пропорции 1.1:1:4 "--- для каждой
%% колонки первый параметр в описании \verb!X[]!. Кроме того, в~таблице
%% убраны отступы слева и справа с помощью \verb!@{}! в
%% преамбуле таблицы. К первому и второму столбцу применяется
%% модификатор 

%% \verb!>{\setlength{\baselineskip}{0.7\baselineskip}}!,

%% \noindent который уменьшает межстрочный интервал в для текста таблиц (иначе
%% заголовок второго столбца значительно шире, а двухстрочное имя
%% сливается с окружающими). Для первой и второй колонки текст в ячейках
%% выравниваются по~центру как по вертикали, так и по горизонтали -
%% задаётся буквами \verb!m! и \verb!c! в~описании столбца \verb!X[]!. 

%% Так как формулы большие "--- используется окружение \verb!alignedat!,
%% чтобы отступ был одинаковый у всех формул "--- он сделан для всех, хотя
%% для большей части можно было и не использовать.  Чтобы формулы
%% занимали поменьше места в~каждом столбце формулы (где надо)
%% используется \verb!\textstyle! "--- он делает дроби меньше, у знаков
%% суммы и произведения "--- индексы сбоку. Иногда формулы слишком большая,
%% сливается со следующей, поэтому после неё ставится небольшой
%% дополнительный отступ \verb!\vspace*{2ex}!  Для штрафных функций "---
%% размер фигурных скобок задан вручную \verb!\Big\{!, т.к. не умеет
%% \verb!alignedat! работать с~\verb!\left! и \verb!\right! через
%% несколько строк/колонок.


%% В примечании к таблице наоборот, окружение \verb!cases! даёт слишком
%% большие промежутки между вариантами, чтобы их уменьшить, в конце
%% каждой строчки окружения использовался отрицательный дополнительный
%% отступ \verb!\\[-0.5em]!.



%% \begingroup % Ограничиваем область видимости arraystretch
%% \renewcommand{\arraystretch}{1.6}%% Увеличение расстояния между рядами, для улучшения восприятия.
%% \begin{longtabu} to \textwidth
%% {%
%% @{}>{\setlength{\baselineskip}{0.7\baselineskip}}X[1.1mc]%
%% >{\setlength{\baselineskip}{0.7\baselineskip}}X[mc]%
%% X[4]@{}%
%% }
%%         \caption{Тестовые функции для оптимизации, $D$ "---
%%           размерность. Для всех функций значение в точке глобального
%%           минимума равно нулю.\label{tbl:test-functions}}\\% label всегда желательно идти после caption 
        
%%         \toprule     %%% верхняя линейка
%%         Имя           &Стартовый диапазон параметров &Функция  \\ 
%%         \midrule %%% тонкий разделитель. Отделяет названия столбцов. Обязателен по ГОСТ 2.105 пункт 4.4.5 
%%         \endfirsthead

%%         \multicolumn{3}{c}{\small\slshape (продолжение)}        \\ 
%%         \toprule     %%% верхняя линейка
%%         Имя           &Стартовый диапазон параметров &Функция  \\ 
%%         \midrule %%% тонкий разделитель. Отделяет названия столбцов. Обязателен по ГОСТ 2.105 пункт 4.4.5 
%%         \endhead
        
%%         \multicolumn{3}{c}{\small\slshape (окончание)}        \\ 
%%         \toprule     %%% верхняя линейка
%%         Имя           &Стартовый диапазон параметров &Функция  \\ 
%%         \midrule %%% тонкий разделитель. Отделяет названия столбцов. Обязателен по ГОСТ 2.105 пункт 4.4.5 
%%         \endlasthead

%%         \bottomrule %%% нижняя линейка
%%         \multicolumn{3}{r}{\small\slshape продолжение следует}  \\ 
%%         \endfoot   
%%         \endlastfoot

%%         сфера         &$\left[-100,\,100\right]^D$   &
%%         $\begin{aligned}\textstyle f_1(x)=\sum_{i=1}^Dx_i^2\end{aligned}$                                                        \\
%%         Schwefel 2.22 &$\left[-10,\,10\right]^D$     &
%%         $\begin{aligned}\textstyle f_2(x)=\sum_{i=1}^D|x_i|+\prod_{i=1}^D|x_i|\end{aligned}$                                     \\
%%         Schwefel 1.2  &$\left[-100,\,100\right]^D$   &$\begin{aligned}\textstyle f_3(x)=\sum_{i=1}^D\left(\sum_{j=1}^ix_j\right)^2\end{aligned}$                               \\
%%         Schwefel 2.21 &$\left[-100,\,100\right]^D$   &$\begin{aligned}\textstyle f_4(x)=\max_i\!\left\{\left|x_i\right|\right\}\end{aligned}$                             \\
%%         Rosenbrock    &$\left[-30,\,30\right]^D$     &$\begin{aligned}\textstyle f_5(x)=\sum_{i=1}^{D-1}\left[100\!\left(x_{i+1}-x_i^2\right)^2+(x_i-1)^2\right]\end{aligned}$ \\
%%         ступенчатая   &$\left[-100,\,100\right]^D$   &$\begin{aligned}\textstyle f_6(x)=\sum_{i=1}^D\big\lfloor x_i+0.5\big\rfloor^2\end{aligned}$                             \\ 
%% зашумлённая квартическая  &$\left[-1.28,\,1.28\right]^D$ &$\begin{aligned}\textstyle f_7(x)=\sum_{i=1}^Dix_i^4+rand[0,1)\end{aligned}$\vspace*{2ex}\\
%%         Schwefel 2.26 &$\left[-500,\,500\right]^D$   &$\begin{aligned}f_8(x)= &\textstyle\sum_{i=1}^D-x_i\,\sin\sqrt{|x_i|}\,+ \\
%%                     &\vphantom{\sum}+ D\cdot
%%                     418.98288727243369 \end{aligned}$\\
%%         Rastrigin     &$\left[-5.12,\,5.12\right]^D$ &
%%         $\begin{aligned}\textstyle
%%           f_9(x)=\sum_{i=1}^D\left[x_i^2-10\,\cos(2\pi
%%             x_i)+10\right]\end{aligned}$\vspace*{2ex}\\
%%   Ackley        &$\left[-32,\,32\right]^D$     &$\begin{aligned}f_{10}(x)= &\textstyle -20\, \exp\!\left(-0.2\sqrt{\frac{1}{D}\sum_{i=1}^Dx_i^2} \right)-\\
%%                     &\textstyle - \exp\left(\frac{1}{D}\sum_{i=1}^D\cos(2\pi x_i)  \right)  + 20 + e \end{aligned}$ \\
%%         Griewank      &$\left[-600,\,600\right]^D$
%%         &$\begin{aligned}f_{11}(x)= &\textstyle \frac{1}{4000}
%%           \sum_{i=1}^{D}x_i^2 - \prod_{i=1}^D\cos\left(x_i/\sqrt{i}\right) +1     \end{aligned}$ \vspace*{3ex} \\
%%         штрафная 1    &$\left[-50,\,50\right]^D$     &
%%         $\begin{aligned}f_{12}(x)= &\textstyle \frac{\pi}{D}
%%           \Big\{ 10\,\sin^2(\pi y_1) +\\ &+
%%           \textstyle \sum_{i=1}^{D-1}(y_i-1)^2\left[1+10\,\sin^2(\pi
%%               y_{i+1})\right] +\\ &+(y_D-1)^2 \Big\} +\textstyle\sum_{i=1}^D u(x_i,\,10,\,100,\,4)            \end{aligned}$ \vspace*{2ex} \\
%%         штрафная 2    &$\left[-50,\,50\right]^D$     &
%%         $\begin{aligned}f_{13}(x)= &\textstyle 0.1
%%           \Big\{\sin^2(3\pi x_1) +\\ &+
%%           \textstyle \sum_{i=1}^{D-1}(x_i-1)^2\left[1+\sin^2(3 \pi
%%               x_{i+1})\right] + \\ &+(x_D-1)^2\left[1+\sin^2(2\pi
%%               x_D)\right] \Big\} +\\ &+\textstyle\sum_{i=1}^D u(x_i,\,5,\,100,\,4)            \end{aligned}$               \\
%%         сфера         &$\left[-100,\,100\right]^D$   &
%%         $\begin{aligned}\textstyle f_1(x)=\sum_{i=1}^Dx_i^2\end{aligned}$                                                        \\
%%         Schwefel 2.22 &$\left[-10,\,10\right]^D$     &
%%         $\begin{aligned}\textstyle f_2(x)=\sum_{i=1}^D|x_i|+\prod_{i=1}^D|x_i|\end{aligned}$                                     \\
%%         Schwefel 1.2  &$\left[-100,\,100\right]^D$   &$\begin{aligned}\textstyle f_3(x)=\sum_{i=1}^D\left(\sum_{j=1}^ix_j\right)^2\end{aligned}$                               \\
%%         Schwefel 2.21 &$\left[-100,\,100\right]^D$   &$\begin{aligned}\textstyle f_4(x)=\max_i\!\left\{\left|x_i\right|\right\}\end{aligned}$                             \\
%%         Rosenbrock    &$\left[-30,\,30\right]^D$     &$\begin{aligned}\textstyle f_5(x)=\sum_{i=1}^{D-1}\left[100\!\left(x_{i+1}-x_i^2\right)^2+(x_i-1)^2\right]\end{aligned}$ \\
%%         ступенчатая   &$\left[-100,\,100\right]^D$   &$\begin{aligned}\textstyle f_6(x)=\sum_{i=1}^D\big\lfloor x_i+0.5\big\rfloor^2\end{aligned}$                             \\ 
%% зашумлённая квартическая  &$\left[-1.28,\,1.28\right]^D$ &$\begin{aligned}\textstyle f_7(x)=\sum_{i=1}^Dix_i^4+rand[0,1)\end{aligned}$\vspace*{2ex}\\
%%         Schwefel 2.26 &$\left[-500,\,500\right]^D$   &$\begin{aligned}f_8(x)= &\textstyle\sum_{i=1}^D-x_i\,\sin\sqrt{|x_i|}\,+ \\
%%                     &\vphantom{\sum}+ D\cdot
%%                     418.98288727243369 \end{aligned}$\\
%%         Rastrigin     &$\left[-5.12,\,5.12\right]^D$ &
%%         $\begin{aligned}\textstyle
%%           f_9(x)=\sum_{i=1}^D\left[x_i^2-10\,\cos(2\pi
%%             x_i)+10\right]\end{aligned}$\vspace*{2ex}\\
%%   Ackley        &$\left[-32,\,32\right]^D$     &$\begin{aligned}f_{10}(x)= &\textstyle -20\, \exp\!\left(-0.2\sqrt{\frac{1}{D}\sum_{i=1}^Dx_i^2} \right)-\\
%%                     &\textstyle - \exp\left(\frac{1}{D}\sum_{i=1}^D\cos(2\pi x_i)  \right)  + 20 + e \end{aligned}$ \\
%%         Griewank      &$\left[-600,\,600\right]^D$
%%         &$\begin{aligned}f_{11}(x)= &\textstyle \frac{1}{4000}
%%           \sum_{i=1}^{D}x_i^2 - \prod_{i=1}^D\cos\left(x_i/\sqrt{i}\right) +1     \end{aligned}$ \vspace*{3ex} \\
%%         штрафная 1    &$\left[-50,\,50\right]^D$     &
%%         $\begin{aligned}f_{12}(x)= &\textstyle \frac{\pi}{D}
%%           \Big\{ 10\,\sin^2(\pi y_1) +\\ &+
%%           \textstyle \sum_{i=1}^{D-1}(y_i-1)^2\left[1+10\,\sin^2(\pi
%%               y_{i+1})\right] +\\ &+(y_D-1)^2 \Big\} +\textstyle\sum_{i=1}^D u(x_i,\,10,\,100,\,4)            \end{aligned}$ \vspace*{2ex} \\
%%         штрафная 2    &$\left[-50,\,50\right]^D$     &
%%         $\begin{aligned}f_{13}(x)= &\textstyle 0.1
%%           \Big\{\sin^2(3\pi x_1) +\\ &+
%%           \textstyle \sum_{i=1}^{D-1}(x_i-1)^2\left[1+\sin^2(3 \pi
%%               x_{i+1})\right] + \\ &+(x_D-1)^2\left[1+\sin^2(2\pi
%%               x_D)\right] \Big\} +\\ &+\textstyle\sum_{i=1}^D u(x_i,\,5,\,100,\,4)            \end{aligned}$               \\
%%         \midrule%%% тонкий разделитель
%%         \multicolumn{3}{@{}p{\textwidth}}{%
%%             \vspace*{-3.5ex}% этим подтягиваем повыше
%%             \hspace*{2.5em}% абзацный отступ - требование ГОСТ 2.105
%%             Примечание "---  Для функций $f_{12}$ и $f_{13}$
%%             используется $y_i = 1 + \frac{1}{4}(x_i+1)$ и
%%             $u(x_i,\,a,\,k,\,m)=\begin{cases}
%% k(x_i-a)^m,\quad &x_i >a\\[-0.5em]
%% 0,\quad &-a\leq x_i \leq a\\[-0.5em]
%% k(-x_i-a)^m,\quad &x_i <-a
%% \end{cases}$  }   \\        \bottomrule %%% нижняя линейка 
%% \end{longtabu} 
%% \endgroup


%% \section{Форматирование внутри таблиц} \label{AppendixB3}

%% В таблице~\ref{tbl:other-row} пример с чересстрочным
%% форматированием. В \verb+userstyles.tex+  задаётся счётчик
%% \verb+\newcounter{rowcnt}+ который увеличивается на 1 после каждой
%% строчки (как указано в преамбуле таблицы). Кроме того, задаётся
%% условный макрос \verb+\altshape+ который выдаёт одно из
%% двух типов форматирования в зависимости от чётности счётчика. 

%% В таблице~\ref{tbl:other-row} каждая чётная строчка --- синяя,
%% нечётная --- с наклоном и слегка поднята вверх. Визуально это приводит
%% к тому, что среднее значение и среднеквадратичное изменение
%% группируются и хорошо выделяются взглядом в таблице. Сохраняется
%% возможность отдельные значения в таблице выделить цветом или
%% шрифтом. К первому и второму столбцу форматирование не применяется по
%% сути таблицы, к шестому общее форматирование не применяетсся для
%% наглядности.

%% Так как заголовок таблицы тоже считается за строчку, то перед ним (для
%% первого, промежуточного и финального варианта) счётчик обнуляется, а в
%% \verb+\altshape+ для нулевого значения счётчика форматирования не
%% применяется. 


%% \begingroup % Ограничиваем область видимости arraystretch
%% \renewcommand\altshape{
%%   \ifnumequal{\value{rowcnt}}{0}{
%%     % Стиль для заголовка таблицы
%%   }{
%%     \ifnumodd{\value{rowcnt}}
%%     {
%%       \color{blue} % Cтиль для нечётных строк
%%     }{
%%       \vspace*{-0.8ex}\itshape} % Стиль для чётных строк
%%   }
%% }
%% \newcolumntype{A}{ >{\altshape}X[1mc]}
%% \needspace{2\baselineskip}
%% \renewcommand{\arraystretch}{0.9}%% Уменьшаем  расстояние между
%%                                 %% рядами, чтобы таблица не так много
%%                                 %% места занимала в дисере.
%% \begin{longtabu} to \textwidth {@{}X[0.2ml]X[0.9mc]AAAX[0.99mc]>{\setlength{\baselineskip}{0.7\baselineskip}}AA<{\stepcounter{rowcnt}}@{}}
%% % \begin{longtabu} to \textwidth {@{}X[0.2ml]X[1mc]X[1mc]X[1mc]X[1mc]X[1mc]>{\setlength{\baselineskip}{0.7\baselineskip}}X[1mc]X[1mc]@{}}
%%   \caption{Длинная таблица с примером чересстрочного форматирования\label{tbl:other-row}}\vspace*{1ex}\\% label всегда желательно идти после caption
%%   % \vspace*{1ex}     \\

%%   \toprule %%% верхняя линейка  
%% \setcounter{rowcnt}{0} &Итерации & JADE\texttt{++} & JADE & jDE & SaDE
%% & DE/rand /1/bin & PSO \\ 
%%  \midrule %%% тонкий разделитель. Отделяет названия столбцов. Обязателен по ГОСТ 2.105 пункт 4.4.5 
%%  \endfirsthead

%%  \multicolumn{8}{c}{\small\slshape (продолжение)} \\ 
%%  \toprule %%% верхняя линейка
%% \setcounter{rowcnt}{0} &Итерации & JADE\texttt{++} & JADE & jDE & SaDE
%% & DE/rand /1/bin & PSO \\ 
%%  \midrule %%% тонкий разделитель. Отделяет названия столбцов. Обязателен по ГОСТ 2.105 пункт 4.4.5 
%%  \endhead
 
%%  \multicolumn{8}{c}{\small\slshape (окончание)} \\ 
%%  \toprule %%% верхняя линейка
%% \setcounter{rowcnt}{0} &Итерации & JADE\texttt{++} & JADE & jDE & SaDE
%% & DE/rand /1/bin & PSO \\ 
%%  \midrule %%% тонкий разделитель. Отделяет названия столбцов. Обязателен по ГОСТ 2.105 пункт 4.4.5 
%%  \endlasthead

%%  \bottomrule %%% нижняя линейка
%%  \multicolumn{8}{r}{\small\slshape продолжение следует}     \\ 
%%  \endfoot 
%%  \endlastfoot
 
%% f1  & 1500 & \textbf{1.8E-60}   & 1.3E-54   & 2.5E-28   & 4.5E-20   & 9.8E-14   & 9.6E-42   \\\nopagebreak
%%     &      & (8.4E-60) & (9.2E-54) & \color{red}(3.5E-28) & (6.9E-20) & (8.4E-14) & (2.7E-41) \\
%% f2  & 2000 & 1.8E-25   & 3.9E-22   & 1.5E-23   & 1.9E-14   & 1.6E-09   & 9.3E-21   \\\nopagebreak
%%     &      & (8.8E-25) & (2.7E-21) & (1.0E-23) & (1.1E-14) & (1.1E-09) & (6.3E-20) \\
%% f3  & 5000 & 5.7E-61   & 6.0E-87   & 5.2E-14   & \color{green}9.0E-37   & 6.6E-11   & 2.5E-19   \\\nopagebreak
%%     &      & (2.7E-60) & (1.9E-86) & (1.1E-13) & (5.4E-36) & (8.8E-11) & (3.9E-19) \\
%% f4  & 5000 & 8.2E-24   & 4.3E-66   & 1.4E-15   & 7.4E-11   & 4.2E-01   & 4.4E-14   \\\nopagebreak
%%     &      & (4.0E-23) & (1.2E-65) & (1.0E-15) & (1.8E-10) & (1.1E+00) & (9.3E-14) \\
%% f5  & 3000 & 8.0E-02   & 3.2E-01   & 1.3E+01   & 2.1E+01   & 2.1E+00   & 2.5E+01   \\\nopagebreak
%%     &      & (5.6E-01) & (1.1E+00) & (1.4E+01) & (7.8E+00) & (1.5E+00) & (3.2E+01) \\
%% f6  & 100  & 2.9E+00   & 5.6E+00   & 1.0E+03   & 9.3E+02   & 4.7E+03   & 4.5E+01   \\\nopagebreak
%%     &      & (1.2E+00) & (1.6E+00) & (2.2E+02) & (1.8E+02) & (1.1E+03) & (2.4E+01) \\
%% f7  & 3000 & 6.4E-04   & 6.8E-04   & 3.3E-03   & 4.8E-03   & 4.7E-03   & 2.5E-03   \\\nopagebreak
%%     &      & (2.5E-04) & (2.5E-04) & (8.5E-04) & (1.2E-03) & (1.2E-03) & (1.4E-03) \\
%% f8  & 1000 & 3.3E-05   & 7.1E+00   & 7.9E-11   & 4.7E+00   & 5.9E+03   & 2.4E+03   \\\nopagebreak
%%     &      & (2.3E-05) & (2.8E+01) & (1.3E-10) & (3.3E+01) & (1.1E+03) & (6.7E+02) \\
%% f9  & 1000 & 1.0E-04   & 1.4E-04   & 1.5E-04   & 1.2E-03   & 1.8E+02   & 5.2E+01   \\\nopagebreak
%%     &      & (6.0E-05) & (6.5E-05) & (2.0E-04) & (6.5E-04) & (1.3E+01) & (1.6E+01) \\
%% f10 & 500  & 8.2E-10   & 3.0E-09   & 3.5E-04   & 2.7E-03   & 1.1E-01   & 4.6E-01   \\\nopagebreak
%%     &      & (6.9E-10) & (2.2E-09) & (1.0E-04) & (5.1E-04) & (3.9E-02) & (6.6E-01) \\
%% f11 & 500  & 9.9E-08   & 2.0E-04   & 1.9E-05   & 7.8E-04)  & 2.0E-01   & 1.3E-02   \\\nopagebreak
%%     &      & (6.0E-07) & (1.4E-03) & (5.8E-05) & (1.2E-03  & (1.1E-01) & (1.7E-02) \\
%% f12 & 500  & 4.6E-17   & 3.8E-16   & 1.6E-07   & 1.9E-05   & 1.2E-02   & 1.9E-01   \\\nopagebreak
%%     &      & (1.9E-16) & (8.3E-16) & (1.5E-07) & (9.2E-06) & (1.0E-02) & (3.9E-01) \\
%% f13 & 500  & 2.0E-16   & 1.2E-15   & 1.5E-06   & 6.1E-05   & 7.5E-02   & 2.9E-03   \\\nopagebreak
%%     &      & (6.5E-16) & (2.8E-15) & (9.8E-07) & (2.0E-05) & (3.8E-02) & (4.8E-03) \\
%% f1  & 1500 & \textbf{1.8E-60}   & 1.3E-54   & 2.5E-28   & 4.5E-20   & 9.8E-14   & 9.6E-42   \\\nopagebreak
%%     &      & (8.4E-60) & (9.2E-54) & \color{red}(3.5E-28) & (6.9E-20) & (8.4E-14) & (2.7E-41) \\
%% f2  & 2000 & 1.8E-25   & 3.9E-22   & 1.5E-23   & 1.9E-14   & 1.6E-09   & 9.3E-21   \\\nopagebreak
%%     &      & (8.8E-25) & (2.7E-21) & (1.0E-23) & (1.1E-14) & (1.1E-09) & (6.3E-20) \\
%% f3  & 5000 & 5.7E-61   & 6.0E-87   & 5.2E-14   & 9.0E-37   & 6.6E-11   & 2.5E-19   \\\nopagebreak
%%     &      & (2.7E-60) & (1.9E-86) & (1.1E-13) & (5.4E-36) & (8.8E-11) & (3.9E-19) \\
%% f4  & 5000 & 8.2E-24   & 4.3E-66   & 1.4E-15   & 7.4E-11   & 4.2E-01   & 4.4E-14   \\\nopagebreak
%%     &      & (4.0E-23) & (1.2E-65) & (1.0E-15) & (1.8E-10) & (1.1E+00) & (9.3E-14) \\
%% f5  & 3000 & 8.0E-02   & 3.2E-01   & 1.3E+01   & 2.1E+01   & 2.1E+00   & 2.5E+01   \\\nopagebreak
%%     &      & (5.6E-01) & (1.1E+00) & (1.4E+01) & (7.8E+00) & (1.5E+00) & (3.2E+01) \\
%% f6  & 100  & 2.9E+00   & 5.6E+00   & 1.0E+03   & 9.3E+02   & 4.7E+03   & 4.5E+01   \\\nopagebreak
%%     &      & (1.2E+00) & (1.6E+00) & (2.2E+02) & (1.8E+02) & (1.1E+03) & (2.4E+01) \\
%% f7  & 3000 & 6.4E-04   & 6.8E-04   & 3.3E-03   & 4.8E-03   & 4.7E-03   & 2.5E-03   \\\nopagebreak
%%     &      & (2.5E-04) & (2.5E-04) & (8.5E-04) & (1.2E-03) & (1.2E-03) & (1.4E-03) \\
%% f8  & 1000 & 3.3E-05   & 7.1E+00   & 7.9E-11   & 4.7E+00   & 5.9E+03   & 2.4E+03   \\\nopagebreak
%%     &      & (2.3E-05) & (2.8E+01) & (1.3E-10) & (3.3E+01) & (1.1E+03) & (6.7E+02) \\
%% f9  & 1000 & 1.0E-04   & 1.4E-04   & 1.5E-04   & 1.2E-03   & 1.8E+02   & 5.2E+01   \\\nopagebreak
%%     &      & (6.0E-05) & (6.5E-05) & (2.0E-04) & (6.5E-04) & (1.3E+01) & (1.6E+01) \\
%% f10 & 500  & 8.2E-10   & 3.0E-09   & 3.5E-04   & 2.7E-03   & 1.1E-01   & 4.6E-01   \\\nopagebreak
%%     &      & (6.9E-10) & (2.2E-09) & (1.0E-04) & (5.1E-04) & (3.9E-02) & (6.6E-01) \\
%% f11 & 500  & 9.9E-08   & 2.0E-04   & 1.9E-05   & 7.8E-04)  & 2.0E-01   & 1.3E-02   \\\nopagebreak
%%     &      & (6.0E-07) & (1.4E-03) & (5.8E-05) & (1.2E-03  & (1.1E-01) & (1.7E-02) \\
%% f12 & 500  & 4.6E-17   & 3.8E-16   & 1.6E-07   & 1.9E-05   & 1.2E-02   & 1.9E-01   \\\nopagebreak
%%     &      & (1.9E-16) & (8.3E-16) & (1.5E-07) & (9.2E-06) & (1.0E-02) & (3.9E-01) \\
%% f13 & 500  & 2.0E-16   & 1.2E-15   & 1.5E-06   & 6.1E-05   & 7.5E-02   & 2.9E-03   \\\nopagebreak
%%     &      & (6.5E-16) & (2.8E-15) & (9.8E-07) & (2.0E-05) & (3.8E-02) & (4.8E-03) \\

%%     % \vspace*{1ex}     \\
%% %         \midrule%%% тонкий разделитель
%% %         \multicolumn{3}{@{}p{\textwidth}}{%
%% %             % \vspace*{-4ex}% этим подтягиваем повыше
%% %             % \hspace*{2.5em}% абзацный отступ - требование ГОСТ 2.105
%% %             Примечание "---  Для функций $f_{12}$ и $f_{13}$
%% %             используется $y_i = 1 + \frac{1}{4}(x_i+1)$ и
%% %             $u(x_i,\,a,\,k,\,m)=\begin{cases}
%% % k(x_i-a)^m,\quad  & x_i >a     \\[-0.5em]
%% % 0,\quad           & -a\leq x_i \leq a        \\[-0.5em]
%% % k(-x_i-a)^m,\quad & x_i <-a
%% % \end{cases}$  }     \\
%% \bottomrule %%% нижняя линейка 
%% \end{longtabu} \endgroup

%% \section{Очередной подраздел приложения} \label{AppendixB3}

%% Нужно больше подразделов приложения!

%% \section{И ещё один подраздел приложения} \label{AppendixB4}

%% Нужно больше подразделов приложения!

