\chapter{Сертификация в симуляции \ARMt~машины}
\label{sec:app:cert}

При доказательстве того, что обещающая машина может симулировать исполнение \ARMt~машины,
мы используем леммы \ref{lem-snd} и \ref{lem-fst}, в которых конструируется шаг обещающей машины.
Согласно определению после каждого шага обещающая машина должна показывать, что находится в \emph{сертифицируемом состоянии}
(предикат $\certifiable$), т.е. что для каждого потока существует последовательность
локальных шагов, при исполнении которых поток
выполняет все данные обещания (предикат $\certifiableTid$):
\[\begin{array}{r l l}
\certifiable(\p) & \triangleq & \forall \tId. \; \certifiableTid(\tIdState(\tId,\p)); \\
\certifiableTid(\tT) & \triangleq & \exists \tT'. \; \tT \promTStep^{*} \tT' \land \tT'.\PromSet = \emptyset; \\
\tIdState(\tId, \p) & \triangleq & \angled{\p.\Mprom, \p.\TSfprom(\tId)}. \\
\end{array}\]

Теорема \ref{cert-thm}, которая является ключевой в данном приложении и используется в доказательстве
лемм \ref{lem-snd} и \ref{lem-fst}, утверждает,
что если состояния обещающей и \ARMt~машин связаны отношением $\simrelBase$, и при этом состояние \ARMt~машины
достижимо из начального, то состояние обещающей машины сертифицируемо.
%% \begin{restatable}{thm}{certThm}
\begin{theorem}
\label{cert-thm}
$\forall (\aT, \p) \in \simrelBase. \; \ainit \armStepP^{*} \aT \Rightarrow \certifiable(\p)$.
%, \lnot \invTidWriteComCERT(\tId, \aT, \p)$.
%% \end{restatable}
\end{theorem}

\section{Структура доказательства теоремы \ref{cert-thm}}

Какие невыполненные обещания есть у потока $\tId$ обещающей машины в состоянии $\p$,
если оно связано с некоторым состоянием $\aT$ машины \ARMt отношением $\simrelBase$?
Согласно отношению $\invMemTwo$, которое 

$\invMemOne$ и , которые 

\app{Неформально описать то, как по плёнке строится исполнение обещающей машины.}
%% \app{Привести формальное доказательство.}

Для реализации описанной выше идеи мы используем отношение
$\invCert(n, \delta, k, \tId, \aT, \tT)$, которое имеет шесть параметров:
\begin{itemize}
  \item $\tId$ --- это идентификатор потока, в контексте сертификации которого используется данный элемент отношения;
  \item $n$ --- это количество экземпляров на пути от указателя потока до последнего завершённого экземпляра записи в
    соответствующей плёнке;
  \item $k$ --- это номер инструкции, следующей за последним завершённым экземпляром записи;
  \item $\delta$ --- это \app{TODO};
  \item $\aT$ --- это состояние машины \ARMt, которое используется для сертификации потока обещающей машины;
  \item $\tT$ --- это состояние потока $\tId$ обещающей машины.
\end{itemize}
Перед тем, как предъявить формальное определение данного отношения, мы приведём утверждения лемм, которые
его используют.

Лемма \ref{cert-lem-three} утверждает, что из $(n, \delta, k, \tId, \aT, \tT) \in \invCert$ следует,
что поток $\tId$ обещающей машины сертифицируем.
\begin{lemma}
\label{cert-lem-three}
$\forall n, \delta, k, \tId, \aT, \tT. \; \invCert(n, \delta, k, \tId, \aT, \tT) \Rightarrow \certifiableTid(\tT)$.
%% \end{restatable}
\end{lemma}
Доказательство данной леммы проводится индукцией по $n$.
Базу индукции мы выделили в лемму \ref{cert-lem-two}, а индукционный переход --- в лемму \ref{cert-lem-one}.

\begin{lemma}
%% \begin{restatable}{lem}{certLemTwo}
\label{cert-lem-two}
$\forall k, \delta, \tId, \aT, \tT, \invCert(0, \delta, k, \tId, \aT, \tT), \tT.\PromSet = \emptyset$.
%% \end{restatable}
\end{lemma}

\begin{lemma}
\label{cert-lem-one}
$\forall n \not = 0, \delta, k, \tId, \aT, \tT, \invCert(n, \delta, k, \tId, \aT, \tT).$ \\
${} \quad \exists \delta', \tT'. \Cfprom(\tId) \vdash \tT \promTStep^{*} \tT' \land \invCert(n - 1, \delta', k, \tId, \aT, \tT')$.
\end{lemma}

\section{Описание отношения $\invCert$}

\[\begin{array}{l}
\invTidWriteComCERT(\tId, \aT, \p) \triangleq \\
\quad \textLet \tape, \cpath \triangleq \aT.\tapef(\tId), \p.\TSfprom(\tId).\cpath \; \textIn \\
\quad \exists \cpath' \ge \cpath. \tape(\cpath') \; \text{is a committed write}. \\
\\
\deltaToView : \Path \rightarrow (\Path \rightharpoonup (\Loc \times \Timestamp)) \rightarrow \View \\
\deltaToView(\cpath, \delta) \triangleq \bigsqcup \{ \delta(\cpath') | \forall \cpath' < \cpath, \delta(\cpath') \not = \bot \}. \\
%% \\
%% \lessUpToDelta{\delta}{\R}{\R'} \triangleq \forall \loc. \R(\loc) < \R'(\loc) + \delta(\loc). \\
%% \\
%% \invViewDeltaCERT(\delta, \tId, \aT, \tT) \triangleq \\
%%   \quad \textLet \cpath, \tape \triangleq \tT.\cpath, \aT.\tapef(\tId) \; \textIn \\
%%   \quad \textLet \cpath^{\LD}, \cpath^{\SY}, \cpath^{\LD\SY} \triangleq
%%      \lastLD(\tape, \cpath), \lastSY(\tape, \cpath), \lastLDSY(\tape, \cpath) \; \textIn \\
%%   \quad (\tT.\Racq \le \bigsqcup \readsCommittedR(\cpath, \tape, \aT.\hmap) \sqcup
%%                        \bigsqcup \opstau(\tId, \cpath, \tape, \aT.\hmap) \sqcup \\
%%   \quad \quad \deltaToView(\cpath, \delta)) \land {} \\
%%   \quad (\tT.\Rcur \le \bigsqcup \readsCommittedR(\cpath^{\LD}, \tape, \aT.\hmap) \sqcup
%%                        \bigsqcup \opstau(\tId, \cpath, \tape, \aT.\hmap) \sqcup \\
%%   \quad \quad \deltaToView(\cpath, \delta)) \land {} \\
%%   \quad (\tT.\Rrel \le \bigsqcup \readsCommittedR(\cpath^{\LD\SY}, \tape, \aT.\hmap) \sqcup \bigsqcup \opstau(\tId, \cpath^{\SY}, \tape, \aT.\hmap)).\\
\end{array}\]

\[\begin{array}{l}
\invViewRelCERT(\tId, \aT, \tT) \triangleq \forall \cpath \ge \tT.\cpath. \\
  \quad \textLet \tape \triangleq \aT.\tapef(\tId) \; \textIn \\
  \quad \textLet \cpath^{\LD} \triangleq \lastLD(\tape, \cpath) \; \textIn \\
  \quad \tT.\Rrel \le
  \bigsqcup \readsCommittedR(\cpath^{\LD}, \tape, \aT.\hmap) \sqcup \bigsqcup \opstau(\tId, \cpath, \tape, \aT.\hmap).\\
\\
\invViewWriteCERT(\tId, \aT, \tT) \triangleq
  \forall \cpath' \ge \tT.\cpath, \loc. \;
    \aT.\tapef(\tId, \cpath') = \tapeWrite{(\tapeWriteCommitted{\_}{\loc}{\_})} \Rightarrow \\
\quad \quad \tT.\Rcur(\loc) < \aT.\tmap(\tId, \cpath') \land {} \\
\quad \quad ((\exists \cpath''. \; \tT.\cpath \le \cpath'' < \cpath' \land 
             \aT.\tapef(\tId, \cpath'') = \tapeFence{\Committed}{\LD}) \Rightarrow \\
\quad \quad \quad \tT.\Racq(\loc) < \aT.\tmap(\tId, \cpath')).\\
%% \\
%% \invWriteTimestampCERT(\delta, \tId, \aT, \tT) \triangleq \\
%% \quad \forall \cpath' < \tT.\cpath, \w, \aT.\tapef(\tId, \cpath') = \tapeRead{(\tapeSatisfied{\Committed}{\w})}. \\
%% \quad (\aT.\tapef(\w.\tId, \w.\cpath) \text{ is committed}) \lor \\
%% \quad (\w.\tId = \tId \land \delta(\cpath) \not = \bot). \\
\end{array}\]

\[\begin{array}{l}
\deltaHmap(\delta, \tId_{\tT}, \aT) \triangleq \lambda \tId, \cpath. \\
  \quad \textIf \tId = \tId_{\tT} \land \aT.\hmap(\tId, \cpath) = \bot \; \textThen \delta(\cpath) \\
  \quad \textElif \exists \tau, \R, \angled{\tau, \_, \R} = \aT.\tmap(\tId, \cpath) \; \textThen \angled{\tau, \R} \\
  \quad \textElse \bot. \\
\\
\invViewReadCERT(\delta, \tId, \aT, \tT) \triangleq
  \forall \cpath' \ge \tT.\cpath, \w, \tau. \\
\quad \aT.\tapef(\tId, \cpath') = \tapeRead{(\tapeSatisfied{\Committed}{\w})} \land
   \angled{\tau, \_} = \deltaHmap(\delta, \tId, \aT, \w.\tId, \w.\cpath) \Rightarrow \\
\quad \quad \tT.\Rcur(\w.\loc) \le \tau \land {} \\
\quad \quad ((\exists \cpath''. \; \tT.\cpath \le \cpath'' < \cpath' \land 
             \aT.\tapef(\tId, \cpath'') = \tapeFence{\Committed}{\LD}) \Rightarrow \tT.\Racq(\w.\loc) \le \tau). \\
%% \\
%% \invViewCERT(\tId, \aT, \tT) \triangleq \exists \delta : \Loc \rightarrow [0, 1). \\
%% \quad \invViewDeltaCERT(\delta, \tId, \aT, \tT),
%%       \invViewWriteCERT(\delta, \tId, \aT, \tT), \invViewReadCERT(\delta, \tId, \aT, \tT). \\
\\
\invStateCERT(\tId, \aT, \tT) \triangleq \\
\quad \textLet \regstcom \triangleq \regstcom(\Cfarm(\tId), \aT.\tapef(\tId), \tT.\cpath) \; \textIn \\
\quad \forall \reg. \; \regstcom(\reg) = \bot \lor \tT.\PromState(\reg) = \regstcom(\reg).\\
\end{array}\]

\[\begin{array}{l}
\invMemZeroCERT(\delta, \tId_{\tT}, \aT, \tT) \triangleq
      \forall \cpath, \tau, \R, \loc, \expr_0, \expr_1. \\
\quad \angled{\tau, \R} = \delta(\cpath) \land
      ``\writeInst{\expr_0}{\expr_1}" = \Cfarm(\tId_{\tT}, \lastInstr{\cpath}) \land \loc = \semfcom{\expr_0}{\cpath} \Rightarrow \\
\quad \exists \stval. \semfcom{\expr_1}{\cpath} \in \{\bot, \stval\} \land
      \writeEvt{\loc}{\stval}{\tau}{\R} \in \tT.\Mprom \setminus \tT.\PromSet. \\
\\
\invMemOneCERT(\tId_{\tT}, \aT, \tT) \triangleq \forall \tId, \loc, \stval, \tau, \R', \cpath. \\
  \quad \tapeWrite{(\tapeWriteCommitted{\_}{\loc}{\stval})} = \aT.\tapef(\tId, \cpath) \land
        \angled{\tau, \_, \R'} = \aT.\hmap(\tId, \cpath) \Rightarrow \\
  \quad \exists \R \le \R'. \; \writeEvt{\loc}{\stval}{\tau}{\R} \in \tT.\Mprom \land {} \\
  \quad \quad (\tId \not = \tId_{\tT} \lor \cpath <   \tT.\cpath \Rightarrow
               \writeEvt{\loc}{\stval}{\tau}{\R} \not \in \tT.\PromSet) \land {} \\
  \quad \quad (\tId = \tId_{\tT} \land \cpath \ge \tT.\cpath \Rightarrow
               \writeEvt{\loc}{\stval}{\tau}{\R} \in \tT.\PromSet).\\
\\
\invMemTwoCERT(\tId, \aT, \tT) \triangleq \forall \writeEvt{\loc}{\stval}{\tau}{\R} \in \tT.\PromSet. \;
     \tau \not = \tstamp{0} \Rightarrow \\
  \quad \exists \R' \ge \R, \cpath \ge \tT.\cpath . \\
  \quad \quad \tapeWrite{(\tapeWriteCommitted{\_}{\loc}{\stval})} = \aT.\tapef(\tId, \cpath) \land
              \angled{\tau, \_, \R'} = \aT.\hmap(\tId, \cpath). \\
\end{array}\]

\[\begin{array}{l}
\invDeltaDefOne(\delta, \tId_{\tT}, \aT, \tT) \triangleq \forall \cpath < \tT.\cpath. \; \delta(\cpath) \not = \bot \Leftrightarrow \\
\quad (\exists \expr_0, \expr_1. \; ``\writeInst{\expr_0}{\expr_1}" = \Cfarm(\tId_{\tT}, \lastInstr{\cpath}) \land {} \\
\quad \quad \aT.\tapef(\tId_{\tT}, \cpath) \text{ isn't committed}). \\
\\
\invDeltaDefTwo(\delta, \tId_{\tT}, \aT, \tT) \triangleq \forall \cpath, \angled{\tau, \R} = \delta(\cpath), \\
\quad ``\writeInst{\expr_0}{\expr_1}" = \Cfarm(\tId_{\tT}, \lastInstr{\cpath}), \loc = \semfcom{\expr_0}{\cpath}. \\
\quad \quad \R = [\loc @ \tau] \sqcup \tT.\Rrel \land \tT.\Rcur(\loc) \ge \tau. \\
\\
\invDeltaDefThree(\delta, \tId_{\tT}, \aT) \triangleq
  \forall \cpath, \cpath' \not = \cpath, \\
\quad \angled{\tau, \_} = \deltaHmap(\delta, \tId_{\tT}, \aT, \tId_{\tT}, \cpath),
  \angled{\tau', \_} = \deltaHmap(\delta, \tId_{\tT}, \aT, \tId_{\tT}, \cpath'), \\
\quad ``\writeInst{\expr_0}{\expr_1}"   = \Cfarm(\tId_{\tT}, \lastInstr{\cpath}),
      ``\writeInst{\expr'_0}{\expr'_1}" = \Cfarm(\tId_{\tT}, \lastInstr{\cpath'}). \\
\quad \semfcom{\expr_0}{\cpath} = \semfcom{\expr'_0}{\cpath'} \Rightarrow \tau \not = \tau'.
\\
\invDeltaDefFour(\delta, \tId_{\tT}, \aT) \triangleq
      \forall \cpath_{\delta} < \cpath_{read} < \cpath_{\LD} < \cpath_{\delta-read}, \\
\quad \angled{\tau, \_} = \delta(\cpath_{\delta}), \w, \loc, \R. \\
\quad \tape(\cpath_{read}) = \tapeRead{(\tapeSatisfied{\Committed}{\w})} \land \tape(\cpath_{\LD}) = \tapeFence{\Committed}{\LD} \land {} \\
\quad \tape(\cpath_{\delta-read}) = \tapeRead{(\tapeSatisfied{\Committed}{\stRequestWrite{\tId_{\tT}}{\cpath_{\delta}}{\loc}{\_}})} \land {} \\
\quad \R = \aT.\rmap(\w.\tId, \w.\cpath) \not = \bot
      \Rightarrow \\
\quad \quad \R(\loc) \le \tau. \\
\end{array}\]

\[\begin{array}{l}
\invCert \subset \mathbb{N} \times (\Path \rightharpoonup (\Timestamp \times \View))
                            \times \mathbb{N} \times \Tid \times \StateARMtau \times \TStateProm \\
\invCert(n, \delta, k, \tId, \aT, \tT) \triangleq \\
\quad \textLet \tape \triangleq \aT.\tapef(\tId) \; \textIn \\
\quad \textLet \cpath_{\textup{\sf last-wcom}} \triangleq \lastCommittedWrite(\tape) \; \textIn \\
\quad \textLet \cpath_{\textup{\sf next-last}} \triangleq \cpath_{\textup{\sf last-wcom}}:k \; \textIn \\
\quad \ainit \armStepP^{*} \aT \land {} \\
\quad \tT.\cpath \le \cpath_{\textup{\sf next-last}} \land (n = \length(\cpath_{\textup{\sf next-last}}) - \length(\tT.\cpath)) \land {} \\
\quad (\forall \cpath'. \; \tT.\cpath \le \cpath' < \cpath_{\textup{\sf next-last}} \Rightarrow \\
\quad \quad (\Cfprom(\tId)[\lastInstr{\cpath'}] \in \{``\ifGotoInst{\_}{\_}", ``\fenceInst{\_}"\} \Rightarrow
             \tape(\cpath') \text{ is committed}) \land {} \\
\quad \quad \tape(\cpath') \text{ has a fully determined address} \land {} \\
\quad \quad \lnot \Cfprom(\tId)[\lastInstr{\cpath'}] = ``\fenceInst{\SY}") \land {} \\
\quad (\forall \cpath'', \cpath', \Rstate. \\
\quad \quad \cpath'' \ge \cpath' \ge \tT.\cpath \land \tape(\cpath') = \tapeRead{\Rstate} \land
       \Rstate \not = \tapeSatisfied{\Committed}{\_} \Rightarrow \\
\quad \quad \quad \tape(\cpath'') \not = \tapeFence{\_}{\_}) \land {} \\
\quad (\forall \cpath' \ge \tT.\cpath, \delta(\cpath') = \bot) \land {} \\
\quad (\tId, \aT, \tT) \in \invMemOneCERT \cap \invMemTwoCERT \cap \invStateCERT \cap \invViewRelCERT \cap \invViewWriteCERT \land {} \\
\quad (\delta, \tId, \aT, \tT) \in \invMemZeroCERT \cap \invDeltaDefOne \cap \invDeltaDefTwo \cap
               %% \invViewDeltaCERT \cap
               %% \invWriteTimestampCERT \cap
               \invViewReadCERT \land {} \\
\quad (\delta, \tId, \aT) \in \invDeltaDefThree \cap \invDeltaDefFour. \\
\end{array}\]

\section{Доказательство вспомогательных лемм и теоремы}

\noindent
{\bf Лемма \ref{cert-lem-two}.}
$\forall k, \delta, \tId, \aT, \tT, \invCert(0, \delta, k, \tId, \aT, \tT), \tT.\PromSet = \emptyset$.
%% \end{restatable}
\begin{proof}%[Доказательство леммы \ref{cert-lem-two}]
  Fix $k, \delta, \tId, \aT, \tT$. From $\invCert(0, \delta, k, \tId, \aT, \tT)$ we know that
  $\tT.\cpath = \lastCommittedWrite(\aT.\tapef(\tId)):k$, as
  $\length(\tT.\cpath) = \length(\lastCommittedWrite(\aT.\tapef(\tId))) + 1$.
  
  Suppose $\exists \writeEvt{\loc}{\stval}{\tau}{\R} \in \tT.\PromSet$. By $\invMemTwoCERT(\tId, \aT, \tT)$,
  $\exists \R' \ge \R, \cpath \ge \tT.\cpath.$
  $\tapeWrite{(\tapeWriteCommitted{\_}{\loc}{\stval})} = \aT.\tapef(\tId, \cpath)$.
  $\cpath \ge \tT.\cpath > \lastCommittedWrite(\aT.\tapef(\tId))$ --- a contradiction.
\end{proof}

\noindent
{\bf Теорема \ref{cert-thm}.}
$\forall (\aT, \p) \in \simrelBase. \ainit \armStepP^{*} \aT \Rightarrow \certifiable(\p)$.
\begin{proof}%[Доказательство теоремы \ref{cert-thm}]
  Fix $\tId, \tape \triangleq \aT.\tapef(\tId)$. \\
  Some notations:
  \[\begin{array}{l l l}
    k & \triangleq & \lastInstr{\lastCommittedWrite(\tape)} + 1; \\
    n & \triangleq & \length(\lastCommittedWrite(\tape):k) - \length(\tT.\cpath). \\
  \end{array}\]

  Let's check that $\invCert(n, \lambda \cpath. \; \bot, k, \tId, \aT, \tT)$ holds.
  More notations:
  \[\begin{array}{l l l}
    \cpath_{\textup{\sf last-wcom}} & \triangleq & \lastCommittedWrite(\tape); \\
    \cpath_{\textup{\sf next-last}} & \triangleq & \cpath_{\textup{\sf last-wcom}}:k; \\
    \delta & \triangleq & \lambda \cpath. \; \bot. \\
  \end{array}\]

  We have to show:
  \[\begin{array}{l}
    \ainit \armStepP^{*} \aT \land {} \\
    \tT.\cpath \le \cpath_{\textup{\sf next-last}} \land (n = \length(\cpath_{\textup{\sf next-last}}) - \length(\tT.\cpath)) \land {} \\
    (\forall \cpath'. \; \tT.\cpath \le \cpath' < \cpath_{\textup{\sf next-last}} \Rightarrow \\
    \quad (\Cprom[\lastInstr{\cpath'}] \in \{``\ifGotoInst{\_}{\_}", ``\fenceInst{\_}"\} \Rightarrow
             \tape(\cpath') \text{ is committed}) \land {} \\
    \quad \tape(\cpath') \text{ has a fully determined address} \land {} \\
    \quad \lnot \Cprom[\lastInstr{\cpath'}] = ``\fenceInst{\SY}") \land {} \\
    (\forall \cpath'', \cpath', \Rstate. \\
    \quad \cpath'' \ge \cpath' \ge \tT.\cpath \land \tape(\cpath') = \tapeRead{\Rstate} \land
       \Rstate \not = \tapeSatisfied{\Committed}{\_} \Rightarrow \\
    \quad \quad \tape(\cpath'') \not = \tapeFence{\_}{\_}) \land {} \\
    (\forall \cpath' \ge \tT.\cpath, \delta(\cpath') = \bot) \land {} \\
    (\tId, \aT, \tT) \in \invMemOneCERT \cap \invMemTwoCERT \cap \invStateCERT \cap \invViewRelCERT \cap \invViewWriteCERT \land {} \\
    (\delta, \tId, \aT, \tT) \in \invMemZeroCERT \cap \invDeltaDefOne \cap \invDeltaDefTwo \cap
               %% \invViewDeltaCERT \cap
               %% \invWriteTimestampCERT \cap
               \invViewReadCERT \land {} \\
    (\delta, \tId, \aT) \in \invDeltaDefThree \cap \invDeltaDefFour. \\
  \end{array}\]

  The following obviously holds:
  \[\begin{array}{l}
    \ainit \armStepP^{*} \aT \land {} \\
    \tT.\cpath \le \cpath_{\textup{\sf next-last}} \land (n = \length(\cpath_{\textup{\sf next-last}}) - \length(\tT.\cpath)) \land {} \\
    (\forall \cpath' \ge \tT.\cpath, \delta(\cpath') = \bot). \\
  \end{array}\]

  The following holds as committed memory instructions can't be restarted by the ARM machine and requirements of
  the {\sf \bf Fence commit} transition:
  \[\begin{array}{l}
    \forall \cpath'', \cpath', \Rstate. \\
    \quad \cpath'' \ge \cpath' \ge \tT.\cpath \land \tape(\cpath') = \tapeRead{\Rstate} \land
       \Rstate \not = \tapeSatisfied{\Committed}{\_} \Rightarrow \\
    \quad \quad \tape(\cpath'') \not = \tapeFence{\_}{\_}. \\
  \end{array}\]

  The following holds as committed memory instructions can't be restarted by the ARM machine, requirements of
  the {\sf \bf Write commit} transition, and $\invComWrite(\aT, \p)$:
  \[\begin{array}{l}
    \forall \cpath'. \; \tT.\cpath \le \cpath' < \cpath_{\textup{\sf next-last}} \Rightarrow \\
    \quad (\Cprom[\lastInstr{\cpath'}] \in \{``\ifGotoInst{\_}{\_}", ``\fenceInst{\_}"\} \Rightarrow
             \tape(\cpath') \text{ is committed}) \land {} \\
    \quad \tape(\cpath') \text{ has a fully determined address} \land
    \quad \lnot \Cprom[\lastInstr{\cpath'}] = ``\fenceInst{\SY}". \\
  \end{array}\]

  $(\delta, \tId, \aT) \in \invDeltaDefThree \cap \invDeltaDefFour$ and
  $(\delta, \tId, \aT, \tT) \in \invDeltaDefOne \cap \invDeltaDefTwo \cap \invMemZeroCERT$
  hold as $\delta = \lambda \cpath. \; \bot$.
  $\invViewReadCERT(\delta, \tId, \aT, \tT)$ follows from $\delta = \lambda \cpath. \; \bot$ and \app{\ref{thm:invAview}}.

  
  $\invMemOneCERT(\tId, \aT, \tT)$, $\invMemTwoCERT(\tId, \aT, \tT)$, and $\invStateCERT(\tId, \aT, \tT)$
  directly follow from $\invMemOne(\aT, \p)$, $\invMemTwo(\aT, \p)$, and $\invState(\aT, \p)$ respectively.
  $\invViewRelCERT(\tId, \aT, \tT)$ follows from $\invView(\aT, \p)$ and a fact that all reads before a committed fence are committed
  in the ARM machine. $\invViewWriteCERT(\tId, \aT, \tT)$ follows $\invView(\aT, \p)$ and \app{\ref{thm:invAview}}.

  After we showed $\invCert(n, \delta, k, \tId, \aT, \tT)$ we apply \app{\ref{cert-lem-three}}.
\end{proof}




%% \begin{restatable}{lem}{certLemOne}
\begin{proof}[Доказательство леммы \ref{cert-lem-one}]
  Fix $n, \tId, \aT, \tT$.
  Some notations:
  \[\begin{array}{l c l}
  \tape & \triangleq & \aT.\tapef(\tId); \\
  \angled{\Mprom, \angled{\cpath, \PromState, \V, \PromSet}} & \triangleq & \tT; \\
  \Cprom & \triangleq & \Cfprom(\tId). \\
  \end{array}\]

  Case analysis on $\tape(\cpath)$:
  \begin{itemize}
    \item $\tape(\cpath) = \tapeNop$.
      Some notations:
      \[\begin{array}{l l l}
        \cpath' & \triangleq & \nextPathProm(\cpath, 1); \\
        \tT'    & \triangleq & \angled{\Mprom, \angled{\cpath', \PromState, \V, \PromSet}}.
      \end{array}\]
      As $\ainit \armStepP^{*} \aT$ and $\tape(\cpath) = \tapeNop$, $\Cprom(\lastInstr{\cpath}) = \nop$.
      Thus, $\tT \promStepNop \tT'$.
      $\invCert(n - 1, \delta, k, \tId, \aT, \tT')$ obviously holds.

    \item $\tape(\cpath) = \tapeAssign$.
      As $\ainit \armStepP^{*} \aT$ and $\tape(\cpath) = \tapeAssign$, \\
      $\exists \reg, \expr. \; \Cprom(\lastInstr{\cpath}) = ``\assignInst{\reg}{\expr}"$.
      \[\begin{array}{l l l}
        \cpath'     & \triangleq & \nextPathProm(\cpath, 1); \\
        \PromState' & \triangleq & \PromState[\reg \mapsto \semState{\expr}{\PromState}] \\
        \tT'        & \triangleq & \angled{\Mprom, \angled{\cpath', \PromState', \V, \PromSet}}.
      \end{array}\]
      Thus, $\tT \promStepNop \tT'$. Let's check $\invCert(n - 1, \delta, k, \tId, \aT, \tT')$.
      The only component of $\invCert$ worth checking is $\invStateCERT(\tId, \aT, \tT')$.
      As $\invStateCERT(\tId, \aT, \tT')$, we know that:
      \[\begin{array}{l}
        \stval \triangleq |[\expr|]^{\Committed}_{\cpath} \in \{\bot, |[\expr|]^{\PromState}\}; \\
        \regstcom(\Cprom(\tId), \tape, \cpath') = \regstcom(\Cprom(\tId), \tape, \cpath)[\reg \mapsto \stval]. \\
      \end{array}\]
      Thus, $\invStateCERT(\tId, \aT, \tT')$ holds.

    \item $\tape(\cpath) = \tapeIfGoto{\IfState}{\z}$. From $\invCert(n, \delta, k, \tId, \aT, \tT')$ we know that
      $\tape(\cpath)$ is committed, so $\IfState \in \{\Taken, \Ignored\}$.
      As $\ainit \armStepP^{*} \aT$ and $\tape(\cpath) = \tapeIfGoto{\IfState}{\z}$, \\
      $\exists \expr. \; \Cprom[\lastInstr{\cpath}] = ``\ifGotoInst{\expr}{\z}"$.
      From \app{\ref{inv:invATapeCfState}} we know that 
      \[\begin{array}{l}
        \exists \stval = \semfcom{\expr}{\cpath} \not = \bot.
          (\IfState = \Taken \land \stval \not = 0) \lor (\IfState = \Ignored \land \stval = 0)).
      \end{array}\]
      Fix $\stval$. As $\stval = \semfcom{\expr}{\cpath} \not = \bot$, $|[\expr|]^{\PromState} = \stval$
      by $\invCert(n, \delta, k, \tId, \aT, \tT)$.
      So $\cpath' \triangleq \nextPathProm(\cpath, \textIf \stval \not = 0 \; \textThen \z \; \textElse 1)$
      $\le \lastCommittedWrite(\tape)$, as the Promise machine chooses the same branch.
      
      By definition, $\tT \promTStepBranch \angled{\Mprom, \angled{\cpath', \PromState, \V, \PromSet}}$.
      $\invCert(n - 1, \delta, k, \tId, \aT, \tT')$ obviously holds.

    \item $\tape(\cpath) = \tapeFence{\Fstate}{\Ftype}$.
      From $\invCert(n, \delta, k, \tId, \aT, \tT)$ we know that $\Fstate = \Committed, \Ftype = \LD$.
      As $\ainit \armStepP^{*} \aT$ and $\tape(\cpath) = \tapeFence{\Committed}{\LD}$,
      $\Cprom(\lastInstr{\cpath}) = \fenceInst{\LD}$.

      Some notations:
      \[\begin{array}{l l l}
        \cpath' & \triangleq & \nextPathProm(\cpath, 1); \\
        \V'     & \triangleq & \angled{\V.\Racq, \V.\Racq, \V.\Rrel} \\
        \tT'    & \triangleq & \angled{\Mprom, \angled{\cpath', \PromState, \V', \PromSet}}.
      \end{array}\]
      $\tT \promTStepAcquire \tT'$ by definition. Thus, we need to check $\invCert(n - 1, \delta, k, \aT, \tT')$.
      %% Obviously, it's enough to check $(\delta, \tId, \aT, \tT') \in \invViewDeltaCERT \cap \invViewWriteCERT \cap \invViewReadCERT$.
      Obviously, it's enough to check $\invViewWriteCERT(\tId, \aT, \tT')$ and $\invViewReadCERT(\delta, \tId, \aT, \tT')$.

      Let's first check $\invViewWriteCERT(\tId, \aT, \tT')$.
      \[\begin{array}{l}
  \forall \cpath'' \ge \tT'.\cpath, \loc. \; \aT.\tapef(\tId, \cpath'') = \tapeWrite{(\tapeWriteCommitted{\_}{\loc}{\_})} \Rightarrow \\
  \quad \tT'.\Rcur(\loc) < \aT.\tmap(\tId, \cpath'') \land {} \\
  \quad ((\exists \cpath'''. \; \tT'.\cpath \le \cpath''' < \cpath'' \land 
             \aT.\tapef(\tId, \cpath''') = \tapeFence{\Committed}{\LD}) \Rightarrow \\
  \quad \quad \tT'.\Racq(\loc) < \aT.\tmap(\tId, \cpath'')).\\
      \end{array}\]
      Simplified:
      \[\begin{array}{l}
  \forall \cpath'' \ge \cpath', \loc. \; \tape(\cpath'') = \tapeWrite{(\tapeWriteCommitted{\_}{\loc}{\_})} \Rightarrow \\
  \quad \Racq(\loc) < \aT.\tmap(\tId, \cpath'') \land {} \\
  \quad ((\exists \cpath'''. \; \cpath' \le \cpath''' < \cpath'' \land 
             \tape(\cpath''') = \tapeFence{\Committed}{\LD}) \Rightarrow \\
  \quad \quad \Racq(\loc) < \aT.\tmap(\tId, \cpath'')).\\
      \end{array}\]
      Obviously, it's enough to show that
      \[\begin{array}{l}
  \forall \cpath'' \ge \cpath', \loc. \; \tape(\cpath'') = \tapeWrite{(\tapeWriteCommitted{\_}{\loc}{\_})} \Rightarrow \\
  \quad \Racq(\loc) < \aT.\tmap(\tId, \cpath'').
      \end{array}\]
      It directly follows from $\invViewWriteCERT(\tId, \aT, \tT)$ (as $\aT.\tapef(\tId, \cpath) = \tapeFence{\Committed}{\LD}$).

      Let's check $\invViewReadCERT(\delta, \tId, \aT, \tT')$.
      \[\begin{array}{l}
  \forall \cpath'' \ge \tT'.\cpath, \w, \tau. \\
\quad \aT.\tapef(\tId, \cpath'') = \tapeRead{(\tapeSatisfied{\Committed}{\w})} \land
   \angled{\tau, \_} = \deltaHmap(\delta, \tId, \aT, \w.\tId, \w.\cpath) \Rightarrow \\
\quad \quad \tT'.\Rcur(\w.\loc) \le \tau \land {} \\
\quad \quad ((\exists \cpath'''. \; \tT'.\cpath \le \cpath''' < \cpath'' \land 
             \aT.\tapef(\tId, \cpath''') = \tapeFence{\Committed}{\LD}) \Rightarrow \tT'.\Racq(\w.\loc) \le \tau). \\
      \end{array}\]
      Simplified:
      \[\begin{array}{l}
  \forall \cpath'' \ge \cpath', \w, \tau. \\
\quad \tape(\cpath'') = \tapeRead{(\tapeSatisfied{\Committed}{\w})} \land
   \angled{\tau, \_} = \deltaHmap(\delta, \tId, \aT, \w.\tId, \w.\cpath) \Rightarrow \\
\quad \quad \Racq(\w.\loc) \le \tau \land {} \\
\quad \quad ((\exists \cpath'''. \; \cpath' \le \cpath''' < \cpath'' \land 
             \tape(\cpath''') = \tapeFence{\Committed}{\LD}) \Rightarrow \Racq(\w.\loc) \le \tau). \\
      \end{array}\]
      Obviously, it's enough to show that
      \[\begin{array}{l}
  \forall \cpath'' \ge \cpath', \w, \tau. \\
\quad \tape(\cpath'') = \tapeRead{(\tapeSatisfied{\Committed}{\w})} \land
   \angled{\tau, \_} = \deltaHmap(\delta, \tId, \aT, \w.\tId, \w.\cpath) \Rightarrow \\
\quad \quad \Racq(\w.\loc) \le \tau.\\
      \end{array}\]
      It directly follows from $\invViewReadCERT(\tId, \aT, \tT)$ (as $\aT.\tapef(\tId, \cpath) = \tapeFence{\Committed}{\LD}$).

  %%     Let's check $\invViewDeltaCERT(\delta, \tId, \aT, \tT')$:
  %%     \[\begin{array}{l}
  %% \textLet \cpath^{\LD}, \cpath^{\SY}, \cpath^{\LD\SY} \triangleq
  %%    \lastLD(\tape, \cpath'), \lastSY(\tape, \cpath'), \lastLDSY(\tape, \cpath') \; \textIn \\
  %% (\Racq \le \bigsqcup \readsCommittedR(\cpath', \tape, \aT.\hmap) \sqcup
  %%                      \bigsqcup \opstau(\tId, \cpath', \tape, \aT.\hmap) \sqcup \\
  %% \quad \deltaToView(\cpath', \delta)) \land {} \\
  %% (\Racq \le \bigsqcup \readsCommittedR(\cpath^{\LD}, \tape, \aT.\hmap) \sqcup
  %%                      \bigsqcup \opstau(\tId, \cpath', \tape, \aT.\hmap) \sqcup \\
  %% \quad \deltaToView(\cpath', \delta)) \land {} \\
  %% (\Rrel \le \bigsqcup \readsCommittedR(\cpath^{\LD\SY}, \tape, \aT.\hmap) \sqcup \bigsqcup \opstau(\tId, \cpath^{\SY}, \tape, \aT.\hmap)).\\
  %%     \end{array}\]
  %%     We know that $\cpath^{\LD} = \cpath$, $\lastSY(\tape, \cpath') = \lastSY(\tape, \cpath)$,
  %%     $\lastLDSY(\tape, \cpath') = \lastLDSY(\tape, \cpath)$, and as $\tape(\cpath)$ isn't a read or a write:
  %%     \[\begin{array}{l}
  %% \textLet \cpath^{\SY}, \cpath^{\LD\SY} \triangleq
  %%    \lastSY(\tape, \cpath), \lastLDSY(\tape, \cpath) \; \textIn \\
  %% (\Racq \le \bigsqcup \readsCommittedR(\cpath, \tape, \aT.\hmap) \sqcup
  %%                      \bigsqcup \opstau(\tId, \cpath, \tape, \aT.\hmap) \sqcup \\
  %% \quad \deltaToView(\cpath, \delta)) \land {} \\
  %% (\Racq \le \bigsqcup \readsCommittedR(\cpath^{\LD}, \tape, \aT.\hmap) \sqcup
  %%                      \bigsqcup \opstau(\tId, \cpath', \tape, \aT.\hmap) \sqcup \\
  %% \quad \deltaToView(\cpath, \delta)) \land {} \\
  %% (\Rrel \le \bigsqcup \readsCommittedR(\cpath^{\LD\SY}, \tape, \aT.\hmap) \sqcup \bigsqcup \opstau(\tId, \cpath^{\SY}, \tape, \aT.\hmap)).\\
  %%     \end{array}\]
  %%     It directly follows from $\invViewDeltaCERT(\delta, \tId, \aT, \tT)$.

    \item $\tape(\cpath) = \tapeRead{(\tapeSatisfied{\Committed}{\w})}$.
      As $\ainit \armStepP^{*} \aT$ and $\tape(\cpath) = \tapeRead{(\tapeSatisfied{\Committed}{\w})}$,
      $\Cprom(\lastInstr{\cpath}) = ``\readInst{\reg}{\expr}"$.
      Some notations:
      \[\begin{array}{l l l}
        \cpath'    & \triangleq & \nextPath{\cpath}{1}; \\
        %% \regstcom  & \triangleq & \regstcom(\aT.\Cfarm(\tId), \tape, \cpath); \\
        \loc       & \triangleq & \semfcom{\expr}{\cpath} \\
                   & =          & \w.\loc \text{ (by \app{\ref{inv:invATapeCfState}})} \\
                   & =          & \semf{\expr}{\PromState} \text{ (by $\invStateCERT(\tId, \aT, \tT)$)}; \\
        \stval  & \triangleq & \w.\stval; \\
        %% \regstcom' & \triangleq & \regstcom(\aT.\Cfarm(\tId), \tape, \cpath') \\
        %%            & =          & \regstcom(\aT.\Cfarm(\tId), \tape, \cpath)[\reg \mapsto \w.\stval] \text{ (by definition)}; \\
        \PromState' & \triangleq & \PromState[\reg \mapsto \stval]. \\
      \end{array}\]
      
      By $\invViewReadCERT(\delta, \tId, \aT, \tT)$, $\exists \tau, \R.$
      $(\tau, \R) \triangleq \deltaHmap(\delta, \aT, \w.\tId, \w.\cpath)$ and \\
      $\Rcur(\loc) \le \tau$.
      $\writeEvt{\loc}{\stval}{\tau}{\R} \in \tT.\Mprom \setminus \tT.\PromSet$
      follows from $\invMemOneCERT(\delta, \tId, \aT, \tT)$ and $\invMemOneCERT(\delta, \tId, \aT, \tT)$.
      
      Denote $\V' = \angled{\Rcur', \Racq', \Rrel'} \triangleq \angled{\Rcur \sqcup [\loc@\tau], \Racq \sqcup \R, \Rrel}.$\\
      Thus, $\tT \promTStepReadLoc \tT' \triangleq \angled{\Mprom, \angled{\cpath', \V', \PromSet}}$.
      
      We need to check $\invCert(n - 1, \delta, k, \aT, \tT')$.
      $(\tId, \aT, \tT') \in \invMemOneCERT \cap \invMemTwoCERT$ follows from 
      $(\tId, \aT, \tT) \in \invMemOneCERT \cap \invMemTwoCERT$.
      $\invStateCERT(\tId, \aT, \tT')$ follows from $\invStateCERT(\tId, \aT, \tT)$ and the definition of $\PromState'$
      as in the $\tapeAssign$ case.
      
      $(\delta, \tId, \aT, \tT') \in \invMemZeroCERT \cap \invDeltaDefOne \cap \invDeltaDefTwo$ follows from
      $(\delta, \tId, \aT, \tT) \in \invMemZeroCERT \cap \invDeltaDefOne \cap \invDeltaDefTwo$.
      
      Let's check $\invViewReadCERT(\delta, \tId, \aT, \tT')$:
      \[\begin{array}{l}
  \forall \cpath'' \ge \tT'.\cpath, \w', \tau'. \\
\quad \aT.\tapef(\tId, \cpath'') = \tapeRead{(\tapeSatisfied{\Committed}{\w'})} \land
   \angled{\tau', \_} = \deltaHmap(\delta, \tId, \aT, \w'.\tId, \w'.\cpath) \Rightarrow \\
\quad \quad \tT'.\Rcur(\w'.\loc) \le \tau' \land {} \\
\quad \quad ((\exists \cpath'''. \; \tT'.\cpath \le \cpath''' < \cpath'' \land 
             \aT.\tapef(\tId, \cpath''') = \tapeFence{\Committed}{\LD}) \Rightarrow \tT'.\Racq(\w'.\loc) \le \tau'). \\
      \end{array}\]
      Simplified:
      \[\begin{array}{l}
  \forall \cpath'' \ge \cpath', \w', \tau'. \\
\quad \tape(\cpath'') = \tapeRead{(\tapeSatisfied{\Committed}{\w'})} \land
   \angled{\tau', \_} = \deltaHmap(\delta, \tId, \aT, \w'.\tId, \w'.\cpath) \Rightarrow \\
\quad \quad (\Rcur \sqcup [\loc @ \tau])(\w'.\loc) \le \tau' \land {} \\
\quad \quad ((\exists \cpath'''. \; \cpath' \le \cpath''' < \cpath'' \land 
             \tape(\cpath''') = \tapeFence{\Committed}{\LD}) \Rightarrow (\Racq \sqcup \R)(\w'.\loc) \le \tau'). \\
      \end{array}\]
      
      From $\invDeltaDefOne(\delta, \tId, \aT, \tT)$ follows that there are two options:
      either $\aT.\tapef(\w.\tId, \w.\cpath)$ is a committed write, or $\w.\tId = \tId$ and $\delta(\w.\cpath) = (\tau, \R)$.
      \begin{itemize}
        \item $\aT.\tapef(\w.\tId, \w.\cpath)$ is a committed write: \\
      Obviously, it's enough to show the statement for the new parts of views, as for the old parts the statement directly
      follows from $\invViewReadCERT(\delta, \tId, \aT, \tT)$:
      \[\begin{array}{l}
  \forall \cpath'' \ge \cpath', \w', \tau'. \\
\quad \tape(\cpath'') = \tapeRead{(\tapeSatisfied{\Committed}{\w'})} \land
   \angled{\tau', \_} = \deltaHmap(\delta, \tId, \aT, \w'.\tId, \w'.\cpath) \Rightarrow \\
\quad \quad [\loc @ \tau](\w'.\loc) \le \tau' \land {} \\
\quad \quad ((\exists \cpath'''. \; \cpath' \le \cpath''' < \cpath'' \land 
             \tape(\cpath''') = \tapeFence{\Committed}{\LD}) \Rightarrow \aT.\rmap(\w.\tId, \w.\cpath)(\w'.\loc) \le \tau'). \\
      \end{array}\]

      Fix $\cpath'', \w', \tau'$. We need to show:
      \[\begin{array}{l}
        [\loc @ \tau](\w'.\loc) \le \tau' \land {} \\
        ((\exists \cpath'''. \; \cpath' \le \cpath''' < \cpath'' \land 
             \tape(\cpath''') = \tapeFence{\Committed}{\LD}) \Rightarrow \aT.\rmap(\w.\tId, \w.\cpath)(\w'.\loc) \le \tau'). \\
      \end{array}\]
      From $\invDeltaDefOne(\delta, \tId, \aT, \tT)$ follows that there are two options:
      either $\aT.\tapef(\w'.\tId, \w'.\cpath)$ is a committed write, or $\w'.\tId = \tId$ and $\delta(\w'.\cpath) = (\tau', \_)$.
      In the first case, the statement holds by \app{\ref{thm:invAview}}. Consider the second option.
      
      Let's check the first conjunct. If $\w'.\loc \not = \loc$, it might be simplified to $0 \le \tau'$, which holds.
      Suppose, $\w'.\loc = \loc$.
      We know that $\w'.\cpath < \cpath$ from $\delta(\w'.\cpath) \not = \bot$ and $\invCert(n, \delta, k, \aT, \tT)$.
      We also know that $\cpath < \cpath''$. By \app{\ref{inv:invAReadRead}}, $\w = \w'$, so $\tau = \tau'$.
      
      Now let's check the second conjunct:
      \[\begin{array}{l}
        (\exists \cpath'''. \; \cpath' \le \cpath''' < \cpath'' \land 
             \tape(\cpath''') = \tapeFence{\Committed}{\LD}) \Rightarrow \aT.\rmap(\w.\tId, \w.\cpath)(\w'.\loc) \le \tau'. \\
      \end{array}\]
      
      Fix $\cpath'''$ and rename it to $\cpath_{LD}$.
      Then we know that
      \[\begin{array}{l}
      \cpath_{\delta} \triangleq \w'.\cpath < \cpath < \cpath' \le \cpath_{LD} < \cpath_{\delta-read} \triangleq \cpath''.
      \end{array}\]
      %% By $\invDeltaDefFour(\delta, \tId, \aT)$, the statement holds.

      $\aT.\rmap(\w.\tId, \w.\cpath)(\w'.\loc) \le \tau'$ holds by $\invDeltaDefFour(\delta, \tId, \aT)$.

        \item $\w.\tId = \tId$ and $\delta(\cpath) = (\tau, \R)$: \\
          $\R = [\loc @ \tau] \sqcup \Rrel$ by $\invDeltaDefTwo(\delta, \tId, \aT, \tT)$.
          We know that $[\loc @ \tau] \le \Rcur$ (from $\invDeltaDefTwo(\delta, \tId, \aT, \tT)$) and $\Rrel \le \Rcur \le \Racq$
          (by an invariant of the Promise machine), so $\Racq \sqcup [\loc @ \tau] \sqcup \Rrel = \Racq$ and
          $\Rcur \sqcup [\loc @ \tau] = \Rcur$.
          Thus, we can simplify the statement we want to prove:
      \[\begin{array}{l}
  \forall \cpath'' \ge \cpath', \w', \tau'. \\
\quad \tape(\cpath'') = \tapeRead{(\tapeSatisfied{\Committed}{\w'})} \land
   \angled{\tau', \_} = \deltaHmap(\delta, \tId, \aT, \w'.\tId, \w'.\cpath) \Rightarrow \\
\quad \quad \Rcur(\w'.\loc) \le \tau' \land {} \\
\quad \quad ((\exists \cpath'''. \; \cpath' \le \cpath''' < \cpath'' \land 
             \tape(\cpath''') = \tapeFence{\Committed}{\LD}) \Rightarrow \Racq(\w'.\loc) \le \tau'). \\
      \end{array}\]
      It directly follows from $\invViewReadCERT(\delta, \tId, \aT, \tT)$.
      \end{itemize}

      Let's check $\invViewWriteCERT(\tId, \aT, \tT')$:
      \[\begin{array}{l}
        \forall \cpath'' \ge \tT'.\cpath, \loc'. \aT.\tapef(\tId, \cpath'') = \tapeWrite{(\tapeWriteCommitted{\_}{\loc}{\_})} \Rightarrow \\
        \quad \tT'.\Rcur(\loc') < \aT.\tmap(\tId, \cpath'') \land {} \\
        \quad ((\exists \cpath'''. \; \tT'.\cpath \le \cpath''' < \cpath'' \land 
             \aT.\tapef(\tId, \cpath''') = \tapeFence{\Committed}{\LD}) \Rightarrow \\
        \quad \quad \tT'.\Racq(\loc') < \aT.\tmap(\tId, \cpath'')).\\
      \end{array}\]
      Simplified:
      \[\begin{array}{l}
        \forall \cpath'' \ge \cpath', \loc'. \tape(\cpath'') = \tapeWrite{(\tapeWriteCommitted{\_}{\loc}{\_})} \Rightarrow \\
        \quad (\Rcur \sqcup [\loc @ \tau])(\loc') < \aT.\tmap(\tId, \cpath'') \land {} \\
        \quad ((\exists \cpath'''. \; \cpath' \le \cpath''' < \cpath'' \land 
             \tape(\cpath''') = \tapeFence{\Committed}{\LD}) \Rightarrow \\
        \quad \quad (\Racq \sqcup \R)(\loc') < \aT.\tmap(\tId, \cpath'')).\\
      \end{array}\]
      
      From $\invDeltaDefOne(\delta, \tId, \aT, \tT)$ follows that there are two options:
      either $\aT.\tapef(\w.\tId, \w.\cpath)$ is a committed write, or $\w.\tId = \tId$ and $\delta(\w.\cpath) = (\tau, \R)$.
      
      \begin{itemize}
        \item $\aT.\tapef(\w.\tId, \w.\cpath)$ is a committed write: \\
      Obviously, it's enough to show the statement for the new parts of views, as for the old parts the statement directly
      follows from $\invViewWriteCERT(\delta, \tId, \aT, \tT)$:
      \[\begin{array}{l}
        \forall \cpath'' \ge \cpath', \loc'. \tape(\cpath'') = \tapeWrite{(\tapeWriteCommitted{\_}{\loc}{\_})} \Rightarrow \\
        \quad [\loc @ \tau](\loc') < \aT.\tmap(\tId, \cpath'') \land {} \\
        \quad ((\exists \cpath'''. \; \cpath' \le \cpath''' < \cpath'' \land 
             \tape(\cpath''') = \tapeFence{\Committed}{\LD}) \Rightarrow \\
        \quad \quad \R(\loc') < \aT.\tmap(\tId, \cpath'')).\\
      \end{array}\]
      It directly follows from \app{\ref{thm:invAview}}.

        \item $\w.\tId = \tId$ and $\delta(\cpath) = (\tau, \R)$: \\
          $\R = [\loc @ \tau] \sqcup \Rrel$ by $\invDeltaDefTwo(\delta, \tId, \aT, \tT)$.
          We know that $[\loc @ \tau] \le \Rcur$ (from $\invDeltaDefTwo(\delta, \tId, \aT, \tT)$) and $\Rrel \le \Rcur \le \Racq$
          (by an invariant of the Promise machine), so $\Racq \sqcup [\loc @ \tau] \sqcup \Rrel = \Racq$ and
          $\Rcur \sqcup [\loc @ \tau] = \Rcur$.
          Thus, we can simplify the statement we want to prove:
      \[\begin{array}{l}
        \forall \cpath'' \ge \cpath', \loc'. \tape(\cpath'') = \tapeWrite{(\tapeWriteCommitted{\_}{\loc}{\_})} \Rightarrow \\
        \quad \Rcur(\loc') < \aT.\tmap(\tId, \cpath'') \land {} \\
        \quad ((\exists \cpath'''. \; \cpath' \le \cpath''' < \cpath'' \land 
             \tape(\cpath''') = \tapeFence{\Committed}{\LD}) \Rightarrow \\
        \quad \quad \Racq(\loc') < \aT.\tmap(\tId, \cpath'')).\\
      \end{array}\]
      It directly follows from $\invViewWriteCERT(\delta, \tId, \aT, \tT)$.

      \end{itemize}
      
    \item $\tape(\cpath) = \tapeRead{\Rstate}$, where $\Rstate$ isn't a committed one. \\
      As $\ainit \armStepP^{*} \aT$ and $\tape(\cpath) = \tapeRead{\Rstate}$,
      $\Cprom(\lastInstr{\cpath}) = ``\readInst{\reg}{\expr}"$.
      Some notations:
      \[\begin{array}{l l l}
        \cpath'    & \triangleq & \nextPath{\cpath}{1}; \\
        \regstcom  & \triangleq & \regstcom(\Cprom, \tape, \cpath); \\
        \regstcom' & \triangleq & \regstcom(\Cprom, \tape, \cpath') \\
                   & =          & \regstcom(\Cprom, \tape, \cpath)[\reg \mapsto \bot] \text{ (by definition)}; \\
        \loc       & \triangleq & \semfcom{\expr}{\cpath} \\
                   & =          & \semf{\expr}{\PromState} \text{ (by $\invStateCERT(\tId, \aT, \tT)$)}; \\
        \tau       & \triangleq & \Rcur(\loc). \\
      \end{array}\]

      From properties of the Promise machine, $\exists \stval, \R. \; \writeEvt{\loc}{\stval}{\tau}{\R} \in \Mprom$. \\
      \[\begin{array}{l l l}
      \V' = \angled{\Rcur', \Racq', \Rrel'} & \triangleq & \angled{\Rcur \sqcup [\loc @ \tau], \Racq \sqcup \R, \Rrel} \\
                                            & =          & \angled{\Rcur, \Racq \sqcup \R, \Rrel}; \\
      \PromState' & \triangleq & \PromState[\reg \mapsto \stval].
      \end{array}\]
      
      By definition, $\tT \promTStepReadLoc \tT' \triangleq \angled{\Mprom, \angled{\cpath', \PromState', \V', \PromSet}}$.
      We need to check $\invCert(n - 1, \delta, k, \aT, \tT')$.

      $(\tId, \aT, \tT') \in \invMemOneCERT \cap \invMemTwoCERT \cap \invViewRelCERT$ follows from 
      $(\tId, \aT, \tT) \in \invMemOneCERT \cap \invMemTwoCERT \cap \invViewRelCERT$.
      $\invStateCERT(\tId, \aT, \tT')$ follows from $\invStateCERT(\tId, \aT, \tT)$ and definitions of
      $\PromState' \triangleq \PromState[\reg \mapsto \stval]$ and $\regstcom'$.

      $(\delta, \tId, \aT, \tT') \in \invMemZeroCERT \cap \invDeltaDefOne \cap \invDeltaDefTwo$ follows from
      $(\delta, \tId, \aT, \tT) \in \invMemZeroCERT \cap \invDeltaDefOne \cap \invDeltaDefTwo$ and
      $\Rcur' = \Rcur, \Rrel' = \Rrel$.

      We know that there is no $\cpath'' > \cpath$, such that $\tape(\cpath'') = \tapeFence{\Committed}{\LD}$, so
      we'll be able to drop parts of the invariants, which are related to committed fences.
      So, $(\delta, \tId, \aT, \tT') \in \invViewReadCERT \cap \invViewWriteCERT(\tId, \aT, \tT')$ follow from
      $(\delta, \tId, \aT, \tT') \in \invViewReadCERT(\delta, \tId, \aT, \tT) \cap \invViewWriteCERT(\tId, \aT, \tT)$,
      and $\Rcur' = \Rcur$.
      
    \item $\tape(\cpath) = \tapeWrite{(\tapeWriteCommitted{\InMemory}{\loc}{\stval})}$. \\
      As $\ainit \armStepP^{*} \aT$ and $\tape(\cpath) = \tapeWrite{(\tapeWriteCommitted{\InMemory}{\loc}{\stval})}$,
      $\Cprom(\lastInstr{\cpath}) = ``\writeInst{\expr_0}{\expr_1}"$.
      Some notations:
      \[\begin{array}{l l l}
        \cpath'    & \triangleq & \nextPath{\cpath}{1}; \\
        \regstcom  & \triangleq & \regstcom(\Cprom, \tape, \cpath); \\
        \angled{\tau, \_, \R'} & \triangleq & \aT.\hmap(\tId, \cpath)
          \text{ (it's defined according to properties of the $\ARMt$ machine)}. \\

      \end{array}\]
      
      By \app{\ref{inv:invATapeCfState}} and $\invStateCERT(\tId, \aT, \tT)$ we know that
      $\loc = \semfcom{\expr_0}{\cpath} = \semf{\expr_0}{\PromState}$ and
      $\stval = \semfcom{\expr_1}{\cpath} = \semf{\expr_1}{\PromState}$. \\
      
      From $\invMemOneCERT(\tId, \aT, \tT)$ we know that \\
      $\exists \R \le \R'. \; \writeEvt{\loc}{\stval}{\tau}{\R} \in \tT.\PromSet \cap \tT.\Mprom$.
      $\Rcur(\loc) < \tau$ follows from $\invViewWriteCERT(\tId, \aT, \tT)$.
      \[\begin{array}{l l l}
        \V' & \triangleq & \angled{\Rcur \sqcup [\loc @ \tau], \Racq \sqcup [\loc @ \tau], \Rrel}; \\
        \PromSet' & \triangleq & \PromSet \setminus \{\writeEvt{\loc}{\stval}{\tau}{\R}\}. \\
      \end{array}\]
      Thus, $\tT \promTStepFulfillLoc \tT' \triangleq \angled{\Mprom, \angled{\cpath', \PromState, \V', \PromSet'}}$.
      We need to check $\invCert(n - 1, \delta, k, \aT, \tT')$.
      \begin{itemize}
        \item $(\tId, \aT, \tT') \in \invMemOneCERT \cap \invMemTwoCERT(\tId, \aT, \tT')$: \\
          It's trivially holds for the fulfilled write and for the other writes.
        \item $(\tId, \aT, \tT') \in \invStateCERT \cap \invViewRelCERT$: \\
          It holds as $\tT'.\PromState = \tT.\PromState$,
          $\regstcom(\Cprom, \tape, \cpath) = \regstcom(\Cprom, \tape, \cpath')$,
          and $\tT'.\Rrel = \tT.\Rrel$.
        \item $\invMemZeroCERT(\delta, \tId, \aT, \tT')$: \\
          We need to show that
          \[\begin{array}{l}
      \forall \cpath'', \tau', \R', \loc', \expr'_0, \expr'_1. \\
\quad \angled{\tau', \R'} = \delta(\cpath'') \land ``\writeInst{\expr'_0}{\expr'_1}" = \Cprom(\lastInstr{\cpath''}) \land
      \loc' = \semfcom{\expr'_0}{\cpath''} \Rightarrow \\
\quad \exists \stval'. \semfcom{\expr'_1}{\cpath''} \in \{\bot, \stval'\} \land
      \writeEvt{\loc'}{\stval'}{\tau'}{\R'} \in \tT'.\Mprom \setminus \tT'.\PromSet. \\
          \end{array}\]
          Fix $\cpath''$ and do minor simplifications:
      %% \forall \cpath'', \angled{\tau', \R'} = \delta(\cpath''), \\
      %% \quad ``\writeInst{\expr'_0}{\expr'_1}" = \aT.\Cfarm(\tId, \lastInstr{\cpath''}), \loc' = \semfcom{\expr'_0}{\cpath''}. \\
          \[\begin{array}{l}
      \exists \stval'. \semfcom{\expr'_1}{\cpath''} \in \{\bot, \stval'\} \land
      \writeEvt{\loc'}{\stval'}{\tau'}{\R'} \in \Mprom \setminus (\PromSet \cup \{\writeEvt{\loc}{\stval}{\tau}{\R}\}). \\
          \end{array}\]
      Suppose, $\cpath'' = \cpath$. By $\invDeltaDefOne(\delta, \tId, \aT, \tT)$, $\tape(\cpath)$ isn't committed. Contradiction.
      So, $\cpath'' \not = \cpath$.
      From $\invDeltaDefThree(\delta, \tId, \aT)$ we know that $(\loc, \tau) \not = (\loc', \tau')$, so we may simplify the
      statement:
          \[\begin{array}{l}
      \exists \stval'. \semfcom{\expr'_1}{\cpath''} \in \{\bot, \stval'\} \land
      \writeEvt{\loc'}{\stval'}{\tau'}{\R'} \in \Mprom \setminus \PromSet. \\
          \end{array}\]
      which directly follows from $\invMemZeroCERT(\delta, \tId, \aT, \tT)$.

        \item $\invDeltaDefOne(\delta, \tId, \aT, \tT')$: \\
          $\delta(\cpath) = \bot$, $\tape(\cpath)$ is committed, so the statement holds.
        \item $\invDeltaDefTwo(\delta, \tId, \aT, \tT')$: \\
          We need to show that
          \[\begin{array}{l}
\forall \cpath', \angled{\tau', \R'} = \delta(\cpath'), \\
\quad ``\writeInst{\expr'_0}{\expr'_1}" = \Cprom(\lastInstr{\cpath'}), \loc' = \semfcom{\expr'_0}{\cpath'}. \\
\quad \quad \R' = [\loc' @ \tau'] \sqcup \tT'.\Rrel \land \tT'.\Rcur(\loc') \ge \tau'. \\
          \end{array}\]
          Simplified:
          \[\begin{array}{l}
\forall \cpath', \angled{\tau', \R'} = \delta(\cpath'), \\
\quad ``\writeInst{\expr'_0}{\expr'_1}" = \Cfarm(\lastInstr{\cpath'}), \loc' = \semfcom{\expr'_0}{\cpath'}. \\
\quad \quad \R' = [\loc' @ \tau'] \sqcup \tT.\Rrel \land [\loc @ \tau] \sqcup \tT.\Rcur(\loc') \ge \tau'. \\
          \end{array}\]
          It directly follows from $\invDeltaDefTwo(\delta, \tId, \aT, \tT')$.
          
          \item $\invViewWriteCERT(\tId, \aT, \tT')$: \\
            We need to show:
            \[\begin{array}{l}
      \forall \cpath'' \ge \tT'.\cpath, \loc'. \; \aT.\tapef(\tId, \cpath'') = \tapeWrite{(\tapeWriteCommitted{\_}{\loc'}{\_})} \Rightarrow \\
      \quad \tT'.\Rcur(\loc') < \aT.\tmap(\tId, \cpath'') \land {} \\
      \quad ((\exists \cpath'''. \; \tT'.\cpath \le \cpath''' < \cpath'' \land 
             \aT.\tapef(\tId, \cpath''') = \tapeFence{\Committed}{\LD}) \Rightarrow \\
      \quad \quad \tT'.\Racq(\loc') < \aT.\tmap(\tId, \cpath'')).\\
            \end{array}\]
            Simplified:
            \[\begin{array}{l}
      \forall \cpath'' \ge \cpath', \loc'. \; \tape(\cpath'') = \tapeWrite{(\tapeWriteCommitted{\_}{\loc'}{\_})} \Rightarrow \\
      \quad (\Rcur \sqcup [\loc @ \tau])(\loc') < \aT.\tmap(\tId, \cpath'') \land {} \\
      \quad ((\exists \cpath'''. \; \cpath' \le \cpath''' < \cpath'' \land 
             \tape(\cpath''') = \tapeFence{\Committed}{\LD}) \Rightarrow \\
      \quad \quad (\Racq \sqcup [\loc @ \tau])(\loc') < \aT.\tmap(\tId, \cpath'')).\\
            \end{array}\]
            As $\invViewWriteCERT(\tId, \aT, \tT)$ holds, it's enough to show that:
            \[\begin{array}{l}
      \forall \cpath'' \ge \cpath', \loc'. \; \tape(\cpath'') = \tapeWrite{(\tapeWriteCommitted{\_}{\loc'}{\_})} \Rightarrow \\
      \quad [\loc @ \tau](\loc') < \aT.\tmap(\tId, \cpath'').
            \end{array}\]
            It follows from \app{\ref{thm:invAview}}.
          \item $\invViewReadCERT(\tId, \aT, \tT')$: \\
            The same proof as for $\invViewWriteCERT(\tId, \aT, \tT')$.
      \end{itemize}

    \item $\tape(\cpath) = \tapeWrite{\Wstate}$, where $\Wstate$ isn't a committed one. \\
      In this case we are going to make two steps via the Promise thread $\tId$: to promise a write
      and to fulfill it.
      As $\ainit \armStepP^{*} \aT$ and $\tape(\cpath) = \tapeWrite{\Wstate}$,
      $\Cprom(\lastInstr{\cpath}) = ``\writeInst{\expr_0}{\expr_1}"$.
      Some notations:
      \[\begin{array}{l l l}
        \cpath'    & \triangleq & \nextPath{\cpath}{1}; \\
        \regstcom  & \triangleq & \regstcom(\Cprom, \tape, \cpath); \\
        \loc       & \triangleq &
          \semfcom{\expr_0}{\cpath} \text{ (addresses are defined according to $\invCert(n, \delta, k, \tId, \aT, \tT)$)} \\
                   & =          & \semf{\expr_0}{\PromState} \text{ (by $\invStateCERT(\tId, \aT, \tT)$)}; \\
        \stval       & \triangleq & \semf{\expr_1}{\PromState}. \\
        %% \angled{\tau, \_, \R'} & \triangleq & \aT.\hmap(\tId, \cpath)
        %%   \text{ (it's defined according to properties of the $\ARMt$ machine)}. \\
      \end{array}\]
      The most sophisticated part of this case is to choose a timestamp for the write properly.
      Let's denote it $\tau$. The following statements have to hold:
      \[\begin{array}{l l}
        1. & \Rcur(\loc) < \tau. \\
        2. & \tau \not \in \{\tau' | \stRequestWrite{\loc}{\_}{\tau'}{\_} \in \Mprom \}. \\
        3. & \forall \cpath_{write} \ge \cpath. \; \tape(\cpath_{write}) = \tapeWrite{(\tapeWriteCommitted{\_}{\loc}{\_})} \Rightarrow \\
           &  \quad \tau < \aT.\tmap(\tId, \cpath_{write}). \\
        4. & \forall \cpath_{read} \ge \cpath. \; \tape(\cpath_{read}) = \tapeRead{(\tapeSatisfied{\Committed}{\w})} \land
                 \aT.\tmap(\w.\tId, \w.\cpath) \not = \bot \Rightarrow \\
           & \quad \tau \le \aT.\tmap(\w.\tId, \w.\cpath). \\
        5. & \forall \cpath_{read} < \cpath_{\LD} < \cpath_{\delta-read}, \w, \R. \cpath < \cpath_{read} \land {} \\
           & \quad \tape(\cpath_{read}) = \tapeRead{(\tapeSatisfied{\Committed}{\w})} \land
             \tape(\cpath_{\LD}) = \tapeFence{\Committed}{\LD} \land {} \\
           & \quad \tape(\cpath_{\delta-read}) = \tapeRead{(\tapeSatisfied{\Committed}{\stRequestWrite{\tId}{\cpath}\loc}{\_})} \land {} \\
           & \quad \R = \aT.\rmap(\w.\tId, \w.\cpath) \not = \bot \Rightarrow \\
           & \quad \quad \R(\loc) \le \tau. \\
      \end{array}\]
      Because timestamps are elements of $\mathbb{Q}^{+}$ (a dense order), by $\invViewWriteCERT(\tId, \aT, \tT)$,
      $\invViewReadCERT(\delta, \tId, \aT, \tT)$, and \app{\ref{thm:invAview}}, there is $\tau$, which satisfies the requirements.

      Some notations:
      \[\begin{array}{l l l}
        \R  & \triangleq & \Rrel \sqcup [\loc @ \tau]; \\
        msg & \triangleq & \writeEvt{\loc}{\stval}{\tau}{\R}; \\
        \Mprom' & \triangleq & \Mprom \cup \{msg\};\\
        \V' = \angled{\Rcur', \Racq', \Rrel'} & \triangleq & \angled{\Rcur \sqcup [\loc @ \tau], \Racq \sqcup [\loc @ \tau], \Rrel}; \\
        \delta' & \triangleq & \delta[\cpath \mapsto \angled{\tau, \R}].\\ 
      \end{array}\]
      Then the following holds:
      \[\begin{array}{l l l}
        \tT & \promTStepPromiseLoc & \angled{\Mprom', \angled{\cpath , \PromState, \V, \PromSet \cup \{msg\}}} \\
            & \promTStepFulfillLoc & \tT' \triangleq \angled{\Mprom', \angled{\cpath', \PromState, \V', \PromSet}}. \\
      \end{array}\]
      We have to show $\invCert(n - 1, \delta', k, \tId, \aT, \tT')$ holds.
      
      \begin{itemize}
        \item $\invMemOneCERT(\tId, \aT, \tT')$: \\
          The statement obviously holds as $\invMemOneCERT(\tId, \aT, \tT)$ holds and $\aT.\hmap(\tId, \cpath) = \bot$.
        \item $\invMemTwoCERT(\tId, \aT, \tT')$: \\
          $\tT'.\PromSet = \tT.\PromSet$, $\tape(\cpath) \not = \tapeWrite{(\tapeWriteCommitted{\_}{\_}{\_})}$,
          so the statement holds.
        \item $\invStateCERT(\tId, \aT, \tT')$: \\
          $\tT'.\PromState = \tT.\PromState$, $\regstcom(\dots, \cpath) = \regstcom(\dots, \cpath')$, so the statement holds.

        \item $\invViewRelCERT(\tId, \aT, \tT')$: \\
          $\tT'.\Rrel = \tT.\Rrel$, so the statement holds.

        \item $\invViewWriteCERT(\tId, \aT, \tT')$: \\
          We need to show that
          \[\begin{array}{l}
            \forall \cpath'' \ge \tT'.\cpath, \loc'. \;
              \aT.\tapef(\tId, \cpath'') = \tapeWrite{(\tapeWriteCommitted{\_}{\loc'}{\_})} \Rightarrow \\
          \quad \quad \tT'.\Rcur(\loc') < \aT.\tmap(\tId, \cpath'') \land {} \\
          \quad \quad ((\exists \cpath'''. \; \tT'.\cpath \le \cpath''' < \cpath'' \land 
                       \aT.\tapef(\tId, \cpath''') = \tapeFence{\Committed}{\LD}) \Rightarrow \\
          \quad \quad \quad \tT'.\Racq(\loc') < \aT.\tmap(\tId, \cpath'')).\\
          \end{array}\]
          Simplified:
          \[\begin{array}{l}
            \forall \cpath'' \ge \cpath', \loc'. \;
              \aT.\tapef(\tId, \cpath'') = \tapeWrite{(\tapeWriteCommitted{\_}{\loc'}{\_})} \Rightarrow \\
          \quad \quad ([\loc @ \tau] \sqcup \Rcur)(\loc') < \aT.\tmap(\tId, \cpath'') \land {} \\
          \quad \quad ((\exists \cpath'''. \; \cpath' \le \cpath''' < \cpath'' \land 
                       \tape(\cpath''') = \tapeFence{\Committed}{\LD}) \Rightarrow \\
          \quad \quad \quad ([\loc @ \tau] \sqcup \Racq)(\loc') < \aT.\tmap(\tId, \cpath'')).\\
          \end{array}\]
          Obviously, it's enough to check only the following:
          \[\begin{array}{l}
            \forall \cpath'' \ge \cpath', \loc'. \;
              \aT.\tapef(\tId, \cpath'') = \tapeWrite{(\tapeWriteCommitted{\_}{\loc'}{\_})} \Rightarrow \\
          \quad \quad [\loc @ \tau](\loc') < \aT.\tmap(\tId, \cpath'') \land {} \\
          \quad \quad ((\exists \cpath'''. \; \cpath' \le \cpath''' < \cpath'' \land 
                       \tape(\cpath''') = \tapeFence{\Committed}{\LD}) \Rightarrow \\
          \quad \quad \quad [\loc @ \tau](\loc') < \aT.\tmap(\tId, \cpath'')).\\
          \end{array}\]
          If $\loc' \not = \loc$, then the statement holds as $[\loc @ \tau](\loc') = 0$.
          If $\loc' = \loc$, the statement is guaranteed by the way we have chosen $\tau$.

        \item $\invMemZeroCERT(\delta', \tId, \aT, \tT')$: \\
          We need to show that
          \[\begin{array}{l}
      \forall \cpath'', \tau', \R', \loc', \expr'_0, \expr'_1. \\
\quad \angled{\tau', \R'} = \delta'(\cpath'') \land ``\writeInst{\expr'_0}{\expr'_1}" = \Cprom(\lastInstr{\cpath''}) \land
      \loc' = \semfcom{\expr'_0}{\cpath''} \Rightarrow \\
\quad \exists \stval'. \semfcom{\expr'_1}{\cpath''} \in \{\bot, \stval'\} \land
      \writeEvt{\loc'}{\stval'}{\tau'}{\R'} \in \tT'.\Mprom \setminus \tT'.\PromSet. \\
          \end{array}\]
          Fix $\cpath''$ and do minor simplifications:
      %% \forall \cpath'', \angled{\tau', \R'} = \delta(\cpath''), \\
      %% \quad ``\writeInst{\expr'_0}{\expr'_1}" = \aT.\Cfarm(\tId, \lastInstr{\cpath''}), \loc' = \semfcom{\expr'_0}{\cpath''}. \\
          \[\begin{array}{l}
      \exists \stval'. \semfcom{\expr'_1}{\cpath''} \in \{\bot, \stval'\} \land
      \writeEvt{\loc'}{\stval'}{\tau'}{\R'} \in (\Mprom \setminus \PromSet) \cup \{\writeEvt{\loc}{\stval}{\tau}{\R}\}. \\
          \end{array}\]
          If $\cpath'' \not = \cpath$, then $\delta'(\cpath'') = \delta(\cpath'')$ and the statement holds by
          $\invMemZeroCERT(\delta, \tId, \aT, \tT)$.
          If $\cpath'' = \cpath$, then the statement obviously holds.

        \item $\invDeltaDefOne(\delta', \tId, \aT, \tT')$: \\
          We need to show that
          \[\begin{array}{l}
            \forall \cpath'' < \tT'.\cpath. \; \delta'(\cpath'') \not = \bot \Leftrightarrow \\
            \quad (\exists \expr'_0, \expr'_1. \; ``\writeInst{\expr'_0}{\expr'_1}" = \Cprom(\lastInstr{\cpath''})
              \land {} \\
            \quad \quad \aT.\tapef(\tId, \cpath'') \text{ isn't committed}). \\
          \end{array}\]
          Simplified:
          \[\begin{array}{l}
            \forall \cpath'' < \cpath'. \; \delta'(\cpath'') \not = \bot \Leftrightarrow \\
            \quad (\exists \expr'_0, \expr'_1. \; ``\writeInst{\expr'_0}{\expr'_1}" = \Cprom(\lastInstr{\cpath''})
              \land {} \\
            \quad \quad \tape(\cpath'') \text{ isn't committed}). \\
          \end{array}\]
          If $\cpath'' \not = \cpath$, then $\delta'(\cpath'') = \delta(\cpath'')$, and the statement holds as
          $\invDeltaDefOne(\delta, \tId, \aT, \tT)$ holds.
          If $\cpath'' \not = \cpath$, then the statement holds by definition of $\delta'$.

        \item $\invDeltaDefTwo(\delta', \tId, \aT, \tT')$: \\
          We need to show that
          \[\begin{array}{l}
            \forall \cpath'', \angled{\tau', \R'} = \delta'(\cpath''), \\
            \quad ``\writeInst{\expr'_0}{\expr'_1}" = \Cprom(\lastInstr{\cpath''}), \loc' = \semfcom{\expr'_0}{\cpath''}. \\
            \quad \quad \R' = [\loc' @ \tau'] \sqcup \tT'.\Rrel \land \tT'.\Rcur(\loc') \ge \tau'. \\
          \end{array}\]
          Simplified:
          \[\begin{array}{l}
            \forall \cpath'', \angled{\tau', \R'} = \delta'(\cpath''), \\
            \quad ``\writeInst{\expr'_0}{\expr'_1}" = \Cprom(\lastInstr{\cpath''}), \loc' = \semfcom{\expr'_0}{\cpath''}. \\
            \quad \quad \R' = [\loc' @ \tau'] \sqcup \Rrel \land ([\loc @ \tau] \sqcup \Rcur)(\loc') \ge \tau'. \\
          \end{array}\]
          Fix $\cpath''$. If $\cpath'' \not = \cpath$, then $\delta'(\cpath'') = \delta(\cpath'')$, and the statement holds as
          $\invDeltaDefTwo(\delta, \tId, \aT, \tT)$ holds.
          If $\cpath'' = \cpath$, then the statement obviously holds by construction of $\delta'$.

        \item $\invViewReadCERT(\delta', \tId, \aT, \tT')$: \\
          \[\begin{array}{l}
            \forall \cpath'' \ge \tT'.\cpath, \w', \tau'. \\
            \quad \aT.\tapef(\tId, \cpath'') = \tapeRead{(\tapeSatisfied{\Committed}{\w'})} \land
               \angled{\tau', \_} = \deltaHmap(\delta', \tId, \aT, \w'.\tId, \w'.\cpath) \Rightarrow \\
            \quad \quad \tT'.\Rcur(\w'.\loc) \le \tau' \land {} \\
            \quad \quad ((\exists \cpath'''. \; \tT'.\cpath \le \cpath''' < \cpath'' \land 
                         \aT.\tapef(\tId, \cpath''') = \tapeFence{\Committed}{\LD}) \Rightarrow \tT'.\Racq(\w'.\loc) \le \tau'). \\
          \end{array}\]
          Simplified:
          \[\begin{array}{l}
            \forall \cpath'' \ge \cpath', \w', \tau'. \\
            \quad \tape(\cpath'') = \tapeRead{(\tapeSatisfied{\Committed}{\w'})} \land
               \angled{\tau', \_} = \deltaHmap(\delta', \tId, \aT, \w'.\tId, \w'.\cpath) \Rightarrow \\
            \quad \quad ([\loc @ \tau] \sqcup \Rcur)(\w'.\loc) \le \tau' \land {} \\
            \quad \quad ((\exists \cpath'''. \; \tT'.\cpath \le \cpath''' < \cpath'' \land 
                         \tape(\cpath''') = \tapeFence{\Committed}{\LD}) \Rightarrow
                         ([\loc @ \tau] \sqcup \Racq)(\w'.\loc) \le \tau'). \\
          \end{array}\]
          Fix $\cpath'', \w', \tau'$.
          As $\deltaHmap(\delta', \tId, \aT, \w'.\tId, \w'.\cpath) \not = \bot$,
          either $\aT.\tapef(\w'.\tId, \w'.\cpath)$ is a committed write, or $\w'.\tId = \tId$ and
          $\delta'(\w'.\cpath) \not = \bot$.
          \begin{itemize}
            \item $\aT.\tapef(\w'.\tId, \w'.\cpath)$ is a committed write: \\
              We may simplify the statement:
          \[\begin{array}{l}
            \tape(\cpath'') = \tapeRead{(\tapeSatisfied{\Committed}{\w'})} \land
               \tau' = \aT.\tmap(\w'.\tId, \w'.\cpath) \Rightarrow \\
            \quad ([\loc @ \tau] \sqcup \Rcur)(\w'.\loc) \le \tau' \land {} \\
            \quad ((\exists \cpath'''. \; \tT'.\cpath \le \cpath''' < \cpath'' \land 
                         \tape(\cpath''') = \tapeFence{\Committed}{\LD}) \Rightarrow
                         ([\loc @ \tau] \sqcup \Racq)(\w'.\loc) \le \tau'). \\
          \end{array}\]
          As $\invViewReadCERT(\delta, \tId, \aT, \tT)$ holds, it's enough to check that:
          \[\begin{array}{l}
            \tape(\cpath'') = \tapeRead{(\tapeSatisfied{\Committed}{\w'})} \land
               \tau' = \aT.\tmap(\w'.\tId, \w'.\cpath) \Rightarrow \\
            \quad [\loc @ \tau](\w'.\loc) \le \tau' \land {} \\
            \quad ((\exists \cpath'''. \; \tT'.\cpath \le \cpath''' < \cpath'' \land 
                         \tape(\cpath''') = \tapeFence{\Committed}{\LD}) \Rightarrow
                         [\loc @ \tau](\w'.\loc) \le \tau'). \\
          \end{array}\]
          Or even simplier:
          \[\begin{array}{l}
            \tape(\cpath'') = \tapeRead{(\tapeSatisfied{\Committed}{\w'})} \land
               \tau' = \aT.\tmap(\w'.\tId, \w'.\cpath) \Rightarrow \\
            \quad [\loc @ \tau](\w'.\loc) \le \tau'. \\
          \end{array}\]
          This is guaranteed by the way we have chosen $\tau$.
          
            \item $\w'.\tId = \tId$ and $\delta'(\w'.\cpath) \not = \bot$: \\
              We may simplify the statement:
          \[\begin{array}{l}
            \tape(\cpath'') = \tapeRead{(\tapeSatisfied{\Committed}{\w'})} \land
               \tau' = \delta'(\w'.\cpath) \Rightarrow \\
            \quad ([\loc @ \tau] \sqcup \Rcur)(\w'.\loc) \le \tau' \land {} \\
            \quad ((\exists \cpath'''. \; \tT'.\cpath \le \cpath''' < \cpath'' \land 
                         \tape(\cpath''') = \tapeFence{\Committed}{\LD}) \Rightarrow
                         ([\loc @ \tau] \sqcup \Racq)(\w'.\loc) \le \tau'). \\
          \end{array}\]
              There are two options --- either $\w'.\cpath = \cpath$, or $\w'.\cpath < \cpath$.
              In the first case $\tau' = \tau$ and $\w'.\loc = \loc$, and the statement obviously holds as $\Rcur(\loc) < \tau$
              by the way we have chosen $\tau$.
              
              Suppose, $\w'.\cpath < \cpath$. In this case $\delta'(\w'.\cpath) = \delta(\w'.\cpath)$.
              As $\invViewReadCERT(\delta, \tId, \aT, \tT)$ holds, we may simplify the statement even further:
          \[\begin{array}{l}
            \tape(\cpath'') = \tapeRead{(\tapeSatisfied{\Committed}{\w'})} \land
               \tau' = \delta(\w'.\cpath) \Rightarrow \\
            \quad [\loc @ \tau](\w'.\loc) \le \tau'. \\
          \end{array}\]
              If $\w'.\loc \not = \loc$, then the statement holds, as $[\loc @ \tau](\w'.\loc) = 0$.
              Suppose $\w'.\loc = \loc$. As $\w'.\cpath < \cpath < \cpath''$, it would have meant that
              a committed read $\tape(\cpath'')$ is satisfied from a write $\tape(\w'.\cpath)$, even though
              there is a newer write to the same location ($\tape(\cpath)$) observable to the read. It contradicts
              \app{\ref{inv:invAWriteWriteRead}}.
          \end{itemize}

          %% Obviously, it's enough to check the following as $\invViewReadCERT(\delta, )$:
          %% \[\begin{array}{l}
          %%   \forall \cpath'' \ge \cpath', \w', \tau'. \\
          %%   \quad \tape(\cpath'') = \tapeRead{(\tapeSatisfied{\Committed}{\w'})} \land
          %%      \angled{\tau', \_} = \deltaHmap(\delta', \tId, \aT, \w'.\tId, \w'.\cpath) \Rightarrow \\
          %%   \quad \quad [\loc @ \tau](\w.\loc) \le \tau'. \\
          %% \end{array}\]
          %% Or even simlified:
          
        \item $\invDeltaDefThree(\delta', \tId, \aT)$: \\
          We have to show that
          \[\begin{array}{l}
            \forall \cpath'', \cpath''' \not = \cpath'', \\
            \quad \angled{\tau'', \_} = \deltaHmap(\delta', \tId, \aT, \tId, \cpath''),
                  \angled{\tau''', \_} = \deltaHmap(\delta', \tId, \aT, \tId, \cpath'''), \\
            \quad ``\writeInst{\expr''_0}{\expr''_1}"   = \Cprom(\tId, \lastInstr{\cpath''}),
                  ``\writeInst{\expr'''_0}{\expr'''_1}" = \Cprom(\tId, \lastInstr{\cpath'''}). \\
            \quad \semfcom{\expr''_0}{\cpath''} = \semfcom{\expr'''_0}{\cpath'''} \Rightarrow \tau'' \not = \tau'''.
          \end{array}\]
          As $\invDeltaDefThree(\delta, \tId, \aT)$ holds, it's enough to check only cases $\cpath'' = \cpath$ or $\cpath''' = \cpath$.
          In both cases it's obvious that the statement holds by the way we have choosen $\tau$.

        \item $\invDeltaDefFour(\delta', \tId, \aT)$: \\
          We have to show that
          \[\begin{array}{l}
      \forall \cpath_{\delta} < \cpath_{read} < \cpath_{\LD} < \cpath_{\delta-read}, \\
\quad \angled{\tau', \_} = \delta'(\cpath_{\delta}), \w', \loc', \R'. \\
\quad \tape(\cpath_{read}) = \tapeRead{(\tapeSatisfied{\Committed}{\w'})} \land \tape(\cpath_{\LD}) = \tapeFence{\Committed}{\LD} \land {} \\
\quad \tape(\cpath_{\delta-read}) = \tapeRead{(\tapeSatisfied{\Committed}{\stRequestWrite{\tId}{\cpath_{\delta}}{\loc'}{\_}})} \land {} \\
\quad \R' = \aT.\rmap(\w'.\tId, \w'.\cpath) \not = \bot \Rightarrow \\
\quad \quad \R'(\loc) \le \tau'. \\
          \end{array}\]
          If $\cpath_{\delta} \not = \cpath$, the statement directly follows from $\invDeltaDefFour(\delta, \tId, \aT)$.
          Suppose $\cpath_{\delta} = \cpath$. The simplified statement:
          \[\begin{array}{l}
      \forall \cpath_{read}, \cpath_{\LD}, \cpath_{\delta-read}, \w', \loc', \R'. \\
\quad \cpath < \cpath_{read} < \cpath_{\LD} < \cpath_{\delta-read} \land {} \\
\quad \tape(\cpath_{read}) = \tapeRead{(\tapeSatisfied{\Committed}{\w'})} \land \tape(\cpath_{\LD}) = \tapeFence{\Committed}{\LD} \land {} \\
\quad \tape(\cpath_{\delta-read}) = \tapeRead{(\tapeSatisfied{\Committed}{\stRequestWrite{\tId}{\cpath}{\loc'}{\_}})} \land {} \\
\quad \R' = \aT.\rmap(\w'.\tId, \w'.\cpath) \not = \bot \Rightarrow \\
\quad \quad \R'(\loc) \le \tau. \\
          \end{array}\]
          The statement is one of the requirements we used to choose $\tau$, so it holds.
      \end{itemize}
  \end{itemize}
\end{proof}


