\chapter*{Список сокращений и условных обозначений}             % Заголовок
\addcontentsline{toc}{chapter}{Список сокращений и условных обозначений}  % Добавляем его в оглавление
\noindent
\addtocounter{table}{-1}% Нужно откатить на единицу счетчик номеров таблиц, так как следующая таблица сделана для удобства представления информации по ГОСТ
%\begin{longtabu} to \dimexpr \textwidth-5\tabcolsep {r X}
\begin{longtabu} to \textwidth {r X c}
% Жирное начертание для математических символов может иметь
% дополнительный смысл, поэтому они приводятся как в тексте
% диссертации
  & & Стр. \\
{\bf CAS} & compare-and-set, инструкция атомарного сравнения и записи, частный случай {\bf RMW} & \pageref{acr:cas} \\
{\bf C/C++11 MM} & C/C++11 memory model, модель памяти C/C++11 \cite{Batty-al:POPL11}
         & \pageref{acr:cppmm} \\
%% {\bf CoRR} & coherence of read-read, шаблон ``корректность повторного чтения''
%%          & \pageref{acr:corr} \\
{\bf DR} & data race, гонка по данным
         & \pageref{acr:dr} \\
%% {\bf IRIW} & independent reads of independent writes, шаблон ``независимые чтения независимых записей''
%%          & \pageref{acr:iriw} \\
{\bf JMM} & Java memory model, модель памяти Java \cite{Manson-al:POPL05}
         & \pageref{acr:jmm} \\
%% {\bf ЯП} & язык программирования \\
{\bf LB} & load buffering, шаблон ``буферизация чтения''
         & \pageref{acr:lb} \\
{\bf MM} & memory model, модель памяти
         & \pageref{acr:mm} \\
{\bf MP} & message passing, шаблон ``передача сообщения'' 
         & \pageref{acr:mp} \\
{\bf OOTA} & out-of-thin-air values, ``значения из воздуха''
         & \pageref{acr:oota} \\
{\bf $\OpCpp$ MM} & операционная модель памяти для C/C++11, предложенная диссертантом~\cite{Podkopaev-al:CoRR16}
         & \pageref{acr:opcpp} \\
{\bf POP} & partial order propagation, подсистема памяти модели ARMv8 POP
            \cite{Flur-al:POPL16}, основанная на частичных порядках
         & \pageref{acr:pop} \\
{\bf \Promise MM} & обещающая модель памяти~\cite{Kang-al:POPL17}
         & \pageref{acr:promise} \\
{\bf QSBR} & quiescent state-based reclamation, стратегия ``освобождение памяти в момент затишья''
         & \pageref{acr:qsbr} \\
{\bf RCU} & read-copy-update, структура данных, поддерживающая неблокирующую синхронизацию для ситуации
            ``один писатель много читателей'' \cite{McKenney-Slingwine:PDCS98,McKenney:PhD}
         & \pageref{acr:rcu} \\
{\bf RMW} & read-modify-write, атомарное чтение-запись
         & \pageref{acr:rmw} \\
{\bf SB} & store buffering, шаблон ``буферизация записи''
         & \pageref{acr:sb} \\
{\bf SC} & sequential consistency, модель последовательной консистентности 
         & \pageref{acr:sc} \\
{\bf SE} & speculative execution, шаблон ``спекулятивное исполнение''
         & \pageref{acr:se} \\
%% {\bf WR} & write reorder, шаблон ``независимые записи''
%%          & \pageref{acr:wr} \\
%% {\bf WRC} & write-to-read causality, шаблон ``зависимость запись-чтение'' 
%%          & \pageref{acr:wrc} \\
\end{longtabu}
