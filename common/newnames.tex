% Новые переменные, которые могут использоваться во всём проекте
% ГОСТ 7.0.11-2011
% 9.2 Оформление текста автореферата диссертации
% 9.2.1 Общая характеристика работы включает в себя следующие основные структурные
% элементы:
% актуальность темы исследования;
\newcommand{\actualityTXT}{Актуальность темы.}
% степень ее разработанности;
\newcommand{\progressTXT}{Степень разработанности темы.}
% цели и задачи;
\newcommand{\aimTXT}{Целью}
\newcommand{\tasksTXT}{задачи}
% научную новизну;
\newcommand{\noveltyTXT}{Научная новизна}
% теоретическую и практическую значимость работы;
%\newcommand{\influenceTXT}{Теоретическая и практическая значимость}
% или чаще используют просто
\newcommand{\influenceTXT}{Теоретическая и практическая значимость работы.}
% методологию и методы исследования;
\newcommand{\methodsTXT}{Mетодология и методы исследования.}
% положения, выносимые на защиту;
\newcommand{\defpositionsTXT}{Основные положения, выносимые на~защиту.}
% степень достоверности и апробацию результатов.
\newcommand{\reliabilityTXT}{Достоверность}
\newcommand{\probationTXT}{Апробация работы.}

\newcommand{\contributionTXT}{Личный вклад.}
\newcommand{\publicationsTXT}{Публикации по теме диссертации.}


\newcommand{\authorbibtitle}{Публикации автора по теме диссертации}
%% \newcommand{\vakbibtitle}{В изданиях из списка ВАК РФ}
\newcommand{\vakbibtitle}{Статьи из ``Перечня рецензируемых научных изданий, в которых должны быть опубликованы основные научные результаты диссертаций на соискание ученой степени кандидата наук, на соискание ученой степени доктора наук'', сформированного согласно требованиям, установленным Министерством образования и науки Российской Федерации}
%% \newcommand{\notvakbibtitle}{В прочих изданиях}
\newcommand{\notvakbibtitle}{Статьи в других изданиях}
%% \newcommand{\confbibtitle}{В сборниках трудов конференций}
\newcommand{\confbibtitle}{Статьи в изданиях, входящих в базы цитирования Web of Science и SCOPUS}
\newcommand{\fullbibtitle}{Список литературы} % (ГОСТ Р 7.0.11-2011, 4)


%% Мои макросы

\newcommand{\OpCpp}{\ensuremath{\mathrm{OpC11}}}
%% \newcommand\sembr[1]{\llbracket #1 \rrbracket}

\newcommand{\vakJournals}{``Перечня рецензируемых научных изданий, в которых должны быть опубликованы основные научные результаты диссертаций на соискание ученой степени кандидата наук, на соискание ученой степени доктора наук'', сформированного согласно требованиям, установленным Министерством образования и науки Российской Федерации}

%% \newcommand\ARM{\mathrm{ARM}}
%% \newcommand\ARMt{\mathrm{ARM}{+}\tau}
%% \newcommand\Promise{\mathrm{Promise}}

%% \newcommand\ifGoto{\textsf{if}-\textsf{goto}}
%% \newcommand{\ifGotoInst}[2]{\textsf{if} \; #1 \; \textsf{goto} \; #2}
%% \newcommand{\writeInst}[2]{[#1]\;:=\;#2}
%% \newcommand{\writeExclInst}[3]{\textup{\sf atomic-write}(#1, [#2], #3)}
%% \newcommand{\assignInst}[2]{#1\;:=\;#2}

%% \newcommand{\excl}[1]{}
%% \newcommand{\readInst }[3][\ExclReadType]{#2 \;:=_{\excl{#1}}\;[#3]}
%% \newcommand{\fenceInst}[1]{\dmb \; #1}
%%   \newcommand\dmb{\textsf{dmb}}
%% \newcommand{\dmbSY}{\fenceInst{\SY}}
%% \newcommand{\dmbLD}{\fenceInst{\LD}}
%%   \newcommand\LD{\mathsf{LD}}
%%   \newcommand\SY{\mathsf{SY}}
%%   \newcommand\ST{\mathsf{ST}}

%%   \newcommand\fence[1]{\mathsf{fence}({#1})}
%%   \newcommand\acqFence{\fence{\textsf{acquire}}}
%%   \newcommand\relFence{\fence{\textsf{release}}}
%%   \newcommand\scFence{\fence{\textsf{sc}}}

%% \newcommand{\IssuedSet}{I}
%% \newcommand{\issuable}{{\textsf{Issuable}}}
  
%% %% \newcommand{\travConfigStep}{\rightarrow_{\textrm{TC}}}
%% \newcommand{\travConfigStep}{\rightarrow}
%% \newcommand{\nextset}{{\textsf{Next}}}
%% \newcommand{\coverable}{{\textsf{Coverable}}}
%% \newcommand{\promisable}{{\textsf{Promisable}}}

%% \newcommand{\tup}[1]{{\langle{#1}\rangle}}
%% \newcommand{\nin}{\not\in}
%% \newcommand{\suq}{\subseteq}
%% \newcommand{\sqsuq}{\sqsubseteq}
%% \newcommand{\sqsu}{\sqsubset}
%% \newcommand{\sqslq}{\sqsupseteq}
