%%% Основные сведения %%%
\newcommand{\thesisAuthor}             % Диссертация, ФИО автора
{%
    \texorpdfstring{% \texorpdfstring takes two arguments and uses the first for (La)TeX and the second for pdf
        Подкопаев Антон Викторович% так будет отображаться на титульном листе или в тексте, где будет использоваться переменная
    }{%
        Подкопаев, Антон Викторович% эта запись для свойств pdf-файла. В таком виде, если pdf будет обработан программами для сбора библиографических сведений, будет правильно представлена фамилия.
    }%
}
\newcommand{\thesisAuthorShort}        % Диссертация, ФИО автора инициалами
{А.В.~Подкопаев}

\newcommand{\thesisUdk}                % Диссертация, УДК
{\todo{xxx.xxx}}
\newcommand{\thesisTitle}              % Диссертация, название
{\texorpdfstring{\MakeUppercase{Операционные методы в приложении к слабым моделям памяти}}
                               {Операционные методы в приложении к слабым моделям памяти}}
\newcommand{\thesisSpecialtyNumber}    % Диссертация, специальность, номер
{\texorpdfstring{05.13.11}{05.13.11}}
\newcommand{\thesisSpecialtyTitle}     % Диссертация, специальность, название
{\texorpdfstring{Математическое и программное обеспечение вычислительных машин, комплексов и компьютерных сетей}
{Математическое и программное обеспечение вычислительных машин, комплексов и компьютерных сетей}}
\newcommand{\thesisDegree}             % Диссертация, ученая степень
{кандидата физико-математических наук}
\newcommand{\thesisDegreeShort}        % Диссертация, ученая степень, краткая запись
{канд. физ.-мат. наук}
\newcommand{\thesisCity}               % Диссертация, город написания диссертации
{Санкт-Петербург}
\newcommand{\thesisYear}               % Диссертация, год написания диссертации
{2018}
\newcommand{\thesisOrganization}       % Диссертация, организация
{Санкт-Петербургский государственный университет}
%% Правительство Российской Федерации \\ Федеральное государственное бюджетное образовательное учереждение высшего
%% профессионального образования \\
%% <<Санкт-Петербургский государственный университет>>}
\newcommand{\thesisOrganizationShort}  % Диссертация, краткое название организации для доклада
{СПбГУ}

\newcommand{\thesisInOrganization}     % Диссертация, организация в предложном падеже: Работа выполнена на ...
{кафедре системного программирования Санкт-Петербургского государственного университета}
%% \todo{учреждении, в~котором выполнялась данная диссертационная работа}

\newcommand{\supervisorFio}            % Научный руководитель, ФИО
{КОЗНОВ Дмитрий Владимирович}
\newcommand{\supervisorRegalia}        % Научный руководитель, регалии
{доктор технических наук, доцент, профессор кафедры системного программирования}
\newcommand{\supervisorFioShort}       % Научный руководитель, ФИО
{Д.В.~Кознов}
\newcommand{\supervisorRegaliaShort}   % Научный руководитель, регалии
{д.т.н., доц.}


\newcommand{\opponentOneFio}           % Оппонент 1, ФИО
{ГЕРГЕЛЬ Виктор Павлович}
\newcommand{\opponentOneRegalia}       % Оппонент 1, регалии
{доктор технических наук, профессор}
\newcommand{\opponentOneJobPlace}      % Оппонент 1, место работы
{Федеральное государственное автономное образовательное учреждение высшего образования
<<Нижегородский государственный университет им. Н.И. Лобачевского>>}
%{\todo{Не очень длинное название для места работы}}
\newcommand{\opponentOneJobPost}       % Оппонент 1, должность
{директор института информационных технологий, математики и механики, заведующий кафедрой программной инженерии}%{\todo{старший научный сотрудник}}

\newcommand{\opponentTwoFio}           % Оппонент 2, ФИО
{ЛУКАШИН Алексей Андреевич}
\newcommand{\opponentTwoRegalia}       % Оппонент 2, регалии
{кандидат технических наук}
\newcommand{\opponentTwoJobPlace}      % Оппонент 2, место работы
{Федеральное государственное автономное образовательное учреждение высшего образования
<<Санкт-Петербургский политехнический университет Петра Великого>>}
\newcommand{\opponentTwoJobPost}       % Оппонент 2, должность
{доцент кафедры <<Телематика (при ЦНИИ РТК)>> института прикладной математики и механики}%{\todo{старший научный сотрудник}}

\newcommand{\leadingOrganizationTitle} % Ведущая организация, дополнительные строки
{Федеральное государственное бюджетное учреждение науки Институт системного программирования Российской академии наук (ИСП РАН)}

\newcommand{\defenseDate}              % Защита, дата
{\_\_\_\_\_\_\_\_\_\_\_\_\_\_\_\_\_\_\_\_\_~г.~в~\_\_\_\_\_~часов}
\newcommand{\defenseCouncilNumber}     % Защита, номер диссертационного совета
{Д\,212.232.51}
\newcommand{\defenseCouncilTitle}      % Защита, учреждение диссертационного совета
{Санкт-Петербургского государственного университета}
\newcommand{\defenseCouncilAddress}    % Защита, адрес учреждение диссертационного совета
{198504, Санкт-Петербург, Старый Петергоф, Университетский пр. 28, математико-механический факультет СПбГУ, ауд. 405}
\newcommand{\defenseCouncilPhone}      % Телефон для справок
{\todo{+7~(0000)~00-00-00}}

\newcommand{\defenseSecretaryFio}      % Секретарь диссертационного совета, ФИО
{Демьянович Юрий Казимирович}
\newcommand{\defenseSecretaryRegalia}  % Секретарь диссертационного совета, регалии
{д.ф.-м.н., профессор}            % Для сокращений есть ГОСТы, например: ГОСТ Р 7.0.12-2011 + http://base.garant.ru/179724/#block_30000

\newcommand{\synopsisLibrary}          % Автореферат, название библиотеки
{Санкт-Петербургского государственного университета по адресу: 199034, Санкт-Петербург, Университетская наб., д. 7/9,
а также на сайте СПбГУ: \url{http://spbu.ru/science/disser/}}
\newcommand{\synopsisDate}             % Автореферат, дата рассылки
{\_\_ \_\_\_\_\_\_\_\_\_\_ 20\_\_ года}

% To avoid conflict with beamer class use \providecommand
\providecommand{\keywords}%            % Ключевые слова для метаданных PDF диссертации и автореферата
{}
