\newcommand{\valuecom}[1]{\textcolor{green!60!black}{// #1}}
\newcommand{\valuefail}[1]{\textcolor{red!60!black}{// #1}}
\newcolumntype{L}{>{$}l<{$}} % math-mode version of "l" column type

\declaretheorem[name=Теорема,style=default,numberwithin=section]{thm}
\declaretheorem[name=Лемма,style=default,numberwithin=section]{lem}
%% \declaretheorem[name=Определение,style=default,sibling=thm]{mydefinition}
\declaretheorem[name=Определение,style=definition,numbered=no]{mydefinition}

% enviorments
\theoremstyle{plain}
%% \newtheorem{notation}[theorem]{Notation}
\newtheorem{notation}{Notation}
\newtheorem*{notation*}{Notation}

\newtheorem{definition}{Определение}
\newtheorem*{definition*}{Определение}
\newtheorem{proposition}{Утверждение}
\newtheorem*{proposition*}{Утверждение}
\newtheorem{theorem}{Теорема}
\newtheorem*{theorem*}{Теорема}
\newtheorem{lemma}{Лемма}
\newtheorem*{lemma*}{Лемма}

\crefformat{section}{#2\S{}#1#3}
\Crefname{section}{Section}{Section}
\Crefformat{section}{Section #2#1#3}
\Crefname{figure}{\text{Figure}}{\text{Figures}}
\crefname{corollary}{\text{Corollary}}{\text{corollaries}}
\Crefname{corollary}{\text{Corollary}}{\text{Corollaries}}
\crefname{lemma}{\text{Lemma}}{\text{Lemmas}}
\Crefname{lemma}{\text{Lemma}}{\text{Lemmas}}
\crefname{proposition}{\text{Prop.}}{\text{Propositions}}
\Crefname{proposition}{\text{Proposition}}{\text{Propositions}}
\crefname{definition}{\text{Def.}}{\text{Definitions}}
\Crefname{definition}{\text{Definition}}{\text{Definitions}}
\crefname{notation}{\text{Notation}}{\text{Notations}}
\Crefname{notation}{\text{Notation}}{\text{Notations}}
\crefname{theorem}{\text{Theorem}}{\text{Theorems}}
\Crefname{theorem}{\text{Theorem}}{\text{Theorems}}


\newenvironment{claim}[1]{\par\noindent\underline{Claim:}\space#1}{}
\newenvironment{claimproof}[1]{\par\noindent\underline{Proof:}\space#1}
{\leavevmode\unskip\penalty9999 \hbox{}\nobreak\hfill\quad\hbox{$\square$}}

\newcommand{\textdom}[1]{\mathsf{#1}}
\newcommand{\textcode}[1]{\texorpdfstring{\texttt{#1}}{#1}}
\newcommand{\kw}[1]{\textbf{\textcode{#1}}}
\newcommand{\skipc}{\kw{skip}}
\newcommand{\ite}[3]{\kw{if}\;#1\:\kw{then}\;#2\;\\ \kw{else}\;#3}
\newcommand{\iteml}[3]{
  \kw{if} \; #1\\
  \begin{array}[t]{@{}l@{}l}
    \kw{then}& \begin{array}[t]{l} #2 \end{array} \\
    \kw{else}& \begin{array}[t]{l} #3 \end{array} \\
  \end{array}
}
\newcommand{\itne}[2]{\kw{if}\;#1\:\kw{then}\\ \quad\;{#2}}
\newcommand{\while}[2]{\kw{while}\;#1\;\kw{do}\;#2}
\newcommand{\ALT}{\;\;|\;\;}

\newcommand{\ie}{\emph{i.e.,} }
\newcommand{\eg}{\emph{e.g.,} }
\newcommand{\etal}{\emph{et~al.}}
\newcommand{\wrt}{w.r.t.~}

\newcommand{\inarrC}[1]{\begin{array}{@{}c@{}}#1\end{array}}
\newcommand{\inpar}[1]{\left(\begin{array}{@{}l@{}}#1\end{array}\right)}
\newcommand{\inset}[1]{\left\{\begin{array}{@{}l@{}}#1\end{array}\right\}}
\newcommand{\inarr}[1]{\begin{array}{@{}l@{}}#1\end{array}}
\newcommand{\inarrII}[2]{\begin{array}{@{}l@{~~}||@{~~}l@{}}\inarr{#1}&\inarr{#2}\end{array}}
\newcommand{\inarrIII}[3]{\begin{array}{@{}l@{~~}||@{~~}l@{~~}||@{~~}l@{}}\inarr{#1}&\inarr{#2}&\inarr{#3}\end{array}}
\newcommand{\inarrIV}[4]{\begin{array}{@{}l@{~~}||@{~~}l@{~~}||@{~~}l@{~~}||@{~~}l@{}}\inarr{#1}&\inarr{#2}&\inarr{#3}&\inarr{#4}\end{array}}
\newcommand{\inarrV}[5]{\begin{array}{@{}l@{~~}||@{~~}l@{~~}||@{~~}l@{~~}||@{~~}l@{~~}||@{~~}l@{}}\inarr{#1}&\inarr{#2}&\inarr{#3}&\inarr{#4}&\inarr{#5}\end{array}}

%% \renewcommand{\comment}[1]{\color{teal}{~~\texttt{/\!\!/}\textit{#1}}}


\newcommand{\set}[1]{\{{#1}\}}
\newcommand{\sem}[1]{\llbracket #1 \rrbracket}
\newcommand{\pfn}{\rightharpoonup}
\newcommand{\suchthat}{\; | \;}
\newcommand{\N}{{\mathbb{N}}}
\newcommand{\dom}[1]{\textit{dom}{({#1})}}
\newcommand{\codom}[1]{\textit{codom}{({#1})}}
\newcommand{\before}[2]{{#1}_{#2}^\uparrow}
\newcommand{\after}[2]{{#1}_{#2}^\downarrow}
%\newcommand{\fv}[1]{fv{[{#1}]}}
\newcommand{\tup}[1]{{\langle{#1}\rangle}}
\newcommand{\nin}{\not\in}
\newcommand{\suq}{\subseteq}
\newcommand{\sqsuq}{\sqsubseteq}
\newcommand{\sqsu}{\sqsubset}
\newcommand{\sqslq}{\sqsupseteq}
\newcommand{\size}[1]{|{#1}|}
\newcommand{\block}[1]{\langle {#1}\rangle}
\newcommand{\true}{\top}
\newcommand{\maketil}[1]{{#1}\ldots{#1}}
\newcommand{\til}{\maketil{,}}
\newcommand{\cuptil}{\maketil{\cup}}
\newcommand{\uplustil}{\maketil{\uplus}}
\renewcommand*{\mathellipsis}{\mathinner{{\ldotp}{\ldotp}{\ldotp}}}
\newcommand{\rst}[1]{|_{#1}}
\newcommand{\imm}[1]{{#1}{\rst{\text{imm}}}}
\renewcommand{\succ}[2]{\text{succ}_{#1}(#2)}
\newcommand{\aite}[3]{(#1?#2,#3)}
%\newcommand{\defeq}{\mathrel{\stackrel{\mathsf{def}}{=}}}
\newcommand{\defeq}{\triangleq}
\newcommand{\powerset}[1]{\mathcal{P}({#1})}
\newcommand{\finpowerset}[1]{\mathcal{P}_{<\omega}({#1})}
\renewcommand{\implies}{\Rightarrow}


\colorlet{colorPO}{gray!60!black}
\colorlet{colorRF}{green!60!black}
\colorlet{colorMO}{orange}
\colorlet{colorFR}{purple}
\colorlet{colorECO}{red!80!black}
\colorlet{colorSYN}{green!40!black}
\colorlet{colorHB}{blue}
\colorlet{colorPPO}{magenta}
\colorlet{colorPB}{olive}
\colorlet{colorSBRF}{olive}
\colorlet{colorRMW}{olive!70!black}
\colorlet{colorRSEQ}{blue}
\colorlet{colorSC}{violet}
\colorlet{colorPSC}{violet}
\colorlet{colorREL}{olive}
\colorlet{colorCONFLICT}{olive}
\colorlet{colorRACE}{olive}
\colorlet{colorWB}{orange!70!black}
\colorlet{colorPSC}{violet}
\colorlet{colorSCB}{violet}
\colorlet{colorDEPS}{violet}

\tikzset{
   every path/.style={>=stealth},
   sb/.style={->,color=colorSB,thin,shorten >=-0.5mm,shorten <=-0.5mm},
   sw/.style={->,color=colorSYN,shorten >=-0.5mm,shorten <=-0.5mm},
   rf/.style={->,color=colorRF,dashed,,shorten >=-0.5mm,shorten <=-0.5mm},
   hb/.style={->,color=colorHB,thick,shorten >=-0.5mm,shorten <=-0.5mm},
   mo/.style={->,color=colorMO,dotted,very thick,shorten >=-0.5mm,shorten <=-0.5mm},
   no/.style={->,dotted,thick,shorten >=-0.5mm,shorten <=-0.5mm},
   fr/.style={->,color=colorFR,dotted,thick,shorten >=-0.5mm,shorten <=-0.5mm},
   deps/.style={->,color=colorDEPS,dotted,thick,shorten >=-0.5mm,shorten <=-0.5mm},
   rmw/.style={->,color=colorRMW,thick,shorten >=-0.5mm,shorten <=-0.5mm},
}

%% Orders
\newcommand{\na}{\mathtt{na}}
\newcommand{\pln}{\mathtt{pln}}
\newcommand{\rlx}{\mathtt{rlx}}
\newcommand{\rel}{{\mathtt{rel}}}
\newcommand{\acq}{{\mathtt{acq}}}
\newcommand{\acqrel}{{\mathtt{acqrel}}}
\newcommand{\sco}{{\mathtt{sc}}}
\newcommand{\sto}{{\mathtt{st}}}
\newcommand{\full}{{\mathtt{sy}}}
\newcommand{\ld}{{\mathtt{ld}}}
\newcommand{\isb}{{\mathtt{isb}}}

%% Event labels

\newcommand{\evlab}[4]{{#1}({#3},{#4})}
\newcommand{\evflab}[1]{{\flab}^{#1}}
\newcommand{\evulab}[4]{{\ulab}^{#1}({#2},{#3},{#4})}
\newcommand{\rlab}[3]{{\lR}^{#1}({#2},{#3})}
\newcommand{\erlab}[4]{{\lR}^{#1}({#2},{#3},{#4})}
\newcommand{\wlab}[3]{{\lW}^{#1}({#2},{#3})}
\newcommand{\flab}[1]{{\lF}(#1)}
%\newcommand{\ulab}[4]{{\lU}^{#1}({#2},{#3},{#4})}

\newcommand{\lE}{{\mathtt{E}}}
\newcommand{\lC}{{\mathtt{C}}}
%\newcommand{\lP}{{\mathit{P}}}
\newcommand{\lM}{{\mathtt{M}}}
\newcommand{\lS}{{\mathtt{S}}}
\newcommand{\lR}{{\mathtt{R}}}
\newcommand{\lW}{{\mathtt{W}}}
\newcommand{\lA}{{\mathtt{A}}}
\newcommand{\lQ}{{\mathtt{Q}}}
\newcommand{\lL}{{\mathtt{L}}}
\newcommand{\lU}{{\mathtt{RMW}}}
\newcommand{\lF}{{\mathtt{F}}}
\newcommand{\lRES}{{\mathtt{Res}}}
\newcommand{\lAT}{{\mathtt{At}}}
\newcommand{\lATR}{{\mathtt{AtR}}}

\newcommand{\lLAB}{{\mathtt{lab}}}
\newcommand{\lTID}{{\mathtt{tid}}}
\newcommand{\lTYP}{{\mathtt{typ}}}
\newcommand{\lLOC}{{\mathtt{loc}}}
\newcommand{\lMOD}{{\mathtt{mod}}}
\newcommand{\lVALR}{{\mathtt{val_r}}}
\newcommand{\lVALW}{{\mathtt{val_w}}}
\newcommand{\lELAB}{{\mathtt{elab}}}
\newcommand{\lAVALS}{{\mathtt{atvals}}}

%% Relations

\newcommand{\lX}{\mathtt{X}}
\newcommand{\lPO}{{\color{colorPO}\mathtt{po}}}
\newcommand{\lRF}{{\color{colorRF} \mathtt{rf}}}
\newcommand{\lRMW}{{\color{colorRMW} \mathtt{rmw}}}
\newcommand{\lMO}{{\color{colorMO} \mathtt{mo}}}
\newcommand{\lMOx}{{\color{colorMO} \mathtt{mo}}_x}
\newcommand{\lMOy}{{\color{colorMO} \mathtt{mo}}_y}
\newcommand{\lCO}{{\color{colorMO} \mathtt{co}}}
\newcommand{\lCOx}{{\color{colorMO} \mathtt{co}}_x}
\newcommand{\lCOy}{{\color{colorMO} \mathtt{co}}_y}
\newcommand{\lFR}{{\color{colorFR} \mathtt{rb}}}
\newcommand{\lFRx}{{\color{colorFR} \mathtt{rb}}_x}
\newcommand{\lFRy}{{\color{colorFR} \mathtt{rb}}_y}
\newcommand{\lECO}{{\color{colorECO} \mathtt{eco}}}
\newcommand{\lSBRF}{{\color{colorSBRF} \mathtt{sbrf}}}
\newcommand{\lRSEQ}{{\color{colorRSEQ}\mathtt{rseq}}}
\newcommand{\lSW}{{\color{colorSYN}\mathtt{sw}}}
\newcommand{\lHB}{{\color{colorHB}\mathtt{hb}}}
%\newcommand{\lWB}{{\color{colorWB} \mathtt{wb}}}
\newcommand{\lDOB}{{\mathtt{dob}}}
\newcommand{\lBOB}{{\mathtt{bob}}}
\newcommand{\lAOB}{{\mathtt{aob}}}
\newcommand{\lOBS}{{\mathtt{obs}}}
\newcommand{\lEORD}{{\mathtt{eord}}}
\newcommand{\lTORD}{{\mathtt{tord}}}
\newcommand{\lSC}{{\mathtt{sc}}}

\newcommand{\lSCB}{{\color{colorSCB} \mathtt{scb}}}
\newcommand{\lPSC}{{\color{colorPSC} \mathtt{psc}}}
\newcommand{\lPSCB}{\lPSC_{\rm base}}
\newcommand{\lPSCF}{\lPSC_\lF}
\newcommand{\lCONFLICT}{{\color{colorCONFLICT} \mathtt{conflict}}}
\newcommand{\lRACE}{{\color{colorRACE} \mathtt{race}}}
\newcommand{\lNARACE}{{\color{colorRACE} \mathtt{na-race}}}

\newcommand{\lmakeW}[1]{\mathtt{w}#1}
\newcommand{\lWFR}{\lmakeW{\lFR}}
\newcommand{\lWECO}{\lmakeW{\lECO}}
\newcommand{\lWSCB}{\lmakeW{\lSCB}}
\newcommand{\lWPSCB}{\lmakeW{\lPSCB}}
\newcommand{\lWPSCF}{\lmakeW{\lPSCF}}
\newcommand{\lWPSC}{\lmakeW{\lPSC}}
\newcommand{\lWB}{\lmakeW{\lMO}}

\newcommand{\lDEPS}{{{\color{colorDEPS}\mathtt{deps}}}}
\newcommand{\lCTRL}{{{\color{colorDEPS}\mathtt{ctrl}}}}
\newcommand{\lCTRLISYNC}{{{\color{colorDEPS}\mathtt{ctrl_{isync}}}}}
\newcommand{\lDATA}{{{\color{colorDEPS}\mathtt{data}}}}
\newcommand{\lADDR}{{{\color{colorDEPS}\mathtt{addr}}}}

\newcommand{\lmakeE}[1]{#1\mathtt{e}}
\newcommand{\lRFE}{\lmakeE{\lRF}}
\newcommand{\lCOE}{\lmakeE{\lCO}}
\newcommand{\lFRE}{\lmakeE{\lFR}}
\newcommand{\lMOE}{\lmakeE{\lMO}}
\newcommand{\lmakeI}[1]{#1\mathtt{i}}
\newcommand{\lRFI}{\lmakeI{\lRF}}
\newcommand{\lCOI}{\lmakeI{\lCO}}
\newcommand{\lFRI}{\lmakeI{\lFR}}

\newcommand{\Tid}{\mathsf{Tid}}
\newcommand{\Loc}{\mathsf{Loc}}
\newcommand{\Val}{\mathsf{Val}}
\newcommand{\Lab}{\mathsf{Lab}}
\newcommand{\Mod}{\mathsf{Mod}}
\newcommand{\Modr}{\mathsf{Mod}_{\lR}}
\newcommand{\Modw}{\mathsf{Mod}_{\lW}}
\newcommand{\Modf}{\mathsf{Mod}_{\lF}}
\newcommand{\Modrmw}{\mathsf{Mod}_{\lU}}

\newcommand{\reorder}[1]{\mathsf{Reorder}({#1})}
%\newcommand{\deorder}[1]{\mathsf{Deorder}({#1})}
\newcommand{\remove}[1]{\mathsf{RemoveWR}({#1})}
\newcommand{\reorderWR}[1]{\mathsf{ReorderWR}({#1})}

\newcommand{\travConfigStep}{\rightarrow_{\rm TC}}
\newcommand{\ARM}{\ensuremath{\mathsf{ARM}}\xspace}
\newcommand{\correctTmap}{{\rm \bf correct\text{-}tmap}}
\newcommand{\rmwCovering}{{\rm \bf rmw\text{-}covering}}
\newcommand{\invWsc}{{\rm \bf write\text{-}sc\text{-}covering}}
\newcommand{\invViewRel}{\mathcal{I}_{\rm \bf view\text{-}rel}}
\newcommand{\setView}{{\rm set\text{-}view}}
\newcommand{\domView}{{\rm dom\text{-}view}}
\newcommand{\scSet}{{\rm sc\text{-}set}}
\newcommand{\scSetF}{{\rm sc\text{-}set\text{-}fun}}
\newcommand{\scSynchSet}{{\rm sc\text{-}synch}}
\newcommand{\acqSynchSet}{{\rm acq\text{-}synch}}
\newcommand{\relSynchFSet}{{\rm rel\text{-}synchf}}
\newcommand{\acqSet}{{\rm acq\text{-}set}}
\newcommand{\acqSetF}{{\rm acq\text{-}set\text{-}fun}}
\newcommand{\curSet}{{\rm cur\text{-}set}}
\newcommand{\curSetF}{{\rm cur\text{-}set\text{-}fun}}
\newcommand{\relSetFF}{{\rm rel\text{-}setf\text{-}fun}}
\newcommand{\msgSetF}{{\rm msg\text{-}set\text{-}fun}}
\newcommand{\msgSet}{{\rm msg\text{-}set}}
\newcommand{\relFunSet}{{\rm rel\text{-}setf}}
\newcommand{\lessView}{{\sf \bf less\text{-}view}}
\newcommand{\TSf}{\ensuremath{\mathcal{TS}}\xspace}
\newcommand{\TS}{\ensuremath{\mathit{TS}}\xspace}
\newcommand{\View}{\ensuremath{\mathcal{V}}\xspace}
\newcommand{\SCview}{\ensuremath{\mathcal{S}}\xspace}
\newcommand{\letdef}[2]{\kw{let} \; #1 \defeq #2 \; \kw{in}}
\newcommand{\nextset}{{\sf Next}}
\newcommand{\coverable}{{\sf Coverable}}
\newcommand{\promisable}{{\sf Promisable}}
%% \newcommand{\addToMemory}[2]{#1 \buildrel{\text{A}}\over\hookleftarrow #2}
\newcommand{\addToMemory}[2]{#1 \cup \{#2\}}

\newcommand{\stepa}{\leadsto_{\ARM}}
\newcommand{\stepp}{\promStepgen{}}
%% \leadsto_{\Promise}}
\newcommand{\stepptid}{\leadsto_{\Promise \; tid}}
\newcommand{\lISB}{\mathtt{F^{\isb}}}
\newcommand{\lDMB}{\mathtt{DMB}}
\newcommand{\lDMBSY}{\flab{\SY}}
\newcommand{\lDMBLD}{\flab{\LD}}
\newcommand{\lDMBST}{\flab{\ST}}
\newcommand{\SY}{\mathtt{sy}}
\newcommand{\LD}{\mathtt{ld}}
\newcommand{\ST}{\mathtt{st}}
\newcommand{\removeCR}[1]{\mathsf{RemoveCR}({#1})}
%% \spnewtheorem{defn}{Definition}{\bfseries}{}
%% \crefname{defn}{Definition}{Definitions}
\newcommand{\Timestamp}{\mathtt{Time}}
\newcommand{\TimestampMap}{\mathit{TimeMap}}

\newcommand{\Prog}{\ensuremath{Prog}\xspace}
\newcommand{\ProgARM}{\ensuremath{Prog_\ARM}\xspace}

% Promise definitions
\newcommand{\msg}[4]{\tup{#1:#2@#3,#4}}
\newcommand{\Promise}{\ensuremath{\mathsf{Promise}}\xspace}
%% \newcommand\PromSet{\mathit{promises}}
\newcommand\PromSet{\mathtt{P}}
\newcommand\PromState{\mathit{st}}
\newcommand{\V}{V}
\newcommand{\view}{view}
%% \newcommand{\viewCur}{view_{\sf cur}}
%% \newcommand{\viewAcq}{view_{\sf acq}}
%% \newcommand{\viewRel}{view_{\sf rel}}
\newcommand{\viewCur}{{\sf cur}}
\newcommand{\viewAcq}{{\sf acq}}
\newcommand{\viewRel}{{\sf rel}}
\newcommand{\viewfRel}{{\sf rel}}
\newcommand{\loc}{\ell}
\newcommand{\val}{v}
%% \newcommand{\tstamp}{\tau}
\newcommand{\tstampRange}[2]{(#1,#2]}
\newcommand{\simRel}{\mathcal{I}}
\newcommand{\invMemOne}{\mathcal{I}_{\rm \bf mem1}}
\newcommand{\invMemTwo}{\mathcal{I}_{\rm \bf mem2}}
\newcommand{\invMemThree}{\mathcal{I}_{\rm \bf mem3}}
\newcommand{\invView}{\mathcal{I}_{\rm \bf view}}
\newcommand{\invState}{\mathcal{I}_{\rm \bf state}}
\newcommand{\pstate}{\sigma}
\newcommand{\labelF}{{\rm label}}
\newcommand{\ltsReq}{{\rm lts\text{-}req}}
\newcommand{\maxF}{{\rm max}}
\newcommand{\compileF}{{\rm compile}}
\newcommand{\compileMod}{{\rm mod\text{-}compile}}
\newcommand{\compileFenceMod}{{\rm fence\text{-}mod\text{-}compile}}
%% \newcommand{\writeLbl}[3]{W \; (#1, #2, #3)}
\newcommand{\writeLbl}[2]{W \; (#1, #2)}
\newcommand{\writeBlankLbl}{W \; \_}
\newcommand{\fenceLbl}[1]{F \; #1}
%% \newcommand{\readLbl}[3]{R \; (#1, #2, #3)}
\newcommand{\readLbl}[2]{R \; (#1, #2)}
\newcommand{\readBlankLbl}{R \; \_}
\newcommand{\casLbl}[5]{U \; (#1, #2, #3, #4, #5)}
\newcommand{\svaleq}{\asymp}
\newcommand{\strans}[1]{\xrightarrow{#1}}

%% \newcommand\readWriteTau{\textsf{access-tmap}}
%% \newcommand\readsView{\textsf{reads-view}}

%\newcommand{\RC}{\ensuremath{\mathsf{RC11}}\xspace}
%\newcommand{\WRC}{\ensuremath{\mathsf{WRC11}}\xspace}

\newcounter{mylabelcounter}

\makeatletter
\newcommand{\labelAxiom}[2]{%
\hfill{\normalfont\textsc{(#1)}}\refstepcounter{mylabelcounter}
\immediate\write\@auxout{%
  %% \string\newlabel{#2}{{1}{\thepage}{{#1}}{mylabelcounter.\number\value{mylabelcounter}}{}}
  \string\newlabel{#2}{{\unexpanded{\normalfont\textsc{#1}}}{\thepage}{{\unexpanded{\normalfont\textsc{#1}}}}{mylabelcounter.\number\value{mylabelcounter}}{}}
}%
}
\makeatother

\newcommand{\squishlist}[1][$\bullet$]{%[$\tinybullet$]{
 \begin{list}{#1}
  { \setlength{\itemsep}{0pt}
     \setlength{\parsep}{0pt}
     \setlength{\topsep}{1pt}
     \setlength{\partopsep}{0pt}
     \setlength{\leftmargin}{1.2em}
     \setlength{\labelwidth}{0.5em}
     \setlength{\labelsep}{0.4em} } }
\newcommand{\squishend}{
  \end{list}  }

\newcommand{\app}[1]{{\color{blue}\textbf{ANTON: #1}}}

%% Old defs.tex content

%% \newcommand{\excl}[1]{{\color{purple} #1}}
\newcommand{\ifext}[2]{\ifdefined\extflag{#1}\else{#2}\fi}

\newcommand{\Reads}{{\sf Reads}}
%% \newcommand{\Reads}{\{ \angled{\tId, \cpath, \loc} \mid \tId, \cpath, \loc \}}

\newcommand{\excl}[1]{}
\newcommand\tT{\mathbf{t}}

\newcommand{\ListOf}[1]{\mathit{List}\;{#1}}
\newcommand{\Label}{\mathit{Label}}
\newcommand{\lab}{\mathit{lab}}
\newcommand{\Pset}{\mathbb{P}}
\newcommand{\dep}{\mathit{dep}}
\newcommand{\regs}{\mathrm{regs}}

\newcommand\viewf{\textbf{\sf viewf}}
\newcommand{\nthf}{\mathrm{nth}}

\newcommand{\deltaToView}{\delta\textup{\sf -to-view}}
\newcommand{\deltaHmap}{\textup{\sf comb-time}}
\newcommand{\invDeltaDefOne}{\inv_{\delta\textup{\sf -con-1}}}
\newcommand{\invDeltaDefTwo}{\inv_{\delta\textup{\sf -con-2}}}
\newcommand{\invDeltaDefThree}{\inv_{\delta\textup{\sf -con-3}}}
\newcommand{\invDeltaDefFour}{\inv_{\delta\textup{\sf -con-4}}}

\newcommand{\tIdState}{\textup{\sf thread-state}}
\newcommand{\lastCommittedWrite}{\textup{\sf last-write-com}}

\newcommand\certifiableTid{\mathsf{certifiable}_{\tId}}
\newcommand\certifiable{\mathsf{certifiable}}

\newcommand\lastInstr[1]{#1.\textup{\sf last}}

\newcommand\length{\textup{\sf length}}
\newcommand\prefix{\textup{\sf prefix}}

\newcommand{\ExclReadType}{\mathit{er}}
\newcommand{\ExclWriteType}{\mathit{ew}}
\newcommand{\ComWriteState}{\mathit{im}}

\newcommand\InMemory{im}
\newcommand{\NotInMemory}{\textup{\sf no-mem}}
\newcommand{\IssuedToMemory}{\textup{\sf mem}}
\newcommand{\ExclIssuedToMemory}{\textup{\sf excl-mem}}

\newcommand{\readInst }[3][\ExclReadType]{#2 \;:=_{\excl{#1}}\;[#3]}
\newcommand{\fenceInst}[1]{\fence{#1}}
\newcommand{\dmbSY}{\fenceInst{\SY}}
\newcommand{\dmbLD}{\fenceInst{\LD}}

\newcommand\ifGoto{\textsf{if}-\textsf{goto}}
\newcommand{\ifGotoInst}[2]{\textsf{if} \; #1 \; \textsf{goto} \; #2}
\newcommand{\writeInst}[2]{[#1]\;:=\;#2}
\newcommand{\writeExclInst}[3]{\textup{\sf atomic-write}(#1, [#2], #3)}
\newcommand{\assignInst}[2]{#1\;:=\;#2}

\newcommand{\restrict}[2]{#1{\restriction_{#2}}}

%% \newcommand{\writeRI}[3][, \ExclType]{#2:#3\excl{, #1}}
\newcommand{\writeRI}[3][]{#2:#3\excl{#1}}

%% \newcommand\ARM{\mathrm{ARM}}
\newcommand\ARMt{\mathrm{ARM}{+}\tau}
%% \newcommand\Promise{\mathrm{Promise}}

\newcommand{\tstamp}[1]{\mbox{\small\color{brown!60!black}\bf{#1}}}

\newcommand{\OrdPrevRequest}{\textup{\sf prev-Ord-req}}

\newcommand{\uniqueTimeLoc}{\textup{\sf uniq-time-loc}}
\newcommand{\ordPrevRequest}{\textup{\sf ord-prev-req}}
\newcommand{\coherentThread}{\textup{\sf coherent-thread}}
\newcommand{\timeRangeCondition}{\textup{\sf time-range}}
\newcommand{\sameMemory}{\textup{\sf same-memory}}
\newcommand{\instToLbl}{\textup{\sf inst-to-lbl}}
\newcommand{\cmdsToLbls}{\textup{\sf cmds-to-lbls}}
\newcommand{\cmdsToLblsAux}{\textup{\sf cmds-to-lbls-aux}}

\newcommand{\instToVrtx}{\textup{\sf inst-to-vertex}}
\newcommand{\cmdsToVrtxs}{\textup{\sf cmds-to-vertices}}
\newcommand{\cmdsToVrtxsAux}{\textup{\sf cmds-to-vertices-aux}}

\newcommand{\prevInstrCommitted}{\textup{\sf prev-instr-committed}}
\newcommand{\prevReadsCommitted}{\textup{\sf prev-reads-committed}}
\newcommand{\prevFencesCommitted}{\textup{\sf prev-fences-committed}}
\newcommand{\prevBrCommitted}{\textup{\sf prev-branches-committed}}
\newcommand{\prevBrFencesCommitted}{\textup{\sf prev-branches-and-fences-committed}}
\newcommand{\prevCmdDetermined}{\textup{\sf prev-fully-determined}}
\newcommand{\prevNoRestart}{\textup{\sf no-prev-restartable-reads-from-loc}}
\newcommand{\prevExclCommitted}{\textup{\sf prev-excl-to-loc-committed}}
\newcommand{\noFollowingWcom}{\textup{\sf no-following-com-writes-to-loc}}

%% \newcommand\reorderableRel{\leftrightarrow} 
%% \newcommand\notReorderableRel{\not \leftrightarrow} 
\newcommand\reorderableRel[2]{#1 \hookrightarrow #2} 
\newcommand\notReorderableRel[2]{#1 \not \hookrightarrow #2} 
%% \newcommand\ReorderingFunction{is\_reorderable}
%% \newcommand{\checkReorderings}[1]{\textup{\sf check-reorderings}(#1)}
\newcommand{\checkReorderings}[1]{#1 \setminus {\reorderableRel{}{}}}

\newcommand{\deleteUpdReads}{\textup{\sf delete-upd-reads}}
\newcommand{\acceptRequest}{\textup{\sf accept-request}}
\newcommand{\acceptExclWrite}{\textup{\sf accept-excl-write}}

\newcommand{\readsBetweenCommitted}{\textup{\sf reads-in-between-committed}}
\newcommand{\noWritesBetween}{\textup{\sf no-writes-to-loc-in-between}}
\newcommand{\noDiffReadsBetween}{\textup{\sf no-different-write-reads-in-between}}
\newcommand{\samePropagated}{\textup{\sf propagated-to-same-threads}}
\newcommand{\fullyPropagated}{\textup{\sf fully-propagated}}
\newcommand{\getNewTapeCell}{\textup{\sf get-new-tapecell}}
\newcommand{\prevReadFromOther}{\textup{\sf prev-read-from-other-write}}
%\newcommand{\nextPath}{\textup{\sf next-path}}
\newcommand{\tapeUpdRestart}{\textup{\sf tape-upd-restart}}
\newcommand{\doesntPreventExcl}{\textup{\sf doesnt-prevent-excl}}
\newcommand{\tapeUpdWcom}{\textup{\sf tape-upd-Wcom}}
\newcommand{\tapeUpdRsat}{\textup{\sf tape-upd-Rsat}}
\newcommand{\tapeUpdIf}{\textup{\sf tape-upd-IfGoto}}
\newcommand{\noExclInBetween}{\textup{\sf no-excl-in-between}}
\newcommand{\getLoc}{\textup{\sf get-loc}}

%% PROBLEMS WITH: \C, \next, \a
  \newcommand\armStepWriteCommit{\armStepgen{\transenv{Write commit} \; \tId \; \cpath \; x \; \stval \; \tau}}
  \newcommand\armStepWriteCommitLoc{\armStepgen{\transenv{Write commit} \; \tId \; \cpath \; \loc \; \stval \; \tau}}
  \newcommand\armStepWriteCommitPrime{\armStepgen{\transenv{Write commit} \; \tId' \; \cpath' \; y \; \stval' \; \tau}}
  \newcommand\armStepWriteCommitP{\armStepPgen{\transenv{Write commit} \; \tId \; \cpath \; x \; \stval}}
  \newcommand\armStepWriteCommitPLoc{\armStepPgen{\transenv{Write commit} \; \tId \; \cpath \; \loc \; \stval}}
  \newcommand\armStepWriteCommitPrimeP{\armStepPgen{\transenv{Write commit} \; \tId' \; \cpath' \; y \; \stval'}}


  \newcommand\promStepBranch{\promStepgen{\transenv{Branch commit} \; \tId}}
  \newcommand\promTStepBranch{\promTStepgen{\transenv{Branch commit}}}

  \newcommand\promStepAcquire{\promStepgen{\transenv{Acquire fence commit} \; \tId}}
  \newcommand\promTStepAcquire{\promTStepgen{\transenv{Acquire fence commit}}}
  \newcommand\promStepRelease{\promStepgen{\transenv{Release fence commit} \; \tId}}
  \newcommand\promTStepRelease{\promTStepgen{\transenv{Release fence commit}}}

  \newcommand\promStepRead{\promStepgen{\transenv{Read from memory} \; \tId \; \writeEvt{x}{\stval}{\tau}{\R}}}
  \newcommand\promTStepRead{\promTStepgen{\transenv{Read from memory} \; \writeEvt{x}{\stval}{\tau}{\R}}}
  \newcommand\promTStepReadLoc{\promTStepgen{\transenv{Read from memory} \; \writeEvt{\loc}{\stval}{\tau}{\R}}}

  \newcommand\promStepPromise{\promStepgen{\transenv{Promise write} \; \tId \; \writeEvt{x}{\stval}{\tau}{\R}}}
  \newcommand\promStepPromiseRPrime{\promStepgen{\transenv{Promise write} \; \tId \; \writeEvt{x}{\stval}{\tau}{\R'}}}
  \newcommand\promTStepPromise{\promTStepgen{\transenv{Promise write} \; \writeEvt{x}{\stval}{\tau}{\R}}}
  \newcommand\promTStepPromiseLoc{\promTStepgen{\transenv{Promise write} \; \writeEvt{\loc}{\stval}{\tau}{\R}}}

  \newcommand\promStepFulfill{\promStepgen{\transenv{Fulfill promise} \; \tId \; \writeEvt{x}{\stval}{\tau}{\R}}}
  \newcommand\promTStepFulfill{\promTStepgen{\transenv{Fulfill promise} \; \writeEvt{x}{\stval}{\tau}{\R}}}
  \newcommand\promTStepFulfillLoc{\promTStepgen{\transenv{Fulfill promise} \; \writeEvt{\loc}{\stval}{\tau}{\R}}}

  \newcommand\promStepAssign{\promStepgen{\transenv{Local variable assignment} \; \tId}}
  \newcommand\promTStepAssign{\promTStepgen{\transenv{Local variable assignment}}}

  \newcommand\promStepNop{\promStepgen{\transenv{Execution of $\nop$} \; \tId}}
  \newcommand\promTStepNop{\promTStepgen{\transenv{Execution of $\nop$}}}

  \newcommand{\tapeToCertificate}{\mathsf{tape}\textsf{-}\mathsf{to}\textsf{-}\mathsf{certificate}}
  %% \newcommand{\Timestamp}{\mathit{Time}}

  \newcommand{\graybox}[1]{$\colorbox{gray!20}{$#1\!$}$}

  %\newcommand\Rarm{\R_{\ARM}}
  \newcommand\Rarm{\mathsf{view}_{\ARM}}
  \newcommand\transenv[1]{\textcolor{darkgray}{\textup{\textsf{\textbf{\mathversion{bold}#1}}}}}

  \newcommand\promMessage{\mathit{msg}}
  \newcommand\promMessageSet{\mathit{Msg}}

  \newcommand\FtypeARM{\mathit{fmod}_{\ARM}}
  %% \newcommand\FtypeProm{\mathit{ftype}_{\Promise}}
  \newcommand\FtypeProm{\mathit{fmod}}

  \newcommand\RtypeProm{\mathit{rtype}_{\Promise}}
  \newcommand\WtypeProm{\mathit{wtype}_{\Promise}}

  %% \newcommand\V{\mathit{V}}
  \newcommand\R{\mathit{view}}
  %% \newcommand\View{\mathit{View}}
  \newcommand\Rsc{\R_{\mathrm{sc}}}
  \newcommand\Rcur{\R_{\mathrm{cur}}}
  \newcommand\Racq{\R_{\mathrm{acq}}}
  \newcommand\Rrel{\R_{\mathrm{rel}}}
  
  \newcommand\lessUpToDelta[3]{#2 <_{#1} #3}

  \newcommand\StmtARM{S}
  \newcommand\StmtProm{S_{\Promise}}


  %% \newcommand\C{C}
  \newcommand\Cf{\mathit{Prog}}%\textsf{Prog}
  \newcommand\Carm{{cmds}}
  \newcommand\CARM{{Cmds}}
  \newcommand\Cfarm{\Cf}
  \newcommand\Cprom{{cmds}}
  \newcommand\Cfprom{\Cf}
  %% \newcommand\Carm{\textsf{C}_{\ARM}}
  %% \newcommand\Cfarm{\Cf_{\ARM}}
  %% \newcommand\Cprom{\textsf{C}_{\Promise}}
  %% \newcommand\Cfprom{\Cf_{\Promise}}
  
  %% \newcommand\PromSet{\mathit{promises}}
  %% \newcommand\PromState{\mathit{st}}

  \newcommand\scAcqRead[2]{#1 := [#2]_{\sf LDAR}}
  \newcommand\scRelWrite[2]{[#1]_{\sf STLR} := #2}

  \newcommand\acqFence{\fence{\sf acquire}}
  \newcommand\relFence{\fence{\sf release}}
  \newcommand\scFence{\fence{\sf sc}}
  \newcommand\fence[1]{\mathsf{fence}({#1})}

  \newcommand\syFence{\fence{\sf sy}}
  \newcommand\ldFence{\fence{\sf ld}}
  
  \newcommand\reqInfoRead[1]{{\sf rd} \; #1}
  \newcommand\reqInfoWrite[2]{{\sf wr} \; #1:#2}
  \newcommand\reqInfoFence{{\sf dmb}}

  \newcommand\stRequest[3]{\angled{#1, #2, #3}}
  \newcommand\stRequestWrite[4]{\angled{#1, #2, \reqInfoWrite{#3}{#4}}}
  \newcommand\stRequestRead[3]{\angled{#1, #2, \reqInfoRead{#3}}}
  \newcommand\stRequestFence[2]{\angled{#1, #2, \reqInfoFence}}

  \newcommand\moTau{\textsf{tedges}}
  \newcommand\invTT{\textsf{inv}}

  % \newcommand\opstau{op\_\loc\_\tau}
  \newcommand\opstau{\textsf{com-writes-time}}
  \newcommand\readsSatisfiedR{\textsf{sat-reads-view}}
  \newcommand\readsCommittedR{\textsf{com-reads-view}}

  \newcommand\hmap{\mathit{H}}
  \newcommand\tmap{\mathit{H}_{\tau}}
  \newcommand\rmap{\mathit{H}_{\mathsf{view}}}
  \newcommand\omap{\mathit{H}_{\le}}

  \newcommand\StateARM{\mathsf{State}_{\ARM}}
  \newcommand\StateARMtau{\mathsf{State}_{\ARMt}}
  \newcommand\StateProm{\mathsf{State}_{\Promise}}
  \newcommand\TStateProm{\mathsf{TState}_{\Promise}}

  \newcommand\armStep{\xrightarrow[\ARMt]{}}
  \newcommand\armStepl{\xrightarrow[\ARMt]{}}
  \newcommand\armStepgen[1]{\xrightarrow[\ARMt]{#1}}
  \newcommand\armStepP{\xrightarrow[\ARM]{}}
  \newcommand\armStepPgen[1]{\xrightarrow[\ARM]{#1}}

  \newcommand\armStepPrSat{\armStepPgen{\transenv{Read satisfy} \; \tId \; \cpath \; \tId' \; \cpath' \; x \; \stval}}
  \newcommand\armStepPrSatLoc{\armStepPgen{\transenv{Read satisfy} \; \tId \; \cpath \; \tId' \; \cpath' \; \loc \; \stval}}
  \newcommand\armStepRSat{\armStepgen{\transenv{Read satisfy} \; \tId \; \cpath \; \tId' \; \cpath' \; x \; \stval}}
  \newcommand\armStepPrSatFail{\armStepPgen{\transenv{Read satisfy (fail)} \; \tId \; \cpath \; \tId' \; \cpath' \; x \; \stval}}
  \newcommand\armStepPrSatFailLoc{\armStepPgen{\transenv{Read satisfy (fail)} \; \tId \; \cpath \; \tId' \; \cpath' \; \loc \; \stval}}
  \newcommand\armStepRSatFail{\armStepgen{\transenv{Read satisfy (fail)} \; \tId \; \cpath \; \tId' \; \cpath' \; x \; \stval}}
  \newcommand\armStepPrInFlightSat{\armStepPgen{\transenv{Read satisfy from in-flight write} \; \tId \; \cpath \; \cpath' \; x \; \stval}}
  \newcommand\armStepPrInFlightSatLoc{\armStepPgen{\transenv{Read satisfy from in-flight write} \; \tId \; \cpath \; \cpath' \; \loc \; \stval}}
  \newcommand\armStepRInFlightSat{\armStepgen{\transenv{Read satisfy from in-flight write} \; \tId \; \cpath \; \cpath' \; x \; \stval}}

  \newcommand\promStep{\xrightarrow[\Promise]{}}
  \newcommand\promStepl{\xrightarrow[\Promise]{}}
  \newcommand\promStepgen[1]{\xrightarrow[\Promise]{#1}}
  \newcommand\promTStepgen[1]{\xrightarrow[\Promise \; \tId]{#1}}
  \newcommand\promTStep{\xrightarrow[\Promise \; \tId]{}}

  \newcommand\semState[2]{|[#1|] ^{#2}}
  \newcommand\semf[2]{|[#1|] ^{#2}}
  \newcommand\semfcom[2]{|[#1|] ^{#2}_\mathsf{com}}
  \newcommand\sembr[1]{|[#1|]}
  \newcommand\PreExecutions{{\rm PreExecs}}

  \newcommand\textIf{\text{\underline{if}} \;}
  \newcommand\textElif{\text{\underline{elif}} \;}
  \newcommand\textElse{\text{\underline{else}} \;}
  \newcommand\textThen{\text{\underline{then}} \;}
  \newcommand\textLet{\text{\underline{let}} \;}
  \newcommand\textIn{\text{\underline{in}}}

  %% \newcommand\nextPathCom[3]{nextPath_{com}(#1, #2, #3)}
  \newcommand\nextPathCom[3]{\mathsf{next}\textsf{-}\mathsf{path}(#1, #2, #3)}
  \newcommand\nextPathProm{\mathsf{next}\textsf{-}\mathsf{path}_{\Promise}}

  \newcommand\comShift{\phantom{{}_\mathsf{com}}}

  \newcommand\extendExpr{\expr\_extend}

  \newcommand\regst{\mathsf{regf}}
  \newcommand\regstcom{\mathsf{regf}_\mathsf{com}}
  \newcommand\regf{\mathit{regf}}
  
  \newcommand\Mpop{\mathit{M}_{\mathrm{POP}}}
  %% \newcommand\Mprom{\mathit{M}_{\mathrm{Promise}}}
  \newcommand\Mprom{\mathit{M}}
  \newcommand{\Mcomp}[3]{\angled{#1, #2, #3}}


  \newcommand\Request{\textit{req}}
  \newcommand\RequestSet{\mathit{ReqSet}}
  \newcommand\RequestInfo{\mathit{reqinfo}}
  \newcommand\RequestInfoSet{\mathit{ReqInfoSet}}
  \newcommand\Evt{\mathit{Evt}}
  \newcommand\Ord{\mathit{Ord}}
  \newcommand\Prop{\mathit{Prop}}
  \newcommand\ExclMap{\mathit{Excl}}

  \newcommand\IssuingOrder{\mathit{iord}}
  \newcommand\IssuingOrderf{\mathit{iordf}}

  \newcommand\Issued[1]{\mathsf{requested} \; #1}
  \newcommand\Satisfied{\mathsf{sat}}
  \newcommand\SatisfiedInFlight{\mathsf{inflight}}
  \newcommand\Committed{\mathsf{com}}
  \newcommand\Plain{\mathsf{pln}}
  \newcommand\Exclusive{\mathsf{excl}}
  \newcommand\None{\mathsf{none}}
  \newcommand\Any{\mathsf{any}}

  %% \newcommand\LD{\mathsf{LD}}
  %% \newcommand\SY{\mathsf{SY}}
  %% \newcommand\ST{\mathsf{ST}}

  \newcommand\tId{\mathit{tid}}
  %% \newcommand\Tid{\mathit{Tid}}

  \newcommand\Tape{\mathit{Tape}}
  \newcommand\tape{\mathit{tape}}
  \newcommand\tapef{\mathit{tapef}}
  \newcommand\TapeCell{\mathit{TapeCell}}
  \newcommand\tapeCell{\mathit{tapecell}}
  \newcommand\Taken{\mathsf{taken}}
  \newcommand\Ignored{\mathsf{ignored}}

  \newcommand\cpath{\mathit{path}}
  \newcommand\cpathSY{\cpath^{\SY}}
  \newcommand\cpathLD{\cpath^{\LD}}
  \newcommand\cpathLDSY{\cpath^{\LD\SY}}
  \newcommand\Path{Path}
  
  \newcommand\tapeRead[1]{\textsf{R} \; #1}
  \newcommand\tapeFence[2]{\textsf{F} \; #1 \; #2}
  \newcommand\tapeWrite[1]{\textsf{W} \; #1}
  \newcommand\tapeIfGoto[2]{\textsf{If} \; #1 \; #2}
  \newcommand\tapeNop{\textsf{Nop}}
  \newcommand\tapeAssign{\textsf{Assign}}

  \newcommand\tapeSatisfied[2]{\Satisfied \; #1 \; #2}
  \newcommand\satisfiedState{\mathit{sat\text{-}state}}

  \newcommand\tapePending[2]{\textsf{pending} \; #1 \; #2}
  \newcommand\tapeWriteCommitted[3]{\Committed \; #1 \; #2 \; #3}

  \newcommand\Fstate{\mathit{st}_{\textup{\rm fence}}}
  \newcommand\Rstate{\mathit{st}_{\textup{\rm read}}}
  \newcommand\Wstate{\mathit{st}_{\textup{\rm write}}}
  \newcommand\IfState{\mathit{st}_{\textup{\rm ifgoto}}}

  \newcommand\locvar{\iota}
  %% \newcommand\loc{\ell}
  %% \newcommand\Loc{Loc}
  \newcommand\Reg{\mathit{Reg}}
  \newcommand\reg{\mathit{reg}}
  %% \newcommand\val{\mathit{val}}
  \newcommand\stval{\mathit{val}}
  \newcommand\Stval{\mathit{Val}}
  \newcommand\expr{\mathit{expr}}
  \newcommand\Expr{\mathit{Expr}}
  \newcommand\z{\mathit{k}}

  \newcommand\dmb{\textsf{dmb}}
  \newcommand\nop{\textsf{nop}}

  \newcommand\ImmediateEdge{ImmediateEdge}
   
  \newcommand\armState[1]{\ARM_{state}(#1)}
  \newcommand\angled[1]{\langle #1 \rangle}
  %% \newcommand\TSfprom{TSf}
  %% \newcommand\TSfprom{TSf_{\Promise}}
  %% \newcommand\TS{TS_{\Promise}}
  %% \newcommand\TSfprom{\textsf{tsf}}
  %% \newcommand\TS{\textsf{ts}}
  %% \newcommand\TSfprom{\mathit{tsf}}
  %% \newcommand\TS{\mathit{ts}}

  %% \newcommand\M{M}
  \newcommand\e{e}
  \newcommand\w{w}
  %% \newcommand\dom[1]{\mathsf{dom}(#1)}
  \newcommand\taumapping{\mathsf{map}_{\tau}}

  \newcommand\Nop{Nop}
  \newcommand\Write[1]{Write \; #1}
  \newcommand\WritePending[1]{\Write(\Pending \; #1)}
  \newcommand\Read[1]{Read \; #1}
  \newcommand\ReadIssued[1]{Read \; (\Issued \; #1)}
  \newcommand\ReadSatisfied[4]{Read \; (\Satisfied \; #1 \; #2 \; #3 \; #4)}
  %% \newcommand\next[2]{next(#1, #2)}
  \newcommand\nextPath[2]{\mathsf{next}\textsf{-}\mathsf{path}(#1, #2)}
  \newcommand\lastIndex{\mathsf{last}\textsf{-}\mathsf{index}}
  \newcommand\last[1]{last(#1)}
  \newcommand\lastSY{\mathsf{last}\SY}
  \newcommand\lastCF{\mathsf{lastCF}}
  \newcommand\lastLD{\mathsf{last}\LD}
  \newcommand\lastLDSY{\mathsf{last}\LD\SY}
  \newcommand\Fence[2]{Fence \; #1 \; #2}
  \newcommand\FenceSY[1]{\Fence{#1}{SY}}
  \newcommand\FenceLD[1]{\Fence{#1}{LD}}
  \newcommand\IfGoto[2]{IfGoto \; #1 \; #2}
  \newcommand\IfGotoK[1]{\IfGoto{#1}{k}}
  \newcommand\tick{✓}
  \newcommand\Certificate{certificate}
  \newcommand\CommandState{instrPlan}
  %% \newcommand\State{State}
  \newcommand\writeEvt[4]{\angled{#1:#2@#3,#4}}
  %% \newcommand\writeEvt[4]{#1:#2@#3,#4}
  \newcommand\simrel{\mathcal{I}}
  \newcommand\simrelPre{\mathcal{I}_{\textup{\rm pre}}}
  \newcommand\simrelBase{\mathcal{I}_{\textup{\rm base}}}
  \newcommand\inv{\mathcal{I}}
  \newcommand\s{\mathbf{s}}
  \newcommand\sfst{\mathbf{s}_0}
  \newcommand\ssnd{\mathbf{s}_1}
  \newcommand\aT{\mathbf{a}}
  \newcommand\afst{\mathbf{a}_0}
  \newcommand\asnd{\mathbf{a}_1}
  \newcommand\p{\mathbf{p}}
  \newcommand\ptid{\mathbf{p}\textsf{-}\tId}
%  \newcommand\ordPlusMO{S_{\Ord \cup mo}}
  \newcommand\ordPlusMO{\textsf{S-Ord-mo}}

  \newcommand\finalStateP{\mathsf{Final}^{\ARM}}
  \newcommand\finalStateA{\mathsf{Final}^{\ARMt}}
  \newcommand\finalStateProm{\mathsf{Final}^{\Promise}}

  \newcommand\sinit{\mathbf{s}^{\rm init}}
  \newcommand\ainit{\mathbf{a}^{\rm init}}
  \newcommand\pinit{\mathbf{p}^{\rm init}}

  \newcommand\invI[1]{\inv_{\textup{\rm #1}}}
  \newcommand\invARM[1]{\inv^\ARM_{\textup{\rm #1}}}

  \newcommand\invTidWriteComCERT{\inv^{\tId}_{\mathsf{w-cert}}}
  \newcommand\invStateCERT{\invI{state-cert}}
  \newcommand\invViewDeltaCERT{\invI{view-cert}}
  \newcommand\invViewWriteCERT{\invI{view-write-cert}}
  \newcommand\invViewRelCERT{\invI{write-rel-cert}}
  \newcommand\invViewReadCERT{\invI{view-read-cert}}
  \newcommand\invWriteTimestampCERT{\invI{write-time-cert}}
  %% \newcommand\invViewCERT{\invI{view-cert}}
  \newcommand\invMemZeroCERT{\invI{mem-1-tid-cert}}
  \newcommand\invMemOneCERT{\invI{mem-1-com-cert}}
  \newcommand\invMemTwoCERT{\invI{mem-2-cert}}
  \newcommand\simrelBaseCERT{\mathcal{I}_{\textup{\rm base-cert}}}

  \newcommand\invCf{\invI{prg}}
  \newcommand\invTId{\invI{tid}}
  \newcommand\invPrefix{\invI{prefix}}
  \newcommand\invSPrefix{\invI{strong prefix}}
  %% \newcommand\invMemOne{\invI{mem1}}
  %% \newcommand\invMemTwo{\invI{mem2}}
  %% \newcommand\invMemThree{\invI{mem3}}
  %% \newcommand\invView{\invI{view}}
  %% \newcommand\invState{\invI{state}}
  \newcommand\invReach{\invI{reach}}
  \newcommand\invComWrite{\invI{com-SY}}
  \newcommand\invCert{\invI{cert}} %\Certificate}}
  \newcommand\invCertTid{\inv_{cert \; \tId}} %\Certificate}}
  \newcommand\simrelTid{\inv_{exec \; \tId}} %\Certificate}}
  \newcommand\invApartialOrderOrd{\inv^{ARM}_{\Ord \; \text{is a partial order}}}
  \newcommand\invAuniqWrite{\invARM{unique write}}
  \newcommand\invAtransClosedOrd{\invARM{\Ord = transitive\_closure(\Ord)}}
  \newcommand\invAimmediateEdge{\invARM{Immediate edge}}
  \newcommand\invAimmediatePath{\invARM{Immediate path}}
  \newcommand\invAmaxPath{\invARM{max \; \cpath}}
  \newcommand\invPromUptoARM{\invI{Promise is up to ARM}}
  \newcommand\invPromUptoARMtId{\inv^{\tId}_{\textup{Promise is up to ARM}}}
  \newcommand\invPromUptoARMnot{\invI{Promise isn't up to ARM}}


  \newcommand\correctStateA{\inv^{\ARMt}_{\textup{\rm correct}}}
  \newcommand\correctStateP{\invARM{correct}}
  \newcommand\invATapeCf{\invARM{tape-Prg}}
  \newcommand\invATapeCfState{\invARM{{tape-Prg-State}}}
  \newcommand\invAReadWrite{\invARM{{Read-Write}}}
  \newcommand\invAReadRead{\invARM{{Read-Read}}}
  \newcommand\invAWriteWriteRead{\invARM{{Write-Write-Read}}}
  \newcommand\invAReadCommittedWrite{\invARM{{Read-Write-Committed}}}
  \newcommand\invAview{\invARM{{View}}}
  \newcommand\invAviewWrite{\invARM{{View-Write}}}
  \newcommand\invAviewRead{\invARM{{View-Read}}}
  \newcommand\invAWriteView{\invARM{{Write-View}}}
  \newcommand\invAnextCommitted{\invARM{{Next-Committed}}}
  \newcommand\invACf{\invARM{Prg}}
  \newcommand\invAtId{\invARM{tid}}
  \newcommand\invAldRead{\invARM{LD-Read}}
  \newcommand\invAtypePreservation{\invARM{{tape-Type}}}
  \newcommand\invAcommittedPreservation{\invARM{{Committed-Preserve}}}
  \newcommand\invAstatePreservation{\invARM{{State-Preserve}}}
  \newcommand\invAstateCom{\invARM{S-Scom}}
  \newcommand\invAcomFences{\invARM{{Committed-Fences}}}
  \newcommand\invAordMOacyclic{\invARM{{Ord-mo-acyclic}}}
  \newcommand\invAordProp{\invARM{{Ord-Prop}}}
  \newcommand\invApropOrd{\invARM{{Prop-Ord}}}
  \newcommand\invAord{\invARM{{Ord-acyclic}}}
  \newcommand\invAevtTape{\invARM{{Evt-tape}}}
  \newcommand\invAtapeEvt{\invARM{{tape-Evt}}}
  \newcommand\invAtapeOrd{\invARM{{tape-Ord}}}
  \newcommand\invAReadWriteOne{\invARM{{Read-Write-1}}}
  \newcommand\invAReadWriteTwo{\invARM{{Read-Write-2}}}

  \newcommand\invPMP{\inv^{\Promise}_{\textsf{M-P}}}
  \newcommand\invPmessageView{\inv^{\Promise}_{\textup{\rm Message-View}}}

  \newcommand\armStepFetch{\overset{fetch \; \tId \; \cpath}{\armStepl}}
  \newcommand\armStepFetchPrime{\overset{fetch \; \tId' \; \cpath'}{\armStepl}}
  \newcommand\armStepProp{\overset{e \rightsquigarrow \tId}{\armStepl}}
  \newcommand\armStepPropPrime{\overset{e' \rightsquigarrow \tId'}{\armStepl}}
  \newcommand\armStepWritePending{\overset{\dashrightarrow \stRequestWrite{\tId}{ \cpath}{ x}{\stval}}{\armStepl}}
  \newcommand\armStepWritePendingPrime{\overset{\dashrightarrow \stRequestWrite{\tId'}{ \cpath'}{ y}{\stval'}}{\armStepl}}

  \newcommand\armStepReadRequest{\overset{\dashrightarrow \stRequestRead{\tId}{ \cpath}{ x}}{\armStepl}}
  \newcommand\armStepReadRequestPrime{\overset{\dashrightarrow \stRequestRead{\tId'}{ \cpath'}{ y}}{\armStepl}}
  \newcommand\armStepCondBranch{\overset{\dashrightarrow \tId, \cpath, \text{choose branch}}{\armStepl}}
  \newcommand\armStepCondBranchPrime{\overset{\dashrightarrow \tId', \cpath', \text{choose branch}}{\armStepl}}
  \newcommand\armStepFenceCommit{\overset{\dashrightarrow \stRequestFence{\tId}{ \cpath}}{\armStepl}}
  \newcommand\armStepFenceCommitPrime{\overset{\dashrightarrow \stRequestFence{\tId'}{ \cpath'}}{\armStepl}}
  \newcommand\armStepReadSatisfy{\overset{\dashrightarrow \stRequestRead{\tId}{ \cpath}{ x}, \stRequestWrite{\tId'}{ \cpath'}{ x}{\stval}}{\armStepl}}
  \newcommand\armStepReadSatisfyPrime{\overset{\dashrightarrow \stRequestRead{\tId'}{ \cpath'}{ y}, \stRequestWrite{\tId''}{ \cpath''}{ y}{\stval'}}{\armStepl}}
  \newcommand\armStepReadSatisfyFail{\overset{\not \dashrightarrow \stRequestRead{\tId}{ \cpath}{ x}, \stRequestWrite{\tId'}{ \cpath'}{ x}{\stval}}{\armStepl}}
  \newcommand\armStepReadSatisfyFailPrime{\overset{\not \dashrightarrow \stRequestRead{\tId'}{ \cpath'}{ y}, \stRequestWrite{\tId''}{ \cpath''}{ y}{\stval'}}{\armStepl}}
  \newcommand\armStepReadSatisfyInF{\overset{\dashrightarrow \stRequestRead{\tId}{ \cpath}{ x}, \stRequestWrite{\tId}{ \cpath'}{ x}{\stval}}{\armStepl}}
  \newcommand\armStepReadSatisfyInFPrime{\overset{\dashrightarrow \stRequestRead{\tId'}{ \cpath'}{ y}, \stRequestWrite{\tId'}{ \cpath''}{ y}{\stval'}}{\armStepl}}
  \newcommand\armStepReadCommit{\overset{\dashrightarrow \tId, \cpath, \text{read commit}}{\armStepl}}
  \newcommand\armStepReadCommitPrime{\overset{\dashrightarrow \tId', \cpath', \text{read commit}}{\armStepl}}

  \newcommand\armStepFetchP{\overset{\text{fetch} \; \tId \; \cpath}{\armStepP}}
  \newcommand\armStepFetchPrimeP{\overset{\text{fetch} \; \tId' \; \cpath'}{\armStepP}}
  \newcommand\armStepPropP{\overset{e \rightsquigarrow \tId}{\armStepP}}
  \newcommand\armStepPropPrimeP{\overset{e' \rightsquigarrow \tId'}{\armStepP}}
  \newcommand\armStepWritePendingP{\overset{\dashrightarrow \stRequestWrite{\tId}{ \cpath}{ x}{\stval}}{\armStepP}}
  \newcommand\armStepWritePendingPrimeP{\overset{\dashrightarrow \stRequestWrite{\tId'}{ \cpath'}{ y}{\stval'}}{\armStepP}}
  \newcommand\armStepReadRequestP{\overset{\dashrightarrow \stRequestRead{\tId}{ \cpath}{ x}}{\armStepP}}
  \newcommand\armStepReadRequestPrimeP{\overset{\dashrightarrow \stRequestRead{\tId'}{ \cpath'}{ y}}{\armStepP}}
  \newcommand\armStepCondBranchP{\overset{\dashrightarrow \tId, \cpath, \text{choose branch}}{\armStepP}}
  \newcommand\armStepCondBranchPrimeP{\overset{\dashrightarrow \tId', \cpath', \text{choose branch}}{\armStepP}}
  \newcommand\armStepFenceCommitP{\overset{\dashrightarrow \stRequestFence{\tId}{ \cpath}}{\armStepP}}
  \newcommand\armStepFenceCommitPrimeP{\overset{\dashrightarrow \stRequestFence{\tId'}{ \cpath'}}{\armStepP}}
  \newcommand\armStepReadSatisfyP{\overset{\dashrightarrow \stRequestRead{\tId}{ \cpath}{ x}, \stRequestWrite{\tId'}{ \cpath'}{ x}{\stval}}{\armStepP}}
  \newcommand\armStepReadSatisfyPrimeP{\overset{\dashrightarrow \stRequestRead{\tId'}{ \cpath'}{ y}, \stRequestWrite{\tId''}{ \cpath''}{ y}{\stval'}}{\armStepP}}
  \newcommand\armStepReadSatisfyFailP{\overset{\not \dashrightarrow \stRequestRead{\tId}{ \cpath}{ x}, \stRequestWrite{\tId'}{ \cpath'}{ x}{\stval}}{\armStepP}}
  \newcommand\armStepReadSatisfyFailPrimeP{\overset{\not \dashrightarrow \stRequestRead{\tId'}{ \cpath'}{ y}, \stRequestWrite{\tId''}{ \cpath''}{ y}{\stval'}}{\armStepP}}
  \newcommand\armStepReadSatisfyInFP{\overset{\dashrightarrow \stRequestRead{\tId}{ \cpath}{ x}, \stRequestWrite{\tId}{ \cpath'}{ x}{\stval}}{\armStepP}}
  \newcommand\armStepReadSatisfyInFPrimeP{\overset{\dashrightarrow \stRequestRead{\tId'}{ \cpath'}{ y}, \stRequestWrite{\tId'}{ \cpath''}{ y}{\stval'}}{\armStepP}}
  \newcommand\armStepReadCommitP{\overset{\dashrightarrow \tId, \cpath, \text{read commit}}{\armStepP}}
  \newcommand\armStepReadCommitPrimeP{\overset{\dashrightarrow \tId', \cpath', \text{read commit}}{\armStepP}}


\newcommand{\event}[3]{#1#2#3}
\tikzset{
   every path/.style={>=stealth},
   po/.style={->,color=brown,,shorten >=-0.5mm,shorten <=-0.5mm},
   rf/.style={->,color=green!60!black,dashed,,shorten >=-0.5mm,shorten <=-0.5mm},
   fr/.style={->,color=red,thick,shorten >=-0.5mm,shorten <=-0.5mm},
   mo/.style={->,color=orange!60!red,dotted,thick,shorten >=-0.5mm,shorten <=-0.5mm},
   no/.style={->,dotted,thick,shorten >=-0.5mm,shorten <=-0.5mm},
   deps/.style={->,color=violet,dotted,thick,shorten >=-0.5mm,shorten <=-0.5mm},
}

\newcommand{\IssuedSet}{I}
\newcommand{\issuable}{{\sf Issuable}}

%% \newcommand{\comment}[1]{\color{teal}{~~\texttt{/\!\!/}\textit{#1}}}

\newcommand{\lEID}{\lE_{tid}}
\newcommand{\TransSet}{{\rm Transitions}}
\newcommand{\LabelSet}{{\rm Labels}}

\newcommand{\pseudoCompileF}{{\rm pseudo\text{-}compile}}
\newcommand{\pseudoCompileReq}{{\rm pseudo\text{-}compile\text{-}req}}
\newcommand{\compileReq}{{\rm compile\text{-}req}}

\newcommand{\scRel}{{\rm sc\text{-}rel}}
\newcommand{\scRelF}{{\rm sc\text{-}rel\text{-}fun}}
\newcommand{\acqRel}{{\rm acq\text{-}rel}}
\newcommand{\acqRelF}{{\rm acq\text{-}rel\text{-}fun}}
\newcommand{\curRel}{{\rm cur\text{-}rel}}
\newcommand{\curRelF}{{\rm cur\text{-}rel\text{-}fun}}
\newcommand{\relRelFF}{{\rm rel\text{-}rel\text{-}fun}}
\newcommand{\relRel}{{\rm rel\text{-}rel}}
\newcommand{\msgRelF}{{\rm msg\text{-}rel\text{-}fun}}
\newcommand{\msgRel}{{\rm msg\text{-}rel}}
