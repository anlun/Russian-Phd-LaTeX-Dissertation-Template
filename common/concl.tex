%% Согласно ГОСТ Р 7.0.11-2011:
%% 5.3.3 В заключении диссертации излагают итоги выполненного исследования, рекомендации, перспективы дальнейшей разработки темы.
%% 9.2.3 В заключении автореферата диссертации излагают итоги данного исследования, рекомендации и перспективы дальнейшей разработки темы.
\begin{enumerate}
  \item Разработана операционная модель памяти C/C++11.
    Данная модель допускает такие же сценарии поведения, что и модель C/С++11 на большинстве
    тестов, приведенных в литературе, но не обладает сценариями поведения со ``значениями из воздуха''.
    В отличие от ``обещающей'' модели, предлагаемая модель является запускаемой, что упрощает
    разработку средств анализа программ для неё. Недостатком модели является то,
    она накладывает синтаксические ограничения на поведения программ.
  \item Доказана корректность компиляции из существенного подмножества ``обещающей'' модели в операционную модель
    памяти ARMv8 POP.
  \item Доказана корректность компиляции из существенного подмножества ``обещающей'' модели в
    аксиоматическую модель памяти ARMv8.3.
\end{enumerate}

В рамках \textbf{рекомендации по применению результатов работы} в индустрии и научных исследованиях указывается,
что модель памяти промышленного языка программирования должна быть лишена сценариев поведения, имеющих ``значения из воздуха'',
а также либо быть представленной в операционной форме, либо иметь эквивалентный операционный аналог.
Последнее позволяет реализовать интерпретатор модели и выполнять отладку программ в рамках модели.

Также были определены \textbf{перспективы дальнейшей разработки тематики}, основным из которых является
разработка обобщенной аксиоматической модели памяти для процессорных архитектур, которая будет
определена для синтаксиса модели C/C++11 и окажется строгим надмножеством существующих моделей памяти
x86, Power и ARM, а также для которой будет применим предложенный метод доказательства корректности компиляции
из ``обещающей'' модели памяти. Это позволит свести дальнейшие доказательства корректности компиляции из
``обещающей'' модели к доказательству корректности компиляции в обобщенную аксиоматическую модель, что
сводится к рассуждениям об ацикличности и вложенности путей на графах.
Кроме того, актуальной является задача разработки эффективной программной логики на базе логики многопоточного разделения
(CSL, concurrent separation logic) для операционного аналога модели памяти C/С++11 и ``обещающей'' модели памяти.
Такая логика позволит формально доказывать  в рамках моделей сложные свойства программ, такие как соответствие спецификации.
