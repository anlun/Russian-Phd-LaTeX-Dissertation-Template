%% Согласно ГОСТ Р 7.0.11-2011:
%% 5.3.3 В заключении диссертации излагают итоги выполненного исследования, рекомендации, перспективы дальнейшей разработки темы.
%% 9.2.3 В заключении автореферата диссертации излагают итоги данного исследования, рекомендации и перспективы дальнейшей разработки темы.
\begin{enumerate}
  \item Разработан операционный аналог модели памяти C/C++11.
    Данный аналог показывает такие же сценарии поведения, что и модель C/С++11, на большинстве
    тестов, приведенных в литературе, но не обладает сценариями поведения со значениями из воздуха.
    В отличие от ``обещающей'' семантики, предлагаемый аналог является запускаемым, что упрощает
    разработку средств анализа программ для него. Негативным отличием от ``обещающей'' семантики является то,
    предлагаемый аналог накладывает синтаксические ограничения на поведения программ.
  \item Доказана корректность компиляции из существенного подмножества ``обещающей'' семантики в операционную модель
    памяти ARMv8 POP.
  \item Доказана корректность компиляции из существенного подмножества ``обещающей'' семантики в
    аксиоматическую модель памяти ARMv8.3.
\end{enumerate}

В рамках \textbf{рекомендации по применению результатов работы} в индустрии и научных исследованиях указывается,
что модель памяти промышленного языка программирования должна быть лишена сценариев поведения, имеющих значения
из воздуха, а также либо быть представленной в операционной форме, либо иметь эквивалетный ей операционный аналог.
Последнее позволяет реализовать интерпретатор модели и выполнять отладку программ в рамках модели.

Также были определены \textbf{перспективы дальнейшей разработки тематики}, основным из которых является
разработка обобщенной аксиоматической модели памяти для процессорных архитектур, которая будет
определена для синтаксиса модели C/C++11 и будет строгим надмножеством существующих моделей памяти
x86, Power и ARM, а также для которой будет применим предложенный метод доказательства корректности компиляции
из ``обещающей'' модели памяти. Это позволит свести дальнейшие доказательства корректности компиляции из
``обещающей'' модели к доказательству корректности компиляции в обобщенную аксиоматическую модель, что
сводится к рассуждениям об ацикличности и вложенности путей на графах.

