\documentclass[14pt]{beamer}
\usepackage[T2A]{fontenc}
\usepackage[utf8]{inputenc}
\usepackage[english,russian]{babel}
\usepackage{amssymb,amsfonts,amsmath,mathtext}
\usepackage{cite,enumerate,float,indentfirst}

\graphicspath{{../images/}{images/}} 



% \usetheme[secheader]{Boadilla}
% \usecolortheme{seahorse}
\mode<presentation>
{
%% \usetheme{Frankfurt}
%% \useoutertheme{split}
	%% \usetheme{CambridgeUS}
	%% \usetheme{Hannover}
  \usetheme{Singapore}
	\usecolortheme{rose}
}

\beamertemplatenavigationsymbolsempty

\newcommand{\todo}{\alert}
%%% Основные сведения %%%
\newcommand{\thesisAuthor}             % Диссертация, ФИО автора
{%
    \texorpdfstring{% \texorpdfstring takes two arguments and uses the first for (La)TeX and the second for pdf
        Подкопаев Антон Викторович% так будет отображаться на титульном листе или в тексте, где будет использоваться переменная
    }{%
        Подкопаев, Антон Викторович% эта запись для свойств pdf-файла. В таком виде, если pdf будет обработан программами для сбора библиографических сведений, будет правильно представлена фамилия.
    }%
}
\newcommand{\thesisAuthorShort}        % Диссертация, ФИО автора инициалами
{А.В.~Подкопаев}

\newcommand{\thesisUdk}                % Диссертация, УДК
{\todo{xxx.xxx}}
\newcommand{\thesisTitle}              % Диссертация, название
{\texorpdfstring{\MakeUppercase{Операционные методы в приложении к слабым моделям памяти}}
                               {Операционные методы в приложении к слабым моделям памяти}}
\newcommand{\thesisSpecialtyNumber}    % Диссертация, специальность, номер
{\texorpdfstring{05.13.11}{05.13.11}}
\newcommand{\thesisSpecialtyTitle}     % Диссертация, специальность, название
{\texorpdfstring{Математическое и программное обеспечение вычислительных машин, комплексов и компьютерных сетей}
{Математическое и программное обеспечение вычислительных машин, комплексов и компьютерных сетей}}
\newcommand{\thesisDegree}             % Диссертация, ученая степень
{кандидата физико-математических наук}
\newcommand{\thesisDegreeShort}        % Диссертация, ученая степень, краткая запись
{канд. физ.-мат. наук}
\newcommand{\thesisCity}               % Диссертация, город написания диссертации
{Санкт-Петербург}
\newcommand{\thesisYear}               % Диссертация, год написания диссертации
{2018}
\newcommand{\thesisOrganization}       % Диссертация, организация
{Санкт-Петербургский государственный университет}
%% Правительство Российской Федерации \\ Федеральное государственное бюджетное образовательное учереждение высшего
%% профессионального образования \\
%% <<Санкт-Петербургский государственный университет>>}
\newcommand{\thesisOrganizationShort}  % Диссертация, краткое название организации для доклада
{СПбГУ}

\newcommand{\thesisInOrganization}     % Диссертация, организация в предложном падеже: Работа выполнена на ...
{кафедре системного программирования Санкт-Петербургского государственного университета}
%% \todo{учреждении, в~котором выполнялась данная диссертационная работа}

\newcommand{\supervisorFio}            % Научный руководитель, ФИО
{КОЗНОВ Дмитрий Владимирович}
\newcommand{\supervisorRegalia}        % Научный руководитель, регалии
{доктор технических наук, доцент, профессор кафедры системного программирования}
\newcommand{\supervisorFioShort}       % Научный руководитель, ФИО
{Д.В.~Кознов}
\newcommand{\supervisorRegaliaShort}   % Научный руководитель, регалии
{д.т.н., доц.}


\newcommand{\opponentOneFio}           % Оппонент 1, ФИО
{ГЕРГЕЛЬ Виктор Павлович}
\newcommand{\opponentOneRegalia}       % Оппонент 1, регалии
{доктор технических наук, профессор, заведующий кафедрой программной инженерии
федерального государственного автономного образовательного учреждения высшего образования
<<Нижегородский государственный университет им. Н.И. Лобачевского>>}
\newcommand{\opponentOneJobPlace}      % Оппонент 1, место работы
{}%{\todo{Не очень длинное название для места работы}}
\newcommand{\opponentOneJobPost}       % Оппонент 1, должность
{}%{\todo{старший научный сотрудник}}

\newcommand{\opponentTwoFio}           % Оппонент 2, ФИО
{}%{\todo{Фамилия Имя Отчество}}
\newcommand{\opponentTwoRegalia}       % Оппонент 2, регалии
{}%{\todo{кандидат физико-математических наук}}
\newcommand{\opponentTwoJobPlace}      % Оппонент 2, место работы
{}%{\todo{Основное место работы c длинным длинным длинным длинным названием}}
\newcommand{\opponentTwoJobPost}       % Оппонент 2, должность
{}%{\todo{старший научный сотрудник}}

\newcommand{\leadingOrganizationTitle} % Ведущая организация, дополнительные строки
{Федеральное государственное бюджетное учреждение науки Институт системного программирования Российской академии наук (ИСП РАН)}

\newcommand{\defenseDate}              % Защита, дата
{\_\_\_\_\_\_\_\_\_\_\_\_\_\_\_\_\_\_\_\_\_~г.~в~\_\_\_\_\_~часов}
\newcommand{\defenseCouncilNumber}     % Защита, номер диссертационного совета
{Д\,212.232.51}
\newcommand{\defenseCouncilTitle}      % Защита, учреждение диссертационного совета
{Санкт-Петербургского государственного университета}
\newcommand{\defenseCouncilAddress}    % Защита, адрес учреждение диссертационного совета
{198504, Санкт-Петербург, Старый Петергоф, Университетский пр. 28, математико-механический факультет СПбГУ, ауд. 405}
\newcommand{\defenseCouncilPhone}      % Телефон для справок
{\todo{+7~(0000)~00-00-00}}

\newcommand{\defenseSecretaryFio}      % Секретарь диссертационного совета, ФИО
{Демьянович Юрий Казимирович}
\newcommand{\defenseSecretaryRegalia}  % Секретарь диссертационного совета, регалии
{д.ф.-м.н., профессор}            % Для сокращений есть ГОСТы, например: ГОСТ Р 7.0.12-2011 + http://base.garant.ru/179724/#block_30000

\newcommand{\synopsisLibrary}          % Автореферат, название библиотеки
{Санкт-Петербургского государственного университета по адресу: 199034, Санкт-Петербург, Университетская наб., д. 7/9,
а также на сайте СПбГУ: \url{http://spbu.ru/science/disser/}}
\newcommand{\synopsisDate}             % Автореферат, дата рассылки
{\_\_ \_\_\_\_\_\_\_\_\_\_ 20\_\_ года}

% To avoid conflict with beamer class use \providecommand
\providecommand{\keywords}%            % Ключевые слова для метаданных PDF диссертации и автореферата
{}
      % Основные сведения

\setbeamercolor{footline}{}
\setbeamertemplate{footline}{
  \leavevmode%
  \hbox{%
  \begin{beamercolorbox}[wd=.333333\paperwidth,ht=2.25ex,dp=1ex,center]{}%
    % И. О. Фамилия, Организация кратко
    \thesisAuthorShort, \thesisOrganizationShort
  \end{beamercolorbox}%
  \begin{beamercolorbox}[wd=.333333\paperwidth,ht=2.25ex,dp=1ex,center]{}%
    % Город, 20XX
    \thesisCity, \thesisYear
  \end{beamercolorbox}%
  \begin{beamercolorbox}[wd=.333333\paperwidth,ht=2.25ex,dp=1ex,right]{}%
  %Стр.
  {\huge \insertframenumber{}} %из \inserttotalframenumber \hspace*{2ex}
  \end{beamercolorbox}}%
  \vskip0pt%
}

\newcommand{\itemi}{\item[\checkmark]}

%\title{\small{Название презентации}}
\title{\small{\thesisTitle}}
\author{\small{%
\emph{Выступающий:}~\thesisAuthorShort\\%
\emph{Руководитель:}~\supervisorRegaliaShort~\supervisorFioShort}\\%
\vspace{30pt}%
\thesisOrganization%
\vspace{20pt}%
}
\date{\small{\thesisCity, \thesisYear}}

\begin{document}

\maketitle

\begin{frame}
\frametitle{Цели и задачи}
\begin{itemize}
  \item \textbf{Предмет исследования:} 
  \item \textbf{Исследуемые характеристики:} 
  \item \textbf{Цель исследования:} 
  \item \textbf{Актуальность:} 
\end{itemize}
\end{frame}

\begin{frame}
\frametitle{Проблемы}
\begin{itemize}
  \item Проблема 1
  \item Проблема 2
  \item Проблема 3    
\end{itemize}
\end{frame}

\begin{frame}
\frametitle{План работ}
\begin{enumerate}
  \item \textbf{Задача 1}
  \begin{itemize}
    \item Подзадача 1-1
    \item Подзадача 1-2
  \end{itemize}
  \item \textbf{Задача 2}
  \begin{itemize}
    \item Подзадача 2-1
    \item Подзадача 2-2
    \item Подзадача 2-3
  \end{itemize}
  \item \textbf{Задача 3}
  \begin{itemize}
    \item Подзадача 3-1
    \item Подзадача 3-2
    \item Подзадача 3-3
  \end{itemize}
\end{enumerate}
\end{frame}

\begin{frame}
\frametitle{Список обыкновенный}
\begin{itemize}
  \item Пункт 1
  \item Пункт 2
  \item Пункт 3
\end{itemize}
\end{frame}

\begin{frame}
\frametitle{Одиночное изображение}
\begin{figure}[H]
  \center
  \includegraphics[width=0.8\linewidth]{latex}
\end{figure}
\end{frame}

\begin{frame}
\frametitle{Формулы}
$$
\left\{
  \begin{array}{rl}
    \dot x = & \sigma (y-x) \\
    \dot y = & x (r - z) - y \\
    \dot z = & xy - bz
  \end{array}
\right.
$$
\end{frame}

\begin{frame}
\frametitle{Составное изображение}
\begin{figure}[h]
  \begin{minipage}[h]{0.49\linewidth}
    \textbf{Составная \\ подпись 1}
    \center{\includegraphics[width=1\linewidth]{knuth1}}
  \end{minipage}
  \hfill
  \begin{minipage}[h]{0.49\linewidth}
    \textbf{Составная \\ подпись 2}
    \center{\includegraphics[width=1\linewidth]{knuth2}}
  \end{minipage}
\end{figure}
\end{frame}

\begin{frame}
\frametitle{Таблица}
\begin{tabular}{|l|l|}
\hline
\textbf{Заголовок 1} & \textbf{Заголовок 2} \\
\hline
Сумма & $b+a$ \\
\hline
Разность & $a-b$ \\
\hline
Произведение & $a*b$ \\
\hline
\end{tabular}
\end{frame}

\begin{frame}
\frametitle{Большой многоуровневый список}
\begin{itemize}
  \item \textbf{Пункт 1}
    \begin{itemize}
      \itemi Подпункт 1-1
      \itemi Подпункт 1-2
    \end{itemize}
  \item \textbf{Пункт 2}
    \begin{itemize}
      \itemi Подпункт 2-1
    \end{itemize}
  \item \textbf{Пункт 3}
    \begin{itemize}
      \itemi Подпункт 3-1
      \itemi Подпункт 3-2
    \end{itemize}
  \item \textbf{Пункт 4}
    \begin{itemize}
      \itemi Подпункт 4-1
    \end{itemize}
  \item \textbf{Пункт 5}
    \begin{itemize}
      \itemi Подпункт 5-1
      \itemi Подпункт 5-2
      \itemi Подпункт 5-3
    \end{itemize}
\end{itemize}
\end{frame}

\begin{frame}
\frametitle{Четыре изображения}
\begin{figure}[H]
  \center
    \includegraphics[width=0.4\linewidth]{latex}
    \includegraphics[width=0.4\linewidth]{latex}\\
    \includegraphics[width=0.4\linewidth]{latex}
    \includegraphics[width=0.4\linewidth]{latex}
\end{figure}
\end{frame}

%%%%%%%%%%%%%%%%%%%%%%%%%%%%%%
\begin{frame}
\frametitle{Перспективы развития проекта}
\begin{itemize}
  \item Перспектива 1
  \item Перспектива 2
  \item Перспектива 3
  \item Перспектива 4
  \item Перспектива 5
\end{itemize}
\end{frame}

\begin{frame}
\frametitle{Результаты работы}
\begin{itemize}
  \item Результат 1
  \item Результат 2
  \item Результат 3
  \item Результат 4
\end{itemize}
\pause
Спасибо за внимание!
\end{frame}

%% \begin{frame}
%% \begin{center}
%% \end{center}
%% \end{frame}

\end{document} 
